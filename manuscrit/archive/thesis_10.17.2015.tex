%============================================================================%
% Antoine Géré (gereantoine@gmail.com).
% Ph.D. thesis. (2013--2015)
% (TITLE) %%TODO TITLE
% --
% Università degli Studi di Genova.
% http://www.unige.it
% --
% Dipartimento di Matematica,
% Via Dodecaneso 35, 16146 Genova, Italia.
%============================================================================%
% LaTeX environment, Kile : available at http://kile.sourceforge.net/
% --
% Package : available at http://www.ctan.org/
% --
% The comprehensive latex symbole list : available at http://www.ctan.org/tex-archive/info/symbols/comprehensive/
% Detexif (an attempt to simplify the search in the latex symbole list) : available at http://detexify.kirelabs.org/classify.html
% --
% Bibliography with Zotero : available at http://www.zotero.org/
% Bibtex style ????? 
% --
% Latex font catalogue : available at http://www.tug.dk/FontCatalogue/
%============================================================================%


\documentclass[11pt]{book}


%----------------------------------------------------------------------------%


%a
\usepackage{amscd}
\usepackage{amsmath}
\usepackage{amsfonts}
\usepackage{amssymb}
\usepackage{amsxtra}
\usepackage{array} 
%b
\usepackage[english]{babel}
%c
\usepackage{calligra}
\usepackage{cite}
\usepackage{color}
%e
\usepackage{enumitem}
%f
\usepackage{fancyhdr}
\usepackage{filecontents}
\usepackage[T1]{fontenc}
%g
\usepackage{geometry}
%h
\usepackage{hyperref}
%i
\usepackage[indentunit=1em,columnsep=1.5\parindent,totoc]{idxlayout}
\usepackage[utf8]{inputenc}
%m
\usepackage{makeidx}
%n
\usepackage[amsmath,amsthm,thmmarks]{ntheorem}
%p
\usepackage{palatino}
%t
\usepackage{tikz}
%u
\usepackage{upgreek}
%w
\usepackage{wrapfig}


%----------------------------------------------------------------------------%


%%%%%%%%%%%%%%%%%%%%%%%COMMENTS%%%%%%%%%%%%%%%%%%%%%%
\usepackage[normalem]{ulem}
\def\bcom{{\color{red}\bf}\def\ecom{}}
\newcommand{\com}[1]{{\color{red}\bf #1}}
\newcommand{\sbar}[1]{\sout{\color{red} #1}}
%%%%%%%%%%%%%%%%%%%%%%%%%%%%%%%%%%%%%%%%%%%%%%%%%%%%%


%----------------------------------------------------------------------------%


%\begin{filecontents}{biblio.bib}


%\end{filecontents}


%----------------------------------------------------------------------------%


\geometry{
a4paper,
left=27mm,
right=27mm,
top=24mm,
bottom=24mm,
}


%----------------------------------------------------------------------------%


\abovedisplayskip=12pt plus 3pt minus 9pt
\abovedisplayshortskip=0pt plus 3pt
\belowdisplayskip=12pt plus 3pt minus 9pt
\belowdisplayshortskip=7pt plus 3pt minus 4pt

 
%----------------------------------------------------------------------------%


\setcounter{tocdepth}{8}  
\setcounter{secnumdepth}{8}
\setlength\parindent{8pt}
\pdfoptionpdfminorversion=6
\renewcommand{\headrulewidth}{0pt}
\renewcommand{\footrulewidth}{0pt}
\setlength{\headheight}{22pt} 
\pagestyle{fancy}
\renewcommand{\chaptermark}[1]{ \markboth{#1}{} }
\renewcommand{\sectionmark}[1]{ \markright{#1} }
\fancyhf{}
\fancyhead[LE,RO]{\thepage}
\fancyhead[RE,CE]{}
\fancyhead[LO,CO]{}
\fancypagestyle{plain}{%
\fancyhf{}
}%
\DeclareMathAlphabet{\mathcalligra}{T1}{calligra}{m}{n}
\makeindex


%----------------------------------------------------------------------------%


\newcommand{\supp}{\mathsf{supp}}
\newcommand{\singsupp}{\mathsf{singsupp}}
\newcommand{\WF}{\mathsf{WF}}
\newcommand{\id}{\mathsf{id}}
\newcommand{\loc}{\mathsf{loc}}
\newcommand{\reg}{\mathsf{reg}}
\newcommand{\pp}{\mathsf{pp}}
\newcommand{\ms}{\mathsf{ms}}
\newcommand{\sd}{\mathsf{sd}}
\newcommand{\vol}{\mathsf{vol}}
\newcommand{\tr}{\mathsf{tr}}
\renewcommand{\Re}{\mathsf{Re}}
\newcommand{\MS}{\textbf{MS}}
\renewcommand{\det}{\mathsf{det}}
\renewcommand{\sup}{\mathsf{sup}}
\renewcommand{\inf}{\mathsf{inf}}
\renewcommand{\exp}{\mathsf{exp}}
\newcommand{\alphabd}{\boldsymbol{\alpha}}
\newcommand{\betabd}{\boldsymbol{\beta}}
\renewcommand{\log}{\mathsf{log}}
\newcommand{\bigint}{\displaystyle\int}
\newcommand{\muc}{\mu\csf}


\newcommand{\abs}[1]{\left|#1\right|}
\newcommand{\norm}[1]{\left|\left|#1\right|\right|}
\newcommand{\sm}[1]{\left\langle#1\right\rangle}
\newcommand{\wick}[1]{:\!{#1}\!:}
\newcommand{\exte}[1]{\overset{\circ}{#1}}


%----------------------------------------------------------------------------%


\newcommand{\Acal}{\mathcal{A}}
\newcommand{\Bcal}{\mathcal{B}}
\newcommand{\Ccal}{\mathcal{C}}
\newcommand{\Dcal}{\mathcal{D}}
\newcommand{\Ecal}{\mathcal{E}}
\newcommand{\Fcal}{\mathcal{F}}
\newcommand{\Gcal}{\mathcal{G}}
\newcommand{\Hcal}{\mathcal{H}}
\newcommand{\Ical}{\mathcal{I}}
\newcommand{\Jcal}{\mathcal{J}}
\newcommand{\Kcal}{\mathcal{K}}
\newcommand{\Lcal}{\mathcal{L}}
\newcommand{\Mcal}{\mathcal{M}}
\newcommand{\Ncal}{\mathcal{N}}
\newcommand{\Ocal}{\mathcal{O}}
\newcommand{\Pcal}{\mathcal{P}}
\newcommand{\Qcal}{\mathcal{Q}}
\newcommand{\Rcal}{\mathcal{R}}
\newcommand{\Scal}{\mathcal{S}}
\newcommand{\Tcal}{\mathcal{T}}
\newcommand{\Ucal}{\mathcal{U}}
\newcommand{\Vcal}{\mathcal{V}}
\newcommand{\Wcal}{\mathcal{W}}
\newcommand{\Xcal}{\mathcal{X}}
\newcommand{\Ycal}{\mathcal{Y}}
\newcommand{\Zcal}{\mathcal{Z}}


%----------------------------------------------------------------------------%


\newcommand{\Abb}{\mathbb{A}}
\renewcommand{\Bbb}{\mathbb{B}}
\newcommand{\Cbb}{\mathbb{C}}
\newcommand{\Dbb}{\mathbb{D}}
\newcommand{\Ebb}{\mathbb{E}}
\newcommand{\Fbb}{\mathbb{F}}
\newcommand{\Gbb}{\mathbb{G}}
\newcommand{\Hbb}{\mathbb{H}}
\newcommand{\Ibb}{\mathbb{I}}
\newcommand{\Jbb}{\mathbb{J}}
\newcommand{\Kbb}{\mathbb{K}}
\newcommand{\Lbb}{\mathbb{L}}
\newcommand{\Mbb}{\mathbb{M}}
\newcommand{\Nbb}{\mathbb{N}}
\newcommand{\Obb}{\mathbb{O}}
\newcommand{\Pbb}{\mathbb{P}}
\newcommand{\Qbb}{\mathbb{Q}}
\newcommand{\Rbb}{\mathbb{R}}
\newcommand{\Sbb}{\mathbb{S}}
\newcommand{\Tbb}{\mathbb{T}}
\newcommand{\Ubb}{\mathbb{U}}
\newcommand{\Vbb}{\mathbb{V}}
\newcommand{\Wbb}{\mathbb{W}}
\newcommand{\Xbb}{\mathbb{X}}
\newcommand{\Ybb}{\mathbb{Y}}
\newcommand{\Zbb}{\mathbb{Z}}


%----------------------------------------------------------------------------%


\newcommand{\Arak}{\mathfrak{A}}
\newcommand{\Brak}{\mathfrak{B}}
\newcommand{\Crak}{\mathfrak{C}}
\newcommand{\Drak}{\mathfrak{D}}
\newcommand{\Erak}{\mathfrak{E}}
\newcommand{\Frak}{\mathfrak{F}}
\newcommand{\Grak}{\mathfrak{G}}
\newcommand{\Hrak}{\mathfrak{H}}
\newcommand{\Irak}{\mathfrak{I}}
\newcommand{\Jrak}{\mathfrak{J}}
\newcommand{\Krak}{\mathfrak{K}}
\newcommand{\Lrak}{\mathfrak{L}}
\newcommand{\Mrak}{\mathfrak{M}}
\newcommand{\Nrak}{\mathfrak{N}}
\newcommand{\Orak}{\mathfrak{O}}
\newcommand{\Prak}{\mathfrak{P}}
\newcommand{\Qrak}{\mathfrak{Q}}
\newcommand{\Rrak}{\mathfrak{R}}
\newcommand{\Srak}{\mathfrak{S}}
\newcommand{\Trak}{\mathfrak{T}}
\newcommand{\Urak}{\mathfrak{U}}
\newcommand{\Vrak}{\mathfrak{V}}
\newcommand{\Wrak}{\mathfrak{W}}
\newcommand{\Xrak}{\mathfrak{X}}
\newcommand{\Yrak}{\mathfrak{Y}}
\newcommand{\Zrak}{\mathfrak{Z}}


\newcommand{\arak}{\mathfrak{a}}
\newcommand{\brak}{\mathfrak{b}}
\newcommand{\crak}{\mathfrak{c}}
\newcommand{\drak}{\mathfrak{d}}
\newcommand{\erak}{\mathfrak{e}}
\renewcommand{\frak}{\mathfrak{f}}
\newcommand{\grak}{\mathfrak{g}}
\newcommand{\hrak}{\mathfrak{h}}
\newcommand{\irak}{\mathfrak{i}}
\newcommand{\jrak}{\mathfrak{j}}
\newcommand{\krak}{\mathfrak{k}}
\newcommand{\lrak}{\mathfrak{l}}
\newcommand{\mrak}{\mathfrak{m}}
\newcommand{\nrak}{\mathfrak{n}}
\newcommand{\orak}{\mathfrak{o}}
\newcommand{\prak}{\mathfrak{p}}
\newcommand{\qrak}{\mathfrak{q}}
\newcommand{\rrak}{\mathfrak{r}}
\newcommand{\srak}{\mathfrak{s}}
\newcommand{\trak}{\mathfrak{t}}
\newcommand{\urak}{\mathfrak{u}}
\newcommand{\vrak}{\mathfrak{v}}
\newcommand{\wrak}{\mathfrak{w}}
\newcommand{\xrak}{\mathfrak{x}}
\newcommand{\yrak}{\mathfrak{y}}
\newcommand{\zrak}{\mathfrak{z}}


%----------------------------------------------------------------------------%


\newcommand{\Asf}{\mathsf{A}}
\newcommand{\Bsf}{\mathsf{B}}
\newcommand{\Csf}{\mathsf{C}}
\newcommand{\Dsf}{\mathsf{D}}
\newcommand{\Esf}{\mathsf{E}}
\newcommand{\Fsf}{\mathsf{F}}
\newcommand{\Gsf}{\mathsf{G}}
\newcommand{\Hsf}{\mathsf{H}}
\newcommand{\Isf}{\mathsf{I}}
\newcommand{\Jsf}{\mathsf{J}}
\newcommand{\Ksf}{\mathsf{K}}
\newcommand{\Lsf}{\mathsf{L}}
\newcommand{\Msf}{\mathsf{M}}
\newcommand{\Nsf}{\mathsf{N}}
\newcommand{\Osf}{\mathsf{O}}
\newcommand{\Psf}{\mathsf{P}}
\newcommand{\Qsf}{\mathsf{Q}}
\newcommand{\Rsf}{\mathsf{R}}
\newcommand{\Ssf}{\mathsf{S}}
\newcommand{\Tsf}{\mathsf{T}}
\newcommand{\Usf}{\mathsf{U}}
\newcommand{\Vsf}{\mathsf{V}}
\newcommand{\Wsf}{\mathsf{W}}
\newcommand{\Xsf}{\mathsf{X}}
\newcommand{\Ysf}{\mathsf{Y}}
\newcommand{\Zsf}{\mathsf{Z}}


\newcommand{\asf}{\mathsf{a}}
\newcommand{\bsf}{\mathsf{b}}
\newcommand{\csf}{\mathsf{c}}
\newcommand{\dsf}{\mathsf{d}}
\newcommand{\esf}{\mathsf{e}}
\newcommand{\fsf}{\mathsf{f}}
\newcommand{\gsf}{\mathsf{g}}
\newcommand{\hsf}{\mathsf{h}}
\newcommand{\isf}{\mathsf{i}}
\newcommand{\jsf}{\mathsf{j}}
\newcommand{\ksf}{\mathsf{k}}
\newcommand{\lsf}{\mathsf{l}}
\newcommand{\msf}{\mathsf{m}}
\newcommand{\nsf}{\mathsf{n}}
\newcommand{\osf}{\mathsf{o}}
\newcommand{\psf}{\mathsf{p}}
\newcommand{\qsf}{\mathsf{q}}
\newcommand{\rsf}{\mathsf{r}}
\newcommand{\ssf}{\mathsf{s}}
\newcommand{\tsf}{\mathsf{t}}
\newcommand{\usf}{\mathsf{u}}
\newcommand{\vsf}{\mathsf{v}}
\newcommand{\wsf}{\mathsf{w}}
\newcommand{\xsf}{\mathsf{x}}
\newcommand{\ysf}{\mathsf{y}}
\newcommand{\zsf}{\mathsf{z}}


%----------------------------------------------------------------------------%


\newcommand*{\makepagetitle}{%
%
{\raggedright% 
%
%
%
%
\thispagestyle{empty}%
%
\vspace*{50pt}
%
{\Large Antoine Géré}\\% 
%
\vspace*{120pt}%
%
{\Huge\bfseries TITLE}\\[\baselineskip]% %%TODO TITLE
%
\vspace*{60pt}%
%
{\Large Ph.D. thesis}\\[\baselineskip]% 
%
\vspace*{80pt}%
%
{\Large Dipartimento di Matematica}\\[\baselineskip]% 
%
\vspace*{1pt}
%
{\Large Università degli Studi di Genova}\\[\baselineskip]% 
%
\vfill% 
%
%
\newpage%
%
\thispagestyle{empty}%
%
\ \vfill%
%
%
\textbf{TITLE} \\[2pt] %%TODO TITLE
Ph.D. thesis submitted by \href{mailto:gere@dima.unige.it}{Antoine Géré} \\[1pt]
\href{http://www.comune.genova.it/}{Genova}, ???? 2016 \\[10pt]
%
%
\begin{minipage}{0.1\linewidth}
\includegraphics[scale=1]{fig_unige.pdf}
% unige.pdf: 29x39 pixel, 72dpi, 1.02x1.38 cm, bb=0 0 29 39
\end{minipage}
%
\begin{minipage}{0.85\linewidth}
\href{http://www.dima.unige.it/}{Dipartimento di Matematica} \\[1pt]
\href{http://www.unige.it/}{Università degli Studi di Genova}
\end{minipage}
%
%
\vspace*{10pt} \\
Supervisor: \href{mailto:pinamont@dima.unige.it}{Prof. Dr. Nicola Pinamonti} \\[1pt]
%
Examiner: \href{mailto:????@????.com}{????}
%
%
%
%
}%
%
}%


%----------------------------------------------------------------------------%


\theoremclass{LaTeX}
\theoremstyle{break}
\theoremheaderfont{\normalfont\bfseries}
\theorembodyfont{\normalfont}
\theoremseparator{}
\theoremsymbol{\ensuremath{\blacktriangleright}}
\newtheorem{theorem}{Theorem}[chapter]
\newtheorem{proposition}{Proposition}
\newtheorem{lemma}{Lemma}[chapter]
\newtheorem{corollary}{Corollary}[chapter]
\theoremsymbol{\ensuremath{\blacklozenge}}
\newtheorem{example}{Example}[chapter]
\newtheorem{remark}{Remark}[chapter]
\newtheorem{definition}{Definition}[chapter]
\theoremsymbol{\ensuremath{\blacksquare}}
\renewtheorem{proof}{Proof}[chapter]
\qedsymbol{\ensuremath{_\blacksquare}}
\newtheorem{sketch}{Sketch of the proof}[chapter]
\theoremsymbol{\ensuremath{_\blacksquare}}


%----------------------------------------------------------------------------%


\definecolor{hypercolor}{rgb}{0.1,0.2,0.6}


\hypersetup{     
pdftoolbar=true,    
pdfmenubar=true,    
pdffitwindow=true,  
pdfstartview={FitH},
pdftitle={PhD thesis},    
pdfauthor={Antoine Géré},     
pdfsubject={Mathematical Physics},
pdfcreator={LaTeX},  
pdfproducer={pdfTex},
pdfkeywords={Quantum Field Theory},  
pdfnewwindow=true,  
colorlinks=true, 
linkcolor=hypercolor, 
urlcolor=hypercolor, 
citecolor=hypercolor,
filecolor=hypercolor,         
}


%============================================================================%
\begin{document}
%============================================================================%


\pagenumbering{arabic}


\makepagetitle

%----------------------------------------------------------------------------%


\newpage


\ \vfill


(blablabla)



\vfill


%----------------------------------------------------------------------------%


\newpage


\vspace*{100pt}


\section*{Abstract}


%----------------------------------------------------------------------------%


(blablabla)


%----------------------------------------------------------------------------%


\tableofcontents


%----------------------------------------------------------------------------%


\chapter*{Introduction}


\addcontentsline{toc}{chapter}{Introduction}


%----------------------------------------------------------------------------%


(blablabla)


\vspace*{50pt}
\noindent
\textbf{Comments}


\begin{enumerate}


\item We should use another symbol when the $=$ is used to define something. A possibility is $\doteq$ or $:=$.


\item In chapter 1 there is a bit of confusion between ``tensor fields'' and tensors over a point of a manifold. It has to be clarified.


\item Improve section \ref{p:FLRW}.


\item Complte the proofs.


\item notice that it has in general meaningless to write $u(x)$ unless the distribution is described by an ordinary function. Despite of this fact the notation $u(x)$ is sometime useful.... you should be more precise on this point...


\end{enumerate}


%----------------------------------------------------------------------------%
\chapter{Spacetime}
\label{p:SPACETIME}
%----------------------------------------------------------------------------%



The starting block for a physical theory is the notion of spacetime, it is a set of points (events) located in time and space. 


In Newton physics, space and time are usually treated separately and, moreover, their role encoded in the axioms of absolute time and of absolute space is passive. Actually this two axioms simply describe the ambient where everything happens. At the beginning of the last century Einstein and others introduced a completely new point of view of these two entities.


Actually, already with the theory of special relativity introduced by Einstein, space and time are treated on the same level, namely as elements of an unique entity. This new point of view has been an important turning point in the understanding of ``the laws of nature''.


After that, in the theory of General Relativity, which is nowadays used to describe gravitational interactions, the physical background (i.e. the spacetime) is an ``active actor''. Indeed gravitation roughly speaking can be viewed as a deformation of the spacetime. Therefore we shall introduce the notion of spacetime starting from the very beginning.


We shall introduce the mathematical framework to describe the spacetime. First we shall define what is a manifold starting from topological considerations, then implement differential structures, and finally equipe the manifold with a causality.


%----------------------------------------------------------------------------%
\section{From topology to manifold}
%----------------------------------------------------------------------------%


The most fundamental way to define a space is to use the notion of topology. It permits to study spaces that are preserved under different type of deformations. We choose to start with the topological structure in oreder to define the notion of manifold.


\begin{definition}[Topological space] \index{topological space}
Let $\Xsf$ be a set\footnote{It is simply a collection of objects.}. A topology\index{topology} on $\Xsf$ is a collection $\Tcal$ of subsets satisfying the three following axioms,%
%
\begin{itemize}
\item \textbf{conventions} : $\emptyset , \ \Xsf \in \Tcal$ ;
\item \textbf{arbitrary union} : $U_i \in \Tcal \mbox{ for } i \in I \Longrightarrow \bigcup_{i\in I} U_i \in \Tcal$ ;
\item \textbf{finite intersection} : $U_1 , \dots , U_n \in \Tcal \Longrightarrow U_1 \cap \dots \cap U_n \in \Tcal$ ,
\end{itemize}
%
where $I$ is an arbitrary index set.
\end{definition}
%
The pair $(\Xsf,\Tcal)$ is called a \textbf{topological space}. The element of $\Tcal$ are the open sets of $\Xsf$. We shall often omit to specify the topology $\Tcal$, and simply say that $\Xsf$ is a topological space. 


\bigskip


We shall illustrate this definitions by few examples. If $\Xsf$ is a set and $\Tcal$ is the collection of all the subsets of X , then $(\Xsf,\Tcal)$ is a topological space and this topology is called the \textbf{discrete topology}. When $\Tcal$ is just the empty set and $\Xsf$ is the entire space, then $(\Xsf,\Tcal)$ is called the \textbf{trivial topology}. In the case of $\Xsf$ corresponding to the real line $\Rbb$ all open intervals $(a,b)$ and their unions define a topology called the \textbf{usual topology}. 


\bigskip


The notion of topology is still general. It does not implement a lot of structureon theses spaces. For instance the operations between elements in $\Xsf$ are not for now considered. Nonetheless it already characterizes maps between different topological spaces.


\begin{definition}[Continuous map and homeomorphism]\index{continuous map}\index{homeomorphism}
%
Let $\Xsf$ and $\Ysf$ be topological spaces. We consider a map $f : \Xsf \to \Ysf$. We say
%
\begin{itemize}
\item $f$ is \textbf{continuous} if $f^{-1}(U) \subset X$ is open for every open $U \subset\Ysf$ ;
\item $f$ is a \textbf{homeomorphism} if $f$ is bijective and both $f$ and $f^{-1}$ are continuous.
\end{itemize}
%
\end{definition}


If we consider two continuous maps $f : \Xsf \to \Ysf$ and $g : \Ysf \to \Zsf$ between topological spaces, then $f \circ g$ is continuous. An example of a map which is not continuous is a function a function from $\Rbb$ to $\Rbb$.
%
\begin{equation*}
f(x) = \left\{
\begin{array}{ll}
x & \mbox{ if } \ x \leq 0 \ , \\
x + 2 & \mbox{ if } \ x > 0 \ .
\end{array}
\right.
\end{equation*}
%
In ``usual'' calculus we say that $f$ is discontinuous in $0$. Let us check that this is also the case with the previous definition. The inverse by $f$ of the open $(-1,1)$ is the set $(-1,0]$ which is not an open anymore. Thus the function $f$ is not continuous.


Notice that the previous definition permits to show that any open interval of $\Rbb$ is homeomorphic to any other open interval. In order to prove it consider the function $g : (-1,1) \to (0,2)$ such as
%
\begin{equation*}
g(x) = x + 1 \ . 
\end{equation*}
%
This function is bijective and, $g$ and $g^{-1}$ are continuous. Another possible example for a homeomorphism is the function tangent hyperbolic which maps $\Rbb$ to $(-1,1)$.


\bigskip


Suppose to have two different continuous functions $f$ and $g$, which map a topological space $\Xsf$ to another topological space $\Ysf$, and a continuous function $h : \Xsf \times [0,1] \to \Ysf$, with  $h(x,0) = f(x)$ and $h(x,1) = g(x)$ for all $x \in \Xsf$, we say that $h$ is an \textbf{homotopy}\index{homotopy}. 

\bigskip


Let us give some generic definitions which shall appear to be useful later on. Below, we shall denote by $\Xsf$ a topological space by $\Tcal$ its topology. Furthermore, $\Zsf$ is a subspace of $\Xsf$.


\begin{itemize}
%
\item $\overline{\Zsf}$ is the \textbf{closure}\index{closure} of $\Zsf \subset \Xsf$, which is obtained as the intersection of all closed sets\footnote{A set $\Csf \subset \Xsf$ is \textbf{closed} if $\Xsf \setminus \Csf$ is open.} containing $\Zsf$.% 
%
\item $\Zsf$ is \textbf{dense}\index{dense} in $\Xsf$ if $\overline{\Zsf} = \Xsf$.%
%
\item $\Bcal \subset \Tcal$ is a \textbf{topological basis}\index{topological basis} if every elements in $\Tcal$ can be written as the union of elements of the basis $\Bcal$.%
%
\item $\Xsf$ is \textbf{second countable}\index{second countable} if it has a countable\footnote{A set is said to be \textbf{countable}\index{countable} if there exists a one to one correspondence between the set considered and the set of natural numbers.} topological basis.%
%
\item $\Xsf$ is \textbf{separable}\index{separable} if there exists a dense and countable subset.%
%
\item $\Ksf \subset \Xsf$ is \textbf{compact}\index{compact} if any covering of it admits a finite subcovering.%
%
\item $\Xsf$ is \textbf{locally compact}\index{locally compact} if every point in $\Xsf$ admits a neighborhood which has compact closure.%
%
\item $\Xsf$ is \textbf{connected}\index{connected} if it cannot be written as a disjoint union of two nonempty open subsets.%
%
\item $\Xsf$ is \textbf{Hausdorff}\index{Hausdorff} if every pair of points have disjoint neighborhoods.%
%
\item $\Xsf$ is \textbf{paracompact}\index{paracompact} if every open cover has a refinement\footnote{A refinement $C$ of a cover $\Ccal$ is a cover such that every element in $C$ is a subset of an element in $\Ccal$.} covering that is locally finite\footnote{A cover $\left\{U_\alpha\right\}$ of $\Xsf$ is \textbf{locally finite} if every points in $\Xsf$ has a neighborhood which has a nonempty intersection with a finite numbers of $U_\alpha$.}.% 
%
\item $\Xsf$ is said to be \textbf{contractible}\index{contractible} if the identity map $\Xsf \to \Xsf$ is homotopic to a constant map.
\end{itemize}


In order to restrict ourselves to a less general picture, we require that the topological spaces we are considering satisfy some separability (Hausdorff) and countability (second countable) condition, so that they look locally like $\Rbb^n$. We have now enough background to introduce the notion of manifold. We start with topological manifolds and we shall implement the differential structure in the next section.%


\bigskip


\begin{definition}[Topological manifold]\index{manifold!topological}
A topological manifold $\Mcal$ of dimension $n$ is a Hausdorff and second countable topological space in which every points admits an open neighborhood homeomorphic to a subset of $\Rbb^n$.
\end{definition}


It is possible to show that instead of requiring $\Mcal$ to be Hausdorff and second countable, we could equivalently demand $\Mcal$ to be separable and metrizable. However, these last mentioned conditions are global extents, while what is really important for describing physical theories, is the requirement that $\Mcal$ admit locally the same topological properties as $\Rbb^n$, or in other words that  $\Mcal$ is locally compact, connected, and contractible.


An important tool to pass from a local to a global point of view is the \textbf{partition of unity} of $\Mcal$. 


\begin{definition}[Partition of unity]\index{partition of unity}
Let $\Xsf$ be a topological space. A partition of unity on $\Xsf$ is a collection $\{g_i\}$ of continuous functions $g_i : \Xsf \to [0,1]$ such that
%
\begin{itemize}
\item $\sum_i g_i(x) = 1$, for all $x \in \Xsf$ ,
\item for every $x \in \Xsf$, there is only a finite number of maps $g_i$ such that $g_i(x) \neq 0$.
\end{itemize}
%
\end{definition}


A partition of unity defines an open cover of $\Xsf$. We call, $\{g_i\}_{i \in I}$, \textbf{partition of unity subordinate to an open cover}\index{partition of unity!subordinate to an open cover} $U=(U_i)_{i \in I}$, if for all $i \in I$, the support of $g_i$ is contained in $U_i$. And we have that a topological manifold $\Mcal$ is paracompact if and only if it admit a partition of unity subordinate to every open cover of $\Mcal$. This is the reason why we shall later require to work with paracompact manifold.


%----------------------------------------------------------------------------%
\section{Lorentzian manifold}
%----------------------------------------------------------------------------%


%----------------------------------------------------------------------------%
\subsection{Smoothness}
%----------------------------------------------------------------------------%


We concluded the previous section introducing the notion of \textbf{topological manifold}. Now we would like to implement a \textbf{differential structure} on it in order to be able for instance to define the notion of tangent space. In the next, we shall restrict our analysis to the case of smooth manifold. This particular case shall appear to be enough to introduce the notion of Lorentzian manifold.


\bigskip


Let $\Mcal$ be a topological manifold of dimension $n$. We would like to be able to ``localize'' points on a manifold, to this end we shall introduce the notion of chart. A \textbf{chart}\index{chart} on $\Mcal$ is a pair $(U,\Phi)$, where $U$ is an open subset of $\Mcal$ and $\Phi$ is a homeomorphism from $U$ to an open subset $\Phi(U) \subset \Rbb^n$. We call $U$ a \textbf{coordinate neighborhood}\index{coordinate neighborhood}, $\Phi$ a \textbf{coordinate map}\index{coordinate map}, and the component functions of $\Phi=(\Phi_1,\dots,\Phi_n)$ \textbf{local coordinates}\index{local coordinates} on $U$.


\begin{figure}[ht]
\begin{center}
\begin{tikzpicture}[thick,scale=0.7] 
%
%
\draw[color=black] (0,0) to [out=50,in=190] (6,3);
\draw[color=black] (6,3) to [out=10,in=90] (10,0);
\draw[color=black] (10,0) to [out=170,in=30] (3,-3);
\draw[color=black] (3,-3) to [out=90,in=10] (0,0);
%
\filldraw (6.5,2.2) circle (0pt) node[above] {$\Mcal$}; 
%
\def\firstellipse{(4,0.5) ellipse (1.6 and 0.6)};
\def\secondellipse{(6,1) ellipse (1.6 and 0.6)};
\draw[color=black] \firstellipse \secondellipse;
%
\filldraw (3.5,1) circle (0pt) node[above] {$U$}; 
\filldraw (8,01) circle (0pt) node[above] {$V$}; 
%
\begin{scope}
\clip \firstellipse;
\fill[white!40!black] \secondellipse;
\end{scope}
%
%
\draw[dashed] (-3,-7) -- (-1,-5) -- (3,-5) -- (1,-7) -- (-3,-7);
\filldraw (1,-7) circle (0pt) node[below] {$\Rbb^n$}; 
\draw[color=black] (0,-6) ellipse (1.6 and 0.6);
\filldraw (-1.8,-7.3) circle (0pt) node[above] {$\Phi(U)$}; 
%
%
\draw[dashed] (7,-7) -- (9,-5) -- (13,-5) -- (11,-7) -- (7,-7);
\filldraw (11,-7) circle (0pt) node[below] {$\Rbb^n$}; 
\draw[color=black] (10,-6) ellipse (1.6 and 0.6); 
\filldraw (8.2,-7.3) circle (0pt) node[above] {$\Psi(V)$};
%
%
\draw[->,dashed,color=black] (1.5,-1.5) -- (0.5,-4.5);
\filldraw (1,-3.2) circle (0pt) node[right] {$\Phi$}; 
%
%
\draw[->,dashed,color=black] (7.5,-1) -- (9.5,-4.5);
\filldraw (8.9,-3.2) circle (0pt) node[right] {$\Psi$}; 
%
%
\draw[->,dashed,color=black] (3,-6) -- (7,-6);
\filldraw (5,-6) circle (0pt) node[above] {$\Psi \circ \Phi^{-1}$}; 
%
%
\end{tikzpicture}
\end{center}
\caption{Charts and transition function.}
\end{figure}


We want to define smooth manifold, therefore we need to define one more notion. Let $(\Phi,U)$ and $(\Psi,V)$ be two charts on $\Mcal$ such that $U \cap V \neq \emptyset$. We call \textbf{transition map}\index{transition map}, the application
%
\begin{equation*}
\Psi \circ \Phi^{-1} \ : \ \Phi(U \cap V) \subset \Rbb^n \ \to \ \Psi(U \cap V) \subset \Rbb^n \ .
\end{equation*}
%
Since $\Mcal$ is a topological manifold, the previous map is surely a homeomorphism. We say that $(\Phi,U)$ and $(\Psi,V)$ are \textbf{smoothly compatible}\index{smoothly compatible} if $U \cap V = \emptyset$ or if furthermore the transition map $\Psi \circ \Phi^{-1}$ is a diffeomorphism\index{diffeomorphism}, i.e. bijective with smooth inverse.


%\bigskip


We call an \textbf{atlas}\index{atlas} of $\Mcal$ a set of chart $\left\{ (U, \Phi) \right\}$ which covers $\Mcal$. An atlas $\Acal$ is called \textbf{smooth} \index{smooth atlas} if any two charts in $\Acal$ are smoothly compatible. A smooth atlas $\Acal$ is called \textbf{maximal} \index{maximal smooth atlas} if it is not contained in any strictly larger smooth atlas. We now can give the definition of a smooth manifold.


\begin{definition}[Smooth manifold]\index{manifold!smooth}
$\Mcal$ is a smooth manifold if it is a topological manifold with a smooth maximal atlas $\left\{(U,\phi)\right\}$.
\end{definition}


One of the first characterization of a smooth manifold $\Mcal$ we can give is its \textbf{orientability}\index{orientability}. An \textbf{orientation}\index{orientation} of a smooth manifold is the choice of a maximal smooth oriented atlas. A smooth atlas $\{(U,\Phi)\}$ is called oriented if the determinant of the derivatives of all transition maps is positive.


It shall appear useful to define smooth maps between manifolds. We shall also characterize real valued maps on smooth manifolds, and say on which conditions they are smooth and compactly supported.


\begin{definition}[Smooth map]\index{smooth map}
A map $f : \Mcal \to \Ncal$ between two smooth manifolds is said to be smooth if for every charts $(U,\Phi)$ and $(V,\Psi)$ on $\Mcal$ and $\Ncal$ respectively, the transition function $\Psi \circ f \circ \Phi^{-1}$ is  smooth.
\end{definition}


We notice in particular that if the two manifolds are of the same dimension then $f$ is said to be a \textbf{diffeomorphism}\index{diffeomorphism}.


\begin{figure}[ht!]
\begin{center}
\begin{tikzpicture}[thick,scale=0.5] 
\draw[color=black] (0,0) to [out=50,in=190] (6,3);
\draw[color=black] (6,3) to [out=10,in=90] (10,0);
\draw[color=black] (10,0) to [out=170,in=30] (3,-3);
\draw[color=black] (3,-3) to [out=90,in=10] (0,0);
\draw[color=black] (5.2,1) ellipse (2 and 1);
\filldraw (10,0) circle (0pt) node[right] {$\Mcal$}; 
\filldraw (3,0) circle (0pt) node[above] {$U$};
\filldraw (5,1) circle (2pt) node[above] {$x$};
%
\draw[color=black] (15,0) to [out=50,in=190] (21,3);
\draw[color=black] (21,3) to [out=10,in=90] (25,0);
\draw[color=black] (25,0) to [out=170,in=30] (18,-3);
\draw[color=black] (18,-3) to [out=90,in=10] (15,0);
\draw[color=black] (20.2,1) ellipse (2 and 1);
\filldraw (25,0) circle (0pt) node[right] {$\Ncal$}; 
\filldraw (18,0) circle (0pt) node[above] {$V$}; 
\filldraw (20,1) circle (2pt) node[above] {$f(x)$};
%
\draw[->,color=black,dashed] (5,-2.5) -- (5,-4.5);
\filldraw (5,-3.5) circle (0pt) node[right] {$\Phi$}; 
%
\draw[dashed] (1,-7) -- (3,-5) -- (9,-5) -- (7,-7) -- (1,-7);
\draw[color=black] (5,-6) ellipse (1.6 and 0.6);
\filldraw (9,-5) circle (0pt) node[right] {$\Rbb^n$}; 
\filldraw (2.5,-7) circle (0pt) node[above] {$\Phi(U)$}; 
%
\draw[->,color=black,dashed] (20,-2.5) -- (20,-4.5);
\filldraw (20,-3.5) circle (0pt) node[right] {$\Psi$}; 
%
\draw[dashed] (16,-7) -- (18,-5) -- (24,-5) -- (22,-7) -- (16,-7);
\draw[color=black] (20,-6) ellipse (1.6 and 0.6);
\filldraw (24,-5) circle (0pt) node[right] {$\Rbb^n$}; 
\filldraw (17.5,-7) circle (0pt) node[above] {$\Psi(V)$}; 
%
\draw[->,color=black,dashed] (9,-6) -- (16,-6);
\filldraw (12.5,-6) circle (0pt) node[above] {$\Psi \circ f \circ \Phi^{-1}$}; 
%
\draw[->,color=black,dashed] (10,2) -- (16,2);
\filldraw (13,2) circle (0pt) node[above] {$f$}; 
%
\end{tikzpicture}
\end{center}
\caption{Smooth map between manifolds.}
\end{figure}


We shall now proceed defining what is a real valued smooth (compactly supported) function.


\begin{definition}[Smooth - compactly supported - function]
A function $f : \Mcal \to \Rbb$ is said to be \textbf{smooth}\index{smooth function} if and only if $f \circ \Phi^{-1} : \Phi(U) \subset \Rbb^n \to f(U) \subset \Rbb$ is smooth for each coordinate chart in the atlas.\par%
A function $f : \Mcal \to \Rbb$ is said to be \textbf{compactly supported}\index{compactly supported function} if the support of $f : \Mcal \to \Rbb$ (i.e. the closure of the set where $f$ does not vanish) is compact. 
\end{definition}


The set of all real valued smooth function on $\Mcal$ is denoted by $\Ecal(\Mcal)$, and the one of all real valued smooth compactly supported functions on $\Mcal$ by $\Dcal(\Mcal)$. 


\begin{figure}[ht!]
\begin{center}
\begin{tikzpicture}[thick,scale=0.7] 
\draw[color=black] (0,0) to [out=50,in=190] (6,3);
\draw[color=black] (6,3) to [out=10,in=90] (10,0);
\draw[color=black] (10,0) to [out=170,in=30] (3,-3);
\draw[color=black] (3,-3) to [out=90,in=10] (0,0);
\draw[color=black] (5.2,1) ellipse (2 and 1);
\filldraw (10,0) circle (0pt) node[right] {$\Mcal$}; 
\filldraw (3,0) circle (0pt) node[above] {$U$}; 
%
\draw[->,color=black,dashed] (5,-2.5) -- (5,-4.5);
\filldraw (5,-3.5) circle (0pt) node[right] {$\phi$}; 
%
\draw[dashed] (1,-7) -- (3,-5) -- (9,-5) -- (7,-7) -- (1,-7);
\draw[color=black] (5,-6) ellipse (1.6 and 0.6);
\filldraw (9,-5) circle (0pt) node[right] {$\Rbb^n$}; 
\filldraw (3,-7) circle (0pt) node[above] {$\Phi(U)$}; 
%
\draw[->,color=black,dashed] (5,-7.5) -- (5,-9.5);
\filldraw (5,-8.5) circle (0pt) node[right] {$f \circ \Phi^{-1}$}; 
%
\draw[line width=0.8mm,color=black] (3,-10) -- (7,-10);
\draw[color=black] (0,-10) -- (10,-10);
\filldraw (10,-10) circle (0pt) node[right] {$\Rbb$};
\filldraw (3,-10) circle (0pt) node[above] {$f(U)$}; 
%
\draw[->,color=black,dashed] (0,-0.5) -- (0,-9.5);
\filldraw (0,-4.5) circle (0pt) node[right] {$f$}; 
%
\end{tikzpicture}
\end{center}
\caption{Real valued smooth function.}
\end{figure}


For now on $\Mcal$ shall be understood as a smooth manifold of dimension $n$. 


%----------------------------------------------------------------------------%
\subsection{Lorentzian structure}
\label{p:LORENTZIAN_STRUCTURE}
%----------------------------------------------------------------------------%


Let us look locally at $\Mcal$, and define a \textbf{curve}\index{curve} $\gamma$ passing through $x$ as the map  
%
\begin{equation*}
\gamma : [-1,1] \to \Mcal \ , \ \ \mbox{ such that } \ \gamma(0) = x \in \Mcal \ .
\end{equation*}
%
We consider $\epsilon > 0$ small enough such that $\gamma([-1,1]) \subset U$, for a coordinate neighborhood $U$ of a chart $(U,\Phi)$. We say that two curves $\gamma$ and $\gamma^\prime$ are equivalent if
%
\begin{equation*}
\underset{t \to 0}{\lim} \ \frac{1}{t} \left( \gamma(x+t) - \gamma(x) \right) = \underset{t \to 0}{\lim} \ \frac{1}{t} \left( \gamma^\prime(x+t) - \gamma^\prime(x) \right) \ .
\end{equation*}
%
We call \textbf{tangent space}\index{tangent space} at $x$, denoted by $T_x\Mcal$, the equivalence class of the curves at $x$. $T_x\Mcal$ can be defined in another way. Let us consider the set of real valued smooth functions on $\Mcal$, $\Ecal(\Mcal)$. We say that two functions $f, g \in \Ecal(\Mcal)$ are equivalent on a coordinate neighborhood $U$ if the restriction of $f$ and $g$ to $U$ are equal for all points in $U$. The new equivalence class at a point $x$ is denoted by $\Ecal_x(\Mcal)$. Then we define a \textbf{derivation}\index{derivation} $V_x$ as a linear map from $\Ecal_x(\Mcal)$ to $\Rbb$ which satisfies the \textbf{Leibniz rule}\index{Leibniz rule},
%
\begin{equation*}
V_x(fg) = f(x) V_x(g) + g(x) V_x(f) \ .
\end{equation*}
%
The \textbf{tangent space}\index{tangent space} at $x$ is then the vector space of the derivations on $\Ecal_x(\Mcal)$. We notice that the equivalence relation defined on $\Ecal(\Mcal)$ is local, $V_x(f)$ depend only on the value of $f$ around $x$. The only thing we can know about $f$ looking at $V_x(f)$ is its behavior in a neighborhood of $x$. Then the Leibniz rule assures that it depends at most on the first derivative of $f$. It can be shown that this two definitions are equivalent. A last remark is that the tangent space to a point of a manifold has the same dimension as the given manifold.


\begin{figure}[ht!]
\begin{center}
\begin{tikzpicture}[scale=0.7] 
\draw[color=black] (0,0) to [out=50,in=190] (6,3);
\draw[color=black] (6,3) to [out=10,in=90] (10,0);
\draw[color=black] (10,0) to [out=170,in=30] (3,-3);
\draw[color=black] (3,-3) to [out=90,in=10] (0,0);
\filldraw (10,0) circle (0pt) node[right] {$\Mcal$};  
\draw[color=black] (2,0) -- (4,2) -- (8,2) -- (6,0) -- (2,0);
\filldraw (7.3,1) circle (0pt) node[right] {$T_x\Mcal$};
\draw [color=black] plot [smooth,tension=1] coordinates {(3,0.5) (4.2,1.4) (5.4,0.2) (7,1.5)};
\filldraw (4.2,1.4) circle (1pt) node[below] {$x$};
\end{tikzpicture}
\end{center}
\caption{Tangent space}
\end{figure}


Before adding more structure on $T_x \Mcal$ we shall introduce the notion of \textbf{vetor bundle}\index{vector bundle}. A \textbf{smooth real vector bundle}\index{vector bundle!smooth} is a triple $(E,\Mcal,\pi)$, where $E$ (\textbf{total space}\index{total space}) and $\Mcal$ (\textbf{base space}\index{base space} of dimension $n$) are smooth manifolds and 
%
\begin{equation*}
\pi : E \to \Mcal , 
\end{equation*}
%
a smooth surjection\index{surjection}\footnote{A map $f : A \to B$ is a surjection if for any $b \in B$ there exists an $a\in A$ such that $b=f(a)$.} which defines a $n$ dimensional vector space $E_x = \pi^{-1}(\{x\})$ for every $x \in \Mcal$, called the \textbf{fibre}\index{fibre} of $E$ at $x\in\Mcal$. Additionally we require that for every $x\in\Mcal$ there exists an open neighborhood $U \subset \Mcal$ and a smooth diffeomorphism $\psi : \pi^{-1}(U) \to U \times E_x$ such that its projection on the first component $\psf\rsf_1$ gives the image of $\pi$. This conditions corresponds to require the ``commutation'' of the diagram \ref{fig:vect_bund_strut}.


\begin{figure}[ht!]
\begin{center}
\begin{tikzpicture}[scale=1]
\draw[->,color=black] (0,0) -- (3,0);
\filldraw (1.5,0) circle (0pt) node[above] {$\psi$};
\draw[->,color=black] (0,-0.2) -- (1.3,-1.5);
\filldraw (0.5,-0.8) circle (0pt) node[left] {$\pi$};
\draw[->,color=black] (3,-0.2) -- (1.7,-1.5);
\filldraw (2.5,-0.8) circle (0pt) node[right] {$\psf\rsf_1$};
%
\filldraw (0,0) circle (0pt) node[left] {$\pi^{-1}(U)$};
\filldraw (3,0) circle (0pt) node[right] {$U \times \{ E_x \}$};
\filldraw (1.5,-1.5) circle (0pt) node[below] {$U$};
\end{tikzpicture}
\end{center}
\caption{Vector bundle structure.}
\label{fig:vect_bund_strut}
\end{figure}


We shall omit to indicate the corresponding triple when speaking of a vector bundle, and specify only the total space $E$, and if necessary the base space.


%\bigskip


We call \textbf{smooth section}\index{section} of a vector bundle a smooth map $s : \Mcal \to E$, such that $\pi \circ s = \id$. The corresponding vector space is denoted $\Gamma(\Mcal)$. We shall later see the importance of this notion in physics.


%\bigskip


There are important particular smooth real vector bundles, the tangent bundle and the real line bundle. This two shall appear to be useful later on. 


%\bigskip


The \textbf{tangent bundle}\index{tangent bundle} is the triple $T\Mcal=(T_x\Mcal, \Mcal, \pi_t)$ with $\pi_t : T_x\Mcal \to \Mcal$. The section of the tangent bundle is $v : \Mcal \to T_x\Mcal$, it is called the \textbf{vector field}\index{vector field}. 


%\bigskip


And the \textbf{real line bundle}\index{real line bundle} is the triple $(\Rbb, \Mcal, \pi_\ell)$ with $\pi_\ell : \Rbb \to \Mcal$. The section of the real line bundle is $\phi : \Mcal \to \Rbb$, it is called the \textbf{real scalar field}\index{scalar field}. 


%\bigskip


Let us come back to the notion of tangent space. It is a vector space therefore we can consider the dual of it. It is called the \textbf{cotangent space}\index{cotangent space} of $\Mcal$ at $x$, and is denoted by $T^\ast_x\Mcal$. We recall that the dual of vector space is the vector space of the linear applications from this space to $\Rbb$, and it forms also a vector space. We can then consider the \textbf{cotangent bundle}, it is the triple
%
\begin{equation*}
T^\ast\Mcal=(T^\ast_x\Mcal, \Mcal, \pi^\ast_t) \ \mbox{ with } \ \ \pi^\ast_t : T^\ast_x\Mcal \to \Mcal \ .
\end{equation*}
%
A section of $T^\ast\Mcal$ is called a \textbf{covector field}\index{covector field}.


%\bigskip


Let $E$, $F$, and $G$ be vector spaces. The vector space of the multilinear forms $E \times F \to G$ is called \textbf{tensor product}\index{tensor product} and denoted $E \otimes F$. We can generalize this notion to the case of $r$ vector spaces, i.e the set of multilinear forms
%
\begin{equation*}
\underbrace{E_1 \times \dots \times E_r}_{r \ \mbox{ times}} \to G \ .
\end{equation*}
%
In the particular case where $E_1 = \dots = E_p = E$ and $E_{p+1} = \dots = E_r = E^\ast$ all these multilinear forms form the tensor product 
%
\begin{equation*}
E_1 \otimes \dots \otimes E_r = E^{\otimes p} \otimes (E^\ast)^{\otimes q} \ ,
\end{equation*}
%
with $q=r-p$. An element of this set is a tensor of type $(p,q)$, $p$ times covariant, and $q$ times contravariant. 


A tensor product is said to be symmetric if it is invariant under permutations of its arguments and it is said to be antisymmetric if it is invariant under even permutations and invariant up to a sign under odd permutations.


A $q$ \textbf{differential form}\index{differential form} on $\Mcal$ is a antisymmetric tensor of type $(0,q)$. We shall denote by $\Omega^q(\Mcal)$ the vector space of the $q$ differential forms on $\Mcal$. If $q > n$ with $n$ the dimension of $\Mcal$ then $\Omega^q(\Mcal)=\{0\}$, and $\Omega^0(\Mcal)$ is the space of the smooth functions on $\Mcal$.


On the space of differential form we can define a product, called the \textbf{exterior product}\index{exterior product}. For every copy of differential forms, $\omega \in \Omega^r(\Mcal)$ and $\eta \in \Omega^s(\Mcal)$, the exterior product is denoted by 
$\omega \wedge \eta \in \Omega^{r+s}(\Mcal)$ and it is defined  as
%
\begin{equation*}
(\omega \wedge \eta)(x_1,\dots,x_{r+s}) = \frac{1}{r!s!} \sum_{\sigma \in \Orak_{r+s}} (-1)^{\mathsf{sign}(\sigma)} \omega(x_{\sigma{1}},\dots,x_{\sigma{r}}) \cdot \eta(x_{\sigma{r+1}},\dots,x_{\sigma{r+s}}) \ .
\end{equation*}


\bigskip


In order to be able to measure the length of tangent vectors, the angles among them and eventually the distance between different points we have to equip our spacetime with a \textbf{metric}\index{metric}. Locally a metric is a \textbf{scalar product}\index{scalar product}. Let us consider a vector space $E$ of dimension $n$ endowed with a non degenerate scalar product $X,Y \mapsto (X,Y)$. If $\{(e_a)\}$ is an orthonormal basis of $E$ an element $X$ of $E$ can be written as $X = X^a e_a$, where the summation over the indice $a$ is understood. The scalar product can be seen as a bilinear, symmetric, and non degenerate function $g$ such that
%
\begin{equation*}
g : \left\{ 
\begin{array}{lcl}
E \times E & \to & \Rbb \ , \\
(X,Y) & \mapsto & g_{ab} X^a Y^b
\end{array}
\right. \ ,
\end{equation*}
%
where $(g_{ab})$ is a diagonal matrix of size $n \times n$. We notice that in order to implement causality in the genral case the scalar product does not need to be positive (look at \ref{p:GLOB_HYPERBOL}). In the present definition $(g_{ab})$ is diagonal and has only $\pm 1$ on it. The signature of a scalar product is the pair $(r,s)$ where $r$ is the number of $+1$ on the diagonal, and $s$ the number $-1$ on the same diagonal. 


We shall genealized the notion of a metric tensor on a manifold. It is a nondegenerate bilinear symmetric tensor of type $(0,2)$, denoted $g$, such that for smooth vector fields $X$ and $Y$, $g(X,Y)$ is a smooth function on $\Mcal$. We can show that the smoothness implies that the signature is constant on any connected component of $\Mcal$. On each point $x$ on $\Mcal$ the metric tensor $g$ defines a scalar product on $T_x\Mcal$. It is common practice to call a metric tensor a metric.


Using the metric we can relate a vector space $E$ to its dual $E^\ast$. Indeed $Y \mapsto g(X,Y) \in \Rbb$ is an element of $E^\ast$. Since $g$ is linear and non degenerate, it is a linear isomorphism. It follows that on a manifold we can use the metric $g$ to define a linear isomorphism between vector fields and one forms. 


We call $g$ a \textbf{rimannian metric}\index{rimannian metric} if for every point of the manifold $g$ has signature $n$, the dimension of $\Mcal$. If the associated scalar product has signature equal to $n-1$ for every point of the spacetime we call $g$ a \textbf{pseudo riemannian}\index{pseudo riemannian metric} (or \textbf{lorentzian}\index{lorentzian metric}) metric. We shall use for now on only lorentzian metric $g$.

\begin{definition}[Lorentzian manifold]\label{def:lorentzian_M}
$(\Mcal,g)$ is a Lorentzian manifold, where $\Mcal$ is a $n$ dimensional smooth manifold, and $g$ is a Lorentzian metric tensor.
\end{definition}


We shall omit to say tensor metric and simply say metric.


%----------------------------------------------------------------------------%
\subsection{Integration}
\label{p:INTEGRATION}
%----------------------------------------------------------------------------%


We define the differential $\dsf$ on a form as the linear application
%
\begin{equation*}
\dsf : \Omega^r(\Mcal) \to \Omega^{r+1}(\Mcal) \ .
\end{equation*}
%
The differential $\dsf : \Omega^0(\Mcal) \to \Omega^1(\Mcal)$ is the differential of functions, i.e. the map from the tangent space to the scalars.


We first consider the particular case $\Mcal=\Rbb^n$, where we have only one chart covering the whole manifold, $\{x_i\}$ being the local coordinates of the coordinate map. Let $\omega$ be a $n$ differential form on $\Rbb^n$, then it can be written as
%
\begin{equation*}
\omega = \omega_{1,\dots,n} \ \dsf x_{1} \wedge \dots \wedge \dsf x_{n} \ .
\end{equation*}
%
The integral over a subset $U$ of $\Rbb^n$ is defined as 
%
\begin{equation*}
\int_U \omega = \int_U \omega_{1,\dots,n} \ \dsf x_{1} \dots \dsf x_{n} \ ,
\end{equation*}
%
where we omit to write the symbol $\wedge$ for the exterior product.
%
In this definition an orientation has been implicitly chosen, i.e. an order in the exterior product of $\{x_i\}$ is taken. Taking another orientation will change the sign of the integral. This is one of the reason while it is important to work with an orientable manifold.


Let us go back to the general situation, where $\Mcal$ is defined as in \ref{def:lorentzian_M}. Furthermore we require $\Mcal$ to be paracompact and oriented. As we know $\Mcal$ looks locally like $\Rbb^n$, actually this was one of the requests in our construction. Therefore it should be possible to define locally an integral on $\Mcal$. In order to map $\Mcal$ to $\Rbb^n$ we consider an atlas $\{(\Phi_i,U_i)\}$ of $\Mcal$. The support of $\omega$ does not lie entirely in one coordinate neighborhood, thus we need to use a partition of unity $\{g_i\}$ subordinate to the covering $\{U_i\}$. Then any other $n$ form $\omega$ can be written as
%
\begin{equation*}
\omega = \sum_{i} g_i \omega \ .
\end{equation*}
%
where the sum is finite because $\supp(\omega)$ is compact. We map $g_i \omega$ on each coordinate neighborhood $U_i$ with $\phi_i : U_i \to W_i \subset \Rbb^n$. Therefore the $n$ form $\phi_i ^\ast (g_i\omega)$ on $W_i$ can be integrated as we have done on $\Rbb^n$. Meaning
%
\begin{equation*}
\int_\Mcal \omega = \sum_i \int_{U_i} g_i \omega = \sum_i \int_{W_i} \phi_i ^\ast (g_i\omega) = \sum_i \int_{W_i} F_i(x_1,\dots,x_n) \ \dsf x_1 \dots \dsf x_n \ ,
\end{equation*} 
%
with $F_i(x_1,\dots,x_n) \ \dsf x_1 \dots \dsf x_n = \phi_i ^\ast (g_i\omega)$. The map $\phi_i ^\ast$ is the pull back of $\phi_i ^\ast$, we have $\phi_i ^\ast(g_i\omega) = (g_i\omega) \circ \phi_i$. It can be shown that this construction does not depend on the chosen partition of unity.


Because the manifold is assumed paracompact, $\Mcal$ is orientable if there exits a non zero smooth $n$ form $\mu$. It is called the volume form of $\Mcal$. Due to the fact that $\Mcal$ is a Lorentzian manifold with tensor metric $g$ we have 
%
\begin{equation*}
\mu = \sqrt{\abs{\det\left(g\right)}} \ \dsf^n x = \sqrt{\abs{g}} \ \dsf^n x \ ,
\end{equation*}
%
We then define the integral of a function $f$ over $\Mcal$ as follow
%
\begin{equation}
\int_\Mcal f = \sum_i \int_{W_i} (g_i f)(x_1,\dots,x_n) \ \sqrt{\abs{g}} \ \dsf x_1 \dots \dsf x_n \ .
\label{eq:int_manifold}
\end{equation}


%----------------------------------------------------------------------------%
\subsection{Covariant derivative, geodesic and curvature}
%----------------------------------------------------------------------------%


We shall introduce the notion of covariant derivative. A \textbf{linear connection}\index{connection} $\nabla$ is a map 
%
\begin{equation*}
\nabla : \left\{
\begin{array}{ccl}
\Gamma(\Mcal) \times \Gamma(\Mcal) & \to & \Gamma(\Mcal) \\
(X,Y) & \mapsto & \nabla_X Y 
\end{array}
\right. \ ,
\end{equation*}
%
such that for every sections $X, Y, Z \in \Gamma(\Mcal)$ and any real valued smooth fucntion $f \in \Ecal(\Mcal)$ we have
%
\begin{eqnarray*}
&& \nabla_{X + Y} Z = \nabla_X Z + \nabla_Y Z \ ; \\ 
&& \nabla_{f X} Y = f \nabla_X Y \ ;\\
&& \nabla_X(Y+Z) = \nabla_X Y + \nabla_X Z \ ;\\
&& \nabla_X(fY) = f \nabla_X Y + (X \cdot f) Y \ .
\end{eqnarray*}
%
The connection $\nabla_X Y$ is called the \textbf{covariant derivative}\index{covariant derivate} of $Y$ with respect to $X$. $\nabla$ is said to be compatible with respect to the pseudo riemannian metric $g$ if for all $X, Y, Z \in \Gamma(\Mcal)$ we have
%
\begin{equation*}
X \cdot g(Y,Z) = g(\nabla_X Y, Z) + g(Y,\nabla_X Z) \ .
\end{equation*}
%
We shall say that $\nabla$ is tensor metric connection. The \textbf{torsion}\index{torsion} of $\nabla$, $T$, is a tensor of type $(1,2)$ such that for every vector fields $X, Y \in \Gamma(T\Mcal)$ 
%
\begin{equation*}
T(X,Y) = \nabla_X Y - \nabla_Y X - \left[ X,Y\right] \ .
\end{equation*}
%
The connection $\nabla$ is said to be torsion free if its torsion tensor is the zero tensor. A fundamental theorem in pseudo riemannian geometry (see \cite{oneill_semi-riemannian_1983}) tells us that on $\Mcal$ there exists a unique linear connection that is compatible with the metric and torsion free. It is called the \textbf{Levi Civita connection}\index{Levi Civita connection}.


\begin{figure}[ht!]
\centering
\begin{tikzpicture}[thick,scale=0.8] 
\draw[color=black] (0,0) to [out=50,in=190] (6,3);
\draw[color=black] (6,3) to [out=10,in=90] (10,0);
\draw[color=black] (10,0) to [out=170,in=30] (3,-3);
\draw[color=black] (3,-3) to [out=90,in=10] (0,0);
\draw[semithick] (5,1) ellipse (1.6 and 0.6); 
\draw[dashed] (2,0) -- (4,2) -- (8,2) -- (6,0) -- (2,0);
\draw[color=black,->] (2,0.2) to [out=120,in=190] (2,4.2);
\draw[semithick] (5,5) ellipse (1.3 and 0.6); 
\draw[dashed] (2,4) -- (4,6) -- (8,6) -- (6,4) -- (2,4);
\filldraw (3.5,-2) circle (0pt) node[above] {$\Mcal$}; 
\filldraw (5,1) circle (2pt) node[above] {$x$};
\filldraw (6.5,1.2) circle (0pt) node[above] {$\Ocal_x$};
\filldraw (6.4,5.1) circle (0pt) node[above] {$\Ocal^\prime_x$};
\filldraw (2.5,4.8) circle (0pt) node[above] {$T_x\Mcal$};
\filldraw (0.5,2.2) circle (0pt) node[above] {$\mathsf{exp}_x$};
\end{tikzpicture}
\caption{Exponential map.}
\end{figure}


A neighborhood $\Ocal_x \subset \Mcal$ is called a \textbf{geodesically starshaped} with respect to $x \in \Mcal$ if there is an open subset $\Ocal^{\prime}_x$ in $T_x\Msf$ which is starshaped with respect to $0 \in T_x\Msf$ such that $\mathsf{exp}_x \ : \ \Ocal^{\prime}_x \ \to \ \Ocal_x$ is a diffeomorphism. $\Ocal \subset \Mcal$ is \textbf{geodesically convex} if it is starshaped with respect to all its points. 
Notice in particular that every two points $x,y$ in $\Ocal$ are connected by a unique geodesic which is completely contained in $\Ocal$.


%\bigskip


If we consider $\Ocal \subset \Mcal$ to be geodesically convex, it makes sense to introduce the Synge's world function (or half of the square geodesic distance) due to the uniqueness of the geodesic between two points.


\begin{figure}[ht]
\begin{center}
\begin{tikzpicture}[thick,scale=1] 
\draw[color=black] (-2,-1) to [out=50,in=190] (8,2);
\draw[color=black] (8,2) to [out=10,in=90] (9,-1);
\draw[color=black] (9,-1) to [out=170,in=30] (0,-3);
\draw[color=black] (0,-3) to [out=90,in=10] (-2,-1); 
\draw[color=black] (1.5,-0.5) to [out=70,in=-60] (7,1);
\filldraw (1.5,-0.5) circle (1pt) node[below] {$x$};
\filldraw (7,1) circle (1pt) node[above] {$x^\prime$}; 
\filldraw (3,0.36) circle (1pt) node[below] {$z(t)$};
\draw[->] (3,0.36) -- (5,0.58) node[above] {$\dot{z}(t)$};
\filldraw (6,0.2) circle (0pt) node[above] {$\gamma$};
\end{tikzpicture}
\end{center}
\caption{Geodesic on a geodesically convex domain.}
\end{figure}



In order to introduce its precise definition, let us fix two points $x'$ and $x$ in a $\Ocal$.
The point $x^\prime$ is called the base point, and $x$ the field point. The geodesic segment $\gamma$ that links $x$ to $x^\prime$ is described by the parametric function $z(t)$, in which $t$ is an affine parameter, $t \in \left[ t_1 , t_2 \right]$, such that $z(t_1) = x$ and $z(t_2) = x^\prime$. The tangent vector to the geodesic on a point $z$ is denoted by $\dot{z}$. The Synge's world function is a scalar function of the base point $x^\prime$ and the field point $x$. It is defined by
%
\begin{equation*}
\sigma(x,x^\prime) =  \frac{1}{2} (t_1 - t_2) \int_{t_1}^{t_2} \dsf t \ g_{\mu \nu} \left( z(t) \right) \ \dot{z}^\mu(t) \ \dot{z}^{\nu}(t) \ .
\end{equation*}
%
In flat space time we have $g_{\mu \nu} \left( z(t) \right) = \eta_{\mu \nu}$, in this case the Synge's world function simplifies to
%
\begin{eqnarray*}
\sigma(x,x^\prime) &=& \frac{1}{2} (t_1 - t_2) \ \eta_{\mu \nu} \ \int_{t_1}^{t_2} \dsf t \ \dot{z}^\mu(t) \ \dot{z}^{\nu}(t) \\
&=& \frac{1}{2} (t_1 - t_2) \ \eta_{\mu \nu} \ (y-x)^\mu \ (x^\prime-x)^\nu \ .
\end{eqnarray*}
%
On the geodesic, the quantity $g_{\mu \nu} \dot{z}^\mu \dot{z}^\nu \doteq \epsilon$ is constant, then $\sigma(x,x^\prime) = \frac{1}{2} (t_2-t_1)^2 \ \epsilon$.


%\bigskip

The quantity which characterize how ``deformed'' is $\Mcal$ is called the \textbf{curvature}\index{curvature} of $\Mcal$. It is a tensor of type $(1,3)$ such that for every vector fields $X, Y, Z \in \Gamma(T\Mcal)$ and any linear connection we have
%
\begin{equation*}
R(X,Y)Z = \left( \nabla_X \nabla_Y - \nabla_Y \nabla_X - \nabla_{[X,Y]} \right) Z \ .
\end{equation*}
%
We recall that we shall use for now on the Levi Civita connection. %It has the following symmetry properties
%
%\begin{eqnarray*}
%&& R(X,Y)Z = - R(Y,X)Z \ ; \\
%&& R(X,Y)Z + R(Y,Z)X + R(Z,X)Y = 0 \ ; \\
%&& g\left(R(X,Y)Z,W\right) = - g\left(Z,R(X,Y)W\right) \ ; \\
%&& g\left(R(X,Y)Z,W\right) = g\left(R(Z,W)X,Y\right) \ ,
%\end{eqnarray*}
%
%with $X,Y,Z,X$ vector fields over $\Mcal$.


%----------------------------------------------------------------------------%
\section{Causality}
%----------------------------------------------------------------------------%


%----------------------------------------------------------------------------%
\subsection{Global hyperbolicity}
\label{p:GLOB_HYPERBOL}
%----------------------------------------------------------------------------%


We shall now work with the pair $(\Mcal,g)$ which denotes a Lorentzian manifold of dimension $n \geq 2$ together with a Lorentzian metric $g$. We associate to each point $x$ of the manifold its corresponding tangent space $T_x\Mcal$. Considering a vector over a point, namely we can evaluate its Lorentzian scalar product with itself, using the metric $g$. It divides the tangent space in three different regions.


\begin{eqnarray*}
g(v,v) &>& 0 , \ \mbox{ then $v$ is called timelike vector}, \\ 
g(v,v) &=& 0 , \ \mbox{ then $v$ is called null vector}, \\ 
g(v,v) &<& 0 , \ \mbox{ then $v$ is called spacelike vector}.
\end{eqnarray*}


In every tangent space $T_x\Mcal$ the set of timelike vectors, is called light cone, it consists of two connected components. A time orientation on $\Mcal$ is a choice of one of these connected components. Then the light cone is refered as the union of the forward and backward lightcones, 
%
\begin{equation*}
\Vcal=\Vcal^{+} \ \cup \ \Vcal^{-} \ , \quad \mbox{with } \ \Vcal^{\pm}=\left\{ x\in\Mcal \ | \ x^{2}>0, \ \pm x^{0}>0 \right\} \ . 
\end{equation*}


\begin{figure}[ht!]
%
\begin{center}
%
\begin{tikzpicture}[scale=0.8]
\fill[left color=gray!50!black,right color=gray!50!black,middle color=gray!50,shading=axis,opacity=0.25] (2,6) -- (0,3) -- (-2,6) arc (180:360:2cm and 0.5cm);
\draw (-2,6) arc (180:360:2cm and 0.5cm) -- (0,3) -- cycle;
\draw[densely dashed] (-2,6) arc (180:0:2cm and 0.5cm);
%
\fill[left color=gray!50!black,right color=gray!50!black,middle color=gray!50,shading=axis,opacity=0.25] (2,0) -- (0,3) -- (-2,0) arc (180:360:2cm and 0.5cm);
\draw (-2,0) arc (180:360:2cm and 0.5cm) -- (0,3) -- cycle;
\draw[densely dashed] (-2,0) arc (180:0:2cm and 0.5cm);
%
\filldraw[black] (0,4.5) circle (0pt) node[right] {$\Vcal^{+}$};
\filldraw[black] (0,1.5) circle (0pt) node[left] {$\Vcal^{-}$};
\end{tikzpicture}
%
\end{center}
%
\caption{Light cone.}
%
\end{figure}


A vector $v \in T_x\Mcal$ is \textbf{future} (respectively \textbf{past}) \textbf{directed} if $v$ is a non spacelike vector and contained in $\Vcal^+$ (respectively in $\Vcal^-$). 


%\bigskip


A differentiable curve $\gamma(\lambda)$ is said to be 
\begin{itemize}
\item a \textbf{future} (respectively \textbf{past}) \textbf{directed timelike curve} if at each point $x(\lambda) \in \gamma$ the tangent vector $v$ is a future (respectively past) directed timelike vector ;
\item a \textbf{future} (respectively \textbf{past}) \textbf{directed causal curve} if at each point $x(\lambda) \in \gamma$ the tangent vector $v$ is either a future (respectively past) directed timelike or null vector. 
\end{itemize}

%\bigskip

%\begin{definition}[Chronological future/past]
The \textbf{chronological future} (respectively \textbf{past}) of $x \in \Mcal$ is denoted by $I^{+}(x)$. It is defined as the sets of points which can be reached by a future (respectively past) directed timelike curve starting from $x$,
%
\begin{equation*}
I^{\pm}(x) = \left\{ y \in \Mcal \ \bigg| \ \begin{array}{l} \text{There exists a future (resp. past) directed timelike} \\ \text{curve $\lambda(t)$, with $\lambda(0)=x$ and $\lambda(1)=y$} \end{array} \ \right\}.
\end{equation*}

We define $I^{+}(S) \ = \ \bigcup_{x \in S} I^{\pm}(x)$ for any subset $S \subset \Mcal$.
%
%\end{definition}

%\bigskip

The causal future/past of a point of the spacetime is defined in a similar way as the chronological future/past of this point, using this time the notion of the causal curve.

%\bigskip

%Let $\gamma : [a,b] \subset \Rbb \to \Mcal$ be a smooth curve joining $x$ to $y$ in $\Mcal$, such that $\gamma(a)=x$ and $\gamma(b)=y$, and $\Gamma(\gamma)$ the space of smooth tangent vector fields along $\gamma$. The \textbf{length} of the curve $\gamma$ between $x$ and $y$ is equal to
%
%\begin{equation*}
%L(x,y) = \int_a^b \dsf t \  \sqrt{g\left(X(t),X(t)\right)} \ ,
%\end{equation*}
%
%with $X \in \Gamma(\gamma)$. The \textbf{geodesic}\index{geodesic} between $x$ and $y$ is the curve with the minimum lenght such that for any vector field $X$ in $\Gamma(\gamma)$ we have
%
%\begin{equation*}
%\nabla(X,X) = 0 \ .
%\end{equation*}
%
%A geodesic on $\Mcal$ is then a curve $\gamma$ such that parallel transport along the curve preserves its tangent vectors to the curve.
%\com{... These two definitions (length and geodesic) works only when the manifold is riemannian. The metric is Lorentzian hence you have to modify both definitions!!! ... }



%\begin{definition}[Causal future/past] 
%
The \textbf{causal future} (respectively \textbf{past}) of $x \in \Mcal$, denoted by $J^{+}(x)$, is defined as the sets of points that can be reached by a future (respectively past) directed causal curve starting from $x$,
%
\begin{equation*}
J^{\pm}(x) = \left\{ y \in M \ \bigg| \ \begin{array}{l} \text{There exists a future (respectively past)} \\ \text{directed causal curve $\lambda(t)$, with $\lambda(0)=x$ and $\lambda(1)=y$} \end{array} \; \right\},
\end{equation*}
We define $J^{\pm}(S) \ = \ \bigcup_{x \in S} J^{\pm}(x)$ for any subset $S \subset \Mcal$.
%
%\end{definition}


%\bigskip


%If $x \in J^+(y)$, then the Lorentzian distance $d(x,y)$ is the supremum of the Lorentzian lengths of all the future directed causal curves from $x$ to $y$. 


%\begin{definition}[Achronal set]




A subset $S \subset M$ is said to be \textbf{achronal} if there do not exist $x, y \in S$ such that $y \in I^{+}(x)$, i.e., if $I^{+}(S) \cap S = \emptyset$. 
%\end{definition}


%\bigskip


%\begin{definition}[Domains of dependance]
We define the \textbf{future} (respectively \textbf{past}) domain of dependence of $S$, denoted by $D^{+}(S)$, by
%
\begin{equation*}
D^{\pm}(S) = \left\{ x \in M \ \bigg| \ \begin{array}{l} \text{Every past (respectively future) inextendible causal curve} \\ \text{through $x$ intersects $S$} \end{array} \; \right\}.
\end{equation*}
%
The (full) \textbf{domain of dependence} of $S$, denoted by $D(S)$, is defined as,
\begin{equation*}
D(S) \ = \ D^{+}(S) \ \cup \ D^{-}(S).
\end{equation*}
The set $S$ is a closed achronal set.


\bigskip


\begin{definition}[Cauchy surface]
A closed achronal set $\Sigma$ for which $D(\Sigma) = M$ is called a Cauchy surface. 
\end{definition}

A spacetime $(\Mcal,\gsf)$ which possesses a Cauchy surface is said to be globally hyperbolic. We invite the reader to look at the end of chapter $8$ of \cite{waldGR} for the equivalence of this definition of global hyperbolicity and the ones of Leray, Hawking, and Ellis. 
We have now enough background to define a curved spacetime.

\begin{definition}[Curved spacetime]\label{def:cst}
A pair $(\Mcal,g)$ is a curved spacetime if $\Mcal$ is a $n \geq 2$ dimensional Lorentzian manifold, endowed with a Lorentzian metric of signature $( - + \dots +)$. The spacetime is required to be orientable, paracompact, time orientable, and globally hyperbolic. 
\end{definition}


\begin{definition}[Minkowski spacetime]\label{def:minkowski}
A pair $(\Mbb,\eta)$ is called Minkowski spacetime if $\Mbb$ is the 4 dimensional euclidean space endowed with a metric $\eta$ of signature $(1,3)$ for which only its diagonal elements are non zero.
\end{definition}


The Minkowski spacetime is a particular case of a globally hyperbolic spacetime. 


%----------------------------------------------------------------------------%
\subsection{Friedmann--Lemaître--Robertson--Walker spacetime}
\label{p:FLRW}
%----------------------------------------------------------------------------%


If we look at our universe from different directions it looks very similar, at least on large scales. Furthermore it seems reasonable to assume that there are no special points in it. For these reason it is common to assume the universe to be homogeneous and isotropic. This two requirements imply that the spacetime describing our universe must be a warped product $I\times \Sigma$ where $I$ is an interval of time and $\Sigma$ a three dimensional riemannian manifold of constant curvature which can be open closed or flat. In this work we shall consider only a spatially flat FLRW spacetime, i.e. a curved spacetime $(\Mcal,g)$ equipped with the following metric
%
\begin{equation*}
ds^2 = dt^2 - a(t)^2 d\vec{x}^2 = a(\tau)^2\left(-d\tau^2+d\vec{x}^2\right) \ ,
\end{equation*}
%
where $t$ is the cosmological time, $\tau$ is conformal time, and $a(t)$ is the scale factor whose expansion rate is the Hubble rate
%
\begin{equation*}
H = \partial_t \log (a) = \frac{\partial_\tau a}{a^2} = \frac{\Hcal}{a} \ .
\end{equation*}
%
The time variables are related by
%
\begin{equation*}
dt = a d\tau = \frac{da}{aH} = -\frac{dz}{(1+z)H} \ . 
\end{equation*}
% 
The curvature can also be written in terms of the Hubble rate
%
\begin{equation}
R=6(\partial_t H + 2 H^2)=\frac{\partial^2_\tau a}{a^3} \ . 
\label{eq:rflrw}
\end{equation}
%
We notice that this metric is invariant under translation and rotation of the space coordinates. We have chosen a \textbf{FLWR} spacetime without spatial curvature in order to simplify computations and because observations are compatible with the assumptions of vanishing spatial curvature. 



%----------------------------------------------------------------------------%
\chapter{Interacting quantum field theory}
%----------------------------------------------------------------------------%


In the last decades quantum field theory \textbf{(QFT)}\index{QFT}\index{quantum field theory} has been tested with very sophisticated experiments, and predictions made by the theory coincide with very high precision to the experimental results. One of the missing block to this robust theoretical framework is implementing gravitation. QFT on curved spacetime \textbf{(CST)}\index{XST}\index{curved spacetime} is a first step in that direction. In this work, for simplicity we will restrict ourselves to the case of a quantum scalar field propagating over a curved background. Despite of the simplicity of this model, its treatment will give us a clear idea of the mathematical structure necessary to describe quantum field theories on curved spacetime.


In this chapter we shall present the mathematical framework necessary to formalize the theory of QFT on CST. In particular we shall use the approach of algebraic quantum field theory. It consists in two steps. First we shall define the algebra of observables and the relations among them, and then identify states, and hence model expectation values.


We shall focus first on the free theory, and describe the mathematical elements necessary to describe quantum field theories on a curved spacetime $\Mcal$. That is to say we shall introduce the notion of fields and observables within this approach, then the classical field theory, and finally present the quantization procedure used.


At the end of this chapter we shall analyze the case of interacting theories, where the interaction will be treated perturbatively. In the perturbative approach to interacting quantum field theories observables are given as formal power series in the coupling constant. These kind theories are affected by three class of problems, ultraviolet divergences, infrared problems, and finally convergence problems of the  perturbative series. At the present time we have the first problem under good mathematical control thanks to the theory of regularization, the second problem can be tackled using for example the adiabatic limit. Unfortunately it is still not known in which sense this series could converge if it converges at all. In the prsent work we shall concentrate on the first class of problems and presenting the status of the art.


%----------------------------------------------------------------------------%
\section{Off shell configuration space}
%----------------------------------------------------------------------------%


%----------------------------------------------------------------------------%
\subsection{Configuration space of a real scalar field theory}
%----------------------------------------------------------------------------%


As stated in the introduction, we aim to describe a quantum field theory over a curved background. In ordinary classical theory the state of a system is described by a section of a particular vector bundle over $\Mcal$. Furthermore, if the field theory we are considering is free, usually the section mentioned above needs to satisfy a linear equation of motion.


In this thesis we would like to describe an interacting real scalar field theory, hence a possible field configuration is described by a scalar function $\phi$ over $\Mcal$. Furthermore, since we shall treat the interaction by means of perturbation theory, we need to enlarge the space of admissible field configurations dropping the requirement that an equation of motion is satisfied.


In order to be able to work easily with the theory, some regularity needs to be requested for the admissible field configurations. In particular, we shall assume that the admissible functions are smooth.   


We may summarize all these requirements in the following definition.


\begin{definition}[Off shell configuration space]\label{def:config_space}
The off shell configuration space over $\Mcal$ is the space of real valued smooth maps, $\phi \in \Ecal(\Mcal)$. It is equipped with the locally convex topology.
\end{definition}


%----------------------------------------------------------------------------%
\subsection{Topological aspects of the configuration space}
\label{p:TOPO_CONFIG_SPACE}
%----------------------------------------------------------------------------%


As we already said in the general case the configuration space over $\Mcal$ can be defined as the space of sections of some vector bundle $E$ over $\Mcal$. Since we shall work only with real scalar fields, the chosen vector bundle is simply the real line bundle $\Ecal(\Mcal)$.


We would like now to discuss some mathematical aspects of $\Ecal(\Mcal)$, i.e. we shall discuss its topology and the notion of convergence on it. We start by looking at the space of functions of the form
%
\begin{equation*}
f : N \subset \Rbb^n \to \Rbb \ 
\end{equation*}
%
Using L. Schwartz's notation we denote the space of real valued smooth functions on $N$ by $\Ecal(N)$. We equip $\Ecal(N)$ with the following family of seminorms
%
\begin{equation}
\Pcal = \left\{ p_{K,r}(f) \ , \ \mbox{ with compact subset } K \subset N \ , \mbox{ and } r \in \Nbb \right\} \ ,
\label{eq:family_seminorm}
\end{equation}
%
where the seminorms $p_{K,r}$ are defined as
%
\begin{eqnarray}
&& p_{K,r}(f) = \sup \left\{ \abs{\partial^\alpha f(x)} \ \mbox{ with } \ x \in N \ \mbox{ and } \ \abs{\alpha} \leq r  \right\} \ , \nonumber \\
&& \mbox{where the multiindex} \ \alpha \in \Nbb^n, \ \abs{\alpha} = \sum_{i=1}^n \alpha_i, \ \mbox{ and } \ \partial^\alpha = \frac{\partial^{\alpha_1}}{\partial x_1^{\alpha_1}} \dots \frac{\partial^{\alpha_n}}{\partial x_n^{\alpha_n}} \ .
\label{eq:conv_multiindex}
\end{eqnarray}
%
The family of seminorms $\Pcal$ endow $\Ecal(N)$ with a locally convex topology. On this space a sequence $(f_n)$ converges with limit $f$ if for all $\alpha$ and all compact $K$,
%
\begin{eqnarray*}
&& (\partial^\alpha f_n) \ \mbox{ converge uniformly on } \ K \ \mbox{ to } \ (\partial^\alpha f) \ , \\
&& \mbox{i.e. for all } \ \epsilon > 0 \ \mbox{ there is } \ r_\epsilon \in \Nbb \ \mbox{ such that for every } \  x \in K \ \mbox{ and any } \ n \geq r_\epsilon \ , \\
&& \mbox{we have } \ \abs{\partial^\alpha f_n(x) - \partial^\alpha f(x)} < \epsilon \ . 
\end{eqnarray*}
%
This space is a locally convex topological vector space which appears to be a Frechet space. We recall that a Frechet space is a complete locally convex topological vector space with a metrizable topology. It can be shown that a locally convex topological vector space is metrizable if and only if the topology can be defined as a countable family of seminorm. Here as defined in \ref{eq:family_seminorm} the topology on $\Ecal(N)$ is defined via a family of seminorm and is also complete. Thus it is true that $\Ecal(N)$ is a Fréchet space.


Now let us come back to the $\Ecal(\Mcal)$, i.e. the space of smooth functions defined as 
%
\begin{equation*}
\phi : \Mcal \to \Rbb \ . 
\end{equation*}
%
As for $\Ecal(\Rbb^n)$ the topology of $\Ecal(\Mcal)$ can be induced by a family of seminorms. Let consider the atlas $\{(U_i,\Phi_i)\}$ on $\Mcal$, a compact set $K \subset \Phi_i(U_i) \subset \Rbb^n$, and $r \in \Nbb$. We define the seminorm $p_\Mcal$ on $\Ecal(\Mcal)$ as follow
%
\begin{equation*}
p_\Mcal(\phi) = p_{K,r}\left( \Psi\left( \phi_{|U_i}(x) \right) \right) \ ,
\end{equation*}
%
with $\phi \in \Ecal\left(\Mcal \right)$. It defines the Frechet topology on $\Ecal\left(\Mcal \right)$. A sequence $(\phi_n)$ converges to $\phi$ with respect to this topology if and only if for any compact set $K \subset U_i$, $\left(\partial^\alpha\phi_m\right)$ converges uniformly on $K$ to $\partial^\alpha\phi$.


For later purposes we consider the space of real valued smooth compactly supported functions $\Dcal(\Mcal)$, endowed with the locally convex topology implemented as follow
%
\begin{equation*}
\Dcal(\Mcal) = \bigcup_{K\subset\Mcal} \Dcal_K(\Mcal) \ ,
\end{equation*}
%
where in the right hand side the union is taken over all compact set $K \subset \Mcal$. $\Dcal_K(\Mcal) \subset \Ecal(\Mcal)$ is the space of all smooth functions supported in $K$, endowed with the topology induced from $\Ecal(\Mcal)$. On $\Dcal(\Mcal)$ we have the inductive limit topology. It is a locally convex topological vector space but non metrizable, therefore it is not a Frechet space. 


%----------------------------------------------------------------------------%
\section{Observables as functionals over field configurations}\label{p:OBS}
%----------------------------------------------------------------------------%


In this section we shall introduce the set of observables of a scalar field theory over a curved background. We recall that an observable models a possible experimental apparatus by means of which we can test the physical system we are considering. In other words, it must be possible to associate an observable to every detectable property of an element of the configuration space over $\Mcal$. Examples of observables are the local energy density, the local temperature, the local momentum and the local density of the field.  Hence, roughly speaking, observables are functions on the configuration space with values in ordinary numbers.


Once we have detected a sensible set of observables which describes a quantum field theory on curved background, we shall analyze the relation among them. Here we aim to individuate the algebraic properties satisfied by the set of observables, this is necessary in order to use standard algebraic method to quantize the system. Notice that the algebra of observables we want to obtain will not depend on the particular state of the system.


We shall use the functional approach to define mathematically the observables, this method is less abstract and more computationally friendly \cite{brunetti_algebraic_2012,brunetti_perturbative_2009}. Hence, we shall view for now on an observable as a functional 
%
\begin{equation*}
\Fsf : \left\{
\begin{array}{ccc}
\Ecal(\Mcal) & \to     & \Cbb \\
\phi  & \mapsto & \Fsf(\phi)
\end{array}
\right. \ . 
\end{equation*}


The generic space of observables is thus the set of all possible functionals and it is denoted by $\Fcal(\Mcal)$. We shall now discuss some further restrictions we have to request in order to be able to work with these objects.


The laboratory where the measurements are made is finite dimensional in space and time, thus we shall consider only functionals which are defined on a finite subspace of $\Mcal$. For this reason we need a concept which permits to localize functionals in a certain region of spacetime.


\begin{definition}[Spacetime support] \label{def:spacetime_supp}
Let $\Fsf$ be a complex valued functional. The spacetime support of $\Fsf$ is defined as follow
%
\begin{equation*}
\supp(\Fsf) \doteq \left\{ x \in \Mcal \ \bigg| \ 
\begin{array}{l}
\forall \ U \ni x , \ \exists \ \phi, \psi \in \Ecal(\Mcal) \ \mbox{ with } \ \supp(\psi) \subset U, \\
\mbox{such that } \Fsf(\phi + \psi) \neq \Fsf(\phi)
\end{array}
\right\} \ .
\end{equation*}
It is a closed subset of $\Mcal$.
%
\end{definition}


According to definition \ref{def:spacetime_supp} a functional $\Fsf$ does not ``feel'' field configurations which are localized outside $\supp(\Fsf)$. In other words the spacetime support of $\Fsf$ is the subset of $\Mcal$ where $\Fsf$ does see the ``fluctuations'' of the field configuration.


As discussed above, the space $\Fcal(\Mcal)$ is still too general. We should consider \textbf{functionals with compact spacetime support} over $\Mcal$, this set is denoted by $\Fcal_0(\Mcal)$. Notice that this new space can be endowed with an algebraic structure, introduced in \cite{brunetti_algebraic_2012}.


\begin{definition}[Algebra of compactly supported functionals] \label{def:algebra_comp_supp_func}
Let $\Fsf$ and $\Gsf$ be compactly supported functionals, i.e. elements of $\Fcal_0(\Mcal)$ then we define the following operations.
%
\begin{itemize}
\item Sum : $(\Fsf+\Gsf)(\phi) = \Fsf(\phi) + \Gsf(\phi)$ ,
\item Multiplication by a scalar $z\in\Cbb$ : $(z \cdot \Fsf)(\phi) = z \Fsf(\phi)$ ,
\item Pointwise product : $(\Fsf \cdot \Gsf)(\phi) = \Fsf(\phi) \cdot \Gsf(\phi)$ ,
\item Involution : $\Fsf^\ast(\phi) = \overline{\Fsf(\phi)}$ ,
\item Unit : $\Ibb = \Fsf(\phi) = 1$ ,
\end{itemize}
%
with $\phi \in \Ecal(\Mcal)$. 
\end{definition}


The space $\Fcal_0(\Mcal)$ equipped with the operations and elements listed above defines a commutative unital $\ast-$algebra.


In order to ensure the well posedness of the previous definition, we can check that the algebraic operations introduced in definition \ref{def:algebra_comp_supp_func} do not modify the spacetime support \ref{def:spacetime_supp}. This lemma has been proved in \cite{brunetti_algebraic_2012}.


\begin{lemma}[``Rigidity'' of the spacetime support]
The algebraic operations introduced in definition \ref{def:algebra_comp_supp_func} preserve the spacetime support of a functional. In particular
%
\begin{itemize}
\item Sum : $\supp(\Fsf + \Gsf) \subseteq \supp(\Fsf) \cup \supp(\Gsf)$ ,
\item Multiplication by a scalar $z\in\Cbb$ : $\supp(z\cdot\Fsf) = \supp(\Fsf)$ ,
\item Pointwise product : $\supp(\Fsf \cdot \Gsf) \subseteq \supp(\Fsf) \cap \supp(\Gsf)$ ,
\item Involution : $\supp(\Fsf^\ast) = \supp(\Fsf)$ ,
\item Scalar multiple of the unit, $z\in\Cbb$ : $\supp(z\cdot\Ibb) = 0 $ ,
\end{itemize}
%
with $\Fsf, \Gsf \in \Fcal_0(\Mcal)$ and $\phi \in \Ecal(\Mcal)$.
\end{lemma}


The set of functionals present in $\Fcal_0(\Mcal)$ is still too general. In order to be able to properly work with a functional we have to be able to compute all its functional derivatives and these functional derivatives have to show some regularity.


Let us start formalizing this discussion giving the definition of functional derivative which will be used all along in the present work. Let $\Xsf$ be a locally convex topological vector space, a subset $\Usf \subseteq \Xsf$, is also said to be locally convex if every point $x \in \Usf$ has a neighborhood contain in $\Usf$. 


\begin{definition}[Functional derivative]\label{def:functional_derivative}
Let $\Xsf$ and $\Ysf$ be two locally convex topological vector spaces, and $\Usf \subseteq \Xsf$ an open subset. The functional derivative (or Gâteau derivative) of a map $\Fsf:  \Usf \to \Ysf$ at $\phi \in \Usf$ in the direction $\psi \in \Xsf$ is defined as the map $\Fsf^{(1)} : \Usf \times \Xsf \to \Ysf$,
%
\begin{equation*}
\Fsf^{(1)}(\phi)[\psi] = \lim_{t \to 0} \ \frac{1}{t} \bigg( \Fsf(\phi_n + t \psi) - \Fsf(\phi) \bigg) \ .
\end{equation*}
% 
The map $\Fsf$ is called \textbf{differentiable} (or Gâteau differentiable) at $\phi \in \Usf$ if the limit exists for all $\psi \in \Xsf$, and \textbf{continously differentiable} if $\Fsf^{(1)}$ is jointly continous on the product space $\Usf \times \Xsf$.
%
%
The generalization to the $n$-th functional derivative of $\Fsf$ at $\phi \in \Usf$ with respect to the directions $\psi_1, \dots, \psi_n \in \Xsf$ is defined as a map $\Fsf : \Usf \times \Xsf^{\otimes n} \to \Ysf$,
%
\begin{equation*}
\Fsf^{(n)}(\phi)[\psi_1,\dots ,\psi_n] = \lim_{t \to 0} \ \frac{1}{t} \bigg( \Fsf^{(n-1)}(\phi_n + t \psi)[\psi_1,\dots ,\psi_{n-1}] - \Fsf^{(n-1)}(\phi)[\psi_1,\dots ,\psi_{n-1}] \bigg) \ .
\end{equation*}
%
The map $\Fsf$ is said to be \textbf{smooth} at $\phi \in \Usf$ if the limit exists for all $\psi_1, \dots, \psi_n \in \Xsf$, and if $\Fsf^{(n)}$ is jointly continuous on the product space $\Usf \times \Xsf^{\otimes n}$.
\end{definition}


We recall that a map 
%
\begin{equation*}
\fsf : \Usf \times \Xsf \to \Ysf
\end{equation*}
%
is \textbf{jointly continuous} at $(x,y) \in \Usf \times \Xsf$, if 
%
\begin{eqnarray*}
&& \mbox{for each neighborhood} \ \Ysf^\prime \ \mbox{of} \ \fsf(x,y) \ , \\
&& \exists \ \mbox{a product of open sets} \ \Xsf^\prime \times \Usf^\prime \subseteq \Xsf \times \Usf \ \mbox{containing} \ (x,y) \ , \\
&& \mbox{such that} \ \fsf(\Xsf^\prime \times \Usf^\prime) \subseteq \Ysf^\prime \ .
\end{eqnarray*}
%
For later purposes we notice that the derivatives of a functionals are distributions.


If instead of considering generic locally convex topological vector spaces $U$ and $V$ in the previous definition, we take $\Ecal(\Mcal)$ or $\Dcal(\Mcal)$ and $\Cbb$ respectively, then we have a precise definition of the derivative of an observable defined as a functional.

\bigskip

We illustrate this definition via a simple example. Here is the first two derivatives of a ``functional potential'' $\phi^4$. Let us consider a test function $\lambda \in \Dcal(\Mcal)$ and 
%
\begin{equation*}
\Vsf(\phi) := \int \dsf x \ \sqrt{\abs{g}} \ \frac{\lambda(x)}{4!} \phi(x)^4 \ ,
\end{equation*}
%
then
%
\begin{eqnarray*}
&& \Vsf^{(1)}(\phi)[\psi] = \int \dsf x \ \sqrt{\abs{g}} \ \frac{\lambda(x)}{3!} \phi(x)^3  \psi(x)\ , \\
%
&& \Vsf^{(2)}(\phi)[\psi_1,\psi_2] = \int \dsf x  \ \sqrt{\abs{g}} \ \frac{\lambda(x)}{2!} \phi(x)^2 \psi_1(x)\psi_2(x) \ ,
\end{eqnarray*}
%
or with a small common abuse of notation
%
\begin{equation*}
\Vsf^{(1)}(\phi) = \frac{\lambda(x)}{3!} \phi(x)^3 \ , \qquad
%
\Vsf^{(2)}(\phi) = \frac{\lambda(x)}{2!} \phi(x)^2 \delta(x,y)  
\end{equation*}
%
where $\delta$ is the Dirac distribution. 


\bigskip


We can show that the main results in differential calculus still hold in this framework.


\begin{lemma}
%
Let $\Fsf$ and $\Gsf$ be two functionals at least continuously differentiable, and let $\phi , \psi_{\sharp}$ contained in a locally convex topological vector space, e.g. $\Ecal(\Mcal)$ or $\Dcal(\Mcal)$. 
%
\begin{itemize}
%
\item fundamental theorem of calculus
%
\begin{equation*}
\Fsf(\phi + \psi) - \Fsf(\phi) = \int_0^1 \dsf t \ \Fsf^{(1)}(\phi+t\psi)[\psi] 
\end{equation*}
%
\item Taylor's formula
\begin{eqnarray*}
\Fsf(\phi + \psi) &=& \Fsf(\phi) + \Fsf^{(1)}(\phi)[\psi] + \dots + \frac{1}{n!} \Fsf^{(n)}(\phi)[\psi_1,\dots,\psi_n] \\
&& + \cfrac{1}{n!} \ \ \bigint_0^1 \dsf t \ (1-t)^n \ \Fsf^{(n+1)}(\phi+t\psi)[\psi^{\otimes n}]
\end{eqnarray*}
%
\item Leibniz formula
\begin{equation*}
\left(\Fsf \cdot \Gsf\right)^{(n)}(\phi)[\psi_1, \dots ,\phi_n] = \sum_{k=0}^{n} \binom{n}{k} \ \Fsf^{(k)}(\phi)[\psi_1, \dots , \psi_k] \ \Gsf^{(n-k)}(\phi)[\psi_1, \dots , \psi_{n-k}] \ .
\end{equation*}
%
\end{itemize}
%
\end{lemma}


We are now ready to introduce the space of observables we shall work with. It is a subset of $\Fcal_0(\Mcal)$, called the space of regular functionals. Let us give its definition.


\begin{definition}[Space of regular functionals]\label{def:obs_reg}
We define the space of regular functional as follow
\begin{equation*}
\Fcal_{\mathsf{reg}}(\Mcal) = \left\{ \Fsf(\phi) \ \bigg| \ \Fsf(\phi) \in \Fcal_0(\Mcal) \ \mbox{ is smooth}, \ \Fsf(\phi)^{(n)} \in \Ecal^\prime(\Mcal^{n}) \right\} \ ,
\end{equation*}
with $\phi$ a test function, i.e. element of $\Ecal(\Mcal)$.
\end{definition}


It is an still algebra. The space $\Ecal^\prime(\Mcal^{n})$ is the dual space of $\Ecal(\Mcal^{n})$\footnote{We will give more details on this space in the section \ref{p:DISTRIB}.}. We shall for now work with regular functionals. 


%----------------------------------------------------------------------------%
\section{Classical free field theory}\label{p:CLASSICAL}
%----------------------------------------------------------------------------%


In the previous sections we have introduced the set of observables of a scalar field theory in the functional approach, we discuss now how to describe a classical field theory in this framework, therefore we have as equation of motion the \textbf{generalised Klein Gordon} equation.
%
\begin{equation} 
\Psf \phi = \left( \Box + V^\prime \right) \phi = 0 \ , \
\mbox{ where } \ V^\prime = \xi R + m^2 + V \ . 
\label{eq:kg_eq}
\end{equation}
%
The coefficient $m$ denotes the (positive real) mass of the theory, $\xi \in \Rbb$ describes the coupling with the scalar curvature $R$, and $V$ is a non linear local potential.


In the present section \ref{p:CLASSICAL} we shall only consider $V=0$, i.e. we shall work only on free theory. We also require in the case of vanishing curvature ($\xi=0$) that the genealized Klein Gordon equation \eqref{eq:kg_eq} reduces itself to the Klein Gordon equation of the free scalar field theory on Minkowski spacetime (cf. definition \ref{def:minkowski}). The case $\xi=0$ is called minimal coupling, and $\xi=\frac16$ conformal coupling (for more details look at \cite{waldGR}).


We recall that $\Mcal$ is chosen to be a globally hyperbolic spacetime (cf. definition \ref{def:cst}), therefore the differential equation \eqref{eq:kg_eq} admits an unique solution once we fix enough initial conditions on a generic Cauchy surface $\Sigma$. It is actually a well posed initial value problem on $\Mcal$,
%
\begin{equation}
\Psf \phi = f \ , \quad \phi|_\Sigma = \phi_1 \ , \quad \nabla_n \phi |_\Sigma = \phi_2 \ ,
\label{eq:init_val_pb}
\end{equation}
%
where $f \in \Dcal(\Mcal)$, $\phi_1, \phi_2 \in \Dcal(\Sigma)$, and $n$ is a future directed timelike vector orthogonal to the Cauchy surface $\Sigma$ of $\Mcal$. The unique solution $\phi \in \Dcal(\Mcal)$ has the following support property
%
%%TODO ??
\begin{equation*}
\supp(\phi) \subset J\bigg( \supp(f) \cap \supp(\phi_1) \cap \supp(\phi_2),\Mcal \bigg) \ .
\end{equation*}
%
It has been shown in \cite{baer_wave_2008} that the \textbf{operator} $\Psf$ has unique \textbf{retarded} $\Delta_\rsf$ and \textbf{advanced} $\Delta_\asf$ \textbf{fundamental solutions}. Meaning there are unique continuous maps 
%
\begin{equation*}
\Delta_{\rsf / \asf} : \Dcal(\Mcal) \to \Dcal(\Mcal)  
\end{equation*}
%
satisfying 
%
\begin{eqnarray}
&& \Psf_x \Delta_{\rsf/\asf}(x,y) = \Delta_{\rsf/\asf}(x,y) \Psf_x = \delta(x,y) \ , 
\quad \Delta_{\asf}(x,y) = \Delta_{\rsf}(y,x) \ , 
\label{eq:identity_adv_ret} \\
&& \mbox{and} \qquad \supp(\Delta_{\rsf/\asf} f) = J^{\pm} \bigg(\supp(f) , \Mcal\bigg) \ , \nonumber
\end{eqnarray}
%
with $\delta$ the Dirac distribution, and $f$ a test function.


\begin{wrapfigure}{r}{0.35\textwidth}
\begin{center}
\begin{tikzpicture}[thick,scale=0.5]
\draw[left color=gray!50!black,right color=gray!50!black,middle color=gray!50,shading=axis,opacity=0.25] (-4,4) -- (4,4) -- (2,0) -- (-2,0) -- cycle;
\draw[left color=gray!50!black,right color=gray!50!black,middle color=gray!50,shading=axis,opacity=0.25] (-4,-4) -- (4,-4) -- (2,0) -- (-2,0) -- cycle;
\draw [left color=gray!50!black,right color=gray!50!black,middle color=gray!50,shading=axis,opacity=0.3] (0,4) ellipse (4cm and 0.6cm);
\draw [left color=gray!50!black,right color=gray!50!black,middle color=gray!50,shading=axis,opacity=0.3] (0,0) ellipse (2cm and 0.4cm);
\draw [left color=gray!50!black,right color=gray!50!black,middle color=gray!50,shading=axis,opacity=0.3] (0,-4) ellipse (4cm and 0.6cm);
\end{tikzpicture} 
\caption{Support of the causal propagator.}
\label{fig:supp_causal_prop}
\end{center}
\end{wrapfigure}


The support of $\Delta_{\mathsf{\rsf/\asf}} f$ is in the causal future (respectively past) of the test function $f$. The difference of the two fundamental solutions is called \textbf{causal propagator} of $\Psf$, actually it is defined as
%
\begin{equation*}
\Delta := \Delta_\rsf - \Delta_\asf \ , \quad \mbox{thus} \qquad \Psf_x \Delta(x,y) = 0 \ .
\end{equation*}
%
Its support is the union of the causal future and the causal past of the element applied to it (cf.figure \ref{fig:supp_causal_prop}). For all solutions $\phi$ of the equation of motion $\Psf \phi =0$ \eqref{eq:kg_eq}, with compactly supported initial conditions on a Cauchy surface $\Sigma$, it is possible to find \cite{baer_wave_2008} a (non unique) compactly supported smooth function $f$ such that
%
\begin{equation}
\phi = \Delta f \ . 
\label{eq:sol_express_prop}
\end{equation}
%
Furthermore if $\Delta f = 0$ then  $f = \Psf g$, with $g\in\Dcal(\Mcal)$. Finally if $\Psf$ is formally self adjoint, i.e. $\Delta(f,g) = - \Delta(g,f)$, where
%
\begin{equation}
\Delta(f,g) := \sm{f,\Delta g} = \int_\Mcal \dsf x \ \sqrt{\abs{g}} \ f(x) \ \left(\Delta g\right)(x) \ , 
\label{eq:smearing}
\end{equation}
%
then $\Delta_\rsf$ is the formal adjoint of $\Delta_\asf$ and vice versa.


\bigskip


Historically classical on shell field theories have been quantized starting from a symplectic structure. A such space is construct from the space of smooth real valued solutions of $\Psf \phi = 0$ \eqref{eq:kg_eq} with compaclty supported initial conditions on a Cauchy surface $\Sigma$, i.e.
%
\begin{equation}
\mathcalligra{S} \ \ (\Mcal) = \left\{ 
\begin{array}{l}
\phi \in \Ecal(\Mcal) \mbox{ such that } \Psf\phi=0 \ , \\
\mbox{ and } \phi \mbox{ satisfies the inital value problem } \eqref{eq:init_val_pb} 
\end{array}
\right\} \ .
\label{eq:sol_space}
\end{equation}
%
We equipe \eqref{eq:sol_space} with a strongly non degenerate map $\tau$, called symplectic form. It is defined as
%
\begin{equation*}
\tau : \Bigg\{
\begin{array}{ccl}
\mathcalligra{S} \ \ (\Mcal) \times \mathcalligra{S} \ \ (\Mcal) & \to & \Rbb \\
(\phi \ , \ \psi) & \mapsto & \bigint_\Sigma  \dsf \Sigma \ \bigg( \psi \ (\nabla_n \phi) - \phi \ (\nabla_n \psi) \bigg)
\end{array}
\ ,
\end{equation*}
%
where $\nabla_n = n_\mu \nabla^\mu$ is still the normal future directed derivative on $\Sigma$, and $\phi_1, \phi_2 \in \Dcal(\Sigma)$  are also still the initial data of the inital value problem \eqref{eq:init_val_pb}. We denote this space, called symplectic space, by 
%
\begin{equation*}
\left(\mathcalligra{S} \ \ (\Mcal),\tau\right) \ . 
\end{equation*}
%
The definition of the symplectic form is independent of the choice of $\Sigma$ we make. Indeed we can identify $\tau$ as the integral over $\Sigma$ of the current $j^\mu$ defined as
%
\begin{equation*}
j^\mu (\phi,\psi) := \psi (\nabla^\mu \phi) - \phi (\nabla^\mu \psi) \ . 
\label{eq:current}
\end{equation*}
%
with $\phi$ and $\psi$ two fields in the space of solutions \eqref{eq:sol_space}. Using the Klein Gordon equation we can show that $j^\mu$ is conserved, i.e. $\nabla_\mu j^\mu = 0$. Hence the divergence theorem implies that the choice of the Cauchy surface $\Sigma$ does not matter. 


Moreover the symplectic form $\tau$ can be written in terms of the causal propagator $\Delta$. Let us collect this result in the following lemma.


\begin{lemma}[Relation between the symplectic form and the causal propagator.]
Let $\phi$ and $\psi$ be two elements of the space of solutions \eqref{eq:sol_space}. Thanks to \eqref{eq:sol_express_prop} we have 
%
\begin{equation*}
\phi = \Delta f \ ,  
\end{equation*}
%
then symplectic form $\tau$ can be written in terms of the causal propagator $\Delta$ such that
%
\begin{equation*}
\tau(\phi,\psi) = \Delta(f,g) \ , \quad \mbox{with} \qquad \Delta(f,g) = \sm{f,\Delta g} \ ,
\end{equation*}
%
for suitable test functions. 
\end{lemma}


\begin{sketch}
Let $\phi$ and $\psi$ be elements of the space \eqref{eq:sol_space}, then we have for suitable test functions $f$ and $g$, $\phi=\Delta f$ and $\psi=\Delta g$. We write the definition of $\Delta(f,g)$ defined in \eqref{eq:sol_express_prop}, where we integrate over a subspace $\Ncal$ of $\Mcal$ according to the support of $f$ and $g$. Then we devide $\Ncal$ in two regions by a Cauchy surface $\Sigma$. Let us recall that due the hyperbolicity property of $\Mcal$ we are sure to find a Cauchy surface. $\Ncal$ is covered by the causal future and past of $\Sigma$. Next step we apply the Stokes' theorem on the two integrals and at the same time we use the identity \eqref{eq:identity_adv_ret}. Finally by having in mind that $\Delta_{\rsf/\asf}f$ is an element of the space \eqref{eq:sol_space} we identify the causal propagator to the symplectic form.
\end{sketch}


We would like to define classical free algebra. The on shell algebra is obtained by taking the quationt of the off shell algebra with respect to the ideal generated by the the equation of motion \eqref{eq:kg_eq}.



However we are going to describe an interacting real scalar field theory, where the interaction shall be treat by means of perturbation theory. Therefore the requirement that an equation of motion is satisfied shall be dropped. We shall work with the classical free off shell algebra of regular functionals.



\begin{definition}[Classical free off shell algebra]\label{def:alg_clas}
We define the classical free off shell algebra as follow
%
\begin{equation*}
\Acal_\reg(\Mcal) = \left(\Fcal_\reg(\Mcal), \cdot\right) \ ,
\end{equation*}
%
with $\Fcal_\reg(\Mcal)$ the space of regular functionals introduced in definition \ref{def:obs_reg}.
\end{definition}


%----------------------------------------------------------------------------%
\section{Wave front set of a distribution}
%----------------------------------------------------------------------------%


We mentioned in the section \ref{p:OBS} that the differentiability of functionals shall be used to define spaces of observables having the correct structure to perform quantization of the theory. We shall see in section \ref{p:Q_DEFORM} that we need to consider pointwise product of derivatives of non regular functionals, i.e. pointwise product of distribution. 


Therefore we shall in this section introduce the concept of distribution on a manifold, and characterize its singularities via the concept of wave front set. Finally we shall discuss a criterion due to Hörmander, see e.g. \cite{hormander_analysis_1990}, which permits to define the pointwise product of distributions.


%----------------------------------------------------------------------------%
\subsection{Distributions on a manifold}
\label{p:DISTRIB}
%----------------------------------------------------------------------------%


Let us first study distributions on $\Rbb^n$ and then generalize it to $\Mcal$. 


\bigskip


The set $\Xsf$ denotes for now on an open subset of $\Rbb^n$. As defined previously the spaces $\Dcal(\Xsf)$ and $\Ecal(\Xsf)$ denote respectively the space of compactly supported smooth functions on $\Xsf$ and the space of smooth functions on $\Xsf$. We call a $u$ \textbf{distribution}\index{distribution} on $\Xsf$,  a linear form on $\Dcal(\Xsf)$ such that for every compact set $\Ksf \subset \Xsf$ there exist constants $C$ and $k$ such that
%
\begin{equation*}
\abs{u(\phi)} \leq C \sum_{\abs{\alpha} \leq k} \sup \abs{\partial^\alpha \phi} \ , \quad \phi \in \Dcal(\Xsf) \ ,
\end{equation*}
%
where we used the same coventions as in \eqref{eq:conv_multiindex}. The set of all distribution in $\Xsf$ is denoted by $\Dcal^\prime(\Xsf)$. It is actually the dual space of $\Dcal(\Xsf)$. Thanks to linearity, we can also characterize a distribution in the following equivalent way. A linear form $u$ on $\Dcal(\Xsf)$ is a distribution if and only if 
%
\begin{equation*}
u(\phi_j) \to 0 \qquad  \mbox{when} \qquad j \to \infty \ ,
\end{equation*}
%
for every sequence $\phi_j \in \Dcal(\Xsf)$ converging to $0$ in the sense that
%
\begin{equation*}
\sup\abs{\partial^\alpha\phi_j} \to 0 \ ,
\end{equation*}
%
for every fixed $\alpha$, and for all $j$ and some fixed compact set $K \subset \Xsf$ such that $\supp(\phi_j) \subset K$. We invite the reader to look at the proof in the monograph of L. Hörmander \cite{hormander_analysis_1990}. 


Notice that it is not possible to evaluate a distribution $u \in \Dcal^\prime(\Xsf)$ on a single point in $\Xsf$, but we can \textbf{restrict} $u$ to a subset $\Usf \subset \Xsf$. 


As for functions we can define the \textbf{support} of $u \in \Dcal^\prime(\Xsf)$, denoted by $\supp(u)$\index{distribution!support}. It is the set of points in $\Xsf$ having no open neighborhood in which the restriction of $u$ is $0$, i.e. 
%
\begin{equation}
\supp(u) = \left\{ x \in \Xsf \ \left|
\begin{array}{l}
\forall u \in \Dcal^\prime(\Xsf) \ \mbox{ we have } \ u(\phi) \neq 0 \ \mbox{ for all test functions } \ \phi \ , \\ 
\mbox{with } \supp(\phi) \subseteq \Usf \ \mbox{ for any open neighborhood } \ \Usf \subseteq \Xsf \ \mbox{ of } \ x 
\end{array}
\right. \right\} \ .
\label{eq:supp_distribution}
\end{equation}
%
If a distribution $u$ has \textbf{compact support}, then $u \in \Ecal^\prime(\Xsf)$ is called \textbf{compactly supported distribution}. Let us give a last important result. Every locally integrable functions $u$ on $\Xsf$, can be seen as a distribution in $\Dcal^\prime(\Xsf)$, denoted by the same symbol $u$, by
%
\begin{equation}
u(f) = \sm{u,\phi} = \int_\Xsf \dsf x \ u(x) f(x) \ , 
\label{eq:distibution_function} 
\end{equation}
%
for $\phi \in \Dcal(\Xsf)$.


\bigskip


Let us now generalize this notion to \textbf{the case of a manifold} $\Mcal$. We recall that $\Ecal(\Mcal,\Ecal)$ and $\Dcal(\Mcal,\Ecal)$ are respectively the sapces of smooth and compaclty supported smooth sections of the vector bundle $E$ (see section \ref{p:LORENTZIAN_STRUCTURE}). Their topologies for the case of the real line bundle have been studied in section \ref{p:TOPO_CONFIG_SPACE}. A \textbf{distribution of the vector bundle $E$ with value in $\Ysf$} is a linear continuous map 
%
\begin{equation*}
u : \Dcal(\Mcal,E^\ast) \to \Ysf \ ,
\end{equation*}
%
where $E^\ast$ is the dual space of $E$. A such linear map $u$ is called continuous if for any $(\phi_n)$ with elements in $\Dcal(\Mcal,E^\ast)$ where $\phi_n \to \phi \in \Dcal(\Mcal,E^\ast)$, we have $u(\pi_n) \to u(\pi)$. The space formed by these continuous linear maps is denoted by $\Dcal^\prime(\Mcal,E,\Ysf)$. We shall restrict ourselves for now to the line vector vetor bundle. Therefore with a small common abuse of notation we shall denote this space by $\Dcal^\prime(\Mcal)$. 
As above we can define the support of a distribution $u \in \Dcal^\prime(\Mcal)$ and it is actually defined in the same way as \eqref{eq:supp_distribution}. If the support of $u \in \Dcal^\prime(\Mcal)$ is compact then $u \in \Ecal^\prime(\Mcal)$. For every submanifold $\Ncal \subset \Mcal$, any distribution $u \in \Dcal^\prime(\Mcal)$ can be restricted to a distribution $\left.u\right|_{\Ncal} \in \Dcal^\prime(\Ncal)$ such that
%
\begin{equation*}
\left.u\right|_{\Ncal}(\phi) = u(\phi) \ , 
\end{equation*}
%
for all $\phi \in \Dcal(\Ncal)$. An important remark is that a distribution on a manifold is completely supported by its restriction. Moreover as know $\Mcal$ looks locally like $\Rbb^n$, thus the definitions and properties obtained for distributions on $\Rbb^n$ are expected ``to be transfered'' to the case of a manifold. In particular \eqref{eq:distibution_function} can be extended to $\Mcal$ owing on Lorentzian manifold we have a distinguished volume element \ref{p:INTEGRATION}.


%----------------------------------------------------------------------------%
\subsection{Schwartz's space and tempered distributions}
%----------------------------------------------------------------------------%


Owing the last remark of section \ref{p:DISTRIB} saying that definitions and properties obtained for distributions on $\Rbb^n$ are expected ``to be transfered'' to $\Mcal$, we shall introduce here the concept of Fourier transform for functions and distributions only on $\Rbb^n$.


\bigskip


Let us consider an integrable function $f$ on $\Rbb^n$. The \textbf{Fourier transform} of $f$ is defined as 
%
\begin{equation*}
\Frak : f \mapsto  \hat{f}(k) = \int_{\Rbb^n} \dsf x \ f(x) \ \esf^{-i x \cdot k} \ , 
\end{equation*}
%
where $k \in \Rbb^n$, and $x \cdot k$ the inner product in $\Rbb^n$. The pointwise product of two Fourier transforms of two integrable functions $f$ and $g$ over $\Rbb^n$ can be written as the Fourier transform of the convolution of $f$ and $g$, i.e.
%
\begin{equation*}
\hat{f} \cdot \hat{g} = \Frak( f \ast g )(x) \ , \quad \mbox{with} \qquad (f \ast g) := \int_{\Rbb^n} \dsf y \ f(y) g(x-y) \ . 
\end{equation*}
%
There exists a particular space of functions which behaves particulary well under the action of $\Frak$. It is the subspace of $\Ecal(\Rbb^n)$ where all functions $\phi$ are \textbf{rapidly decreasing}, i.e.
%
\begin{equation*}
\mbox{for all } \ \phi \in \Ecal(\Rbb^n) \ \mbox{ we have } \ \norm{\phi}_{\alpha\beta} := \sup_x \abs{x^\alpha \partial^\beta \phi(x)} < \infty \ ,
\label{eq:norm_schwartz}
\end{equation*}
%
where we used dor the multiindices $\alpha$ and $\beta$ the same convention as in \eqref{eq:conv_multiindex}. We call this function space the \textbf{Schwartz space}\index{Schwartz space} and denote it by $\Scal(\Rbb^n)$. It can be shown that $\Scal(\Rbb^n)$ is complete. Furthermore the seminorm defined in \eqref{eq:norm_schwartz} genrate a topology on $\Scal(\Rbb^n)$, thererfore this space is metrizable. We can say then the Schwartz space is a Frechet space. In addition of being a subspace of $\Ecal(\Rbb^n)$ this new space is contained in $\Dcal(\Rbb^n)$, i.e.
%
\begin{equation*}
\Dcal(\Rbb^n) \subset \Scal(\Rbb^n) \subset \Ecal(\Rbb^n) \ .
\end{equation*}
%
As main properties we have for $\phi \in \Scal(\Rbb^n)$ 
%
\begin{eqnarray}
&& \Frak(\partial^\alpha \phi)(k) = k^\alpha \hat{\phi}(k) \ , \qquad \Frak(x^\alpha \phi)(k) = \partial^\alpha \hat{\phi}(k) \ , \nonumber \\[6pt]
&\mbox{and}& \phi(x) = (2\pi)^{-n} \int \dsf x \ \hat{f}(k) \ \esf^{i x \cdot \xi} \ .
\label{eq:inversion_formula}
\end{eqnarray}
%
The relation \eqref{eq:inversion_formula} is called the inversion formula. We can show that the map
%
\begin{equation*}
\Frak : \Scal(\Rbb^n) \to \Scal(\Rbb^n) \ , \quad \mbox{and its inverse} \qquad \Frak^{-1} : \Scal(\Rbb^n) \to \Scal(\Rbb^n)
\end{equation*}
%
are two continuous isomorphisms. The Fourier transform $\Frak$ can be used to give a criterion for the smoothness of functions in $\Dcal(\Rbb^n)$. Indeed we have the following equivalence property
%
\begin{equation}
\phi \in \Dcal(\Rbb^n) \ \  \Longleftrightarrow  \ \ \abs{\hat{\phi}(k)} \leq C (1+\abs{k})^{-n} \ , \ \mbox{ for } \ C \in \Rbb \ , \ \ n \in \Nbb \setminus \{0\} \ , \ \mbox{ and } \ \ \xi \in \Rbb^n \ .
\label{eq:criterion_smoothness_function}
\end{equation}
%
A function $\phi$ is smooth compactly supported if and only if its Fourier transform decays faster than any negative power of the dual variable $k$.


We can extend the concept of Fourier transform to the distributional framework. The dual $\Scal^\prime(\Rbb^n)$ is the space of all continuous linear form on $\Scal(\Rbb^n)$. We can show that $\Scal^\prime(\Rbb^n)$ is a space of distribution with the following property
%
\begin{equation*}
\Ecal^\prime(\Rbb^n) \subset \Scal^\prime(\Rbb^n) \subset \Dcal^\prime(\Rbb^n) \ .
\end{equation*}
%
Distributions on $\Scal^\prime(\Rbb^n)$ are called a \textbf{tempered distributions}\index{tempered distribution}. If $u \in \Scal^\prime(\Rbb^n)$, its Fourier transform is defined for $\phi \in \Scal(\Rbb^n)$ as 
%
\begin{equation*}
\sm{\hat{u},\phi} = \langle u,\hat{\phi}\rangle \ ,
\end{equation*}
%
where we used the smearing $u(\phi) = \sm{u,\phi}$. The convolution of $u \in \Scal^\prime(\Rbb^n)$ and $v\in \Ecal^\prime(\Rbb^n)$ is in $\Scal^\prime(\Rbb^n)$ and its Fourier transform is equal to the pointwise product of the Fourier transforms of $u$ and $v$, i.e.
%
\begin{equation*}
\Frak(u \ast v) = \hat{u} \cdot \hat{v} \ .
\end{equation*}
%
The Fourier transform map $\Frak$ which maps $\Scal^\prime(\Rbb^n)$ to itself is a sequentially continuous isomorphism, whereas its inverse is only sequentially continuous. We recall that a map $\fsf : \Xsf \to \Ysf$ where $\Xsf$ and $\Ysf$ are at least toplogical spaces, is said to be sequentially continuous if for any convergent sequence $(x_n) \to x$ in $\Xsf$, we have $(\fsf(x_n)) \to \fsf(x)$ in $\Ysf$. 


\bigskip


Let us illustrate the notions collected above with few examples. A first example is the Dirac distibution $\delta \in \Dcal^\prime(\Rbb)$ defined as 
%
\begin{equation*}
\sm{\delta , \phi} = \phi(0) \ , 
\end{equation*}
%
for $\phi \in \Dcal(\Rbb)$. We now compute its derivatives
%
\begin{equation*}
\sm{\partial^k \delta , \phi} = (-1)^k \sm{\delta , \partial^k \phi} =  (-1)^k \partial^k \phi(0) \ ,
\end{equation*}
%
with $k \in \Nbb \setminus \{0\}$, and where the first equality is obtained after performing $k$ successive integrations by part. A second example can be presented with the Heaviside function which is defined as
%
\begin{equation*}
H(x) = \left\{
\begin{array}{l}
1 \ \mbox{ if } \ x > 0 \\
0 \ \mbox{ if } \ x \leq 0
\end{array}
\right. \ .
\end{equation*}
%
We compute its derivative as
%
\begin{equation*}
\sm{\partial H , \phi} = - \sm{H , \partial \phi} = - \int_\Rbb \dsf x \ H(x) \partial \phi(x) = - \int_0^\infty \dsf x \ \partial \phi(x) = \phi(0) = \sm{\delta , \phi} \ .
\end{equation*}
%
Thus
%
\begin{equation*}
\partial H = \delta \ . 
\end{equation*}
%
A last example is the distribution $u \in \Dcal^\prime(\Rbb)$ given by
%
\begin{equation*}
u(x) = \lim_{\epsilon \downarrow 0} u_\epsilon \ , \quad \mbox{where} \qquad u_\epsilon(x) = \frac{1}{x^2 + i \epsilon} 
\end{equation*}




%----------------------------------------------------------------------------%
\subsection{Singularities and wave front set}\label{p:SING_WF}
%----------------------------------------------------------------------------%


We will give here a way to characterize the singularities of a distribution introducing the notion of wave front set. We shall see that it tells us not only at which points a singularity occurs, but it also indicates the directions in the dual space from which the singularities are ``coming'' from. We shall first analyze the particular case $\Mcal = \Rbb^n$ and then generalize it to a generic curved spacetime $\Mcal$. 


%%TODO ??
Before introducing the precise definition of the wavefront set, let us start recalling some preliminary important notions. We define the \textbf{singular support}\index{distribution!singular support} of a distribution $u$, $\singsupp(u)$, as the set of points in $X$ having no open neighborhood to which the restriction of $u$ is a smooth function. 


A set $\Sigma \subset \Rbb^n \setminus \{0\}$ is called a \textbf{conic set}\index{conic set} if for any point $\xi \in \Sigma$, it contains all the points $a \xi$ with $a > 0$. A conic set is completely determined by its intersection with the unit sphere $S^{n-1}$ in $\Rbb^n$. By a \textbf{conic neighborhood}\index{conic neighborhood} of a point $\xi \in \Xsf \setminus \{0\}$ we mean an open conic set that contains $\xi$. Using this new notion we can define what is a \textbf{regular direction}\index{regular direction} of a compaclty suppoted distribution $u\in\Ecal^\prime(\Xsf)$. It is a vector $\xi \in \Rbb\setminus\{0\}$ such that there exists an open canonical neighborhood $\Sigma$ of $\xi$ and such that \eqref{eq:criterion_smoothness} holds for $\xi$ in this set. Conversely a \textbf{singular direction}\index{singular direction} of a distribution $u$ is the set of all directions which are not regular, this set is denoted by $\Sigma (u)$.


The singular support gives the location of the singularities in space and the set of singular directions gives the high frequencies which are source of these singularities. The idea is to combine these two notions. We do it using the result from L. Hörmander proved in \cite{hormander_analysis_1990} which we recall.
%
\begin{lemma}
Let $u \in \Ecal^\prime(\Rbb^n)$ and $\phi \in \Ecal(\Rbb^n)$, then $\Sigma(\phi u) \subset \Sigma(u)$. 
\end{lemma}
%
Due to this result the set of singular directions of $u\in\Dcal^\prime(\Xsf)$ at a point $x$ is defined as
%
\begin{equation*}
\Sigma_x(u) := \underset{\phi}{\bigcap} \Sigma(\phi u) \ , 
\end{equation*}
%
with $\phi \in \Dcal(X)$ and $f(x) \neq 0$. Since it is an intersection of closed conic sets, $\Sigma_x(u)$ is also a closed conic set. This permits us to define the notion of wave front set of a distribution. 


If $u \in \Dcal^\prime(\Xsf)$, then the closed subset of $\Xsf \times (\Rbb^n \setminus 0)$ defined by
%
\begin{equation*}
\WF(u) := \bigg\{ (x,\xi) \in X \times (\Rbb^n \setminus 0) \ ; \ \xi \in \Sigma_x(u) \bigg\}
\end{equation*}
%
is called the \textbf{wave front set}\index{wave front set} of $u$. The projection of $\WF(u)$ to $X$ is $\mathsf{singsupp}(u)$. If $u \in \Ecal^\prime(\Rbb^n)$ then the projection of $\WF(u)$ on $(\Rbb^n \setminus 0)$ is $\Sigma(u)$. 


The main properties of the wave front set are the following \cite{hormander_analysis_1990}.


\begin{lemma}
\com{Remind the precise lemma/theorem of Hormander}
%
\begin{eqnarray}
&& \WF(\phi u) \subset \WF(u) \ , \nonumber \\
&& \WF(u+v) \subset \WF(u) \cup \WF(v) \ , \\ 
&& \WF(\Psf u) \subset \WF(u) \ , \nonumber
\label{eq:prop_wf}
\end{eqnarray}
%
with $u,v \in \Dcal^\prime(\Xsf)$, $f\in\Dcal(\Xsf)$, and $\Psf$ a \com{generic} linear differential operator. 
\end{lemma}


We would like now to be able to compute the wave front set of the pointwise product of two distributions. Consider $u,v\in \Dcal^\prime$, we notice that the pointwise product $uv$
if meaningful, it could be defined as the projection of the tensor product $u \otimes v$ to the diagonal. Therefore we shall first analyze the wave front set of the tensor product of two distributions. For $\Xsf$ and $\Ysf$ two open subsets of $\Rbb^n$, the tensor product of $u\in\Dcal^\prime(\Xsf)$ and $v\in\Dcal^\prime(\Ysf)$ has the following wave front set (for a proof, see e.g. theorem 8.2.9 in \cite{hormander_analysis_1990}).
%
\begin{eqnarray*}
\WF(u \otimes v) &\subset& \left( \WF(u) \times \WF(v) \right) \cup \\
&& \qquad \left( \left( \supp(u) \times \{0\} \right) \times \WF(v) \right) \cup \left( \WF(u) \times \left( \supp(v) \times \{0\} \right) \right) \ . 
\end{eqnarray*}
%
The product $uv$ is defined as the pullback of $u \otimes v$ under the ``diagonal'' map. We recall that the pullback of $u\otimes v$ by the map $d : \Xsf \times \Xsf \to \Xsf$ is well defined if 
%
\begin{equation*}
\WF(u \otimes v) \cap N_d = \emptyset \ , \quad \mbox{with } \ N_d = \left\{ (f(x,y) , \eta) \in ( \Xsf \times \Xsf )\times \Rbb^n | ^{\tsf}f^\prime(x,y) \eta = 0 \right\} \ .
\end{equation*}
\com{.... I don't understand anything.... what is $f$ what is its relation with $d$ you have not yet defined $d$ ....  }
%
\com{If we choose $d(x,y) \doteq xy$ the previous condition (Antoine check that this is the correct d!!!!)}
\sbar{This condition} tells us that the product $uv$ is well defined if 
%
\begin{equation*}
\com{\mbox{ for every } \ (x,\eta) \in \WF(u), \qquad \ (x,-\eta) \notin \WF(v) \ . }
\end{equation*}


\begin{theorem}
\com{Recall the precise theorem in Hormander!!!}
When the product is well defined we have 
%
\begin{equation*}
\WF(u v) \subset \left\{ (x,\xi+\xi^\prime) | (x,\xi) \in \WF(u) , (x,\xi^\prime) \in \WF(v) \right\} \cup \WF(u) \cup \WF(v) \ . 
\end{equation*} 
\end{theorem}

\com{ADD THEOREM HORMANDER PRODUCT DISTRIBUTUION!!!}
We now generalize this powerfull tool the framework of manifold. The wave front set of a distribution $u \in \Dcal^\prime(\Mcal)$, roughly speaking, is a conic subset of the cotangent bundle $\WF(u) \subset T^\ast\Mcal\setminus\{0\}$ where the first component gives the singular support, $\singsupp(u)$, and the second gives the direction in which the Fourier transform of $u$ does not decrease rapidly. Let us be more precise and consider two coordinate charts ${(U,\phi)}$ and ${(V,\psi)}$ covering both $\Mcal$. The wave front set of a distribution $u\in\Dcal^\prime(M,E)$ whose restriction to $U_i$ is $u_i$ is defined as the union of the wave front set of all restriction on the coordinate neighborhoods. The restriction $u_i$ has the following wave front set
%
\begin{equation*}
\WF(u_i) = \left\{ \left( x,T_x^\ast\phi(\xi) \right) \in U_i \times \left( \Rbb^n \setminus \{0\} \right) \ | \ \left( \phi(x) , \xi \right) \in \WF\left( \psi_i \circ u \circ \phi_i^{-1} \right) \right\} \ . 
\end{equation*}
%
Therefore
%
\begin{equation*}
\WF(u) = \bigcup_i \WF(u_i) \ . 
\end{equation*}


we can adapt the result in \eqref{eq:criterion_smoothness_function} for compaclty supported distributions. Indeed a distribution $u \in \Ecal^\prime(\Rbb^n)$ can also be seen as a function $u \in \Dcal(\Rbb^n)$.



%----------------------------------------------------------------------------%
\subsection{Explicit computations of wave front sets}
%----------------------------------------------------------------------------%


$\bullet$ \ \textbf{Wave front set} : local and covariant under coordinate transformations. \par
\quad $\to$ \ It generalises to CST (in contrast to the Fourier transform). \par

\bigskip

$\bullet$ \ \textbf{Examples} : \par

$\leadsto$ \ $\WF(\delta) = \left\{ (0,k) | k \in \Rbb^n , k \neq 0 \right\} $ \par
\textit{Proof}: The singular support of $\delta(x)$ is $\{0\}$ and $\hat{f\delta}(k) = f(0)$ is not fast decreasing if $f(0) = 0$. \par

$\leadsto$ 
\vspace*{-17pt}
\begin{equation*}
\hspace*{-110pt} u(x) = \frac{1}{x^2 + i \epsilon}, \quad \WF(u) = \{ (0;k) | k<0 \}
\end{equation*}
\textit{Proof}: By contour integration $\hat{u}(k) = -2i\pi \Theta(-k)$, thus
\begin{equation*}
\abs{\hat{fu}(k)} = \abs{ \frac{1}{2\pi} \int_\Rbb dq \ \hat{f}(q) \ \hat{u}(k-q) } = \abs{ \int_k^\infty dq \ \hat{f}(q) }
\end{equation*}
\qquad Fourier transform of a test function, $\hat{f}(q)$, is fast decreasing for $q \geq 0$ !!


%----------------------------------------------------------------------------%
\section{Formal deformation}\label{p:Q_DEFORM}
%----------------------------------------------------------------------------%


We shall now pass to discuss the quantization of the classical free theory we presented in section \ref{p:CLASSICAL}. In other words we shall quantize the algebra $\Acal_\reg(\Mcal)$ of classical observables in the functional approach.


%%TODO ??
For later purposes the quantum observables shall be defined as a power series in $\hbar$ with coefficients in $\Fcal_\reg(\Mcal)$. The space of these quantum observables will be denoted by $\Fcal_\reg(\Mcal)[[\hbar]]$. As for the classical counterparts, an involution $\ast$ is required and it is defined as 
%
\begin{equation*}
\Fsf^\ast(\phi) =  \overline{\Fsf(\overline{\phi})} \ ,
\end{equation*}
%
where the operation considered on the right hand side is the complex conjugation. The quantization procedure we choose is a formal deformation of the pointwise product. We define a product between elements in $\Fcal_\reg(\Mcal)[[\hbar]]$, called the $\star$ product, in the following way
%
\begin{equation}
(\Fsf \star_{\Delta_+} \Gsf)(\phi) := \Fsf(\phi) \cdot \Gsf(\phi) + \sum_{n=1}^\infty \frac{\hbar^n}{n!} \sm{ \Fsf^{(n)} , \Delta_+^{\otimes n} \Gsf^{(n) } } \ ,
\label{eq:q_prod}
\end{equation}
%
where $\Delta_+$ is a Hadamard distribution of the free theory we study. The $\star$ prdouct \eqref{eq:q_prod} satisfies the following properties
%
\begin{equation*}
\overline{\Fsf \star \Gsf} = \overline{\Fsf} \star \overline{\Gsf} \ .
\end{equation*}
%
This so-called Hadamard distribution is a distribution which satisfies the following conditions.
%
\begin{enumerate}
\item The antisymmetric part is proportional to the commutator function $\Delta$\footnote{The commutator function is related to causal propagator introduced in section, indeed \ref{p:CLASSICAL} $\Delta(f,g)=\sm{f,\Delta g}$.}, i.e. 
%
\begin{eqnarray*}
i \Delta(f,g) = \Delta_+(f,g) - \Delta_-(f,g) \ , \quad \mbox{ with } \ \Delta_+(f,g) = \overline{\Delta_-(f,g)} \ ,
\end{eqnarray*}
%
with $f,g \in \Dcal(\Mcal)$.

\item The microlocal spectrum condition holds, namely
%
\begin{equation}
\WF(\Delta_+) \ = \ \bigg\{ \bigg( x, y ; k_x, k_y \bigg) \in T^\ast\Msf^2 \setminus \{0\} \ \bigg| \ (x,k_x) \sim (x,-k_y), \ k_x \triangleright 0 \bigg\} 
\label{eq:wf_hadamard}
\end{equation}
%
where $(x,k_x) \sim (x,k_y)$ implies that there exists a null geodesic $\gamma$ connecting $x$ to $y$ such that $k_x$ is coparallel and cotangent to $\gamma$ at $x$ and $k_y$ is the parallel transport of $k_x$ from $x$ to $y$ along $\gamma$. Furthermore, $k_x \triangleright 0$ means that the covector $k_x$ is future directed.
\end{enumerate}


We also notice that in the product \eqref{eq:q_prod} we get in general products of distributions. As a first observation, we notice that here we are considering regular functionals, hence the distributions involved are actually regular functions and the problems with pointwise multiplications with $\Delta_+^n$ are avoided.


The product introduced in \eqref{eq:q_prod} seems to depend on the choice of $\Delta_+$, however, if we choose a different $\Delta^\prime_+$ with the same properties, we have that $w:=\Delta^\prime_+ - \Delta_+$. Moreover, thanks to the work of Radzikowski \cite{radzikowski_micro-local_1996} $w$ is real, smooth and symmetric and
%
\begin{eqnarray}
&& \Fsf \star_{\Delta^\prime_+} \Gsf = \alpha_w \left(\alpha_{-w}(\Fsf) \star_{\Delta_+} \alpha_{-w}(\Gsf)\right) \ , \nonumber \\[6pt]
\mbox{ with } && \alpha_{w}(\Fsf) = \exp\left(\hbar \sm{ w(x,y) , \frac{\delta}{\delta\phi(x)} \ \frac{\delta}{\delta\phi(y)} } \right) \Fsf \ .
\label{eq:alpha_isomorph}
\end{eqnarray}
%
Thus the algebras constructed with $\star_{\Delta_+}$ and $\star_{\Delta^\prime_+}$ are isomorphic via $\alpha_{w}$ and thus the construction does not really depend on this choice. 


An important remark concerning the definition of the $\star$ product is that the power series in $\hbar$ does not seem to converge, this is one of the reason why we chose to work with formal power series have a finite numbers of non zero functional derivatives, and thus the series in the $\star$ product among them is always convergent. We also notice that applying the limit $\hbar \to 0$ to the $\star$ product we get the classical pointwise product
%
\begin{equation*}
\Fsf \star \Gsf \underset{\hbar \to 0}{=} \Fsf \cdot \Gsf \ . 
\end{equation*}


All in all we have defined the regular quantum algebra as the free off shell $\ast-$algebra of regular functionals endowed with a $\star$ product, denoted
%
\begin{equation}
\Acal_\reg(\Mcal)[[\hbar]] = \left(\Fcal_\reg(\Mcal)[[\hbar]] , ^\ast , \star \right) \ . 
\label{eq:alg_q_reg}
\end{equation}
%
It is a noncommutative, associative $\ast-$algebra.


%----------------------------------------------------------------------------%
\section{Interacting picture}
%----------------------------------------------------------------------------%


We shall present now the perturbative construction of an interacting quantum field theory on a generic curved spacetime $\Mcal$. We shall work in the framework of perturbative algebraic quantum field theory (\textbf{pAQFT}\index{perturbative algebraic quantum field theory (pAQFT)}) which has been recently developed in \cite{brunetti_perturbative_2009,fredenhagen_perturbative_2015,fredenhagen_batalin-vilkovisky_2013} and it is based on earlier ideas of Esptain and Glaser and Steinmann \cite{epstein_glaser,steinmann_perturbation_1971}.


%----------------------------------------------------------------------------%
\subsection{Enlarged observables spaces}
%----------------------------------------------------------------------------%


We shall construct observables of the interacting theory in section \ref{p:INT_Q_ALG} by means of perturbation theory, namely when the free algebra is perturbed by a non linear local potential. However up to now we have worked among regular functionals only, unfortunately the case of interacting potentials constructed with regular functionals are of limited interest. We aim to be able to work with local non linear interacting potentials like
%
\begin{equation}
\Vsf(\phi) = \int_\Mcal \dsf x \ \sqrt{\abs{g}} \ \frac{\lambda_n(x)}{n!} \phi^n(x) \ ,
\label{eq:local_pot}
\end{equation}
%
with $\lambda_n \in \Dcal(\Mcal)$ and $n\geq2$. Hence, we have have to enlarge the space of functionals we are considering. The first new space is  called space of \textbf{microcausal functionals} and it is defined as follows
%
\begin{equation}
\Fcal_{\mu\csf}(\Mcal) := \left\{ 
\Fsf(\phi) \ \bigg| \ 
\begin{array}{l}
\Fsf(\phi) \in \Fcal(\Mcal), \ \Fsf(\phi)^{(n)} \in \Ecal^\prime(\Mcal^{\otimes n}) \\
\mbox{ and } \ \WF(\Fsf^{(n)}(\phi)) \cap \left( \Mcal^n \times ( \overline{V^{n}_{+}} \cup \overline{V^{n}_{-}} ) \right)  = \emptyset 
\end{array}
\right\} \ .
\label{eq:func_micro}
\end{equation}
\index{$\Fcal_{\mu\csf}(\Mcal)$}
%
We shall work with its quantum version $\Fcal_{\mu\csf}(\Mcal)[[\hbar]]$. This space contains the functionals representing the local interaction potentials but not only. For instance the regular functionals are still contained in it. The space which contains only the interaction potentials is called space $\mathcal{F}_\mathsf{loc}$ of local functionals. We define it as the space of microcausal functionals having as support for their derivatives the total diagonal $d_n$
%
\begin{equation*}
\Fcal_\loc(\Mcal) := \left\{ \Fsf(\phi) \in \Fcal_{\mu\csf}(\Mcal) \ \bigg| \ \supp\left(\Fsf(\phi)^{(n)}\right) \subset d_n \right\} \ ,
\label{eq:func_loc}
\end{equation*}
\index{$\Fcal_{\mathsf{loc}}(\Mcal)$}
%
where the \textbf{total diagonal}\index{diagonal!total} $d_n$ is defined as
%
\begin{equation}
d_n = \left\{ (x,\dots,x) \subset \Mcal^n \right\} \ .
\label{eq:total_diag}
\end{equation}
\index{$d_n$}
%
Again the quantum version $\Fcal_{\mathsf{loc}}(\Mcal)[[\hbar]]$ shall be considered.


%----------------------------------------------------------------------------%
\subsection{Quantum product among local functionals}
%----------------------------------------------------------------------------%


In section \ref{p:Q_DEFORM} we have defined a new product among regular functionals $\Fcal_\reg(\Mcal)[[\hbar]]$ which encodes the quantum properties needed.

If we consider two fonctionals in $\Fcal_{\mathsf{\mu c}}(\Mcal)$ we can, using the causal propagator, define the product of them as follow
\begin{equation*}
 (F \star_\Delta G)(\phi) \ \doteq \ F(\phi).G(\phi) + \sum_{n=1}^{\infty} \frac{i^n \hbar^n}{2^n n!} \sm{F^{(n)},\Delta^{\otimes n} G^{(n)}} \ ,
\end{equation*}
%
The space $\left( \Fcal_{\mathsf{loc}}(\Mcal) , \star_\Delta \right)$ with the $\ast$ operation, is the quantized algebra. We should check now if the product of causal propagators is well defined, because as we already said before, products of distribution are not always well defined. We know that the wave front set of $\Delta$ is given by
\begin{equation*}
 \WF(\Delta) \ = \ \bigg\{ \left(x,y;k_x,-k_y\right) \in \Tsf^{\ast}(\Mcal^{2}) \backslash\{0\} \ \bigg| \ (x,k_x) \sim (y,k_y) \bigg\} \ .
\end{equation*}
where $(x,k_x) \sim (y,k_y)$ implies that there exists a null geodesic $\gamma$ connecting $x$ to $y$ such that $k_x$ is coparallel and cotangent to $\gamma$ at $x$ and $k_y$ is the parallel transport of $k_x$ from $x$ to $y$ along $\gamma$. Therefore
\begin{equation*}
 \WF(\Delta) \ \oplus \ \WF(\Delta) \ = \ \bigg\{ \left(x,y;k_x+q_x,-k_y-q_y\right) \in \Tsf^{\ast}(\Mcal^{2}) \backslash\{0\} \ \bigg| \ 
 \begin{array}{l}
  \left(x,y;k_x,-k_y\right) \in \WF(\Delta) \\ 
  \left(x,y;q_x,-q_y\right) \in \WF(\Delta)
 \end{array}
 \bigg\} \ .
\end{equation*}
There is no restriction on the direction of $k_0$ in particular, hence it can intersect the empty set. Therefore the product of causal propagators is ill defined. Another candidate is the Feynman Hadamard distribution because in its wave front set there is a condition on $k_0$, then the product of two of this distribution is well defined. \par


\bigskip



\begin{equation*}
\Fsf^\ast(\phi) =  \overline{\Fsf(\overline{\phi})} \ ,
\end{equation*}
%
where the operation considered on the right hand side is the complex conjugation. The quantization procedure we choose is a formal deformation of the pointwise product. We define a product between elements in $\Fcal_\reg(\Mcal)[[\hbar]]$, called the $\star$ product, in the following way
%
\begin{equation}
(\Fsf \star_{\Delta_+} \Gsf)(\phi) := \Fsf(\phi) \cdot \Gsf(\phi) + \sum_{n=1}^\infty \frac{\hbar^n}{n!} \sm{ \Fsf^{(n)} , \Delta_+^{\otimes n} \Gsf^{(n) } } \ ,
%\label{eq:q_prod}
\end{equation}
%
where $\Delta_+$ is a Hadamard distribution of the free theory we study. The $\star$ prdouct \eqref{eq:q_prod} satisfies the following properties
%
\begin{equation*}
\overline{\Fsf \star \Gsf} = \overline{\Fsf} \star \overline{\Gsf} \ .
\end{equation*}
%
This so-called Hadamard distribution is a distribution which satisfies the following conditions.
%
\begin{enumerate}
\item The antisymmetric part is proportional to the commutator function $\Delta$\footnote{The commutator function is related to causal propagator introduced in section, indeed \ref{p:CLASSICAL} $\Delta(f,g)=\sm{f,\Delta g}$.}, i.e. 
%
\begin{eqnarray*}
i \Delta(f,g) = \Delta_+(f,g) - \Delta_-(f,g) \ , \quad \mbox{ with } \ \Delta_+(f,g) = \overline{\Delta_-(f,g)} \ ,
\end{eqnarray*}
%
with $f,g \in \Dcal(\Mcal)$.

\item The microlocal spectrum condition holds, namely
%
\begin{equation}
\WF(\Delta_+) \ = \ \bigg\{ \bigg( x, y ; k_x, k_y \bigg) \in T^\ast\Msf^2 \setminus \{0\} \ \bigg| \ (x,k_x) \sim (x,-k_y), \ k_x \triangleright 0 \bigg\} 
%\label{eq:wf_hadamard}
\end{equation}
%
where $(x,k_x) \sim (x,k_y)$ implies that there exists a null geodesic $\gamma$ connecting $x$ to $y$ such that $k_x$ is coparallel and cotangent to $\gamma$ at $x$ and $k_y$ is the parallel transport of $k_x$ from $x$ to $y$ along $\gamma$. Furthermore, $k_x \triangleright 0$ means that the covector $k_x$ is future directed.
\end{enumerate}


We also notice that in the product \eqref{eq:q_prod} we get in general products of distributions. As a first observation, we notice that here we are considering regular functionals, hence the distributions involved are actually regular functions and the problems with pointwise multiplications with $\Delta_+^n$ are avoided.


The product introduced in \eqref{eq:q_prod} seems to depend on the choice of $\Delta_+$, however, if we choose a different $\Delta^\prime_+$ with the same properties, we have that $w:=\Delta^\prime_+ - \Delta_+$. Moreover, thanks to the work of Radzikowski \cite{radzikowski_micro-local_1996} $w$ is real, smooth and symmetric and
%
\begin{eqnarray}
&& \Fsf \star_{\Delta^\prime_+} \Gsf = \alpha_w \left(\alpha_{-w}(\Fsf) \star_{\Delta_+} \alpha_{-w}(\Gsf)\right) \ , \nonumber \\[6pt]
\mbox{ with } && \alpha_{w}(\Fsf) = \exp\left(\hbar \sm{ w(x,y) , \frac{\delta}{\delta\phi(x)} \ \frac{\delta}{\delta\phi(y)} } \right) \Fsf \ .
%\label{eq:alpha_isomorph}
\end{eqnarray}
%
Thus the algebras constructed with $\star_{\Delta_+}$ and $\star_{\Delta^\prime_+}$ are isomorphic via $\alpha_{w}$ and thus the construction does not really depend on this choice. 



\bigskip

We see that in the definition of the Bogoliubov formula \eqref{eq:bogoliubov} there are the two products, the $\star$ product and the time oredere product. These two are well defined amaong regular functionals $\Fcal_\reg(\Mcal)[[\hbar]]$, but we need to look at their behavior among local functionals $\Fcal_\loc(\Mcal)[[\hbar]]$. In particular we need to know if the pointwise products of $\Delta_+$ and $\Delta_\fsf$ are well defined or not.


Let us consider an example to illustrate the situation. We choose to work with the following functional
%
\begin{equation*}
\Fsf(\phi) = \int_\Mcal \dsf x \ \sqrt{\abs{g}} \ \phi(x)^2 \ f(x)  \ ,
\end{equation*}
%
with $f \in \Dcal(\Mcal)$. It has for derivatives
%
\begin{eqnarray*}
\Fsf^{(1)}(\phi) = 2 \ \phi(x) \ f(x) \ , \quad \Fsf^{(2)}(\phi) = 2 \ f(x) \ \delta(x,y) \ , \quad \mbox{and} \qquad \Fsf^{(3)}(\phi) = 0 \ .
\end{eqnarray*}
%
Therefore using \eqref{eq:q_prod} and \eqref{eq:time_ordered_prod} we can write explicitely the $\star$ product and the time ordered product for $\Fsf(\phi)$ as follow.
%
%%TODO ??
\begin{eqnarray*}
(\Fsf \star \Fsf)(\phi) &=& \Fsf(\phi) \cdot \Fsf(\phi) + \hbar \sm{ \Fsf^{(1)}(\phi) , \Delta_+ \ \Fsf^{(1)}(\phi) } + \ \frac{\hbar^2}{2} \sm{ \Fsf^{(2)}(\phi) , \Delta_+^{\otimes 2} \ \Fsf^{(2)}(\phi)} \\
%
&=& \int \dsf x \ \dsf y \ f(x) f(y) \ \bigg( \phi^2(x) \phi^2(y) + 4 \hbar \phi(x) \phi(y) \Delta_+(x,y) + 2 \hbar^2 \left(\Delta_+(x,y)\right)^2 \bigg) \ , \\[8pt]
%
(\Fsf \cdot_{\Tsf_{\Delta_\fsf}} \Fsf)(\phi) &=& \Fsf(\phi) \cdot \Fsf(\phi) + \hbar \sm{ \Fsf^{(1)}(\phi) , \Delta_\fsf \ \Fsf^{(1)}(\phi) } + \ \frac{\hbar^2}{2} \sm{ \Fsf^{(2)}(\phi) , \Delta_\fsf^{\otimes 2} \ \Fsf^{(2)}(\phi)} \\
%
&=& \int \dsf x \ \dsf y \ f(x) f(y) \ \bigg( \phi^2(x) \phi^2(y) + 4 \hbar \phi(x) \phi(y) \Delta_\fsf(x,y) + 2 \hbar^2 \left(\Delta_\fsf(x,y)\right)^2 \bigg) \ .
\end{eqnarray*}
%
We see that we in both product we have to consider pointwise product of distribution. We need to know if they are well defined or not. For doing it we shall use the notion of wave front set introduced in section \ref{p:SING_WF}.


%----------------------------------------------------------------------------%
\subsection{Time ordered product}
%----------------------------------------------------------------------------%




In every perturbative construction of an interacting theory there is the need of writing products of the perturbation potential where the factors are ordered in time. Hence, a necessary tool to construct a perturbative interacting theory is the time ordered product. Let us start constructing it in the case of regular functionals where it is completely characterized by the \textbf{causal factorisation property}\index{causal factorisation property}, namely by the condition
%
\begin{equation}
\Fsf \cdot_\Tsf \Gsf = 
\left\{
\begin{array}{ll}
\Fsf \star \Gsf \quad \mbox{if } \ \supp(\Fsf) \ \mbox{ is later than  } \ \supp(\Gsf)  \\
\Gsf \star \Fsf \quad \mbox{if } \ \supp(\Gsf) \ \mbox{ is later than  } \ \supp(\Fsf) 
\end{array}
\right. \ ,
\label{eq:causal_factorization}
\end{equation}
%
where $\supp(\Fsf)$ is said to be  later than   $\supp(\Gsf)$ if $J^+(\supp(\Fsf)) \cap \supp(\Gsf)=\emptyset$.




We see that in the definition of the Bogoliubov formula \eqref{eq:bogoliubov} there are the two products, the $\star$ product and the time oredere product. These two are well defined amaong regular functionals $\Fcal_\reg(\Mcal)[[\hbar]]$, but we need to look at their behavior among local functionals $\Fcal_\loc(\Mcal)[[\hbar]]$. In particular we need to know if the pointwise products of $\Delta_+$ and $\Delta_\fsf$ are well defined or not.


Let us consider an example to illustrate the situation. We choose to work with the following functional
%
\begin{equation*}
\Fsf(\phi) = \int_\Mcal \dsf x \ \sqrt{\abs{g}} \ \phi(x)^2 \ f(x)  \ ,
\end{equation*}
%
with $f \in \Dcal(\Mcal)$. It has for derivatives
%
\begin{eqnarray*}
\Fsf^{(1)}(\phi) = 2 \ \phi(x) \ f(x) \ , \quad \Fsf^{(2)}(\phi) = 2 \ f(x) \ \delta(x,y) \ , \quad \mbox{and} \qquad \Fsf^{(3)}(\phi) = 0 \ .
\end{eqnarray*}
%
Therefore using \eqref{eq:q_prod} and \eqref{eq:time_ordered_prod} we can write explicitely the $\star$ product and the time ordered product for $\Fsf(\phi)$ as follow.
%
%%TODO ??
\begin{eqnarray*}
(\Fsf \star \Fsf)(\phi) &=& \Fsf(\phi) \cdot \Fsf(\phi) + \hbar \sm{ \Fsf^{(1)}(\phi) , \Delta_+ \ \Fsf^{(1)}(\phi) } + \ \frac{\hbar^2}{2} \sm{ \Fsf^{(2)}(\phi) , \Delta_+^{\otimes 2} \ \Fsf^{(2)}(\phi)} \\
%
&=& \int \dsf x \ \dsf y \ f(x) f(y) \ \bigg( \phi^2(x) \phi^2(y) + 4 \hbar \phi(x) \phi(y) \Delta_+(x,y) + 2 \hbar^2 \left(\Delta_+(x,y)\right)^2 \bigg) \ , \\[8pt]
%
(\Fsf \cdot_{\Tsf_{\Delta_\fsf}} \Fsf)(\phi) &=& \Fsf(\phi) \cdot \Fsf(\phi) + \hbar \sm{ \Fsf^{(1)}(\phi) , \Delta_\fsf \ \Fsf^{(1)}(\phi) } + \ \frac{\hbar^2}{2} \sm{ \Fsf^{(2)}(\phi) , \Delta_\fsf^{\otimes 2} \ \Fsf^{(2)}(\phi)} \\
%
&=& \int \dsf x \ \dsf y \ f(x) f(y) \ \bigg( \phi^2(x) \phi^2(y) + 4 \hbar \phi(x) \phi(y) \Delta_\fsf(x,y) + 2 \hbar^2 \left(\Delta_\fsf(x,y)\right)^2 \bigg) \ .
\end{eqnarray*}
%
We see that we in both product we have to consider pointwise product of distribution. We need to know if they are well defined or not. For doing it we shall use the notion of wave front set introduced in section \ref{p:SING_WF}. Let us look first at $\Delta_+$, its wave front set was given in \eqref{eq:wf_hadamard}. Due to the restriction on the ``temporal codirection'' $k_0$, the sum $\WF(\Delta_+) + \WF(\Delta_+)$ cannot intersect the empty set. Therefore the product is well defined even  for local functionals. It remains to discuss the case of the time ordered product, i.e. the pointwise product of $\Delta_\fsf$ which satisfy the following identities.
%
\begin{equation}
\Delta_\fsf(x,y) = \Delta_+(x,y) + i \Delta_\asf(x,y) = \Delta_-(x,y) + i \Delta_\rsf(x,y) 
\label{eq:conv_feynman_prop}
\end{equation}
%
Thanks to the properties \eqref{eq:prop_wf} we can conclude that wave front set of the integral kernel of $\Delta_\asf$ contains the wave front set the Dirac distribution $\delta$ because $\Psf \Delta_\asf = \Ibb$, and then pointwise products of $\Delta_\fsf$ with itself are ill defined on coinciding points.  We see that we need an extension of the time ordered product. Unfortunately this extension is not trivial and it requires the introduction of the theory of regularization. We shall come back to this issue in section \ref{p:EPSTEIN_GLASER} and chapter \ref{p:COV_REG} which is the core of the present work.

\bigskip

Our task here is to give the necessary requirements to build time ordered product in order to get a physically meaningful theory. It has been done in \cite{hollands_local_2001,hollands_existence_2002}. We shall just recall the main ideas adapted to the framework we use.


The  time ordered product of ``causal'' functionals can also be view as the following map
%
\begin{equation}
\Tcal_n \ : \ 
\left\{
\begin{array}{lcl}
\Fcal_\Tsf(\Mcal)[[\hbar]]^{\otimes n} & \to & \Fcal_\Tsf(\Mcal)[[\hbar]] \\
\Fsf_1(\phi) \otimes \ ... \ \otimes \Fsf_n(\phi) & \mapsto & \Fsf_1(\phi) \cdot_{\Tsf} \ ... \ \cdot_{\Tsf} \Fsf_n(\phi)
\end{array}
\right. \ .
\label{eq:time_ordered_op}
\end{equation}
%
We shall now impose some conditions on $\Tcal_n$. 



\begin{description}
%
\item[1 -- Causal factorization.] $\Tcal_n$ has to satisfy \eqref{eq:causal_factorization}.
%
\item[2 -- Symmetry.] $\Tcal_n$ has to be symmetric under a permutations of its arguments.
%
\item[3 -- Unitarity.] $\Tcal_n$ has to be unitary.
%
\item[4 -- Local and covariant.] $\Tcal_n$ has to be local and covariant.
%
\item[5 -- Microlocal spectrum.] $\Tcal_n$ has to satisfy the microlocal spectrum condition.
%
\item[6 -- Field Independence.] $\Tcal_n$ has to be $\phi$ independent.
%
\item[7 -- Leibniz rule.] $\Tcal_n$ has to satisfy the Leibniz rule.
%
\item[8 -- Principle of perturbative agreement.] 
%%TODO ??
$\Tcal_n$ has to satisfies the Principle of Perturbative Agreement for perturbations of the generalised mass term $\mu$ in the free Klein Gordon equation
%
\begin{equation*}
\Psf \phi = \left( - \Box + \mu \right) \phi = 0 \ . 
\end{equation*}
%
\item[9 -- Isometry invariance.] If the spacetime $\Mcal$ has non trivial isometries and if the Feynman propagator $\Hsf_\fsf$ is chosen such as to be invariant under these isometries, then $\Tcal_n$ has to be invariant under these isometries as well.
%
\end{description}



%----------------------------------------------------------------------------%
\subsection{Interacting off shell quantum algebra}
\label{p:INT_Q_ALG}
%----------------------------------------------------------------------------%



Let us discuss how it is possible to construct observables of the interacting theory by means of perturbation theory, namely, when the free algebra is perturbed by a non linear local potential $\Vsf$. In the latter case the interacting algebra is represented on the free algebra by means of the \textbf{Bogoliubov formula} which is given in terms of the local $S$ matrix, 
%
\begin{equation}
S(\Vsf) = \exp_\Tsf\left(\Vsf\right) = \sum^\infty_{n=0} \frac{i^n}{n!\ \hbar^n} \ \underbrace{\Vsf(\phi) \cdot_\Tsf \cdots \cdot_\Tsf \Vsf(\phi)}_{n \mbox{ times }} \ , 
\label{eq:S_matrix}
\end{equation}
%
where $\Vsf(\phi)$ is a local functional representing the interacting potential. The the time ordering product, which is well defined among regular functionals, as the following explicit form
%
\begin{equation}
(\Fsf \cdot_{\Tsf_{\Delta_\fsf}}  \Gsf)(\phi) = \Fsf(\phi) \cdot \Gsf(\phi) + \sum_{n=1}^\infty \frac{\hbar^n}{n!} \sm{ \Fsf(\phi)^{(n)} , \Delta_\fsf^{\otimes n} \ \Gsf(\phi)^{(n)} } \ ,
\label{eq:time_ordered_prod}
\end{equation}
%
where $\Delta_\fsf$ is the time ordered (Feynman) version of $\Delta_+$
%
\begin{equation*}
\Delta_\fsf(x,y) = \Theta(t_x-t_y) \Delta_+(x,y) + \Theta(t_y-t_x) \Delta_-(x,y) \ ,
\end{equation*}
%
with $\Theta$ the Heaviside function. 


\bigskip



As for the $\star$ product we can wonder if the time ordered product introduced in \eqref{eq:time_ordered_prod} depends on the choice of $\Delta_\fsf$. Again if we choose a different $\Delta^\prime_\fsf$ with the same properties, we have that $w:=\Delta^\prime_\fsf - \Delta_\fsf$, and using $\alpha_{w}$ defined in \eqref{eq:alpha_isomorph} the algebras build with $\cdot_{\Tsf_{\Delta_\fsf}}$ and $\cdot_{\Tsf_{\Delta^\prime_\fsf}}$ are isomorphic and thus the construction is independent of this choice. 


\bigskip


%%TODO ??
The time ordered product in \eqref{eq:S_matrix} and \eqref{eq:time_ordered_prod} can be ill defined for local functionals. Nonetheless let us present the Bogoliubov formula  assuming well defined products and then we shall focus on this issue. Therefore the observables are mapped from the interacting off shell algebra to the free off shell algebra with this Bogoliubov formula which is defined as follows
%
\begin{equation}
\Rsf_\Vsf(\Fsf(\phi)) := S(\Vsf)^{\star-1} \star \left( S(\Vsf) \cdot_\Tsf \Fsf(\phi) \right) \ ,
\label{eq:bogoliubov}
\end{equation}
%
where $S^{\star-1}(\Vsf)$ is the inverse of $S(V)$ with respect to the $\star$ product. 


\bigskip
\com{Maybe at this point you should prove that $\Rsf_\Vsf(\Fsf_f(\phi))$ solves the interacting equation of motion weakly.}
\bigskip






We shall construct observables of the interacting theory by means of perturbation theory, namely, when the free algebra is perturbed by a non linear local potential $\Vsf$. 



Considering the time ordered product defined above we can introduced a new space of functionals. The space of ``causal'' functionals denoted $\Fcal_\Tsf(\Mcal)[[\hbar]]$\index{$\Fcal_\Tsf(\Mcal)$}, it is the space of functionals which can be written as time ordered products of quantum local functionals. It is a subspace of the microcausal space. The free off shell quantum algebra corresponding is defined as 
%
\begin{equation}
\Acal_\Tsf(\Mcal)[[\hbar]]\{I\} := \left( \Fcal_\Tsf(\Mcal)[[\hbar]] , ^\ast , \star , \cdot_\Tsf \right) \ . 
\label{eq:alg_free_int}
\end{equation}
\index{$\Acal_\Tsf(\Mcal)[[\hbar]]\{I\}$}
%
Thus using the Bogoliubov formula \eqref{eq:bogoliubov} we can define the interacting quantum algebra as follow
%
\begin{equation}
\Rsf_\Vsf \ : \ \Acal_\Tsf(\Mcal)[[\hbar]]\{I\} \ \to \ \Rcal_\Tsf(\Mcal)[[\hbar]]\{I\} \ .
\label{eq:alg_int}
\end{equation}
\index{$\Rcal_\Tsf(\Mcal)[[\hbar]]\{I\}$}











%----------------------------------------------------------------------------%
\section{States in algebraic quantum field theory}
%----------------------------------------------------------------------------%


Up to now all the construction used to build quantum field theory on curved background is state independent. This is important to have an independent state construction in the case of theories on curved spacetime, because we do know objects on this background only locally, and it turns out that states are related to non local objects.


In a labaratory the physical system is prepared in order to be able to perform an experiments on it. The prescription on how we shall prepare a system is called the \textbf{state} of system. When an experiment is made the corresponding observable gives a value which can vary according to the preparation of the system. Therefore to evaluate the observable $\Fsf$ in a state $\omega$ we consider its ``average value''. We call the evaluation of $\Fsf$ on $\omega$ the expectation value of $\Fsf$ in the given system $\omega$, and denoted $\omega(\Fsf)$. For instance if we prepare the system such that the uncertainty of the value od the measurements is minimal, we have defined a pure state.


\begin{definition}[State]
Let $\Acal$ be a $\ast$--algebra. A state $\omega$ on the algebra $\Acal$ is defined as a continuous linear functional
%
\begin{equation*}
\omega : \Acal \to \Cbb \ , 
\end{equation*}
%
which is normalized and positive, i.e.
%
\begin{equation*}
\omega(\Ibb) =  1 \ , \quad \mbox{and} \qquad \omega(\Fsf^\ast \Fsf) \leq 0 \ . 
\end{equation*}
%
for all $\Fsf \in \Acal$.
\end{definition}


It is important to notice that a state $\omega$ is determined by its $n$--point (correlation) function, it is the functional average (expectation value) of a the product of $n$ observables,

%
\begin{equation*}
\omega( \Fsf_1 , \dots , \Fsf_n ) =  \omega_n\left( \Fsf_1(\phi) \star \dots \star \Fsf_n(\phi) \right) \ . 
\end{equation*}
%
Many different states can be considered on $\Acal$. We shall present few particular types of states in the following definition.


\begin{definition}
Let $\Acal$ be $\ast$--algebra.  
%
\begin{itemize}
%
%
%
\item A state $\omega$ on $\Acal$ is called \textbf{mixed} if it is a convex linear combination of states, i.e.
%
\begin{equation*}
\omega = \lambda \omega_1 + (1-\lambda) \omega_2 
\end{equation*}
%
where $\lambda < 1$, and $\omega_1$, $\omega_2$ are distinct from $\omega$.
%
%
%
\item A state $\omega$ on $\Acal$ is called \textbf{pure} if it is not mixed.
%
%
%
\item A state $\omega$ on $\Acal$ is called \textbf{even} if it is invariant under the following transformation
%
\begin{equation*}
\Fsf(\phi) \mapsto - \Fsf(\phi) \ , 
\end{equation*}
%
i.e. it has vanishing $n$--point functions for all odd $n$.
%
%
%
\item A state $\omega$ on $\Acal$ is called \textbf{quasifree} or \textbf{gaussian} if for all even $n$
%
\begin{equation*}
\omega_n\left( \Fsf_1(\phi) \star \dots \star \Fsf_n(\phi) \right) =  \sum_{\pi \in \Srak_n} \omega_2\left( \Fsf_{\pi(1)}(\phi) \star \Fsf_{\pi(2)}(\phi) \right) \cdot \ \dots \ \cdot \omega_2\left( \Fsf_{\pi(n-1)}(\phi) \star \Fsf_{\pi(n)}(\phi) \right) \ ,
\end{equation*}
%
where $\Srak_n$ is the set of ordered permutations of $n$ elements, i.e. we have 
%
\begin{equation*}
\pi(1) < \pi(3) < \dots < \pi(n-1) \ , \quad \mbox{and} \quad \pi(1) < \pi(2) \ ;  \ \pi(n-1) < \pi(n) \ . 
\end{equation*}
%
%
%
\item A state $\omega$ on $\Acal$ is called \textbf{Hadamard} if its two point function satisfy the microlocal spectrum condition \eqref{eq:wf_hadamard} that we rcall here
%
\begin{equation*}
\WF(\omega_2) \ = \ \bigg\{ \bigg( x, y ; k_x, k_y \bigg) \in T^\ast\Msf^2 \setminus \{0\} \ \bigg| \ (x,k_x) \sim (x,-k_y), \ k_x \triangleright 0 \bigg\}  
\end{equation*}
%
%
%
\end{itemize}
\end{definition}


We shall conclude this section with a practical computation, the two point function of the interacting field in a gaussian Hadamard state $\omega$ of the free field. We consider a quartic potential eqref(eq:pot quartic) and perform the computation up to the second order in $\lambda$. 


In practice we need to compute the $\star$ product of $\Rsf_\Vsf(\phi)$ and $\Rsf_\Vsf(\phi)$ and evaluate it on the state $\omega$. A graphical representation of the Bogoliubov formula \eqref{eq:bogoliubov} is displayed and permits us to write this two point function as a sum of graphs.


The functionals $\Fsf$, $\Gsf$, and the potential $\Vsf$ are given by
%
\begin{equation*}
\Fsf(\phi) = \int \dsf x \ \fsf(x) \ \phi(x) \ , \quad \Gsf(\phi) = \int \dsf x \ \gsf(x) \ \phi(x) \ , \quad \Vsf(\phi) = \int \dsf x \ \left( \frac{\lambda(x)}{4!} + \frac{\mu(x)}{2!} \right) \ \phi(x)^4 \ ,
\end{equation*}
%
where $\mu$ and $\lambda$ are compaclty smooth test functions on $\Mcal$. Then we can compute the $S$ matrix \eqref{eq:S_matrix} for the potential $\Vsf$
%
\begin{eqnarray*}
&& S(\Vsf) = 1 + i \Vsf(\phi) - \frac{1}{2} \Vsf(\phi) \cdot_\Tsf \Vsf(\phi) + \Ocal(\lambda^3) \\
&=&  \ 1 + i \Vsf(\phi) - \frac{1}{2} \Vsf(\phi) \cdot \Vsf(\phi) \\
&& - \int_\Mcal \dsf x \ \dsf y \ \lambda(x) \lambda(y) \ \bigg( 
\frac{\hbar}{72} \phi(x)^3 \phi(x)^3 \Delta_\fsf(x,y) + \frac{\hbar^2}{16} \phi(x)^2 \phi(x)^2 \Delta_\fsf(x,y)^2 \\ 
&& \hspace*{13pt} + \frac{\hbar^3}{6} \phi(x) \phi(x) \Delta_\fsf(x,y)^3 + \frac{\hbar^4}{8} \Delta_\fsf(x,y)^4 \bigg) \\
&& + \int_\Mcal \dsf x \ \dsf y \ \mu(x) \mu(y) \ \bigg( 
\frac{\hbar}{2} \phi(x) \phi(x) \Delta_\fsf(x,y) + \frac{\hbar^2}{4} \Delta_\fsf(x,y)^2 \bigg) + \Ocal(\lambda^3) \ , 
\end{eqnarray*}
%
and also its inverse with respect to the $\star$ product
%
\begin{eqnarray*}
S(\Vsf)^{\star -1} &=& 1 \ - i \ \Vsf(\phi) \ + \frac{1}{2} \ \Vsf(\phi) \cdot_\Tsf \Vsf(\phi) \ - \ \Vsf(\phi) \star \Vsf(\phi) \ + \ \Ocal(\lambda^3) \ .
\end{eqnarray*}
%
We can now compute the corresponding interacting functional for $\Fsf$
%
\begin{eqnarray*}
\Rsf_\Vsf(\Fsf) &=& S(\Vsf)^{\star -1} \star \left( S(\Vsf) \cdot_\Tsf \Fsf(\phi) \right) \\
&=& \Fsf(\phi) - i \Vsf(\phi) \star \Fsf(\phi) + i \Vsf(\phi) \cdot_\Tsf \Fsf(\phi) + \frac{1}{2} \left( \Vsf(\phi) \cdot_\Tsf\Vsf(\phi) \right) \star \Fsf(\phi) \\
&& - \Vsf(\phi) \star \Vsf(\phi) \star \Fsf(\phi) - \frac{1}{2} \Vsf(\phi) \cdot_\Tsf \Vsf(\phi) \cdot_\Tsf\Fsf(\phi) + \Vsf(\phi) \star \left( \Vsf(\phi) \cdot_\Tsf \Fsf(\phi) \right) + \Ocal(\lambda^3) \ ,
\end{eqnarray*}
%
and in the same way we get $\Rsf_\Vsf(\Gsf)$. Then as already said we have to compute the $\star$ product of $\Rsf_\Vsf(\phi)$ and $\Rsf_\Vsf(\phi)$ and evaluate in the state $\omega$. In the computation many expressions can be shortened by using the relation \eqref{eq:conv_feynman_prop}. The resulting Feynman diagrams are depicted in figure \ref{fig:2pf} where we used the conventions from \ref{fig:prop_vertices} concerning the various propagators and vertices. Notice we choose $\mu(x)=3\lambda w(x,x)$.


\begin{figure}[ht]
\centering
\includegraphics[scale=0.8]{./fig_propagators.pdf}
% fig_propagators.pdf: 309x75 pixel, 72dpi, 10.90x2.65 cm, bb=0 0 309 75
\caption{Propagators and vertices.}
\label{fig:prop_vertices}
\end{figure}


The up to second order contributions to the two point function of the interacting field with potential $\Vsf$. We omit the labels of the external vertices after the first line using the convention that the left external vertex is always the $x$-vertex, and then the right external vertex the $y$-vertex.


\begin{figure}[ht!]
\centering
\includegraphics[scale=0.8]{./fig_2pf.pdf}
% fig_2pf.pdf: 360x240 pixel, 72dpi, 12.70x8.47 cm, bb=0 0 360 240
\caption{Two point function of the interacting field in Gaussian Hadamard state.}
\label{fig:2pf}
\end{figure}


Each edge corresponds to a propagator (cf. \ref{fig:prop_vertices}), every internal vertices to $\lambda$ or $\mu(x)$ (cf. \ref{fig:prop_vertices}), and any closed loop represent an integration of the propagator(s) in the loop against suitable test function(s). All graph are a pointwise product of ``vertices'' and propagators which are from time to time integrated.

\vspace*{80pt}


$\bullet$ \ Given a $\ast$ algebra $\Acal$, \par
\hspace*{8pt} \textbf{a state} $\omega$ is a positive, normalised, linear functional on $\Acal$. \par

\bigskip
  
$\bullet$ \ Often the algebra of observables $\Acal$, cannot be equipped with a norm \par
\hspace*{8pt} BUT we have unital $\ast$ algebra \par


\bigskip

$\bullet$ \ \textbf{The GNS construction remains possible !} \par
\hspace*{8pt} But in particular it is not garantee that selfadjoint elements of $\Acal$ are \hspace*{8pt} represented by selfadjoint Hilbert space operators ! \par
\begin{center}
\textbf{Hilbert space formalism} \ $\leftarrow$ \textbf{GNS construction} $\rightarrow$ \ \textbf{Algebraic formalism}
\end{center}

$\to$ \textbf{A state} $\omega$ on $\Acal$ vector is represented as a \textbf{``vacuum'' vector}, and elements \hspace*{9pt} of $\Acal$ as \textbf{linear operators}. \par

\bigskip

$\to$ \textbf{Conversely}, any normalised \textbf{Hilbert space vector} is a \textbf{state on the algebra \hspace*{9pt} of linear operators} with the $\ast$ operation given by the Hermitian adjoint.


\bigskip

$\bullet$ \ \textbf{Hadamard states} \par
\qquad $\to$ specify by \textbf{constrainig the singulary} of the two point function

\bigskip

Hadamard condition cite(Radzikowski 1996)\par
A state $\omega$ fulfils the \textbf{Hadamard condition} ($\mu S C$) iff
%\vspace*{-10pt}
\begin{equation*}
 \WF(\omega_{2}) \ = \ \bigg\{ (x,k_x ; y,-k_y) \in T^\ast\Mcal^2 \backslash \{0\} \ \bigg| \ (x,k_x) \sim (y,k_y) , \ k_x \triangleright 0 \bigg\}
\end{equation*}
$\sim$ : $\exists$ a null geodesic connecting $x$ and $x^\prime$, and $k^\prime$ is the parallel transport of $k$. \par
$k_x \triangleright 0$ : $k_x$ is futur directed.


\bigskip

$\bullet$ \textbf{abstract definition} of Hadamard states \par
\qquad $\to$ powerful BUT unconvenient for computations !


\bigskip


The notions of states and observables are very important. A state is in some sence how we prepare the system to do the experiment. For instance we can imagine that we prepare our system at temperature $T$ that we control. Then the observables can be viewed as the experiments. Therefore we see that an observable is a function acting on the system that we put in given state and gives us aa result which is real number, for instance an energy (in electron volt). 

The big difference with field theory is that now each point on the spacetime has an infinite degree of freedom. However we can impose a condition on these numbers of degree of freedom. Indeed let us give a very naive picture of the situation. We can imagine that our spacetime is an ocean (choose the one that you prefer), then if a government has the very smart idea to test some new expolion weapon in this ocean, if it is not a too big nuclear bomb which could desrtoy the entire earth, the explosion won't impact the particle of water of the ocean, only a given quantity of the particle of water will be impact. Therefore from a more serious and forma way, it is thus justify to consider a principle of locality which will tell us that every point of the spacetime is only linked to a neighborhood of points of the whole spacetime. 



\begin{definition}[Physical system.] 
 A physical system $S$ is described by its observables, viewed now as self adjoint elements in a certain $\ast$--algebra $\Acal$
\end{definition}




The algebra $\Usf_s$ is not seen as a concrete $\Csf^\ast$ algebra of operators on a given Hilbert space, but remains an abstract $\Csf^\ast$ algebra. Physically, $\omega(a)$ is the expectation value of the observable $a \in \Usf$ in state $\omega$.



\begin{theorem}[$\mathsf{GNS}$ theorem for $\ast$ algebra with unit.]
 Let $\Usf$ be a $\ast$ algebra with unit $\mathsf{1}$ and $\omega : \Usf \to \Cbb$ a positive linear functional with $\omega(\mathsf{1})=1$. Then \par
 \noindent
 $\mathbf{A.}$ there exists a quadrapule 
 \begin{equation*}
  \bigg(\Hcal_\omega, \ \Dcal_\omega, \ \pi_\omega, \ \Omega_\omega\bigg)
 \end{equation*}
 made of a Hilbert space $\Hcal_\omega$, a subspace $\Dcal_\omega \subset \Hcal_\omega$, a linear map $\pi_\omega : \Usf \to \Lcal(\Dcal_\omega,\Hcal_\omega)$, and element $\Omega_\omega \in \Dcal_\omega$, such that
 
  [(i)] $\Dcal_\omega$ is a $\pi_\omega(a)$ invariant for every $a\in \Usf$, since $\Dcal_\omega=\pi_\omega(\Usf)\Omega_\omega$ \ ;
  [(ii)] $\Omega_\omega$ is cyclic for $\pi_\omega$, $\Dcal_\omega$ is dense in $\Hcal_\omega$ \ ;
  [(iii)] $\pi_\omega : \Usf \to \pi_\omega(\Usf)$ is an algebra homomorphism satisfying $\pi_\omega(\mathsf{1})=\mathsf{1}$ and $\pi_\omega(a^\ast) = \pi_\omega(a)^\ast\upharpoonright_{\Dcal_\omega}$, for $a\in \Usf$ \ ;
  [(iv)] $\left(\Omega_\omega | \pi(a) \psi_\omega\right)=\omega(a)$, $a\in\Usf$.
 
 $\mathbf{B.}$ If the quadrapule
 \begin{equation*}
  \bigg(\Hcal, \ \Dcal, \ \pi, \ \Omega\bigg)
 \end{equation*}
 fulfills (i)-(iv), there exists a unitary operator
 \begin{equation*}
  \Psf:\Hcal_\omega \to \Hcal 
 \end{equation*}
 such that 
 \begin{eqnarray*}
  \psi = \Psf \Omega_\omega \ , \qquad \Dcal = \Psf \Dcal_\omega \ , \qquad \pi(a) = \Psf \pi_\omega(a) \Psf^{-1} 
 \end{eqnarray*}
 for any $a \in \Usf$. 
\end{theorem}

For a proof look at moretti's book

%----------------------------------------------------------------------------%
\section{The regularization problem}\label{p:EPSTEIN_GLASER}
%----------------------------------------------------------------------------%


%----------------------------------------------------------------------------%
\subsection{Extention up to the total diagonal}
%----------------------------------------------------------------------------%


We have build successfully the interacting quantum field theory perturbatively using the functional approach. The interacting observables have been obtained with the Bogoliubov formula \eqref{eq:bogoliubov}. The problem using $\Rcal_\Vsf$ lies in the construction of the time ordered product.


The time ordered product can be ill defined. It is what we call the regularization problem. In particular for the following local observables
%
\begin{equation*}
\Fsf(\phi) = \int_\Mcal \dsf x \ \sqrt{\abs{g}} \ f(x) \phi(x)^2 \ , \ \mbox{ and } \ 
\Gsf(\phi) = \int_\Mcal \dsf x \ \sqrt{\abs{g}} \ g(x) \phi(x)^3 \ ,
\end{equation*}
%
with $f, g \in \Dcal(\Mcal)$, the time ordered product is defined as
%
\begin{equation*}
(\Fsf \cdot._\Tsf \Gsf)(\phi) = \Fsf(\phi) \cdot \Gsf(\phi) + \hbar \sm{ \Fsf^{(1)}(\phi) , \Delta_\fsf \ \Fsf^{(1)}(\phi) } + \frac{\hbar^2}{2} \sm{ \Fsf^{(2)}(\phi) , \Delta_\fsf^{\otimes 2} \ \Fsf^{(2)}(\phi) } \ .
\end{equation*}
%
The term in $\hbar^2$ can be written as
%
\begin{equation*}
\sm{ \Fsf^{(2)}(\phi) , \Delta_\fsf^{\otimes 2} \ \Fsf^{(2)}(\phi) }  = 
12 \int_{\Mcal \times \Mcal} \dsf x \ \dsf y \ f(x) \phi(y) \ \Delta_\fsf(x,y)^2
\end{equation*}
%
We see that we have to know the square of the Feynman propagator on the full space $\Mcal^2$. We have $\WF(\Delta_\fsf) \subset \WF(\Delta_\asf)$ using \eqref{eq:conv_feynman_prop} and \eqref{eq:prop_wf}, furthermore $\Psf\Delta_\asf=\Ibb$ thus $\WF(\Delta_\fsf) \subset \WF(\delta)$. And we knwo that product of the distribution $\delta$ are ill defined on the total diagonal, thus it is the same for $\Delta_\fsf$.


%\bigskip


The regularization problem can be solved using the Epstein Glaser procedure \cite{brunetti_microlocal_2000}, where the time ordered product is build recursively on the full space up to the total diagonal. At each recursion step the causal factorisation property permits to construct the distributions defining the time ordered product up to the total diagonal. Let us present this recursive construction. 


%\bigskip


We consider $\forall i \in \{1,\dots,n\}$ the functionals $\Fsf_i(\phi) \in \Fcal_\Tsf(\Mcal)[[\hbar]]$. We recall the time ordered product of two ``causal'' functionals 
%
\begin{equation*}
\Fsf_1(\phi) \cdot_\Tsf \Fsf_2(\phi) = \Fsf_1(\phi) \star \Fsf_2(\phi) \ ,
\end{equation*}
%
coincide with the $\star$ property for $\supp(\Fsf_1)$ later that $\supp(\Fsf_2)$. It is well defined on $\Mcal^n \setminus d_n$. Now let us assume that we were able to construct the time ordered product of $k$ ``causal'' functionals, with $k \in \{1,\dots,n-1\}$. We supose it is well defined on the full space minus the total diagonal, and that it satisfies the causal factorisation property. We want now to define the time ordered product of $n$ functionals.


%\bigskip

We shall need a partition of $\Mcal^n \setminus d_n$. Let $\Jcal$ be the set of all non empty proper subset $I$ of $\{1,\dots,n\}$, and $I^c$ the complement set of $I$ in $\{1,\dots,n\}$. We define the subset $\Ccal_I$ of $\Mcal^n$ as follow
%
\begin{equation*}
\Ccal_I = \left\{ (x_1,\dots,x_n) \in \Mcal^n \ | \ \forall (i,j) \in I \times I^c , \ x_i \notin J^-(x_j) \right\} \ .
\end{equation*}
%
Then we can show \cite{brunetti_microlocal_2000}
%
\begin{equation*}
\bigcup_{I \in \Jcal} \Ccal_I = \Mcal^n \setminus d_n \ .
\end{equation*}
%
Then on every $\Ccal_I$ we have 
%
\begin{equation*}
\Fsf_1(\phi) \cdot_\Tsf \dots \cdot_\Tsf \Fsf_n(\phi) = T_I(\Fsf) \star T_{I^c}(\Fsf) \ , 
\end{equation*}
%
where $T_I(\Fsf)$ (respectively $T_{I^c}(\Fsf)$) is the time oredered product of all $\Fsf_i(\phi)$ with $i \in I$ (respectively $i \in I^c$). Accordind to the induction hypothesis the time ordered product of $n$ functionals is then well defined on $\Ccal_I$. 


%\bigskip


Two sets $\Ccal_{I_1}$ and $\Ccal_{I_2}$ can overlap, but we can show \cite{brunetti_microlocal_2000} that for any set $I_1 , I_2 \in \Jcal$ such that $\Ccal_{I_1} \cap \Ccal_{I_2} \neq \emptyset$ we have the restrictions on $\Ccal_{I_1} \cap \Ccal_{I_2}$ of $\Tsf_{I_1}(F)$ and $\Tsf_{I_2}(F)$ which coincide. 


%\bigskip


Therefore if we define a smooth partition of unity $\{ g_I\}$  subordinate to $\{\Ccal_I\}$, we have
%
\begin{equation*}
\Fsf_1(\phi) \cdot_\Tsf \dots \cdot_\Tsf \Fsf_n(\phi) = \sum_{I\in\Jcal} g_I \Tsf_I(F) 
\end{equation*}
%
is well defined on the full space up to the total diagonal $\Mcal^n \setminus d_n$. 



%\bigskip


This procedure, knwon as the Epstein Glaser regularization, is theoreticaly clear but quite difficult to implement in practise. The aim of the present work is to discuss a regularization scheme which is suitable for practical computations.

\bigskip


We shall here define the notion of extension of a distribution. When an extension can be found it is not unique therefore we introduce the scaling degree, which in some sense is a constraint in the construction of extensions. An important theorem due to R. Brunetti et K. Fredenhagen presnted in \cite{brunetti_microlocal_2000} shall be recalled and adapted to our situation. It characterize the existence and uniqueness of an extension of an ill defined distribution.


In section \ref{p:REG_FRAMEWORK} we shall give the necessary mathematical framework to construct extension, i.e. to define a regularization procedure.


And finally in section \ref{p:DIFFERENTIAL_EULER} we shall for later on present the genealized  Euler operator which shall be used in our regularization scheme.


%----------------------------------------------------------------------------%
\subsection{Scaling degree and regularization}
%----------------------------------------------------------------------------%



\com{   This section is problematic!!! you are mixing the notion of extension of numerical distribution towards a point with the extensions over the diagonals.... this should be completely rewritten}



We denote by $u$ the distribution for which we want to find an extension. $u \in \Dcal^\prime(\Mcal^n \setminus d_n )$ is a distribution defined for all test functions supported outside the the total diagonal. We call $\exte{u}$ an \textbf{extension} of $u$ if 
%
\begin{equation*}
\mbox{for } \ \exte{u} \in \Dcal^\prime(\Mcal^n), \ \mbox{ we have } \ \forall \phi \in \Dcal\left(\Mcal^n \setminus d_n \right), \ \exte{u}(\phi) = u(\phi) \ .
\end{equation*}
We cannot always find an extension of a distribution \sbar{ill defined}, and \sbar{when} \com{even if extensions exists, they must not be unique.} \sbar{we can \com{find} the extension \com{it might be non} is not unique}. A distribution supported on the total diagonal can be written as a polynomial in the derivatives of the Dirac distribution \com{supported on $d_n$} \com{.... this is not true.... because the diagonal is not a point.... you should also add some extra condition like to say this....!!!}. Therefore we could add a \sbar{polynom} \com{polynomial} in the derivatives of the Dirac distribution to the extension and we would not change the extension of our distribution ill defined. We have to restrict the possibilities of extension by adding a constraint. A possible choice of a constraint is to to require that the distributions we want to extend and their corresponding extensions should have the same scaling degree. \com{ .... but $d_n$ is big... so you should fix some other constraint.   if you want we could discuss this poin//. ....} Let us first introduce a notion of scaling transformation.


As we know for every pair of points $x_1,x_i$ in a normal neighbourhood  $\Ncal \subset \Mcal$ there exists a unique geodesic $\gamma$ connecting $x_1$ and $x_i$. We shall assume that
%
\begin{equation*}
\gamma : \lambda \mapsto x_i(\lambda) 
\end{equation*}
%
is affinely parametrised and that $x_i(0) =x_1$ whereas $x_i(1) = x_i$. For all $\lambda \geq 0$ and all $\phi \in \Dcal(\Ncal_n)$ with $\Ncal_n \subset \Mcal^n$ a normal neighbourhood of the total diagonal $d_n$ \eqref{eq:neighborhood_glob}, the \textbf{geometric scaling transformation} we shall consider is
%
\begin{equation}
\phi_\lambda = \lambda^{4(n-1)} \ \phi\left(x_1,x_2(\lambda ),\dots,x_n(\lambda\right)) \ \prod_{i=2}^n \frac{\sqrt{g(x_i(\lambda ))}}{\sqrt{g(x_i)}} \ ,
%\label{eq:geo_scaling_transfo}
\end{equation}
%
where $g(x)$ is the absolute value of the determinant of the metric expressed in normal coordinates. For $\lambda > 1$ it may happen that $x_i(\lambda)$ lies outside of $\Ncal_n$ and is thus not well defined in general. In this case we set $\phi_\lambda = 0$ which is well defined because $\phi = 0$ outside of $\Ncal_n$. For later purposes, we recall that the \textbf{determinant of the metric} computed in normal coordinates centred at $x_1$ is such that
%
\begin{equation*}
\sqrt{g(x_i)} = \frac{1}{\usf^2(x_1,x_i)} \ , 
\end{equation*}
%
where $\usf$ is the Hadamard coefficient in \eqref{eq:hadamard_rep} and $\usf^2$ is the van Vleck--Morette determinant, see e.g. \cite{poisson_motion_2011}.


By means of the transformation \eqref{eq:geo_scaling_transfo}, relevant information about the behaviour of a distribution in the neighbourhood of the total diagonal $d_n$ can be obtained. 


\com{..... The scaling degree is defined as a scaling towards a point.... you should not mix the concepts too much   ...}

\begin{definition}%\label{def:scaling_degree}
The \textbf{scaling degree} of a distribution $u \in \Dcal^\prime(\Mcal^n)$ or $u \in \Dcal^\prime(\Mcal^n \setminus d_n)$ towards $d_n$ is defined as 
%
\begin{equation*}
\sd(u) \ \doteq \ \inf\bigg\{ \omega \in \Rbb \ \bigg| \ \lim_{\lambda \downarrow 0} \ \lambda^\omega \ \sm{u,\phi_{1/\lambda}} \ = \ 0, \ \forall \phi\in\Dcal(\Ncal_n\setminus d_n) \bigg\} \ .
\end{equation*}
%
\end{definition}


The scaling degree has the the following properties. Let $u, v \in\Dcal^\prime(\Mcal^n)$ and $\alpha \in \Nbb^n$, then
%
\begin{itemize}
\setlength\itemsep{0pt}
\item $\sd(\nabla^\alpha u) \leq \sd(u) + \abs{\alpha}$,
\item $\sd(x^\alpha u) \leq \sd(u) - \abs{\alpha}$,
\item $\sd(\phi u) \leq \sd(u)$, for all $\phi \in \Ecal(\Mcal^n)$, 
\item $\sd(u \otimes v) = \sd(u) + \sd(v)$
\end{itemize}
%
We notice that when the product $uv$ is well defined it has the same scaling degree as $u \otimes v$. The reason is that we define $uv$ as the pullback of $u \otimes v$ by the diagonal map (look at \ref{p:SING_WF}).


The scaling degree permits us to predict the existence and the possible uniqueness of a distribution. According to the value the scaling degree of the distribution $u$ and compairing to the dimension of the total space, we shall be able to conclude on the existence and uniqueness of an extension. Let us present this important result in the following theorem.


\com{the same problem as above. this theorem is given for the extension of distributions towards a point}

\begin{theorem}[Existence and uniqueness of an extension] %\label{theo:extension_distribution}
Let $u \in \Dcal^\prime(\Mcal^n \setminus d_n )$, then
%
\vspace*{-5pt}
\begin{itemize}
\setlength\itemsep{0pt}
\item if $\sd(u) < 4(n-1)$, then there exists a unique extension $\dot{u} \in \Dcal^\prime(\Mcal^n)$ with $\sd(\dot{u})=\sd(u)$,
%
\item if $4(n-1)\leq\sd(u)<\infty$, then there exist several extensions $\dot{u} \in \Dcal^\prime(\Mcal^n)$ with $\sd(\dot{u})=\sd(u)$. They are uniquely defined by their values on a finite set of test functions.
\end{itemize}
%
A distribution which have an infinite scaling degree cannot be extended.
\end{theorem}
\com{... in the previous theorem you are changing notation for the extended distribution!!! ... }



\bigskip


We denote by $u$ the distribution for which we want to find an extension. $u \in \Dcal^\prime(\Mcal^n \setminus d_n )$ is a distribution defined for all test functions supported outside the the total diagonal. We call $\exte{u}$ an \textbf{extension} of $u$ if 
%
\begin{equation*}
\mbox{for } \ \exte{u} \in \Dcal^\prime(\Mcal^n), \ \mbox{ we have } \ \forall \phi \in \Dcal\left(\Mcal^n \setminus d_n \right), \ \exte{u}(\phi) = u(\phi) \ .
\end{equation*}


We cannot always find an extension of a distribution ill defined, and when we can the extension is not unique. A distribution supported on the total diagonal can be written as a polynomial in the derivatives of the Dirac distribution. Therefore we could add a polynom in the derivatives of the Dirac distribution to the extension and we would not change the extension of our distribution ill defined. We have to restrict the possibilities of extension by adding a constraint. A possible choice of a constraint is to to require that the distributions we want to extend and their corresponding extensions should have the same scaling degree. Let us first introduce a notion of scaling transformation.


As we know for every pair of points $x_1,x_i$ in a normal neighbourhood $\Ncal \subset \Mcal$ there exists a unique geodesic $\gamma$ connecting $x_1$ and $x_i$. We shall assume that
%
\begin{equation*}
\gamma : \lambda \mapsto x_i(\lambda) 
\end{equation*}
%
is affinely parametrised and that $x_i(0) =x_1$ whereas $x_i(1) = x_i$. For all $\lambda \geq 0$ and all $\phi \in \Dcal(\Ncal_n)$ with $\Ncal_n \subset \Mcal^n$ a normal neighbourhood of the total diagonal $d_n$ \eqref{eq:neighborhood_glob}, the \textbf{geometric scaling transformation} we shall consider is
%
\begin{equation}
\phi_\lambda = \lambda^{4(n-1)} \ \phi\left(x_1,x_2(\lambda ),\dots,x_n(\lambda\right)) \ \prod_{i=2}^n \frac{\sqrt{g(x_i(\lambda ))}}{\sqrt{g(x_i)}} \ ,
\label{eq:geo_scaling_transfo}
\end{equation}
%
where $g(x)$ is the absolute value of the determinant of the metric expressed in normal coordinates. For $\lambda > 1$ it may happen that $x_i(\lambda)$ lies outside of $\Ncal_n$ and is thus not well defined in general. In this case we set $\phi_\lambda = 0$ which is well defined because $\phi = 0$ outside of $\Ncal_n$. For later purposes, we recall that the \textbf{determinant of the metric} computed in normal coordinates centred at $x_1$ is such that
%
\begin{equation*}
\sqrt{g(x_i)} = \frac{1}{\usf^2(x_1,x_i)} \ , 
\end{equation*}
%
where $\usf$ is the Hadamard coefficient in \eqref{eq:hadamard_rep} and $\usf^2$ is the van Vleck--Morette determinant, see e.g. \cite{poisson_motion_2011}.


By means of the transformation \eqref{eq:geo_scaling_transfo}, relevant information about the behaviour of a distribution in the neighbourhood of the total diagonal $d_n$ can be obtained. 


\begin{definition}\label{def:scaling_degree}
The \textbf{scaling degree} of a distribution $u \in \Dcal^\prime(\Mcal^n)$ or $u \in \Dcal^\prime(\Mcal^n \setminus d_n)$ towards $d_n$ is defined as 
%
\begin{equation*}
\sd(u) \ \doteq \ \inf\bigg\{ \omega \in \Rbb \ \bigg| \ \lim_{\lambda \downarrow 0} \ \lambda^\omega \ \sm{u,\phi_{1/\lambda}} \ = \ 0, \ \forall \phi\in\Dcal(\Ncal_n\setminus d_n) \bigg\} \ .
\end{equation*}
%
\end{definition}


The scaling degree has the the following properties. Let $u, v \in\Dcal^\prime(\Mcal^n)$ and $\alpha \in \Nbb^n$, then
%
\begin{itemize}
\setlength\itemsep{0pt}
\item $\sd(\nabla^\alpha u) \leq \sd(u) + \abs{\alpha}$,
\item $\sd(x^\alpha u) \leq \sd(u) - \abs{\alpha}$,
\item $\sd(\phi u) \leq \sd(u)$, for all $\phi \in \Ecal(\Mcal^n)$, 
\item $\sd(u \otimes v) = \sd(u) + \sd(v)$
\end{itemize}
%
We notice that when the product $uv$ is well defined it has the same scaling degree as $u \otimes v$. The reason is that we define $uv$ as the pullback of $u \otimes v$ by the diagonal map (look at \ref{p:SING_WF}).


The scaling degree permits us to predict the existence and the possible uniqueness of a distribution. According to the value the scaling degree of the distribution $u$ and compairing to the dimension of the total space, we shall be able to conclude on the existence and uniqueness of an extension. Let us present this important result in the following theorem.


\begin{theorem}[Existence and uniqueness of an extension] \label{theo:extension_distribution}
Let $u \in \Dcal^\prime(\Mcal^n \setminus d_n )$, then
%
\vspace*{-5pt}
\begin{itemize}
\setlength\itemsep{0pt}
\item if $\sd(u) < 4(n-1)$, then there exists a unique extension $\dot{u} \in \Dcal^\prime(\Mcal^n)$ with $\sd(\dot{u})=\sd(u)$,
%
\item if $4(n-1)\leq\sd(u)<\infty$, then there exist several extensions $\dot{u} \in \Dcal^\prime(\Mcal^n)$ with $\sd(\dot{u})=\sd(u)$. They are uniquely defined by their values on a finite set of test functions.
\end{itemize}
%
A distribution which have an infinite scaling degree cannot be extended.
\end{theorem}


Let us sketch the proof of theorem \ref{theo:extension_distribution}.


\begin{sketch}
%%TODO PROOF
We will prove the two points of the theorem %\ref{th:Extension-distribution}.
 \begin{center}
  \textbf{1st case:} $\sd(u)<d$. %\cite[p.646]{BF2000}  
 \end{center}
 The difference between two different possible extensions would be a distribution with support at $\{0\}$. Using the theorem %\ref{th:Distribution-singular-point}
 ``this difference'' would be a polynom $P$ in derivatives of the Dirac distribution, but this distribution has sacling degree equal to $d+\deg(P)$, with $\deg(P)$ the degree of the polynom $P$. There is a contradiction thus if the scaling degree of $t_0$ is strictly smaller than the dimension then the extension $t$ of $t_0$ is unique.\par
 Let us consider a smooth function of compact support $\kappa$ such that $\kappa=1$ in a neighborhood of the origin. We set
 \begin{eqnarray*}
  && \kappa_\rho(x) \ \doteq \ \kappa(\rho x) \ , \quad \lambda \in \Rbb \\
  && u^n \ = \ (1-\kappa_{2^n}) \ u \ , \quad n \in \Nbb
 \end{eqnarray*}
 where $u^n$ is sequence of distibutions defined on the whole space $\Rbb^d$. We would like to show this sequence converge, to do that it is sufficient to prove it is a Cauchy sequence. Let $\phi \in \Dcal(\Rbb^d)$ 
 \begin{eqnarray*}
  \left( u^{n+1} \ - \ u^n \right)(\phi) &\doteq& \sm{\left(u^{n+1} - u^n\right) \ , \ \phi} \\
  &=& \sm{\left(\kappa^{n+1} - \kappa^n\right) u \ , \ \phi} \\
  &=& \int dx \ \left(\kappa_{2^n} - \kappa_{2^{n+1}}\right)(x) \ u(x) \ \phi(x) \\
  &=& \int dx \ \bigg(\kappa(2^n x) - \kappa(2^{n+1} x)\bigg) \ u(x) \ \phi(x) \\
  &=& 2^{-nd} \ \int dy \ \bigg(\kappa(y) - \kappa(2 y)\bigg) \ u\left(\frac{y}{2^n}\right) \ \phi\left(\frac{y}{2^n}\right) \\ 
  &=& 2^{-nd} \ \int dy \ \bigg(\kappa(y) - \kappa(2 y)\bigg) \ u\left(2^{-n} y \right) \ \phi\left(2^{-n} y\right) \\
  \sm{\left(u^{n+1} - u^n\right) \ , \ \phi} &=& 2^{-nd} \ \sm{\left(\kappa_1 - \kappa_2\right) \ , \ (\phi u)_{2^{-n}} }
 \end{eqnarray*}
 But we know if $\alpha = \sd(\phi u)$, then 
 \begin{equation*}
  \lim_{n \to \infty} \ 2^{-n\alpha} \ \sm{\left(\kappa_1 - \kappa_2\right) \ , \ (\phi u)_{2^{-n}}} \ = \ 0
 \end{equation*}
 Yet we have this inequality
 \begin{equation*}
  \sd(\phi u) \ < \ \sd(u) \ \doteq \ \omega \label{eq:inequality-sd}
 \end{equation*}
 thus
 \begin{equation*}
  \lim_{n \to \infty} \ 2^{-n\omega} \ \sm{\left(\kappa_1 - \kappa_2\right) \ , \ (\phi u)_{2^{-n}}} \ = \ 0
 \end{equation*}
 which by setting $\omega^\prime = \omega + d$ can be rewritten as follow
 \begin{equation*}
  \lim_{n \to \infty} \ 2^{n(\omega^\prime-d)} \ \sm{\left(\kappa_1 - \kappa_2\right) \ , \ (\phi u)_{2^{-n}}} \ = \ 0  
 \end{equation*}
 Thus the sequence is majorized, for every $\omega^\prime \in (\omega,d)$, by the constant $2^{n(\omega^\prime-d)}$, then $u^n$ is a Cauchy sequence and therefore it converges. The limit
 \begin{equation*}
  \dot{u}(\phi) \ = \ \lim_{n\to \infty} \ u^n(\phi) \ ,  \quad \forall \phi \in \Dcal(\Rbb^d)
 \end{equation*}
 defines an extension of $u$. Because of %\ref{eq:inequality-sd} 
 we know the scaling degree of $\dot{u}$ is smaller or equal to the scaling degree of $u$. We need to prove the scaling degree of $\dot{u}$ is bigger or equal to the one of $u$. We consider the following expression
 \begin{eqnarray*}
  u_\rho(\phi) &=& \rho^{-d} \ \sm{\dot{u} , \phi_{\rho^{-1}}} \\
  &=& \lim_{n \to \infty} \ \rho^{-d} \ \sm{u^n , \phi_{\rho^{-1}}} \\
  &=& \lim_{n \to \infty} \ \rho^{-d} \ \int dx \ \bigg( 1 \ - \ \kappa_{2^n}(x) \bigg) \ u(x) \ \phi_{\rho^{-1}}(x) \\
  &=& \lim_{n \to \infty} \ \rho^{-d} \ \int dx \ \bigg( 1 \ - \ \kappa(2^n x) \bigg) \ u(x) \ \phi_{\rho^{-1}}(x) \\
  &=& \sum_{n=0}^{\infty} \ \rho^{-d} \ \int dx \ \bigg( \kappa(2^n x) \ - \ \kappa(2^{n+1} x) \bigg) \ \phi_{\rho^{-1}}(x) \ u(x) 
 \end{eqnarray*}
 Let $R$ and $\epsilon$ two strictly real positive parameters be such that 
 \begin{eqnarray*}
  && \supp(\phi) \subset \{ x | \abs{x} < R \} \\
  && \kappa(x) = 1 \quad \text{for} \ \abs{x} < \epsilon
 \end{eqnarray*}
 then
 \begin{equation*}
  \bigg( \kappa(2^n x) \ - \ \kappa(2^{n+1} x) \bigg) \ \phi_{\rho^{-1}}(x) \ = \ 0 \quad \mathsf{for} \ 2^{-n} \ \epsilon \ > \abs{x} \geq \rho \ R
 \end{equation*}
 Let us choose $p$ such that 
 \begin{equation*}
  2^{-p} \ \epsilon > \rho \ R \ > 2^{-(p+1)} \ \epsilon
 \end{equation*}
 then we have
 \begin{eqnarray*}
  u_\rho(\phi) &=& \sum_{n=p}^{\infty} \ \rho^{-d} \ \int dx \ \bigg( \kappa(2^n x) \ - \ \kappa(2^{n+1} x) \bigg) \ \phi_{\rho^{-1}}(x) \ u(x) \\
  &=& \sum_{n=p}^{\infty} \ (2^n\rho)^{-d} \ \int dy \ \bigg( \kappa(y) \ - \ \kappa(2 y) \bigg) \ \phi(\rho^{-1}2^{-n}y) \ u(x) 
 \end{eqnarray*}
 thus
 \begin{eqnarray*}
  \abs{u_\rho(\phi)} &\leq& \sum_{n=p}^{\infty} \ (2^n\rho)^{-d} \ c \ 2^{n\omega^\prime} \\
  &\leq& c \ \rho^{-d} \ \frac{2^{-p(d-\omega^\prime)}}{1 - 2^{-(d-\omega^\prime)}} \\
  &\leq& \frac{c \ \rho^{-d}}{1 - 2^{-(d-\omega^\prime)}} \ \left( \frac{2R}{\epsilon} \right) \ \rho^{d-\omega^\prime} \\
  &\leq& c^\prime \ \rho^{-\omega^\prime}
 \end{eqnarray*}
 Thus the scaling degree of $\dot{u}$ is $\omega^\prime$ if $\omega^\prime < 0$. But $\omega^\prime \in (\omega,d)$ therefore the scaling degree of $\dot{u}$ is bigger or equal to the one of u. Yet $\sd(\dot{u})$ is also smaller or equal to $\sd(u)$, we conclude 
 \begin{equation*}
  \sd(\dot{u}) \ = \ \sd(u)  
 \end{equation*}
 
 Therefore we proved if $\sd(u)<d$, then there exists a unique extension $\dot{u} \in \Dcal^\prime(\Rbb^d)$ of $u$, which has the same scaling degree, $\sd(\dot{u})=\sd(u)$. 
 
 \begin{center}
  \textbf{2nd case:} $\sd(u)\geq d$. %\cite[p.647]{BF2000} 
  and %\cite[Appendix B, p. 56]{DF2004}
 \end{center}
 We define the projection $\Wsf$ from $\Dcal(\Rbb^d)$ to $\Dcal_\lambda(\Rbb^d)$. The complement of $\Dcal_\lambda(\Rbb^d)$. The complement of $D_\lambda(\Rbb^d)$ consisting of the derivatives of the the Dirac functions up to order $\lambda$ we can write $\Wsf$ as follow
 \begin{equation*}
  W \ : \ \left\{
  \begin{array}{ccl}
   \Dcal(\Rbb^d) & \to & \Dcal_\lambda(\Rbb^d) \\
   \phi & \mapsto & \phi \ - \ \underset{\abs{\gamma}\leq\lambda}{\sum} \omega_\gamma \ \partial^\gamma \phi(0)
  \end{array}
  \right.
 \end{equation*}
 with $\omega_\gamma$ being smooth functions of compact support such that $\partial^\gamma\omega_\beta(0)=\delta^\gamma_\beta$. Any $\phi \in \Dcal(\Rbb^d)$ can be uniquely decomposed as 
 \begin{eqnarray*}
  \phi &=& \Wsf \phi \ + \ \sum_{\abs{\gamma}\leq\lambda} \omega_\gamma \ \partial^\gamma \phi(0)
 \end{eqnarray*}
 and any element of $\Dcal_\lambda(\Rbb^d)$ can be written in the form
 \begin{eqnarray*}
  \Wsf \phi(x) &=& \sum_{\abs{\gamma}=\lambda^\prime} x^\gamma \varphi_\gamma(x) \ , \quad \varphi_\gamma \in \Dcal(\Rbb^d)
 \end{eqnarray*}
 thus for $u \in \Dcal(\Rbb^d \backslash \{0\})$ 
 \begin{eqnarray*}
  \sm{u,\Wsf\phi} &=& \sum_{\abs{\gamma}=\lambda^\prime} \sm{x^\gamma u \ , \ \varphi_\gamma}
 \end{eqnarray*}
 and since $x^\gamma u(x)$ has scaling degree equal at most to $\lambda - \lambda^\prime + d$ which is strictly smaller bigger than $\lambda$, because $\lambda^\prime$ is the first integer strictly bigger than $\lambda$, therefore $\Wsf \phi$ has an unique extension , it's the one we called the direct extension. Therefore we can define $\dot{u} \in \Dcal(\Rbb^d)$ as
 \begin{eqnarray*}
  \sm{\dot{u},\phi} &=& \sm{u,\Wsf\phi} \ + \ \sum_{\abs{\gamma}\leq\lambda} \partial^\gamma \phi(0) \ \sm{\dot{u},\omega_\gamma} 
 \end{eqnarray*}
 We just proved that $u$ has $\dot{u}$ as extension which is not unique. We still have to prove that $\dot{u}$ has the same scaling degree as $u$.
 
\end{sketch}


The same geometric transformation \eqref{eq:geo_scaling_transfo} can be used to introduce relevant homogeneity properties of a distribution.

\begin{definition}\label{def:homogeneous}
A distribution $u \in \Dcal^\prime(\Mcal^n)$ or $u \in \Dcal^\prime(\Mcal^n\setminus d_n)$ is called {\bf homogeneous of degree} $\delta$, if it satisfies the equality
%
\begin{equation}
\lambda^{\delta} \sm{ u , \phi_\lambda } = \sm{ u , \phi } \ , \ \lambda > 0 \ , \ \mbox{ and } \ \delta \in \Cbb \ ,
\label{eq:homog_id}
\end{equation}
%
where the transformation \eqref{eq:geo_scaling_transfo} are applied for all $\phi \in \Dcal(\Ncal_n\setminus d_n)$.
\end{definition}


The definitions \ref{def:scaling_degree} and \ref{def:homogeneous} imply that a distribution which is homogeneous of degree $\delta$ has scaling degree $-\Re(\delta)$. We further recall the following theorem due to L. Hörmander \cite{hormander_analysis_1990}.


\begin{theorem}
If the distribution $u \in \Dcal(\Mcal^n\setminus d_n)$ is homogeneous of degree $\delta$, and
$\delta$ is not an integer $\leq -(\delta+4(n-1))$, then $u$ has a unique extension $\exte{u} \in \Dcal(\Mcal^n)$ which is homogeneous of degree $\delta$.
\end{theorem}


\begin{proof}
%%TODO PROOF
(blablabla)
\end{proof}


Once we know that a distribution can be extended the next step is to find one. A special case assure us to find an unique extension. For that we need to define the \textbf{partial space} of order $\lambda$. It is the space of functions which vanish up to order $\lambda \in \Rbb$ towards the total diagonal, 
%
\begin{equation*}
\Dcal_{\lambda}(\Mcal^n) \ = \ \left\{ \phi \in \Dcal(\Mcal^n) \ | \ \forall \abs{\alpha} \leq \lambda\ , \ \ \left(\nabla^{\alpha}\phi\right)(0)=0 \right\} \ ,
\end{equation*}
%
with $\Dcal_\lambda(\Mcal^n) = \Dcal(\Mcal^n)$ if $\lambda < 0$. The dual space $\Dcal^\prime_\lambda(\Mcal^n)$ is the corresponding space of distributions. It can been shown that any distribution $u \in \Dcal^\prime (\Mcal^n \setminus d_n )$ has an unique extension $\overline{u} \in \Dcal^\prime_\lambda(\Mcal^n)$, $\lambda = \sd(u) - 4(n-1)$, with same scaling degree. It is called the \textbf{direct extension} of $u$.




\bigskip


We can use this notion of scaling degree using the graphical method introduced in \ref{p:PIC_REG_PB}. The Steinmann scaling degree of $\tsf_\gamma$ is equal to 
%
\begin{equation}
\sd(\tsf_\gamma) = 2 \abs{E(\gamma)} \ . 
\label{eq:sd_graph}
\end{equation}
%
Without giving a general proof we can understand it by looking at a particular graph. Let us consider again the triangular graph with one intern loop, on Minkowski spacetime. The Feynman propagator for free massive scalar fields on Minkowski can be writen as
%
\begin{equation*}
\Delta_\fsf(x) = \frac{1}{\left(2\pi\right)^4} \int_{\Rbb^4} \dsf^4p \ \frac{\esf^{ipx}}{p^2 - m^2 + i \epsilon} \ ,
\end{equation*}
%
and its scaling degree 
%
\begin{equation*}
\sd(\Delta_\fsf) = 2 \ .
\end{equation*}
%
We wrote $\tsf_\gamma$ as the product of Feynman propagator \eqref{eq:kernel}. Thus admiting the product is well defined its scaling degree is the sum of the scaling degree of each term. Therefore $\sd(\tsf_\gamma)$ is given by \eqref{eq:sd_graph}. 


\begin{wrapfigure}{r}{0.3\textwidth}
\begin{tikzpicture}[thick,scale=1] 
\draw[dashed] (0,0) circle (1cm and 0.3cm);
\draw (0,0) circle (1cm and 0.6cm);
\draw[dashed] (-1,0) -- (1,0);
\filldraw (-1,0) circle (2pt) node[left] {$x_1$};
\filldraw (1,0) circle (2pt) node[right] {$x_2$};
\end{tikzpicture}
\end{wrapfigure}


Again looking at the graphs we can define the degree of divergence of a graph $\gamma$ 
%
\begin{equation*}
\omega_\gamma = 2 \abs{E(\gamma)} - 4(\abs{V(\gamma)} - 1) \ .
\end{equation*}
%
We call $\gamma$ supercially convergent if $\omega_\gamma  > 0$, logarithmically divergent if $\omega_\gamma = 0$, and divergent of degree $\omega_\gamma < 0$. In the case of a graph with only two vertices (e.g. the fish or the sunset graph) it is divergent for $\abs{V(\gamma)} \geq 2$. Indeed the scaling degree of $\Delta_\fsf^n$ is lower than $4$ only for $n=1$.


%----------------------------------------------------------------------------%
\bigskip
%\subsection{Regularization framework}
\label{p:REG_FRAMEWORK}
%----------------------------------------------------------------------------%


Theorem \ref{theo:extension_distribution} tells us the extension of a distribution is unique if its scaling degree is strictly smaller than the total dimension. In this case we deduce that the extension is given by the direct extension. Though if the scaling degree is bigger than the total dimension then the extension is not unique, thus we do need a \textbf{procedure of regularization}.


\bigskip


We call \textbf{regularization} of a distribution $u \in \Dcal^\prime(\Mcal^n\setminus d_n)$ a family $\left\{ u^{\alpha}\right\}$ of distributions $u^{\alpha}\in\Dcal^\prime(\Mcal^n)$ if
%
\begin{equation*}
\lim_{\alpha \to 0} \sm{u^{\alpha},\phi} \ = \ \sm{\overline{u},\phi} \ , \ \ \forall \phi \in \Dcal_{\lambda}(\Mcal^n) \ , 
\end{equation*}
%
where $\alpha \in \Omega\setminus\left\{ 0\right\}$ with $\Omega\subset\mathbb{C}$ a neighborhood of the origin. The family $\left\{ u^{\alpha}\right\}$ is called \textbf{analytic regularization} if the map 
% 
\begin{equation*}
\alpha \mapsto \sm{u^{\alpha},\phi} \ , \ \ \forall \phi \in \Dcal(\Mcal^n)
\end{equation*}
%
is analytic in $\alpha\in\Omega\setminus\left\{ 0\right\}$ with pole(s) of finite order at the origin. We recall that an function is analytic\footnote{A synonym for analytic function is holomorphic function.} on $\Omega$ if it is differentiable on the complex plane.  
 
 
The procedure we choose to regularize a distribution is the \textbf{minimal subtraction}. It is defined for a $u \in \Dcal^\prime(\Mcal^n \setminus d_n)$ as 
%
\begin{equation*}
\sm{u^{\mathsf{MS}},\phi} = \lim_{\alpha \to 0} \ \bigg( \sm{u^\alpha , \phi} - \pp\left(\sm{u^\alpha , \phi}\right) \bigg) \ ,
\end{equation*}
%
where $\pp$ is the principal part. We require $\sd(u^{\mathsf{MS}}) = \sd(u)$, then $u^{\mathsf{MS}}$ is called the minimal subtraction of $u$. In order to be able to perform the minimal subtraction we shall need to expand in series a function. This function has to be meromorphic to assure the existence of the series expansion. We recall a function $\phi$ on an open set $\Omega$ is \textbf{meromorphic} if there exists a sequence of points $\{z_0 , z_1 , z_2 , \dots \}$ that has no limit points in $\Omega$\footnote{A point $z \in \Cbb$ is said to be a limit point of the set $\Omega$ if there exists a sequence of points $z_n \in \Omega$ such that $z_n \neq z$ and $\lim_{n \to \infty} z_n = z$.}, and such
that the function $f$ is holomorphic in $\Omega \setminus \{z_0 , z_1 , z_2 , \dots \}$, and $\phi$ has poles at the points $\{z_0 , z_1 , z_2 , \dots \}$.


A generic extension $\dot{u}$ can be defined as follow
% 
\begin{eqnarray*}
\dot{u} = u^{\mathsf{MS}} + P(\delta) \ ,
\end{eqnarray*}
%
with $P(\delta)$ a finite order polinom in the derivatives of the distribution $\delta$.

\bigskip

Let us illustrate the minimal subtraction procedure. We consider the following distribution $u_n \in \Dcal^\prime(\Rbb\setminus\{0\})$
%
\begin{equation*}
u_n(x) = \frac{1}{(x^2)^{n+\alpha}} \ , \ \mbox{ with } \ n \geq 1 \ . 
\end{equation*}
%
Its scaling degree $\sd(u_n)=2n$. Therefore for $n>1$ the extension is not unique. To perform the minimal subtraction procedure we introduce the regularized distribution $u^\alpha_n \in \Dcal^\prime(\Rbb)$ defined as
%
\begin{equation*}
u^\alpha_n(x) = \frac{1}{(x^2)^{n+\alpha}} \ .
\end{equation*}
%
For practical reason we consider the following identity
%
\begin{equation*}
\Box \ u^\alpha_n(x) = - 2 \ (n+\alpha) \ (n+\alpha-2) \ \ u^\alpha_{n+1}(x)  \ .
\end{equation*}
%
Therefore after integrating by part few times we obtain
%
\begin{eqnarray}
\sm{u^\alpha_n \ , \ \phi} &=& \frac{(-1)^{n}}{2^{n-1}} \ \frac{\Gamma(\alpha+1)\Gamma(\alpha-1)}{\Gamma(\alpha+n)\Gamma(\alpha+n-2)} \nonumber \\
&& \cdot \ \int_{\Rbb^4} \dsf x \ \exp\left(-(1+\alpha) \ \log\left(x^2\right)\right) \ \Box^{n-1}\phi(x) \ , 
\label{eq:exo_ms}
\end{eqnarray}
%
where $\Gamma(n)$ is the gamma function. We want to determine the principal part of $\sm{u^\alpha_n \ , \ \phi}$, thus we need to expand in series with respect to $\alpha$ the expression \eqref{eq:exo_ms}. We get
%
\begin{equation*}
\pp\left(\sm{u^\alpha_n \ , \ \phi}\right) = \frac{(-1)^{n}}{2^{n-1}} \ \frac{1}{\Gamma(n-2)\Gamma(n)} \ \int_{\Rbb} dx \ \frac{1}{x^2} \ \Box^{n-1}\phi(x) 
\end{equation*}
%
thus
%
\begin{eqnarray*}
\sm{u_n \ , \ \phi}_\ms &=& \frac{(-1)^{n}}{2^{n-1}} \ \frac{1}{\Gamma(n-2)\Gamma(n)} \\
&& \cdot \ \int_{\Rbb} \dsf x \ \frac{1}{x^2} \ \bigg( -1 + 2 \gamma + \log(x^2) + \Psi^{(0)}(n-2) + \Psi^{(0)}(n) \bigg) \ \Box^{n-1} \phi(x) \ ,
\end{eqnarray*}
%
where $\Psi^{(n)}$ is the $(n+1)$st derivative of the logarithm of the gamma function. Therefore
%
\begin{equation*}
\left(u_n\right)_\ms = \frac{(-1)^{n}}{2^{n-1}} \frac{1}{\Gamma(n-2)\Gamma(n)} \ \Box^{n-1} \left( \frac{-1 + 2 \gamma + \log(x^2) + \Psi^{(0)}(n-2) + \Psi^{(0)}(n)}{x^2} \right) \ .
\end{equation*}
%





\newpage

%%TODO ??
\begin{figure}
\centering
\includegraphics[scale=0.8]{./fig_chap2.pdf}
% chap2.pdf: 595x842 pixel, 72dpi, 20.99x29.70 cm, bb=0 0 595 842
\caption{Summarize.}
\label{fig:chap_2}
\end{figure}


%----------------------------------------------------------------------------%
\chapter{A covariant regularization scheme}
\label{p:COV_REG}
%----------------------------------------------------------------------------%


In the perturbative construction of models in quantum field theory on curved spacetimes we have to construct time--ordered products of field polynomials which are à priori ill defined due to the appearance of UV divergences. Several renormalisation schemes which deal with these divergences in the presence of non--trivial spacetime curvature have been discussed in the literature, as for example the Epstein--Glaser regularisation discussed in \com{section XXX}.  \sbar{This sheme} \com{Although this scheme} is conceptually clear and mathematically rigorous, it is not easily applicable in practical computations. 


\com{In this chapter we} \sbar{We} develop a regularisation scheme for time ordered products in interacting field theories on curved spacetimes, manifestly generally covariant, invariant under any spacetime isometries present and constructed to all orders in perturbation theory. \com{As in the previous chapters} \sbar{In this work}, we discuss only scalar fields in four spacetime dimensions, but we shall argue that this scheme can be directly generalised to other spacetime dimensions and field theories with higher spin, as well as to theories with local gauge invariance. 


The renormalisation scheme we propose is inspired by the works \cite{keller_dimensional_2010,duetsch_dimensional_2014} which deal with perturbative QFT in Minkowski spacetime. In these works, the authors introduce an analytic regularisation of the position space Feynman propagator in Minkowski spacetime which is similar to the one discussed in \cite{bollini_dimensional_2008}. 



%----------------------------------------------------------------------------%
\section{Another approach to the regularization problem}
\label{p:ANOTHER_APPROACH}
%----------------------------------------------------------------------------%

%%TODO ??
Here we \sbar{shall} present another approach to the regularization problem \com{presented/discussed/introduced in section XXXX}, by introducing a ``graphical way'' to write the time ordered product of different observables. 
\com{Due to the causal factorisation property stated as one of the axioms that every time ordered product has to satisfy presented in section XXX, the}
\sbar{The} time ordered product is well defined if the support of $\Fsf_1, \ ... \, \Fsf_n$ are pairwise disjoint
%
\begin{equation*}
\supp(\Fsf_i) \cap \supp(\Fsf_j) = \emptyset \ , \qquad \forall i , j \in \{1,...,n\}, \ i \neq j \ . 
\end{equation*}
%
\com{However, }\sbar{If} we restrict ourselves to functionals which have pairwise disjoint support, \com{we are excluding} \sbar{it is like if we exclude} the functional on $\Fcal_\mathsf{loc}(\Mcal)$ from our study. \sbar{But} \com{Unfortunately, } as we said earlier the functionals which represent the \com{local non linear} interacting \sbar{terms} \com{potentials} of the theory are elements of the space of local functional. Hence we have to extend the definition of the time ordered product for functionals with overlapping supports. \com{This extension is obtained introducing a suitable regularisation scheme.} The regularisation \sbar{sheme} \com{scheme} we developed used the minimal subtraction procedure. It shall appear that these subtraction have to done in a \sbar{specefic} \com{specific} way using the so-called Forest formula \com{to ensures that the axioms stated in section XXX are satisfied}.


\bigskip

Here we shall present another approach to the regularization problem, by introducing a ``graphical way'' to write the time ordered product of different observables. The time ordered product is well defined if the support of $\Fsf_1, \ ... \, \Fsf_n$ are pairwise disjoint
%
\begin{equation*}
\supp(\Fsf_i) \cap \supp(\Fsf_j) = \emptyset \ , \qquad \forall i , j \in \{1,...,n\}, \ i \neq j \ . 
\end{equation*}
%
If we restrict ourselves to functionals which have pairwise disjoint support, it is like if we exclude the functional on $\Fcal_\mathsf{loc}(\Mcal)$ from our study. But as we said earlier the functionals which represent the interacting terms of the theory are elements of the space of local functional. Hence we have to extend the definition of the time ordered product for functionals with overlapping supports. The regularization sheme we developed used the minimal subtraction procedure. It shall appear that these subtraction have to done in a specefic way using the so-called Forest formula.


\bigskip

%%TODO ??
Here we \sbar{shall} present another approach to the regularization problem \com{presented/discussed/introduced in section XXXX}, by introducing a ``graphical way'' to write the time ordered product of different observables. 
\com{Due to the causal factorisation property stated as one of the axioms that every time ordered product has to satisfy presented in section XXX, the}
\sbar{The} time ordered product is well defined if the support of $\Fsf_1, \ ... \, \Fsf_n$ are pairwise disjoint
%
\begin{equation*}
\supp(\Fsf_i) \cap \supp(\Fsf_j) = \emptyset \ , \qquad \forall i , j \in \{1,...,n\}, \ i \neq j \ . 
\end{equation*}
%
\com{However, }\sbar{If} we restrict ourselves to functionals which have pairwise disjoint support, \com{we are excluding} \sbar{it is like if we exclude} the functional on $\Fcal_\mathsf{loc}(\Mcal)$ from our study. \sbar{But} \com{Unfortunately, } as we said earlier the functionals which represent the \com{local non linear} interacting \sbar{terms} \com{potentials} of the theory are elements of the space of local functional. Hence we have to extend the definition of the time ordered product for functionals with overlapping supports. \com{This extension is obtained introducing a suitable regularisation scheme.} The regularisation \sbar{sheme} \com{scheme} we developed used the minimal subtraction procedure. It shall appear that these subtraction have to done in a \sbar{specefic} \com{specific} way using the so-called Forest formula \com{to ensures that the axioms stated in section XXX are satisfied}.


%----------------------------------------------------------------------------%
\subsection{``Pictorial'' perspective of the regularization problem}
\label{p:PIC_REG_PB}
%----------------------------------------------------------------------------%


We saw that the Epstein Glaser procedure \com{illustrated in section} \ref{p:EPSTEIN_GLASER} is not convenient to perform \com{concrete computations}\sbar{in practise}. We shall present another approach to this regularization problem, called the analytic regularization. In order to present this method we shall write the time ordered product in a ``graphical way''. It shall permit us to write the time ordered product of observables as a sum of Feynman diagram. 


The  time ordered product of ``causal'' functionals can be view as the following map described in \eqref{eq:time_ordered_op}. By defining the \com{$n-$}th \com{order} pointwise product $\msf_\nsf$
%
\begin{equation*}
\msf_\nsf \left( \Fsf_1 \otimes \ ... \ \otimes \Fsf_n \right)(\phi) \ = \ \Fsf_1(\phi) \cdot \ ... \ \cdot \Fsf_n(\phi) \ ,
\end{equation*}
%
we can rewrite the time ordering product of $n$\sbar{th} argument in terms of an exponential
%
\begin{equation*}
\Tcal_n (\Fsf_1 \otimes \ ... \ \otimes \Fsf_n)(\phi) \ = \ \msf_\nsf \circ \Tsf_n \bigg( \Fsf_1(\phi_1) \otimes \ ... \ \otimes \Fsf_n(\phi_n) \bigg) \bigg|_{\phi_1 = ... = \phi_n = \phi} \ ,
\end{equation*}
%
with 
%
\begin{eqnarray*}
&&\Tsf_n = \exp\left(\hbar \sum_{1 \leq i < j \leq n} D_{ij}\right) =
\prod_{1 \leq i < j \leq n} \ \sum_{\abs{\ell_{ij}}=1}^\infty \hbar^{\abs{\ell_{ij}}} \ \frac{D_{ij}^{\abs{\ell_{ij}}}}{\abs{\ell_{ij}} !} \ , \\
&& \mbox{ and } \ D_{ij} \ = \ \sm{\Delta_{ij} \ , \ \frac{\delta^2 \hfil}{\delta \phi_i} } \ 
\end{eqnarray*}
%
where $\phi_i = \phi(x_i)$, and $\Delta_{ij}=\Delta_\fsf(x_i,x_j)$. We shall now introduce the graphical procedure we shall use. 

\com{You have to rewrite the following part. You should describe more extensively what is a graph.... a graph is formed by $n$ points called vertices joined by some lines which are called edges. Furthermore, the line are not oriented and the end points of the edges must be different.... at this point you can already introduce  $E(\gamma)$ and $V(\gamma)$ and then you say that the collection of al such graphs with $n$ edges is $\Gcal_n$}


Let $\Gcal_n$ be the set of graphs with \sbar{$n=V(\Gcal_n)$}   \com{$n$} vertices and \com{for every $\gamma\in\Gcal_n$ let us indicate by $V(\gamma)$ the set of vertices of $\gamma$ and by $E(\gamma)$ the set of unoriented} edges \com{of the graph}. We will consider only the subgraphs $\gamma$ of $\Gcal_n$ such that
%
\begin{equation*}
V(\gamma) \ = \ \left\{ 1, ... , k \right\} \ , \quad \mathsf{and} \quad E(\gamma) \ = \ \left\{ \ell_{ij} \ | \  i \in V(\gamma) , \ i\neq j \in \partial \ell_{ij} \right\}
\end{equation*}
\com{ ... Notice that the elements of $\Gcal_n$  have exactly $n$ vertices not less..}
%
with $k\leq n$, and $\partial \ell_{ij} = \{i,j\}$. We will denote by $\abs{E(\gamma)}$ and $\abs{V(\gamma)}$ respectively the number of elements in the set $E(\gamma)$ and $V(\gamma)$. Therefore the time ordered product can be written as follow in order to encode the full combinatorics of the Feynman diagrams,
%
\begin{eqnarray}
&& \Tsf_n \ = \ \sum_{\gamma \in \Gcal_n} \Tsf_\gamma \ , \qquad \mbox{with} \quad \Tsf_\gamma \ = \ \frac{1}{\Nsf(\gamma)} \ \sm{\tsf_\gamma \ , \ \delta_\gamma} \ ,
%\label{eq:time_ordered_prod_graph} 
\\
%
&& \Nsf(\gamma) = \hbar^{-\abs{\Esf(\gamma)}} \prod_{\ell_{ij}\in\Esf(\gamma)} \abs{\ell_{ij}} ! = \hbar^{-\abs{\Esf(\gamma)}} \prod_{(i,j) \in \gamma} \abs{\ell_{ij}} ! \ , \nonumber \\
%
&& \delta_\gamma = \frac{\delta^{2\abs{E(\gamma)}}\hfill}{\underset{\substack{i \in V(\gamma), \\ j \in \partial \ell_{ij}}}{\prod} \delta \phi_i^{\abs{\ell_{ij}}}} \ = \ \frac{\delta^{2\abs{E(\gamma)}}\hfill}{\underset{(i,j)\in\gamma}{\prod} \delta \phi_i^{\abs{\ell_{ij}}}} \ , \nonumber \\
%
\mbox{and} &&\tsf_\gamma = \prod_{\substack{i < j, \\ i \in V(\gamma), \\ j \in \partial \ell_{ij}}} \Delta_{ij}^{\abs{\ell_{ij}}} \ = \ \prod_{(i,j)\in\gamma} \Delta_{ij}^{\abs{\ell_{ij}}} \ , 
%
%\label{eq:kernel}
\end{eqnarray}
%
where $\abs{\ell_{ij}}$ is the number of edges between the vertices $i$ and $j$. Let us illustrate this graphical approach. 


\bigskip

We saw that the Epstein Glaser procedure \ref{p:EPSTEIN_GLASER} is not convenient to perform in practise. We shall present another approach to this regularization problem, called the analytic regularization. In order to present this method we shall write the time ordered product in a ``graphical way''. It shall permit us to write the time ordered product of observables as a sum of Feynman diagram. 


The  time ordered product of ``causal'' functionals can be view as the following map described in \eqref{eq:time_ordered_op}. By defining the $n$th pointwise product $\msf_\nsf$
%
\begin{equation*}
\msf_\nsf \left( \Fsf_1 \otimes \ ... \ \otimes \Fsf_n \right)(\phi) \ = \ \Fsf_1(\phi) \cdot \ ... \ \cdot \Fsf_n(\phi) \ ,
\end{equation*}
%
we can rewrite the time ordering product of $n$th argument in terms of an exponential
%
\begin{equation*}
\Tcal_n (\Fsf_1 \otimes \ ... \ \otimes \Fsf_n)(\phi) \ = \ \msf_\nsf \circ \Tsf_n \bigg( \Fsf_1(\phi_1) \otimes \ ... \ \otimes \Fsf_n(\phi_n) \bigg) \bigg|_{\phi_1 = ... = \phi_n = \phi} \ ,
\end{equation*}
%
with 
%
\begin{eqnarray*}
&&\Tsf_n = \exp\left(\hbar \sum_{1 \leq i < j \leq n} D_{ij}\right) =
\prod_{1 \leq i < j \leq n} \ \sum_{\abs{\ell_{ij}}=1}^\infty \hbar^{\abs{\ell_{ij}}} \ \frac{D_{ij}^{\abs{\ell_{ij}}}}{\abs{\ell_{ij}} !} \ , \\
&& \mbox{ and } \ D_{ij} \ = \ \sm{\Delta_{ij} \ , \ \frac{\delta^2 \hfil}{\delta \phi_i} } \ 
\end{eqnarray*}
%
where $\phi_i = \phi(x_i)$, and $\Delta_{ij}=\Delta_\fsf(x_i,x_j)$. We shall now introduce the graphical procedure we shall use. Let $\Gcal_n$ be the set of graphs with $n=V(\Gcal_n)$ vertices and $E(\Gcal_n)$ edges. We will consider only the subgraphs $\gamma$ of $\Gcal_n$ such that
%
\begin{equation*}
V(\gamma) \ = \ \left\{ 1, ... , k \right\} \ , \quad \mathsf{and} \quad E(\gamma) \ = \ \left\{ \ell_{ij} \ | \  i \in V(\gamma) , \ i\neq j \in \partial \ell_{ij} \right\}
\end{equation*}
%
with $k\leq n$, and $\partial \ell_{ij} = \{i,j\}$. We will denote by $\abs{E(\gamma)}$ and $\abs{V(\gamma)}$ respectively the number of elements in the set $E(\gamma)$ and $V(\gamma)$. Therefore the time ordered product can be written as follow in order to encode the full combinatorics of the Feynman diagrams,
%
\begin{eqnarray}
&& \Tsf_n \ = \ \sum_{\gamma \in \Gcal_n} \Tsf_\gamma \ , \qquad \mbox{with} \quad \Tsf_\gamma \ = \ \frac{1}{\Nsf(\gamma)} \ \sm{\tsf_\gamma \ , \ \delta_\gamma} \ ,
\label{eq:time_ordered_prod_graph} \\
%
&& \Nsf(\gamma) = \hbar^{-\abs{\Esf(\gamma)}} \prod_{\ell_{ij}\in\Esf(\gamma)} \abs{\ell_{ij}} ! = \hbar^{-\abs{\Esf(\gamma)}} \prod_{(i,j) \in \gamma} \abs{\ell_{ij}} ! \ , \nonumber \\
%
&& \delta_\gamma = \frac{\delta^{2\abs{E(\gamma)}}\hfill}{\underset{\substack{i \in V(\gamma), \\ j \in \partial \ell_{ij}}}{\prod} \delta \phi_i^{\abs{\ell_{ij}}}} \ = \ \frac{\delta^{2\abs{E(\gamma)}}\hfill}{\underset{(i,j)\in\gamma}{\prod} \delta \phi_i^{\abs{\ell_{ij}}}} \ , \nonumber \\
%
\mbox{and} &&\tsf_\gamma = \prod_{\substack{i < j, \\ i \in V(\gamma), \\ j \in \partial \ell_{ij}}} \Delta_{ij}^{\abs{\ell_{ij}}} \ = \ \prod_{(i,j)\in\gamma} \Delta_{ij}^{\abs{\ell_{ij}}} \ , 
%
\label{eq:kernel}
\end{eqnarray}
%
where $\abs{\ell_{ij}}$ is the number of edges between the vertices $i$ and $j$. Let us illustrate this graphical approach. 


%%TODO FIGURE
\begin{figure}[ht!]
\centering
\includegraphics[scale=0.1]{./fig_Tprod_2obs.jpg}
% Tprod_2obs.jpg: 3165x321 pixel, 72dpi, 111.65x11.32 cm, bb=0 0 3165 321
\caption{$(\Fsf \cdot_{\Tsf} \Gsf)(\phi) = $}
\end{figure}


%%TODO FIGURE
\begin{figure}[ht!]
\centering
\includegraphics[scale=0.1]{./fig_Tprod_3obs.jpg}
% Tprod_3obs.jpg: 3177x1083 pixel, 72dpi, 112.08x38.21 cm, bb=0 0 3177 1083
\caption{$(\Fsf_1 \cdot_{\Tsf} \Fsf_2 \cdot_{\Tsf} \Fsf_3)(\phi) =$}
\end{figure}



\bigskip







We \com{have} just present\com{ed} a convenient way to represent the product $\cdot_T$ of ``causal'' functionals. \com{However, in the previous expansion and in particular in equation \eqref{eq:kernel}} \sbar{But} $\tsf_\gamma$ \sbar{\eqref{eq:kernel}} is a distribution which is well defined outside of all \textbf{partial diagonals}\index{diagonal!partial}, namely on \sbar{$\Mcal^n\setminus D_n$}  $\Dcal (\Mcal^n\setminus D_n)$, where 
%
\begin{equation}
D_n = \{x_1,\ldots,x_n\,|\, x_i=x_j \text{ for at least one pair } (i,j),\, i\neq j \} \ .
%\label{eq:all_diagonals}
\end{equation}
%
The restricted domain of $\tsf_\gamma$ is the reason why $\Tsf_n$ defined as above is not a well defined operation on $\mathcal{F}_\loc(\Mcal)[[\hbar]]^{\otimes n}$. 


In order to complete the construction we need to extend the obtained distributions $\tsf_\gamma$ to the diagonals $D_n$. The extension is not a straightforward because of the singular structure of the Feynman propagator $\Delta_\fsf$. Indeed the wave front set of $\Delta_\fsf$ contains the one of the $\delta$ distribution, and as already explained it tells us that pointwise products of \sbar{the the} Feynman distribution\com{s} are ill defined on the coinciding points. Consequently we do need a regularization procedure in order to extend $\tsf_\gamma$ to the full space $\Mcal^n$. We shall see that this extension is in general not unique, but subject to a so-called regularization freedom.


\bigskip

Here we shall introduce a procedure to extend the distributions $\tsf_\gamma$ to $D_n$ called \textbf{minimal subtraction (MS)} (look at section %\ref{p:EXT_DISTRIB}
). It makes use of an analytic regularization $\Delta_\fsf^{(\alpha)}$ of the Feynman propagator $\Delta_\fsf$ parametrised by complex parameters $\alpha$ contained in some neighbourhood $\Omega \subset \Cbb$ of the origin. An analytic regularization of the Feynman propagator $\Delta_\fsf$ shall be studied later on \sbar{the present work}. Nonetheless the idea of the $\MS$ scheme is independent of the details of the analytic regularization. Namely, given any analytic regularization $\Delta^{(\alpha)}_\fsf$ of $\Delta_\fsf$ we repeat the formal construction of $\Tsf_n$ by replacing $\Delta_{ij}$ by $\Delta^{(\alpha)}_{ij}$ in \eqref{eq:kernel}. Proceeding in this way we define 
%
\begin{equation}
\Tsf_n^{(\alphabd)} = \sum_{\gamma \in \Gcal_n} \Tsf_\gamma^{(\alphabd)} \ , \qquad \mbox{with} \quad \Tsf_\gamma^{(\alphabd)} \ = \ \frac{1}{\Nsf(\gamma)} \ \sm{\tsf_\gamma^{(\alphabd)} \ , \ \delta_\gamma}
%\label{eq:time_ordered_prod_graph_reg}
\end{equation}
\com{ .... what is $\alphabd$ ??? you have to say that you need to keep track of the different vertices joined by a line. in other words you use $\Delta^{(\alpha_{ij})}_\fsf$  .... you have to say why it is necessary to operate in that way....}


%
and the corresponding integral kernels $\tsf^{(\alphabd)}_\gamma$ of Feynman graphs $\gamma$ 
%
\begin{equation}
\tsf^{(\alphabd)}_\gamma = \prod_{(i,j) \in \gamma} \Delta_{ij}^{\abs{\ell_{ij}}+\alpha_{ij}} \ .
%\label{eq:kernel_reg}
\end{equation}
%
We shall show that the distributions $\tsf^{(\alphabd)}_\gamma$ are \textbf{multivariate meromorphic functions} \com{.... you should give a definition of this concept.... ...} which have poles at the origin for some of the $\alpha_{ij}$. Hence, in order to obtain well defined distributions in the limit $\alpha_{ij}$ to $0$ and consequently a regularized time ordered product, all these poles will \sbar{be} have to be subtracted. The analyticity property of the regularised Feynman propagator shall imply that $\tsf^{(\alphabd)}_\gamma$ is well defined on $ \Mcal^n \setminus D_n $ \eqref{eq:all_diagonals} even if all $\alpha_{ij}$ are vanishing. 


Since $\tsf^{(\alphabd)}_\gamma$ is a multivariate meromorphic function in $\alphabd$  which is analytic if restricted to $\Mcal^n\setminus D_n$, we may deduce that the principal part of $\tsf^{(\alphabd)}_\gamma$ for some $\alpha_{ij}$ must be supported on a partial diagonal of $\Mcal^n$. In fact in order for the time ordered products to fulfil the factorisation property \eqref{eq:causal_factorization}, the subtraction of the principal parts of $\tsf^{(\alphabd)}_\gamma$ needs to be done in such a way that at each step only local terms are subtracted. However \com{the domain of analiticity of $\tsf^{(\alphabd)}_\gamma$ } \sbar{it} only implies that the support of the principal parts is contained in $D_n$, i.e. the union of all the partial diagonals in $\Mcal^n$. \sbar{In} \com{Hence, in} order to satisfy the causal factorisation property, the principal parts need to be removed in a recursive way starting from the partial diagonals corresponding to two vertices and proceeding with the partial diagonals $\drak_{I}$ corresponding to an increasing number $k \leq n$ of vertices 
%
\begin{equation*}
\drak_{I} = \left\{ (x_1,\dots, x_n) \in \Mcal^n, x_i=x_j, \mbox{ with } \ i,j \in I\subset \{1,\dots, n\} , \abs{I} = k \right\} \ . 
\end{equation*}
%
We call $\drak_{I}$ \textbf{partial ordered diagonals}\index{diagonal!partial ordered}. The correct recursion procedure is implemented by the so called \textbf{Epstein Glaser forest formula}, which is a position space analogue of the Zimmermann forest formula \cite{duetsch_dimensional_2014}. \com{Notice that T}\sbar{T}his method does not depend on the graph expansion. \com{In order to present this formula let us start considering} \sbar{We consider} the set of indices 
\begin{equation*}
\overline{n} := \{1,\dots , n\} 
\end{equation*}
and \com{let us} define a forest $F$ \com{as a collection of subsets of  $\overline{n}$ with at least two elements, namely }
%
\begin{equation*}
F = \{ I_1,\dots, I_k\} \ , \qquad I_j \subset \overline{n} \ , \qquad \mbox{and} \qquad \abs{I_j} \geq 2 \ ,
\end{equation*}
%
\sbar{with} \com{where} $\abs{I_j}$ is the number of elements contained in $I_j$. \com{Furthermore, f}\sbar{F}or every pair $I_i,I_j$ of a forest $F$ we require
%
\begin{equation*}
I_i\cap I_j = \emptyset \ , \qquad \text{or} \qquad I_i \subset I_j \ , \qquad \mbox{or} \qquad  I_j\subset I_j \ .
\end{equation*}
%
The set of all forests \sbar{$F$} of $n$ indices together with the empty forest $\{\emptyset\}$ is denoted by $\mathfrak{F}_{\overline{n}}$.


For every subset $I\subset \overline{n}$ we indicate by $\Rsf_I$ the operator which extracts the principal part 
of \com{of a multivariate meromorphic function $f(\alphabd)$ where $\{\alpha_{ij}\}_{1\leq i<j \leq n}$} 
with respect to \com{every $\alpha_{ij}$ with $i,j\in I$} \sbar{$\alpha_I$} \sbar{of a multivariate meromorphic function $f(\alphabd)$}, 
%


\begin{equation}
\com{\Rsf_I f(\alphabd) = - \pp_{\alpha_{ij}, \{i,j\}\subset I} f(\alphabd) \ ,
\qquad \alphabd = \{\alpha_{ij}\}_{1\leq i<j \leq n} \ , }
%\Rsf_I f(\alphabd) = - \pp \lim_{\alphabd \to \alpha_I} f(\alphabd) \ ,
%\qquad \alphabd = \{\alpha_{ij}\}_{1\leq i<j \leq n} \ , 
%\label{eq:pp_op}
\end{equation}
\index{$\Rsf_I $}
%
\com{Antoine, pay attention, the limit is taken at a later stage}
\sbar{where the limit is taken only for the parameters $\alpha_{ij}$ with $i,j \in I$.} 
We set for the case $I=\emptyset$, the operator $\Rsf_\emptyset$ \com{is} equal to the identity.
We can now define the regularized time ordered product in the $\MS$ scheme \cite{duetsch_dimensional_2014}.







\bigskip


We just present a convenient way to represent the product $\cdot_T$ of ``causal'' functionals. But $\tsf_\gamma$ \eqref{eq:kernel} is a distribution which is well defined outside of all \textbf{partial diagonals}\index{diagonal!partial}, namely on $\Mcal^n\setminus D_n$, where 
%
\begin{equation}
D_n = \{x_1,\ldots,x_n\,|\, x_i=x_j \text{ for at least one pair } (i,j),\, i\neq j \} \ .
\label{eq:all_diagonals}
\end{equation}
%
The restricted domain of $\tsf_\gamma$ is the reason why $\Tsf_n$ defined as above is not a well defined operation on $\mathcal{F}_\loc(\Mcal)[[\hbar]]^{\otimes n}$. 


In order to complete the construction we need to extend the obtained distributions $\tsf_\gamma$ to the diagonals $D_n$. The extension is not a straightforward because of the singular structure of the Feynman propagator $\Delta_\fsf$. Indeed the wave front set of $\Delta_\fsf$ contains the one of the $\delta$ distribution, and as already explained it tells us that pointwise products of the the Feynman distribution are ill defined on the coinciding points. Consequently we do need a regularization procedure in order to extend $\tsf_\gamma$ to the full space $\Mcal^n$. We shall see that this extension is in general not unique, but subject to a so-called regularization freedom.


\bigskip

Here we shall introduce a procedure to extend the distributions $\tsf_\gamma$ to $D_n$ called \textbf{minimal subtraction (MS)} (look at section %\ref{p:EXT_DISTRIB}
). It makes use of an analytic regularization $\Delta_\fsf^{(\alpha)}$ of the Feynman propagator $\Delta_\fsf$ parametrised by complex parameters $\alpha$ contained in some neighbourhood $\Omega \subset \Cbb$ of the origin. An analytic regularization of the Feynman propagator $\Delta_\fsf$ shall be studied later on the present work. Nonetheless the idea of the $\MS$ scheme is independent of the details of the analytic regularization. Namely, given any analytic regularization $\Delta^{(\alpha)}_\fsf$ of $\Delta_\fsf$ we repeat the formal construction of $\Tsf_n$ by replacing $\Delta_{ij}$ by $\Delta^{(\alpha)}_{ij}$ in \eqref{eq:kernel}. Proceeding in this way we define 
%
\begin{equation}
\Tsf_n^{(\alphabd)} = \sum_{\gamma \in \Gcal_n} \Tsf_\gamma^{(\alphabd)} \ , \qquad \mbox{with} \quad \Tsf_\gamma^{(\alphabd)} \ = \ \frac{1}{\Nsf(\gamma)} \ \sm{\tsf_\gamma^{(\alphabd)} \ , \ \delta_\gamma}
\label{eq:time_ordered_prod_graph_reg}
\end{equation}
%
and the corresponding integral kernels $\tsf^{(\alphabd)}_\gamma$ of Feynman graphs $\gamma$ 
%
\begin{equation}
\tsf^{(\alphabd)}_\gamma = \prod_{(i,j) \in \gamma} \Delta_{ij}^{\abs{\ell_{ij}}+\alpha_{ij}} \ .
\label{eq:kernel_reg}
\end{equation}
%
We shall show that the distributions $\tsf^{(\alphabd)}_\gamma$ are \textbf{multivariate meromorphic functions} which have poles at the origin for some of the $\alpha_{ij}$. Hence, in order to obtain well defined distributions in the limit $\alpha_{ij}$ to $0$ and consequently a regularized time ordered product, all these poles will be have to be subtracted. The analyticity property of the regularized Feynman propagator shall imply that $\tsf^{(\alphabd)}_\gamma$ is well defined on $ \Mcal^n \setminus D_n $ \eqref{eq:all_diagonals} even if all $\alpha_{ij}$ are vanishing. 


Since $\tsf^{(\alphabd)}_\gamma$ is a multivariate meromorphic function in $\alphabd$  which is analytic if restricted to $\Mcal^n\setminus D_n$, we may deduce that the principal part of $\tsf^{(\alphabd)}_\gamma$ for some $\alpha_{ij}$ must be supported on a partial diagonal of $\Mcal^n$. In fact in order for the time ordered products to fulfil the factorisation property \eqref{eq:causal_factorization}, the subtraction of the principal parts of $\tsf^{(\alphabd)}_\gamma$ needs to be done in such a way that at each step only local terms are subtracted. However it only implies that the support of the principal parts is contained in $D_n$, i.e. the union of all the partial diagonals in $\Mcal^n$. In order to satisfy the causal factorisation property, the principal parts need to be removed in a recursive way starting from the partial diagonals corresponding to two vertices and proceeding with the partial diagonals $\drak_{I}$ corresponding to an increasing number $k \leq n$ of vertices 
%
\begin{equation*}
\drak_{I} = \left\{ (x_1,\dots, x_n) \in \Mcal^n, x_i=x_j, \mbox{ with } \ i,j \in I\subset \{1,\dots, n\} , \abs{I} = k \right\} \ . 
\end{equation*}
%
We call $\drak_{I}$ \textbf{partial ordered diagonals}\index{diagonal!partial ordered}. The correct recursion procedure is implemented by the so called \textbf{Epstein Glaser forest formula}, which is a position space analogue of the Zimmermann forest formula \cite{duetsch_dimensional_2014}. This method does not depend on the graph expansion. We consider the set of indices 
\begin{equation*}
\overline{n} := \{1,\dots , n\} 
\end{equation*}
and define a forest $F$ as 
%
\begin{equation*}
F = \{ I_1,\dots, I_k\} \ , \qquad I_j \subset \overline{n} \ , \qquad \mbox{and} \qquad \abs{I_j} \geq 2 \ ,
\end{equation*}
%
with $\abs{I_j}$ is the number of elements contained in $I_j$. For every pair $I_i,I_j$ of a forest $F$ we require
%
\begin{equation*}
I_i\cap I_j = \emptyset \ , \qquad \text{or} \qquad I_i \subset I_j \ , \qquad \mbox{or} \qquad  I_j\subset I_j \ .
\end{equation*}
%
The set of all forests $F$ of $n$ indices together with the empty forest $\{\emptyset\}$ is denoted by $\mathfrak{F}_{\overline{n}}$.


For every subset $I\subset \overline{n}$ we indicate by $\Rsf_I$ the operator which extracts the principal part with respect to $\alpha_I$ of a multivariate meromorphic function $f(\alphabd)$, 
%
\begin{equation}
\Rsf_I f(\alphabd) = - \pp  \lim_{\alphabd \to \alpha_I} f(\alphabd) \ ,
\qquad \alphabd = \{\alpha_{ij}\}_{1\leq i<j \leq n} \ , 
\label{eq:pp_op}
\end{equation}
\index{$\Rsf_I $}
%
where the limit is taken only for the parameters $\alpha_{ij}$ with $i,j \in I$. We set for the case $I=\emptyset$, the operator $\Rsf_\emptyset$ equal to the identity.


We can now define the regularized time ordered product in the $\MS$ scheme \cite{duetsch_dimensional_2014}.


\begin{theorem}[The regularized time ordered product in the MS scheme] %\label{theo:renorm_t_prod_ms_forest}
The regularized time ordered product of $n$ arguments can be written in the $\MS$ scheme as follow
%
\begin{eqnarray}
\Tcal_n \ = \ \left(\Tcal_n\right)_\ms &=& \lim_{\alphabd \to 0} \msf_\nsf \circ \left( \sum_{F\in\Frak_{\overline{n}}} \prod_{I\in F} \Rsf_I \right) \circ \Tsf^{(\alphabd)}_n \nonumber \\
&=& \lim_{\alphabd \to 0} \msf_\nsf \circ \left( \sum_{F\in\Frak_{\overline{n}}} \prod_{I\in F} \Rsf_I \right) \circ \left( \sum_{\gamma\in\Gcal_n} \Tsf_{\gamma}^{(\alphabd)} \right) \ .
%\label{eq:ms_t_forest}
\end{eqnarray}
%
In the product over $I\in F$, the operator $\Rsf_I$ has to be applied before $\Rsf_J$ if $I\subset J$.
%
The limit $\alphabd \to 0$ is taken in two steps. First we set $\alpha_{I}= \alpha_\gamma$ for $I \in \Frak_{\bar{n}}$ before taking the sum over all forests. Second the limit $\alpha_\gamma \to 0$ is performed.
\end{theorem}


\begin{proof}
%%TODO PROOF
(blablabla)
\end{proof}


The Feynman propagator $\Delta_\fsf$ can be written using the Hadamard representation, denoted $\Hsf_\fsf$. We chose to work in dimension $4$, therefore 
%
\begin{equation}
\Hsf_\fsf(x,y) = \frac{1}{8\pi^2} \bigg( \frac{u(x,y)}{\sigma_\fsf(x,y)} + v(x,y) \log\left( \sigma_\fsf(x,y) \right) + w(x,y) \bigg) 
\label{eq:hadamard_rep}
\end{equation}
%
with $\sigma_\fsf(x,y) = \sigma(x,y) + i \epsilon$ and, $u(x,y)$, $v(x,y)$, and $w(x,y)$ are smooth symmetric biscalars functions regular on coinciding points, and $v(x,y)$, $w(x,y)$ possesse expansion of the form
%
\begin{equation*}
v(x,y) = \sum_{n=0}^{+\infty} v_n(x,y) \sigma(x,y)^n \ , \quad 
w(x,y) = \sum_{n=0}^{+\infty} w_n(x,y) \sigma(x,y)^n \ .
\end{equation*}
% 
The analytic regularization of the local Hadamard expansion \ref{eq:hadamard_rep} of $\Delta_\fsf$ is defined as 
%
\begin{equation}
\Hsf^{(\alpha)}_\fsf = \lim_{\epsilon \downarrow 0} \frac{1}{8\pi^2} \left( \frac{u}{M^{2\alpha} \ \sigma_\fsf^{1+\alpha}} + \frac{v}{\alpha} \left( 1 - \frac{1}{ M^{2\alpha} \ \sigma_\fsf^{\alpha} } \right) \right) + w \ ,
%\label{eq:hadamard_rep_reg}
\end{equation}
%
where we use the (arbitrary but fixed) mass scale $M$ for preserving the mass dimension of $\Hsf_\fsf$ in the regularization. The expressions \ref{eq:hadamard_rep} and \ref{eq:hadamard_rep_reg} are only meaningful on normal neighbourhoods $\Ncal$ of $\Mcal$. In order to define \ref{eq:hadamard_rep_reg} and \ref{eq:kernel_reg} globally, we may employ suitable partitions of unity. Rather than providing general and cumbersome formulas, we prefer to illustrate the idea at the example of the triangular graph.


\begin{wrapfigure}{r}{0.2\textwidth}
\begin{center}
\begin{tikzpicture}[thick,scale=0.8] 
\draw (0,0) -- (2,0);
\draw (2,0) -- (1,1.4);
\draw [bend left] (0,0) edge (1,1.4);
\draw [bend left] (1,1.4) edge (0,0);
\filldraw (0,0) circle (2pt) node[left] {$x_1$};
\filldraw (2,0) circle (2pt) node[right] {$x_3$};
\filldraw (1,1.4) circle (2pt) node[above] {$x_2$};
\end{tikzpicture}
\end{center}
\end{wrapfigure}


The corresponding expression of \ref{eq:kernel} can be written as 
%
\begin{eqnarray*}
\tsf_\gamma &=& \Hsf_{13} \ \Hsf_{23} \ \Hsf_{12}^2 \\
&=& \Hsf_\fsf(x_1,x_3) \ \Hsf_\fsf(x_2,x_3) \ \Hsf_\fsf(x_1,x_2)^2 \ .
\end{eqnarray*}
%
For its regularized version we have to apply the forest formula. The subsets $I$ from which we get the forests formula for any $3$ vertices graph are
%
\begin{equation*}
(12) \ , \quad (13) \ , \quad (23) \ , \ \mbox{ and } \quad (123) \ .
\end{equation*}
%
In this particular case of the triangular graph with one fish graph as subgraph the subsets which correspond to divergent contributions are 
%
\begin{equation*}
(12) \quad \mbox{ and } \quad (123) \ .
\end{equation*}
%
Thus we relevant forests to write \eqref{eq:ms_t_forest} are 
%
\begin{equation*}
\{\emptyset\} \ , \quad \{(12)\} \ , \quad \{(123)\} \ , \quad \{(12),(123)\} \ .
\end{equation*} 
%
The regularized $\tsf_\gamma$ thus reads
\begin{equation}
\left(\tsf_\gamma\right)_\ms = \left(1+\Rsf_{12}+\Rsf_{123}+\Rsf_{123}\Rsf_{12}\right) \tsf^{(\alphabd)}_\gamma = (1+\Rsf_{123})(1+\Rsf_{12}) \tsf^{(\alphabd)}_\gamma
%\label{eq:kernel_trig_ms}
\end{equation}
The regularization of $\tsf_\gamma$ is discussed in details in section \ref{p:COMPLICATED_GRAPH}. 


In order to define $\tsf_\gamma$ globally we shall show that $\Mcal$ admits a covering of open geodesically convex sets.


\begin{lemma}
For any lorentzian manifold $\Mcal$ there exits a cover $\Ccal$ such that every elements $\Ncal_i$ of $\Ccal$ and their overlaps $\Ncal_i \cap \Ncal_j$ are open geodesically convex subsets of $\Mcal$.
\end{lemma}


\begin{proof}
%%TODO PROOF 
(blablabla)
\end{proof}


Therefore we shall consider a such cover $\Ccal$ of $\Mcal$. We define the sets
%
\begin{equation}
\Ncal_{12} = \bigcup_{\Ncal\in\Ccal} \Ncal \times \Ncal \subset \Mcal^2 \ , \ \mbox{ and } \ \  \Ncal_{123} = \bigcup_{\Ncal\in\Ccal} \Ncal \times \Ncal\times \Ncal \subset \Mcal^3 \ . 
%\label{eq:neighborhood_glob}
\end{equation}
%
We call sets of the form $\Ncal_{12}$ and $\Ncal_{123}$ \textbf{normal neighbourhoods of the total diagonal}. 


This definition is essentially motivated by the fact that for every $x \in \Mcal$ we can find a normal neighborhood $\Ncal_x \in \Ccal$ of $x$ in $\Mcal$. 



The squared geodesic distance $\sigma$ is then well defined on $\mathcal{N}_{12}$, whereas the same is in general not true if we replace $\mathcal{C}$ in the previous formula with a covering of $\mathcal{M}$ formed by sets which are not open geodesically convex. We set 
%
\begin{equation*}
\sigma_{ij} = \sigma(x_i,x_j) \ .
\end{equation*}
%
We observe that $\sigma_{12}$ is well defined on $\Ncal_{12}$, and that $\sigma_{12}$, $\sigma_{13}$ and $\sigma_{23}$ are well defined on $\Ncal_{123}$. 


We now consider the following smooth and compactly supported functions
%
\begin{eqnarray*}
&& \chi_{12} \in \Dcal(\Ncal_{12}) \ , \ \mbox{ with } \ \ \chi_{12} = 1 \ \mbox{ on } \ d_2 \subset \Ncal_{12} \ , \\[6pt]
&\mbox{and}& \chi_{123} \in \Dcal(\Ncal_{123}) \ , \ \mbox{ with } \ \ \chi_{123} = 1 \mbox{ on } d_3 \subset \Ncal_{123} \ . 
\end{eqnarray*}
%
Note that by construction $\chi_{12}$ and $\chi_{123}$ vanish outside of $\Ncal_{12}$ and $\Ncal_{123}$ respectively. We may now define the analytically regularized distribution $\tsf^{(\alphabd)}_\gamma$ by setting
%
\begin{eqnarray}
\tsf^{(\alphabd)}_\gamma &=& \Hsf^{(\alpha_{13})}_{13} \ \Hsf^{(\alpha_{23})}_{23} \left(\Hsf^{(\alpha_{12})}_{12}\right)^2 \ \chi_{12} \ \chi_{123} \ + \ \Hsf_{13} \ \Hsf_{23} \ \Hsf_{12}^2 \ (1-\chi_{12}) \nonumber \\[3pt]
&& + \ \Hsf_{13} \ \Hsf_{23} \ \left(\Hsf^{(\alpha_{12})}_{12}\right)^2 \ \chi_{12} \ (1-\chi_{123}) \ , 
%\label{eq:kernel_reg_glob}
\end{eqnarray}
%
where the Feynman propagators are regularized as in \ref{eq:hadamard_rep_reg}. By construction $\tsf^{(\alphabd)}_\gamma$ is globally well defined.


Keeping in mind this approach to define global analytically regularized quantities, we shall for simplicity work only with local quantities in the following.


\begin{theorem}[The regularized time ordered product in the MS scheme] \label{theo:renorm_t_prod_ms_forest}
The regularized time ordered product of $n$ arguments can be written in the $\MS$ scheme as follow
%
\begin{eqnarray}
\Tcal_n \ = \ \left(\Tcal_n\right)_\ms &=& \lim_{\alphabd \to 0} \msf_\nsf \circ \left( \sum_{F\in\Frak_{\overline{n}}} \prod_{I\in F} \Rsf_I \right) \circ \Tsf^{(\alphabd)}_n \nonumber \\
&=& \lim_{\alphabd \to 0} \msf_\nsf \circ \left( \sum_{F\in\Frak_{\overline{n}}} \prod_{I\in F} \Rsf_I \right) \circ \left( \sum_{\gamma\in\Gcal_n} \Tsf_{\gamma}^{(\alphabd)} \right)  \nonumber \\
%\label{eq:ms_t_forest}
&=& \com{ \sum_{\gamma\in\Gcal_n} \lim_{\alphabd \to 0} \msf_\nsf \circ \left( \sum_{F\in\Frak_{\overline{n}}} \prod_{I\in F} \Rsf_I \right) \circ  \Tsf_{\gamma}^{(\alphabd)} \ ,}
\label{eq:ms_t_forest}
\end{eqnarray}
%
\com{where at the last point the linearity of the operators $\Rsf_I$ is used, furthermore in}
\sbar{In} the product over $I\in F$, the operator $\Rsf_I$ has to be applied before $\Rsf_J$ if $I\subset J$.
%
The limit $\alphabd \to 0$ is taken in two steps. First we set $\alpha_{I}= \alpha_\gamma$ for \sbar{$I \in \Frak_{\bar{n}}$} \com{$I \in F$} before taking the sum over all forests. Second the limit $\alpha_\gamma \to 0$ is performed.
\end{theorem}
\com{ .... Check if what it is written in the theorem (the order of the limits in particular) is really what we are doing!}

\begin{proof}
%%TODO PROOF
(blablabla)
\end{proof}
\com{...You should write some explicit examples for some graph....    you should maybe discuss later the problem... with sigmas... In other words 1) write the forest formula for the graph you are proposing below. 2) say that $\Delta$ is $H$ and that $H$ is something of the form $1/\sigma$ and state that you will discuss its rigorous construction later. After this you could discuss the problem with the neighborhoods. Maybe it is better to devote a subsection to that problem namely to the whole problem 2).}

The analytic regularization of the local Hadamard expansion \ref{eq:hadamard_rep} of $\Delta_\fsf$ is defined as 
%
\begin{equation}
\Hsf^{(\alpha)}_\fsf = \lim_{\epsilon \downarrow 0} \frac{1}{8\pi^2} \left( \frac{u}{M^{2\alpha} \ \sigma_\fsf^{1+\alpha}} + \frac{v}{\alpha} \left( 1 - \frac{1}{ M^{2\alpha} \ \sigma_\fsf^{\alpha} } \right) \right) + w \ ,
\label{eq:hadamard_rep_reg}
\end{equation}
%
\com{you are introducing at this point this definition, but you prove much later that this is really an analytic regularization, maybe it is not really a good strategy....}
where we use the (arbitrary but fixed) mass scale $M$ for preserving the mass dimension of $\Hsf_\fsf$ in the regularization. The expressions \com{\eqref{eq:hadamard_rep}} and \com{\eqref{eq:hadamard_rep_reg}} are only meaningful on normal neighbourhoods $\Ncal$ of $\Mcal$. In order to define \ref{eq:hadamard_rep_reg} and \ref{eq:kernel_reg} globally, we may employ suitable partitions of unity. Rather than providing general and cumbersome formulas, we prefer to illustrate the idea at the example of the triangular graph.


\begin{wrapfigure}{r}{0.2\textwidth}
\begin{center}
\begin{tikzpicture}[thick,scale=0.8] 
\draw (0,0) -- (2,0);
\draw (2,0) -- (1,1.4);
\draw [bend left] (0,0) edge (1,1.4);
\draw [bend left] (1,1.4) edge (0,0);
\filldraw (0,0) circle (2pt) node[left] {$x_1$};
\filldraw (2,0) circle (2pt) node[right] {$x_3$};
\filldraw (1,1.4) circle (2pt) node[above] {$x_2$};
\end{tikzpicture}
\end{center}
\end{wrapfigure}


The corresponding expression of \ref{eq:kernel} can be written as 
%
\begin{eqnarray*}
\tsf_\gamma &=& \Hsf_{13} \ \Hsf_{23} \ \Hsf_{12}^2 \\
&=& \Hsf_\fsf(x_1,x_3) \ \Hsf_\fsf(x_2,x_3) \ \Hsf_\fsf(x_1,x_2)^2 \ .
\end{eqnarray*}
%
For its regularized version we have to apply the forest formula. The subsets $I$ from which we get the forests formula for any $3$ vertices graph are
%
\begin{equation*}
(12) \ , \quad (13) \ , \quad (23) \ , \ \mbox{ and } \quad (123) \ .
\end{equation*}
\com{here and below the notation is not correct. You should use $\{1,2\}$ instead of $(12)$....}
%
In this particular case of the triangular graph with one fish graph as subgraph the subsets which correspond to divergent contributions are 
%
\begin{equation*}
(12) \quad \mbox{ and } \quad (123) \ .
\end{equation*}
%
Thus we relevant forests to write \eqref{eq:ms_t_forest} are 
%
\begin{equation*}
\{\emptyset\} \ , \quad \{(12)\} \ , \quad \{(123)\} \ , \quad \{(12),(123)\} \ .
\end{equation*} 
%
The regularized $\tsf_\gamma$ thus reads
\begin{equation}
\left(\tsf_\gamma\right)_\ms = \left(1+\Rsf_{12}+\Rsf_{123}+\Rsf_{123}\Rsf_{12}\right) \tsf^{(\alphabd)}_\gamma = (1+\Rsf_{123})(1+\Rsf_{12}) \tsf^{(\alphabd)}_\gamma
\label{eq:kernel_trig_ms}
\end{equation}
\com{...the limit is missing....}
The regularization of $\tsf_\gamma$ is discussed in details in section \ref{p:COMPLICATED_GRAPH}. 


In order to define $\tsf_\gamma$ globally we shall show that $\Mcal$ admits a covering of open geodesically convex sets.


\begin{lemma}
For any lorentzian manifold $\Mcal$ there exits a cover $\Ccal$ such that every elements $\Ncal_i$ of $\Ccal$ and their overlaps $\Ncal_i \cap \Ncal_j$ are open geodesically convex subsets of $\Mcal$.
\end{lemma}


\begin{proof}
%%TODO PROOF 
(blablabla)
\end{proof}


Therefore we shall consider a such cover $\Ccal$ of $\Mcal$. We define the sets
%
\begin{equation}
\Ncal_{12} = \bigcup_{\Ncal\in\Ccal} \Ncal \times \Ncal \subset \Mcal^2 \ , \ \mbox{ and } \ \  \Ncal_{123} = \bigcup_{\Ncal\in\Ccal} \Ncal \times \Ncal\times \Ncal \subset \Mcal^3 \ . 
\label{eq:neighborhood_glob}
\end{equation}
%
We call sets of the form $\Ncal_{12}$ and $\Ncal_{123}$ \textbf{normal neighbourhoods of the total diagonal}. 


This definition is essentially motivated by the fact that for every $x \in \Mcal$ we can find a normal neighborhood $\Ncal_x \in \Ccal$ of $x$ in $\Mcal$. 



The squared geodesic distance $\sigma$ is then well defined on $\mathcal{N}_{12}$, whereas the same is in general not true if we replace $\mathcal{C}$ in the previous formula with a covering of $\mathcal{M}$ formed by sets which are not open geodesically convex. We set 
%
\begin{equation*}
\sigma_{ij} = \sigma(x_i,x_j) \ .
\end{equation*}
%
We observe that $\sigma_{12}$ is well defined on $\Ncal_{12}$, and that $\sigma_{12}$, $\sigma_{13}$ and $\sigma_{23}$ are well defined on $\Ncal_{123}$. 


We now consider the following smooth and compactly supported functions
%
\begin{eqnarray*}
&& \chi_{12} \in \Dcal(\Ncal_{12}) \ , \ \mbox{ with } \ \ \chi_{12} = 1 \ \mbox{ on } \ d_2 \subset \Ncal_{12} \ , \\[6pt]
&\mbox{and}& \chi_{123} \in \Dcal(\Ncal_{123}) \ , \ \mbox{ with } \ \ \chi_{123} = 1 \mbox{ on } d_3 \subset \Ncal_{123} \ . 
\end{eqnarray*}
%
Note that by construction $\chi_{12}$ and $\chi_{123}$ vanish outside of $\Ncal_{12}$ and $\Ncal_{123}$ respectively. We may now define the analytically regularised distribution $\tsf^{(\alphabd)}_\gamma$ by setting
%
\begin{eqnarray}
\tsf^{(\alphabd)}_\gamma &=& \Hsf^{(\alpha_{13})}_{13} \ \Hsf^{(\alpha_{23})}_{23} \left(\Hsf^{(\alpha_{12})}_{12}\right)^2 \ \chi_{12} \ \chi_{123} \ + \ \Hsf_{13} \ \Hsf_{23} \ \Hsf_{12}^2 \ (1-\chi_{12}) \nonumber \\[3pt]
&& + \ \Hsf_{13} \ \Hsf_{23} \ \left(\Hsf^{(\alpha_{12})}_{12}\right)^2 \ \chi_{12} \ (1-\chi_{123}) \ , 
\label{eq:kernel_reg_glob}
\end{eqnarray}
%
where the Feynman propagators are regularised as in \sbar{\ref{eq:hadamard_rep_reg}} \com{\eqref{eq:hadamard_rep_reg}}. By construction $\tsf^{(\alphabd)}_\gamma$ is globally well defined.


Keeping in mind this approach to define global analytically regularised quantities, we shall for simplicity work only with local quantities in the following.


%----------------------------------------------------------------------------%
\subsection{Program} 
\label{p:PROGRAM}
%----------------------------------------------------------------------------%


For implementing the minimal subtraction scheme as presented in section \ref{p:PIC_REG_PB} we first need to specify an analytic regularization $\Delta^{(\alpha)}_\fsf$ of the Feynman propagator $\Delta_\fsf$ on generic curved spacetimes. Then we have to show that for all graphs $\gamma \in \Gcal_n$ the \com{analytically regularized} integral kernels \com{appearing in} \eqref{eq:kernel_reg} \sbar{analytically regularized appearing} satisfy the necessary properties for the implementation of the $\MS$ scheme of theorem \ref{theo:renorm_t_prod_ms_forest}. 


In particular we need to \sbar{shox} \com{show} that the distribution $\tsf^{(\alphabd)}_\gamma$ \eqref{eq:kernel_reg}, which is a priori defined only on $\Mcal^n \setminus D_n$, can be uniquely extended to the full space $\Mcal^n$ without regularization. The uniqueness of this extension is important in order to obtain a definite regularization scheme. 


Moreover, we need to show that the distribution $\tsf^{(\alphabd)}_\gamma \in \Dcal^\prime(\Mcal^n)$ is weakly meromorphic in $\alphabd$ in a neighbourhood $\Omega \subset \Cbb$ of 0 \com{....this is not correct.... $\alphabd$ is a multivariable function...}. We work with the forest formula therefore we just need to show that, setting $\alpha_{ij} = \alpha_I$ for all $i,j\in I$, \eqref{eq:kernel_reg} is weakly meromorphic in $\alpha_I$. 


Additionally, we need to prove that if $\tsf_\gamma$ \com{in} \eqref{eq:kernel} is well defined outside all the partial ordered diagonal $\drak_I$, then the pole of $\tsf^{(\alphabd)}_\gamma$ \com{in} \eqref{eq:kernel_reg} with $\alpha_{ij} = \alpha_I$ for all $i,j\in I$ in $\alpha_I$ is supported on $\drak_I$ and thus local. 


The local pole contributions are independent of the choice of $\chi_{ij}$, and $\Ncal_{ij}$ in \sbar{oreder} \com{order} to write \com{\eqref{eq:kernel_reg_glob}}, such that the $\MS$ regularised amplitude $(\tsf_\gamma)_\ms$ is both globally well defined and independent of the quantities used for the global definition of the analytic regularization. 


Finally, we need to prove that our $\MS$ scheme satisfies all properties given in \cite{hollands_local_2001,hollands_existence_2002} \com{and discussed in section XXXX} which a physically meaningful regularization scheme on curved spacetimes should satisfy.

\bigskip


For implementing the minimal subtraction scheme as presented in section \ref{p:PIC_REG_PB} we first need to specify an analytic regularization $\Delta^{(\alpha)}_\fsf$ of the Feynman propagator $\Delta_\fsf$ on generic curved spacetimes. Then we have to show that for all graphs $\gamma \in \Gcal_n$ the integral kernels \eqref{eq:kernel_reg} analytically regularized appearing satisfy the necessary properties for the implementation of the $\MS$ scheme of theorem \ref{theo:renorm_t_prod_ms_forest}. 


In particular we need to shox that the distribution $\tsf^{(\alphabd)}_\gamma$ \eqref{eq:kernel_reg}, which is a priori defined only on $\Mcal^n \setminus D_n$, can be uniquely extended to the full space $\Mcal^n$ without regularization. The uniqueness of this extension is important in order to obtain a definite regularization scheme. 


Moreover, we need to show that the distribution $\tsf^{(\alphabd)}_\gamma \in \Dcal^\prime(\Mcal^n)$ is weakly meromorphic in $\alphabd$ in a neighbourhood $\Omega \subset \Cbb$ of 0. We work with the forest formula therefore we just need to show that, setting $\alpha_{ij} = \alpha_I$ for all $i,j\in I$, \eqref{eq:kernel_reg} is weakly meromorphic in $\alpha_I$. 


Additionally, we need to prove that if $\tsf_\gamma$ \eqref{eq:kernel} is well defined outside all the partial ordered diagonal $\drak_I$, then the pole of $\tsf^{(\alphabd)}_\gamma$ \eqref{eq:kernel_reg} with $\alpha_{ij} = \alpha_I$ for all $i,j\in I$ in $\alpha_I$ is supported on $\drak_I$ and thus local. 


The local pole contributions are independent of the choice of $\chi_{ij}$, and $\Ncal_{ij}$ in oreder to write \ref{eq:kernel_reg_glob}, such that the $\MS$ regularized amplitude $(\tsf_\gamma)_\ms$ is both globally well defined and independent of the quantities used for the global definition of the analytic regularization. 


Finally, we need to prove that our $\MS$ scheme satisfies all properties given in \cite{hollands_local_2001,hollands_existence_2002} which a physically meaningful regularization scheme on curved spacetimes should satisfy.


Our plan to construct the mentioned quantities and to prove their required properties is as follows.


\begin{itemize}


\item In section \ref{p:REG_FEYNMAN_PROP} we build an analytic regularization $\Delta^{(\alpha)}_\fsf$ \eqref{eq:hadamard_rep_reg} of the Feynman propagator based on the observation that locally $\Delta_\fsf$ can be written using the Hadamard representation \eqref{eq:hadamard_rep}. 


\item We then prove in proposition \ref{prop:amplitude_sigma_prop_analyt} that the relevant distributions 
%
\begin{equation}
\Asf_\gamma^{(\alphabd)} = \prod_{(i,j)\in\gamma} \frac{1}{\sigma_\fsf^{\ell_{ij}(1+\alpha_{ij})}} \in \Dcal^\prime(\Mcal^n\setminus D_n)
\label{eq:amplitude_sigma_reg}
\end{equation}
\index{$\Asf_\gamma^{(\alphabd)}$}%
%
are multivariate analytic functions. The distribution \eqref{eq:amplitude_sigma_reg} only shows the most singular contribution of \eqref{eq:hadamard_rep_reg}. The others contributions are of the same form up to replacing some of the factors $(1+\alpha_{ij})$ in the exponents by $\alpha_{ij}$ or $0$.


\item In order to show that $\Asf_\gamma^{(\alphabd)}$ \eqref{eq:amplitude_sigma_reg} can be uniquely extended from $\Mcal^n\setminus D_n$ to $\Mcal^n$ in a weakly meromorphic way, i.e. that the singularities relevant for the forest formula are poles of finite order, we follow a strategy similar to the one used in \cite{hollands_local_2001} and consider a scaling expansion with respect to a suitable scaling transformation. 

We first argue in proposition \ref{prop:regularization} that an analytically regularized distribution $u^{(\alpha)}\in\Dcal^\prime(\Mcal^n\setminus d_n)$, which can be written as a sum of homogeneous terms with respect to this scaling transformation plus a sufficiently regular remainder, can be uniquely extended to $\Mcal^n$ in a weakly meromorphic way.

In proposition \ref{prop:set} we give a sufficient condition for the existence of such a homogeneous expansion and we show in proposition \ref{prop:almost_homo} that the distributions $\Asf_\gamma^{(\alphabd)}$ satisfy this condition.


\item 
%%TODO ??
The above mentioned results are proved by means of generalised Euler operators which can be written abstractly in terms of a scaling transformation, but also in terms of covariant differential operators whose explicit form can be straightforwardly computed as we argue in section \ref{p:DIFFERENTIAL_EULER}. 

In proposition \ref{prop:expose_poles} we use these operators in order to demonstrate how the full relevant pole structure of $\Asf_\gamma^{(\alphabd)}$ can be computed, thus showing the practical feasibility of the $\MS$ scheme. We find that our regularization scheme corresponds in fact to a particular form of differential regularization.


\item Finally, in proposition \ref{prop:properties_scheme} we prove that the $\MS$ scheme satisfies the axioms of \cite{hollands_local_2001,hollands_existence_2002} for time ordered products and in addition preserves invariance under any spacetime isometries present.


\end{itemize}


%----------------------------------------------------------------------------%
\section{Regularization via the genealized Euler operator}
\label{p:DIFFERENTIAL_EULER}
%----------------------------------------------------------------------------%


In this section we analyse analytically regularized distributions satisfying a weaker homogeneity condition, provide sufficient conditions for this weaker homogeneity to hold, and show how the principal part of a distribution of this type can be efficiently computed.


\bigskip


We consider a normal neighbourhood $\Ncal_n$ \eqref{eq:neighborhood_glob} of the total diagonal $d_n$ and define the \textbf{generalized Euler operator}\index{generalized Euler operator}\index{$\Esf_p$}
%
\begin{equation}
\Esf_p : \left\{
\begin{array}{lcl}
\Dcal(\Ncal_n) & \to & \Dcal(\Ncal_n) \\
\phi(x_1,\dots, x_n) & \mapsto & \left. (-1)^p \ \lambda^{p+4(n-1)} \ \dfrac{d^p}{d\lambda^p} \bigg( \lambda^{-4(n-1)}  \phi_\lambda(x) \bigg) \right|_{\lambda = 1}
\end{array}
\right. \ ,
\label{eq:euler_op}
%
\end{equation}
%
where the scaling transformation \eqref{eq:geo_scaling_transfo} is used. We then consider a family of distributions $u^{(\alpha)} \in \Dcal^\prime(\Ncal_n\setminus d_n)$ defined for $\alpha$ in some neighbourhood $\Omega \subset \Cbb$ of $0$ and assume that $u^{(\alpha)}$ can be expanded as
%
\begin{equation*}
u^{(\alpha)}  = \sum_{k=0}^m u^{(\alpha)}_k + r^{(\alpha)} \ . 
\end{equation*}
%
where $u^{(\alpha)}_k$ are homogeneous with degree with degree $a_k = - \delta_\alpha + k$, thus 
%
\begin{equation*}
\sd(u^{(\alpha)}_k) = \Re\left(\delta_\alpha\right) - k \geq 4(n-1) \ .
\end{equation*}
%
The remainder $r^{(\alpha)}\in \Dcal^\prime(\Ncal_n\setminus d_n)$ has scaling degree smaller than $4(n-1)$ and can thus be uniquely extended to $d_n$ for every $\alpha \in \Omega$ by theorem \ref{theo:extension_distribution}. Owing to its homogeneity, every $u^{(\alpha)}_k$ can be rewritten by means of the relation \eqref{eq:homog_id} and the generalised Euler operator $\Esf_p$ \eqref{eq:euler_op} as
%
\begin{equation}
\sm{ u^{(\alpha)}_k, \phi } = \frac{1}{\overset{p-1}{\ \underset{j=0}{\prod}} \ (a_k+j+4(n-1))}   \sm{ u^{(\alpha)}_k, \Esf_p \phi } \ .
\label{eq:expose_poles}
\end{equation}
%
Note that, $\Esf_p \phi(x_1, \dots x_n)$ is smooth and vanishes for
%
\begin{equation*}
y = (y_1 , \dots , y_n) \to x = (x_1, \dots , x_n) \ , \ \mbox{ as } \ \ C|y-x|^p \ , 
\end{equation*}
%
i.e. it is in the class $\Ocal(|y-x|^{p})$. For this reason, if  $p$ is chosen sufficiently large as $p > -a_k-4(n-1)$ then 
\begin{equation*}
u^{(\alpha)}_k \circ \Esf_p 
\end{equation*}
%
possesses a unique extension to $d_n$. 


\bigskip


We recall that, in order to regularize $u^{(\alpha)}$ for $\alpha=0$ in the $\MS$ scheme, we have to subtract its principal part before computing the limit of vanishing $\alpha$
%
\begin{equation*}
\sm{ (u_k)_\ms, \phi } = \lim_{\alpha \to 0} \left( \sm{ u^{(\alpha)}_k, \phi } - \pp\sm{ u^{(\alpha)}_k , \phi } \right) \ .
\end{equation*}


However, if we use the representation of $u^{(\alpha)}_k$ provided by the right hand side of equation \eqref{eq:expose_poles}, its poles are manifestly exposed and can be easily subtracted. We recall that, since the original distribution $u^{(\alpha)}_k$ is well defined on $\Ncal_n\setminus d_n$ even for $\alpha=0$, the principal part we are subtracting can only be supported on $d_n$. We summarise this discussion in the following proposition.


\begin{proposition}\label{prop:regularization}
Consider a normal neighbourhood $\Ncal_n$ of the total diagonal $d_n$, and $\Omega \subset \Cbb$ a neighbourhood of the origin. Assume that $u^{(\alpha)}\in\Dcal^\prime(\Ncal_n\setminus d_n)$ is an \textbf{analytic regularization} of $u\in\Dcal^\prime(\Ncal_n\setminus d_n)$, \textbf{i.e.}
%
\begin{center}
$u^{(\alpha)}$ is weakly analytic for $\alpha \in \Omega$, and $\ \underset{\alpha\to 0}{\lim} u^{(\alpha)} = u$.
\end{center}
%
Moreover, assume that $u^{(\alpha)}$ can be decomposed as 
%
%%TODO ??
\begin{equation*}
u^{(\alpha)} = \sum_{k=0}^m u^{(\alpha)}_k + r^{(\alpha)} \ ,
\end{equation*}
%
where $u^{(\alpha)}_k$ are weakly analytic distributions which scale homogeneously under transformations of the form \eqref{eq:geo_scaling_transfo} with degree $a_k = -\delta_\alpha + k$, and where $r^{(\alpha)}_k$ is a weakly analytic distribution whose scaling degree towards $d_n$ is strictly smaller than $4(n-1)$. Then the following statements hold.
%
\begin{enumerate}
\item $u^{(\alpha)}$ can be extended to $\exte{u}^{(\alpha)} \in \Dcal^\prime(\Ncal_n)$ for every $\alpha \in \Omega \setminus \{0\}$.
%
\item $\exte{u}^{(\alpha)}$ is weakly meromorphic for $\alpha \in \Omega$ with possible poles for $\alpha=0$ and it is the unique weakly meromorphic extension of $u^{(\alpha)}$.
%
\item The pole of $\exte{u}^{(\alpha)}$ in $0$ is supported on $d_n$.
%
\item The limit $\alpha \to 0$ can be considered after subtracting the pole part, namely
%
\begin{equation*}
\sm{ u_\ms, \phi } = \lim_{\alpha \to 0} \left( \sm{ \exte{u}^{(\alpha)} , \phi } - \pp\sm{ \exte{u}^{(\alpha)} , \phi } \right) 
\end{equation*}
%
is well defined for all $\phi \in \Dcal(\Ncal_n)$ and $u_\ms$ is an extension of $u$ which preserves the scaling degree.
\end{enumerate}
%
\end{proposition}


\begin{proof}
%%TODO PROOF
The proof of $a)$ and $b)$ is an application of %\cite[Theorem 3.2.3]{Hormander} 
to every $t^{(\alpha)}_k$. Furthermore, since the scaling degree of $r^{(\alpha)}$ is strictly smaller than $4(n-1)$, $r^{(\alpha)}$ possesses an unique extension towards $d_n$, cf. %\cite[Theorem 5.2]{Brunetti-Fredenhagen:2000}.%

\bigskip

In order to prove $c)$ we note that the original distribution $t^{(\alpha)}$ defined on $\Ncal_n\setminus d_n$ is weakly analytic and that an explicit construction of the weakly meromorphic extension $\dot{t}^{(\alpha)}$ to $\Ncal_n$ is provided by 
%\eqref{eq:expose-poles0}
, choosing for every component $t^{(\alpha)}_k$ a sufficiently large $p$ and using %\cite[Theorem 5.2]{Brunetti-Fredenhagen:2000}
. Hence, the poles of $\dot{t}^{(\alpha)}$ can only be supported on $d_n$. For this reason, after subtracting the principal part of the distribution the limit $\alpha \to 0$ can be safely taken. The such obtained distribution prior to considering the limit $\alpha \to 0$ coincides with $t^{(\alpha)}$ on $\Ncal_n\setminus d_n$ and the same holds in the limit $\alpha\to 0$. Consequently $t_\ms$ is an extension of $t$. Finally, $\sd(t_\ms)=\sd(t)$, because our assumptions and the above analysis imply that $\sd(\dot{t}^{(\alpha)})=\sd(t^{(\alpha)})=\Re(\delta_\alpha)$, $\sd(\pp(\dot{t}^{(\alpha)}))\le \Re(\delta_\alpha)$ and $\lim_{\alpha\to 0}\Re(\delta_\alpha) = \sd(t)$.
\end{proof}


We now discuss how equation \eqref{eq:expose_poles} can be used in order to regularize the most singular part of a distribution $u^{(\alpha)}$ which is known to be of the form
%
\begin{equation*}
u^{(\alpha)} = \sum_{k=0}^m u^{(\alpha)}_k + r^{(\alpha)} \ ,
\end{equation*}
%
but where the distributions $u^{(\alpha)}_k$ are not explicitly known. To this end, observe that equation \eqref{eq:expose_poles} implies
%
\begin{equation*}
\sm{ u^{(\alpha)} , \Esf_p \phi } = \sum_{k=0}^m \left( \prod_{j=0}^{p-1} (a_k+j+4(n-1)) \right) \sm{ u^{(\alpha)}_k , \phi } + \sm{ r^{(\alpha)} , \Esf_p \phi } \ .
\end{equation*}
%
Moreover, we may assume without loss of generality as in proposition \ref{prop:regularization} that the homogeneity degrees $a_k$ of $u^{(\alpha)}_k$ are of the form $a_k = -\delta_\alpha + k$ where $\Re(\delta_\alpha)$ is the scaling degree of $u^{(\alpha)}$. Consequently $u^{(\alpha)}_0$ is the contribution with the highest scaling degree which may be extracted by introducing the coefficients
%
\begin{equation*}
c_k = \prod_{j=0}^{p-1} \left(a_k+j+4(n-1)\right) 
\end{equation*}
%
and considering
%
\begin{equation}
\sm{ u^{(\alpha)} , \Esf_p \phi } - c_0 \sm{ u^{(\alpha)} , \phi } = \sum_{k=1}^m (c_k-c_0) \sm{ u^{(\alpha)}_k , \phi } + \sm{ r^{(\alpha)} , \Esf_p \phi } - c_0 \sm{ r^{(\alpha)}, \phi } \ ,
\label{eq:decrease_scaling_degree}
\end{equation}
%
where the distributions on the right hand side of \eqref{eq:decrease_scaling_degree} have a scaling degree smaller than $\Re(\delta_\alpha) = - \Re (a_0)$. Hence, although in general the distribution $u^{(\alpha)}$ does not scale homogeneously, equation \eqref{eq:expose_poles} still holds up to distributions with a lower scaling degree. Knowing the decreasing degree of homogeneity of the components in the expansion of $u^{(\alpha)}$, we may use a recursive procedure in order to expose the pole part of this distribution. In fact, the previous discussion straightforwardly implies the validity of the following proposition.


\begin{proposition}\label{prop:expose_poles}
We consider a distribution $u^{(\alpha)}$ with the properties assumed in proposition \ref{prop:regularization} and set
%
\begin{equation*}
U_0 = u^{(\alpha)} \ , \qquad U_{k+1} = c_k U_k - U_k \circ \Esf_{p_k} \ , \qquad 0 \leq k < m 
\end{equation*}
%
where $p_k$ are the smallest natural numbers chosen in such a way that 
%
\begin{equation*}
p_k+\Re(a_k)+4(n-1)>0 \ , \ \mbox{ and } \ \ c_k = \prod_{j=0}^{p_k-1} \left(a_k+j+4(n-1)\right) \ .
\end{equation*}
%
Then in order to expose the poles of $u^{(\alpha)}$, we may invert the recursive definition of $U_k$ obtaining
%
\begin{equation*}
u^{(\alpha)} = \frac{1}{c_0} \left( U_0\circ \Esf_{p_0} +  \frac{1}{c_1} \left( U_1 \circ \Esf_{p_1} +\dots + \frac{1}{c_n} \left( U_n \circ \Esf_{p_n} + U_{n+1} \right) \right) \right) \ .
\end{equation*}
%
\end{proposition}


In order to be able use the previous results for our purposes, we provide in the next proposition a criterion which is sufficient to ensure that a generic distribution can be decomposed into the sum of a homogeneous distribution and a remainder with lower scaling degree. We shall use this criterion in order to prove that the distributions $\Asf_\gamma^{(\alphabd)}$ defined in \eqref{eq:amplitude_sigma_reg} have the desired property.


\begin{proposition}\label{prop:set}
Let $\Ncal_n$ be a normal neighbourhood of the total diagonal $d_n$ and $u \in \Dcal^\prime(\Ncal_n)$ with scaling degree $s_1$ towards $d_n$. If there exists an $\delta\in\Cbb$ satisfying $\Re(\alpha)=-s_1$ such that
%
\begin{equation*}
u\circ(\Esf_1+\delta+4(n-1)) 
\end{equation*}
%
has scaling degree $s_2 < s_1$, then $u$ can be decomposed into the sum of a homogeneous distribution with degree $\delta$ and a remainder with scaling degree smaller than or equal to $s_2$.
\end{proposition}


\begin{proof}
%%TODO PROOF





We start by observing that, for every test function $f\in\Dcal(\Ncal_n)$, 
\[
F(\lambda, f) := \left\langle t, f_{1/\lambda} \right\rangle
\] 
is a continuous linear functional of $f$ which is smooth in $\lambda$ for $\lambda>0$. Moreover, since the scaling degree of $t$ is $s_1$, $\lambda^{a}F(\lambda, f)$ vanishes in the limit $\lambda \to 0$ for every $a > s_1$ and for every $f\in\Dcal(\Ncal_n)$.
Let us now consider 
\[
G(\lambda, f)  :=  \left\langle(-E_1+\alpha+4(n-1))t, f_{1/\lambda}\right\rangle. 
\]
$G(\lambda, \cdot)$ is again a family of distributions on $\Ncal_n$ which depends smoothly on $\lambda$ for positive $\lambda$. Furthermore, $\lambda^{a}G(\lambda, f)$ vanishes in the limit $\lambda\to 0$ for every $a > s_2$ and every $f\in\Dcal(N_n)$. Hence,  
$\lambda^{\alpha-1} G(\lambda, \cdot)$ tends to $0$ in $\Dcal^\prime(\Ncal_0)$ for $\lambda\to 0$ and, additionally, the Banach--Steinhaus theorem implies that 
\begin{equation}\label{eq-bound-dist}
\left| \lambda^{a} G(\lambda,f) \right|  \leq C \sum_{\alpha\leq k} |\partial^{\alpha} f|,
\end{equation}
for every $a>s_2$, uniformly for $f$ supported in a compact set $K\subset \Ncal_n$ and for suitable $C$ and $k$ which do not depend on $\lambda$.

After these preparatory considerations, we observe that $G$ and $F$ are related by the generalised Euler operator in the following way
\[
G(\lambda, f) =  \lambda^{-\alpha+1} \frac{d}{d\lambda} \lambda^{\alpha} F(\lambda, f).
\]
We can invert this relation to obtain
\[
F(\lambda, f) =  \frac{C(f)}{\lambda^{\alpha}} + \frac{1}{\lambda^\alpha} \int_0^\lambda  \tilde{\lambda}^{\alpha-1} G(\tilde{\lambda}, f) d\tilde{\lambda},
\]
where, $C(f)$ is a suitable constant which depends on $f$. We want to prove that $C(\cdot)$ is in fact a distribution. To this end, we note that, owing to the bound \eqref{eq-bound-dist}, the integral in $\tilde\lambda$ can be performed and the result of this integration is a distribution for every $\lambda > 0$ because $\Re(\alpha) > s_2$.  
This implies that 
%%TODO ??
\[
C(f) = F(1, f)  - \int_0^1  \tilde{\lambda}^{\alpha-1} G(\tilde{\lambda}, f) d\tilde{\lambda}
\]
is a distribution because it is a linear combination of distributions. By construction $F(1,f_{1/\lambda})=F(\lambda,f)$ and $C\circ (E_1+\alpha) =0$, hence $C$ is a homogeneous distribution of degree $\alpha$. By means of a direct computation we also find that the scaling degree of the remainder $F(1,f)-C(f)$ is smaller than or equal to the scaling degree of $G$ which is $s_2$.






\end{proof}


We are going to analyse the action of the generalized Euler operators $\Esf_p$ appearing in \eqref{eq:expose_poles} on test functions. In fact, we shall see that $\Esf_p$ corresponds to a particular geometric partial differential operator. To this end, we observe that
%
\begin{equation*}
\Esf_p = (\Esf_1 + (p-1)) \Esf_{p-1} \ . 
\end{equation*}
%
Hence, knowing the differential form of the generalized Euler operator $\Esf_1$, it is possible to construct recursively every $\Esf_p$. Regarding the differential form of $\Esf_1$, we note that it can be written in terms of the geodesic distance and the van Vleck--Morette determinant\footnote{Recall that the square root of the van Vleck--Morette determinant coincides with the Hadamard coefficient $\usf$ appearing in \eqref{eq:hadamard_rep}.} $\usf^2$  as
%
\begin{equation*}
\Esf_1 \phi(x_1 , \dots , x_n) = \sum_{j=2}^n \bigg( \sigma^a(x_j) \nabla^{x_j}_a  + 2  \sigma^a(x_j) \nabla^{x_j}_a  \log\left(\usf(x_j,x_1)\right) \bigg) \phi(x_1 , \dots , x_n) \ , 
\end{equation*}
%
where $\nabla^{x_j}_a$ indicates the $a$--th component of the covariant derivative computed in $x_j$ and $\sigma^a(x_j) = {\nabla^{x_j}}^a \sigma(x_1,x_j)$. Considering the adjoint $\Esf^\dagger_p$ of $\Esf_p$, we have $u \circ \Esf_p = \Esf^\dagger_p u$ where, using the relation
\begin{equation*}
\Box \sigma + 2 \sigma^a \ \nabla_a\left( \log (u) \right) = 4 \ , 
\end{equation*}
%
we find for $p=1$
%
\begin{eqnarray}
\Esf_1^\dagger  u(x_1,\dots,x_n) &=& \sum_{j=2}^n \Bigg( - \nabla^{x_j}_a \sigma^a(x_j) + 2 \sigma^a(x_j) \bigg( \nabla^{x_j}_a \log\left(u(x_j,x_1)\right) \bigg) \Bigg) u(x_1,\dots,x_n) \nonumber \\
&=& -\left( 4(n-1) + \sum_{j=2}^n \sigma^a(x_j) \nabla^{x_j}_a \right) u(x_1,\dots,x_n) \ .
\label{eq:euler_operator}
\end{eqnarray}
%
We finally observe that the recursive identity for $\Esf_p$ implies that also $\Esf^\dagger_p$ can be constructed recursively starting from $\Esf^\dagger_1$ as 
%
\begin{equation*}
\Esf_p^\dagger = \Esf_{p-1}^\dagger \left(\Esf_1^\dagger-(p-1)\right) \ . 
\end{equation*}
%
Therefore knowing $\Esf_1^\dagger$ gives us every differential operator $\Esf_p$.


\bigskip


The next step in the strategy outlined in section \ref{p:PROGRAM} is to extend the distributions $\Asf^{(\alphabd)}_\gamma$, which are à priori defined only outside of the union of all partial diagonals $D_n$ in a normal neighbourhood $\Ncal$ of the total diagonal to $D_n \cap \Ncal$ and show that this extension is weakly meromorphic in $\alpha_I$ upon setting $\alpha_{ij}=\alpha_I$ for all $i,j\in I$. We shall prove this by using the genealized Euler operator studied in the present section. It shall be done in section \ref{p:EULER_OP_PRACTICE}.


%----------------------------------------------------------------------------%
\section{The minimal subtraction scheme}
%----------------------------------------------------------------------------%

We presented in section \ref{p:ANOTHER_APPROACH} the regularization procedure we use in this work, then in section %\ref{p:EXT_DISTRIB}
we recalled and introduced definitions and properties in the general case to construct an extension of an ill defined distribution. Because as we know the problem we have is using the Bogoliubov formula \eqref{eq:bogoliubov} for constructing interacting fields perturbatively, it gives terms of the $S$-S, which is the time--ordered exponential \eqref{eq:S_matrix}. Unfortunately, the time ordered product defined in terms of a ``deformation'' written by means of a Feynman propagator $\Delta_\fsf$ is well defined only on regular functionals because the singularities present in $\Delta_\fsf$ forbid their application to more general functionals. Therefore we shall here look at the regularization sheme applied to distributions build from the Feynman propagator.


%----------------------------------------------------------------------------%
\subsection{Regularization of the Feynman propagator}
\label{p:REG_FEYNMAN_PROP}
%----------------------------------------------------------------------------%


Following the plan outlined in section \ref{p:PROGRAM}, we would like to define an analytic regularization $\Hsf^{(\alpha)}_\fsf$ of $\Hsf_\fsf$ given by \eqref{eq:hadamard_rep}. To this end we start our analysis by constructing the distribution 
%
\begin{equation}
\frac{1}{\sigma_\fsf^{1+\alpha}}
\label{eq:sigma_reg}
\end{equation}
%
in $\Mcal^2$ for $\alpha \in \Cbb \setminus \Nbb$. We shall make use of scaling properties of
\eqref{eq:sigma_reg} and the induced quantities $\Asf^{(\alphabd)}_\gamma$ \eqref{eq:amplitude_sigma_reg} with respect to the geometric scaling transformation \eqref{eq:geo_scaling_transfo}.


Therefore we introduce the distributions we shall use as building blocks for the construction of regularized Feynman propagators on lorentzian manifolds. The construction we are going to present is in some way similar to the extension of Riesz distributions to curved spaces presented in \cite{baer_wave_2008}. In particular, here we shall discuss the boundary value of 
%
\begin{equation}
\frac{1}{\sigma_\fsf^\alpha} = \frac{1}{(\sigma+i\epsilon)^{\alpha}} 
\label{eq:inverse_sigma_f}
\end{equation}
%
for $\epsilon\to0$ while ordinary Riesz distributions are related to the antisymmetric part of a different boundary value of the functions $1/\sigma^\alpha$.


\begin{proposition}\label{prop:sigma_1}
Consider a normal neighbourhood $\Ncal_2 \subset\Mcal^2$ of $d_2$ and the following expression for $\alpha \in \Cbb$ and $\phi \in \Dcal(\Ncal_2)$
%
\begin{equation*}
\sm{ \frac{1}{\sigma^\alpha_\fsf} , \phi } = \lim_{\epsilon \to 0^+ } \ \int_{\Mcal^2} \dsf x \ \dsf y \ \phi(x,y) \ \frac{1}{(\sigma(x,y)+i\epsilon)^{\alpha}} \ .
\end{equation*}
%
Then the following statements hold.
%
\begin{enumerate}
%
\item The distribution \eqref{eq:inverse_sigma_f} restricted to $\Dcal(\Ncal_2 \setminus d_2)$ is a distribution which is weakly analytic in $\alpha$.
%
\item The distribution \eqref{eq:inverse_sigma_f} is homogeneous of degree $-2\alpha$ with respect to transformations of the form \eqref{eq:geo_scaling_transfo} for $\phi \in \Dcal(\Ncal_2\setminus d_2)$.
%
\item The distribution \eqref{eq:inverse_sigma_f} is well defined as a distribution on $\Ncal_2$ for $2\alpha-4 \notin \Nbb$. Furthermore
%
\begin{equation*}
\mbox{for all } \ \phi \in \Dcal(\Ncal_2) \ , \ \ \sm{ \frac{1}{\sigma^\alpha_\fsf} , \phi } \ \mbox{ is analytic}
\end{equation*}
%
for $2\alpha-4\notin \Nbb$ and meromorphic for $\alpha \in \Cbb$ with simple poles at $2\alpha-4\in \mathbb{N}$. 
%
\end{enumerate}
%
\end{proposition}


\begin{proof}
%%TODO PROOF
$a)$ For every $x \in \Mcal$ we fix a normal coordinate system $\xi_x:y\to{\mathbb{R}^4}$ in order to parametrise points $y$ in a normal neighbourhood of $x$. Consequently, on $\Ncal_2$ the squared geodesic distance divided by $2$ can be easily expressed as
\[
\sigma(x,y)= \frac{1}{2}\eta(\xi_x(y),\xi_x(y)) = \frac{1}{2}\xi_x^{a}{\xi_x}_{a}\,,
\]
where $\eta$ is the standard Minkowski metric given in Cartesian coordinates. Furthermore\[%\label{eq:normal-integral}
\left\langle \frac{1}{\sigma^\alpha_F}, f \right\rangle = 
\lim_{\epsilon\to0^+ } \int_\Mcal \int_{\mathbb{R}^4}  \frac{2^{\alpha}}{(\xi_x^a{\xi_{x}}_a+i\epsilon)^{\alpha}} f(x,\xi_x) \sqrt{g(\xi_x)}\; d^4\xi_x\; d\mu_g (x).
\]which is well defined for $f\in\Dcal(\Ncal_2)$.Observe that $1/(\xi^{a}{\xi}_{a})^\alpha$  for $\xi^{a}\in \{z\in \mathbb{C}^4 \,|\,\Im(z) \in V^\pm \}$, where $V^\pm$ is the forward or past light cone with respect to the Minkowski metric, is  analytic both in $\xi$ and $\alpha$.  Furthermore, in the limit $\epsilon\to0^+$,  $1/(\xi^{a}{\xi}_{a} + i \epsilon)^\alpha$ can be seen as the boundary value of that analytic function. Since this function grows at most polynomially for large $1/\Im(\xi^a\xi_a)$ its boundary value defines a distribution, see e.g. %\cite[Theorem 3.1.15]{Hormander}
. The analytic dependence on $\alpha$ is weakly preserved in the limit $\epsilon\to0^+$, and thus the resulting distribution is weakly analytic. \par%
%
\bigskip
%
$b)$  The transformation defined in %\eqref{eq:n-dim-scaling}
acts on points parametrised by normal coordinates as $\xi \to \lambda \xi $. Furthermore, $1/(\xi^{a}{\xi}_{a})^\alpha$ on $A \subset \mathbb{C}^4$ is homogenous of degree $2\alpha$ with respect to the transformation $\xi\to \lambda \xi$. The statement follows from this observation, taking into account %\eqref{eq:n-dim-scaling} 
and %\eqref{eq:normal-integral}
.

\bigskip

$c)$ Theorem 3.2.3 in %\cite{Hormander} 
ensures that the distribution $\dot{t}\doteq1/(\xi^{a}{\xi}_{a})^\alpha\in\Dcal^\prime(\Rbb^4\setminus  0)$ has a unique extension $t$ to $0$ preserving the degree of homogeneity for every $2\alpha-4 \notin \Nbb$. 
Hence, $\mathbb{I}\otimes t$ defines a distribution on $\Mcal \times \Rbb^4$.
Finally, notice that there exists a neighbourhood $\Ocal$ of the diagonal in $\Mcal_2$
where $f(x,\xi_x) \sqrt{g(\xi_x)}$ is a smooth and compactly supported function for every $f\in C^\infty_0(\Ocal)$. Consequently, the statement follows by choosing a partition of unity adapted to $\Ocal$.








\end{proof}


The proposition \ref{prop:sigma_1} guarantees that \eqref{eq:sigma_reg} is weakly meromorphic in $\alpha$ with simple poles at $2\alpha-4\in\Nbb$. This property is preserved under taking linear combinations and multiplication by smooth functions. Consequently, the analytically regularized Feynman propagator $\Hsf^{(\alpha)}_\fsf$ defined by \eqref{eq:hadamard_rep_reg} is well defined on a normal neighbourhood of the total diagonal and weakly meromorphic in $\alpha$. 


The singularities of $\Hsf^{(\alpha)}_\fsf$ \eqref{eq:hadamard_rep_reg} constains in its wave front set are constained in the wave front set of $\Hsf_\fsf$ \eqref{eq:hadamard_rep}. It guarantees that the regularization does not add new divergences. The scaling degree of $\Hsf^{(\alpha)}_\fsf$  which characterize the existence and uniqueness of an extension of $\Hsf^{(\alpha)}_\fsf$ is infinite when the real part of the complex parameter $\alpha$ tends to $\infty$.


\begin{proposition}
Consider a normal neighbourhood $\Ncal_2$ of the diagonal $d_2\in\Mcal^2$. The following statements hold for the analytically continued Feynman propagator $\Hsf^{(\alpha)}_\fsf\in\Dcal^\prime(\Ncal_2)$ defined in \eqref{eq:hadamard_rep_reg}.
%
\begin{eqnarray*}
&\mbox{1.}& \underset{\alpha \to 0}{\lim} \ \Hsf^{(\alpha)}_\fsf = \Hsf_\fsf \ . \\
&\mbox{2.}& \WF\left(\Hsf^{(\alpha)}_\fsf\right) \subset \WF\left(\Hsf_\fsf\right) \ . \\
&\mbox{3.}& \sd\left(\Hsf^{(\alpha)}_\fsf\right) \to - \infty \ \mbox{ when } \ \Re\left(\alpha\right) \to \infty \ .  
\end{eqnarray*}
%
\end{proposition}


\begin{proof}


%%TODO PROOF


The proof of this proposition follows from the properties of $\sigma_F^{1+\alpha}$ obtained in Proposition %\ref{pr:sigma-1}
. In particular, $a)$ and $c)$ can be directly obtained from the weak analyticity, while $b)$ follows from the fact that the distribution $1/\sigma_F^\alpha$ is well defined on $\Ncal_2\setminus d_2$ where it coincides either with $1/\sigma_+^\alpha$ or with $1/\sigma_-^\alpha$. 
In order to analyse the wave front sets of $1/\sigma_\pm^\alpha$, we pass to a normal coordinate system and obtain $1/\sigma_\pm^\alpha=2/(\xi^a\xi_a\pm i\epsilon \xi^0)^\alpha$. This distribution can be extended to a tempered  distribution for every $\alpha$ and thus its Fourier transform can be directly computed. One finds that for $1/\sigma_\pm^\alpha$, only the null future/past directed directions do not decay rapidly, consequently $H_F^{(\alpha)}$ restricted to $\Ncal_2\setminus d_2$ has $\WF(H^{(\alpha)}_F)\subset \WF(H_F)$. Finally, we observe that the extension of $H^{(\alpha)}_F$ to $\Ncal_2$ may possess further singularities supported on the diagonal with singular directions orthogonal to $d_2$. Hence, $\WF(H^{(\alpha)}_F)\subset \WF(H_F)$ still holds for $H_F^{(\alpha)}\in\Dcal^\prime(\Ncal_2)$.



\end{proof}


We are now able to discuss the analytical regularization $\tsf^{(\alphabd)}_\gamma$ \eqref{eq:kernel_reg} of the distributions $\tsf_\gamma$ given in \eqref{eq:kernel} which appear in the graph expansion \eqref{eq:time_ordered_prod_graph} of the time ordered products $\Tcal_n$ \eqref{eq:time_ordered_op}. 


Owing to the form of $\Hsf^{(\alpha)}_\fsf$ given in \eqref{eq:hadamard_rep_reg} the relevant distributions which need to be discussed are $\Asf^{(\alphabd)}_\gamma$ introduced in \eqref{eq:amplitude_sigma_reg}. We analysed \eqref{eq:amplitude_sigma_reg} in the following proposition.


\begin{proposition}\label{prop:amplitude_sigma_prop_analyt}
Let consider $\Ncal$ a normal neighbourhood of the union of all partial diagonals $D_n$. The operation
%
\begin{equation*}
\sm{ \Asf_\gamma^{(\alphabd)} , \phi } = \int_{\Mcal^n} \ \dsf x_1 \ \dots \ \dsf x_n \ \phi(\mathbf{x}) \prod_{(i,j)\in\gamma} \frac{1}{\sigma_{ij}^{\ell_{ij}(1+ \alpha_{ij})}}  \ , 
\end{equation*}
%
for $\sigma_{ij}=\sigma_\fsf(x_i,x_j)$ and $\phi\in \Dcal(\Mcal^n\setminus D_n\cap \Ncal)$, has the following properties.
%
\begin{enumerate}
%
\item $\Asf_\gamma^{(\alphabd)}$ is a distribution on $\Mcal^n\setminus D_n\cap \Ncal$.
%
\item $\sm{ \Asf_\gamma^{(\alphabd)} , \phi }$ is a continuous function for $\alphabd \in \Cbb^{n(n-1)/2}$.
%
\item $\sm{ \Asf_\gamma^{(\alphabd)} , \phi }$ is analytic for every $\alpha_{ij}$ with $i<j$ and thus a multivariate analytic function.
%
\end{enumerate}
%
\end{proposition}


\begin{proof}


%%TODO PROOF






$a)$ The domain $\Mcal^n\setminus D_n\cap \Ncal$ is a disjoint union of connected components. On every connected component $\mathcal{C}$ $\sigma_F(x_i,x_j)$ equals either $\sigma_+(x_i,x_j)$ or $\sigma_+(x_j,x_i)$ depending on the causal relation between $x_i$ and $x_j$ which is fixed in $\mathcal{C}$.
Hence, on $\mathcal{C}$, the wave front set of $\sigma_F(x_i,x_j)^{-1}$ is contained either in $\mathcal{V}_+$ or $\mathcal{V}_-$, where $\mathcal{V}_{+/-} = \{(x,x',k,k') \in T^*\Mcal^2\setminus 0, (x,k)\sim (x',-k'),  k \triangleleft / \triangleright 0  \}$. Consequently, $\sigma_F(x_i,x_j)^{-1}$ satisfies the Hadamard condition up to a permutation of the arguments. The very same holds for the distributions $\sigma_F(x_i,x_j)^{l_{ij}(1+\alpha_{ij})}$ for every $l_{ij}$ and every $\alpha_{ij}$ which have been discussed in Proposition %\ref{pr:sigma-1}
.

Owing to the form of their wave front set, the pointwise products of these distributions present in $t_\Gamma^{(\alphabd)}$ are well--defined because the H\"ormander--criterion for multiplication of distributions is satisfied. In fact, up to some fixed permutation of the arguments $(x_1,\dots x_n)$, $t_\Gamma^{(\alphabd)}$ satisfies the micro local spectrum condition introduced in %\cite{bfk:1996}
. 
Hence $t_\Gamma^{(\alphabd)}$ is a well--defined distribution on every connected component $\mathcal{C}$ of $\Mcal^n\setminus D_n\cap \Ncal$ and thus it is well--defined also on $\Mcal^n\setminus D_n\cap \Ncal$. 

\bigskip

$b)$ 
In order to check continuity for $\alphabd= \{\alpha_{ij}\}_{i<j}\in \mathbb{R}^{n(n-1)/2}$ in a fixed point $\overline{\alphabd}$ we may analyse the distribution on a fixed connected component $\mathcal{C}$ of the domain of $t_\Gamma^{(\alphabd)}$ and factorize the distribution in two parts. 
In fact, due to the wave front set of $t_\Gamma^{(\alphabd)}$ on $\mathcal{C}$ the factorisation $t_\Gamma^{(\alphabd)} =t_\Gamma^{(\overline{\alphabd})} \cdot \tau_\Gamma^{(\betabd)}$ is unique where the integral kernel of $\tau_\Gamma^{(\betabd)}$ is $\prod_{1\leq i < j \leq n } \frac{1}{\sigma_F(x_i,x_j)^{\beta_{ij}}}$.
For $\betabd$ in a sufficiently small neighbourhood of $0$, $\tau_\Gamma^{(\betabd)}$ is an integrable function which is differentiable for $\betabd=0$ as can be obtained by dominated convergence. Finally, the continuity is preserved by pointwise multiplication with $t_\Gamma^{(\overline{\alphabd})}$.


\bigskip


$c)$ 
For an arbitrary but fixed pair of indices $i,j$, $\alpha_{ij}$ appears in the product displayed in %\eqref{eq:prod-sigma} 
as $1/\sigma_F(x_1,x_j)^{l_{ij}(1+\alpha_{ij})}$ and we have already analysed the analyticity property of such a distribution in Proposition %\ref{pr:sigma-1}
. 
We shall thus interpret $t_\Gamma^{(\alphabd)}$ as a composition of distributions, namely as $1/\sigma_F^{\alpha_{ij}}\circ z $
 where $z$ is an operator which maps $\Dcal(\Mcal^{n}\setminus D_{n}\cap \Ncal)$ to $ \Dcal^\prime(\Mcal^{2}\setminus D_2\cap \Ncal_2)$ for a suitable $\Ncal_2\supset D_2=d_2$. The $\epsilon$--regularized integral kernel of $z$ corresponds to the product present in %\eqref{eq:prod-sigma}
 with the factor  $1/\sigma_F^{\alpha_{ij}}$ removed. Because of the singular structure of $z$, for every $f\in \Dcal(\Mcal^{n}\setminus D_{n}\cap \Ncal)$, $\langle z,f\rangle$ is in fact a compactly supported smooth function supported on $\Mcal^{2}\setminus D_2\cap \Ncal_2$. Hence, the analysis of its composition with $1/\sigma_F^{\alpha_{ij}}$ is straightforward. These considerations imply separate analyticity of $t_\Gamma^{(\alphabd)}$ in each $\alpha_{ij}$ whereas joint analyticity follows from the continuity proved in $b)$.
\end{proof}


The weakly meromorphicity of $\Asf_\gamma^{(\alphabd)}$ is guaranted by means of proposition \ref{prop:amplitude_sigma_prop_analyt}. It allows to appply the machinery developed using the genearlized Euler operator to $\Asf_\gamma^{(\alphabd)}$, in order to apply the minimal subtraction procedure, i.e. to be able to explicitly subtract the principal part.


%----------------------------------------------------------------------------%
\subsection{The genealized Euler operator in practice}
\label{p:EULER_OP_PRACTICE}
%----------------------------------------------------------------------------%


The next step is to extend the distributions $\Asf^{(\alphabd)}_\gamma$, which are defined only outside of the union of all partial diagonals $D_n$ in a normal neighbourhood $\Ncal$ of the total diagonal to $D_n \cap \Ncal$, and weakly meromorphic with respect to $\alpha_I$ upon setting $\alpha_{ij}=\alpha_I$ for all $i,j\in I\subset \{1,\ldots,n\}$. 


In section \ref{p:DIFFERENTIAL_EULER} we defined the genralized Euler operator, and give a recursive way to obtain the differential form of every Euler operators. We also analysed the action of $\Esf_p$ on test functions. Here we shall look at a particular case. 


The distribution we shall consider $\Asf^{(\alphabd)}_\gamma$ \eqref{eq:amplitude_sigma_reg} is of the form
%
\begin{equation*}
\Asf_\gamma^{(\alphabd)}=\prod_{(i,j)\in\gamma} \frac{1}{\sigma_{ij}^{\ell_{ij}(1+ \alpha_{ij})}} 
\end{equation*}
%
and has scaling degree 
%
\begin{equation*}
\sd(\Asf_\gamma^{(\alphabd)}) = \sum_{(i,j)\in\gamma} 2 \ell_{ij}\left(1+ \Re(\alpha_{ij})\right) 
\end{equation*}
%
towards the thin diagonal $d_n$. When we apply $\Esf^\dagger_1$, introduced in \eqref{eq:euler_operator}, to distribution $\Asf_\gamma^{(\alphabd)}$ , the result is a term proportional to $\Asf_\gamma^{(\alphabd)}$ plus a remainder which has lower scaling degree as foreseen in \eqref{eq:decrease_scaling_degree}. Hence proposition \ref{prop:set} implies that $\Asf_\gamma^{(\alphabd)}$ can be written as a homogeneous distribution plus a remainder with lower scaling degree. If the scaling degree of the remainder is not sufficiently low, we reiterate the procedure in order to obtain a full almost homogeneous expansion of the desired form.


In order to analyse this issue we shall only consider the relevant differential operator on $\Mcal^n$ appearing in $\Esf_1^\dagger$ \eqref{eq:euler_operator},
%
\begin{equation*}
\Esf_1^\dagger u = - \left( 4(n-1) + \rho \right) u \ , \quad \mbox{with } \quad 
\rho = - \sum_{j=2}^n \sigma^a(x_j) \nabla^{x_j}_a \ .    
\end{equation*}


We start by looking at the ``three dimensional case'', i.e.  $\Mcal^3$. We analyse the action of $\rho$ on $\sigma(x_2,x_3)$ for $x_2,x_3$ in a normal neighbourhood $\Ncal_{x_1}$ of the point $x_1$.


\begin{lemma}\label{lem:rho_over_squared}
Let $\Ncal_{x_1}$ be a normal neighbourhood of the point $x_1$ and let $x_2,x_3 \in \Ncal_{x_1}$. Then,
%
\begin{equation*}
\rho \sigma(x_2,x_3) = 2\sigma(x_2,x_3) + G(x_1,x_2,x_3) \ ,
\end{equation*}
%
where $G$ is a smooth function which vanishes in the limit $x_2,x_3 \to x_1$ as a monomial of order $4$ in the normal coordinates of $x_2$ and $x_3$ centred in $x_1$. 
\end{lemma}


\begin{proof}
%%TODO PROOF





Using the notation in the proof of Proposition %\ref{pr:sigma-1}
we write the action of $\rho$ on $\sigma_{23}:=\sigma(x_2,x_3)$ as
\[
\rho \sigma_{23}  =    \xi_a(x_2) \sigma^a_{23} + \xi_{b^\prime}(x_3)\sigma^{b^\prime}_{23}\,.
\]
Recall that $\sigma_{23}^a$ is the covector in $T^*_{x_2}M$ cotangent to the unique geodesic joining $x_2$ and $x_3$, that $-\sigma^{b'}_{23}$ is equal to the parallel transport of $\sigma^a_{23}$ from $x_2$ to $x_3$ along the geodesic $\gamma$ joining the two points, and that $\xi^c(x_i):=\sigma^c(x_1,x_i)$.


Let us parametrise the image of $\gamma$ with an affine parameter $\lambda$ such that $x(0) = x_2$ and $x(1) = x_3$. In order to simplify the notation, we indicate by $t(\lambda)$ the tangent vector of the geodesic in $x(\lambda)$. As argued before, we have
\[
t^a(0)=\sigma^a_{23},\qquad \text{and}\qquad t^{b^\prime}(1)=-\sigma^{b^\prime}_{23}.
\]   
Consequently,
\begin{eqnarray*}
\rho \sigma_{23}
&=& \xi^a t_a (0) - \xi^b t_b(1) 
= - \int_{0}^{1} \frac{d}{d\lambda} (\xi^a t_a)(\lambda) d\lambda \\
&=& - \int_{0}^{1} t^a \nabla_t  \xi_a   d\lambda  
= \int_{0}^{1} t^a t^b \sigma_{ab}(x(\lambda),x_1) d\lambda\,,
\end{eqnarray*}
where $\sigma_{ab} := \nabla_a\nabla_b \sigma$. If we now consider the covariant Taylor expansion of $\sigma_{ab}(x(\lambda),x_1)$ around $x(\lambda)$ (see e.g. %\cite{Poisson:2011nh}
), we find that $E_{ab}(x,x_1):=\sigma_{ab}(x,x_1) - g_{ab}(x) $ is a smooth function that vanishes for $x\to x_1$ as $O(\sigma(x,x_1))$, hence
\[
\rho\sigma_{23}  = \int_{0}^{1} t^a t^b g_{ab}(x(\lambda))    d\lambda +  \int_{0}^{1} t^a t^b E_{ab}(x(\lambda),x_1)    d\lambda  = 2\sigma(x_2,x_3) + G(x_1,x_2,x_3)\,,
\]
where the remainder is smooth because of the smoothness of the metric $g$ and can be further expanded as 
\begin{eqnarray}\label{eq:remainder-sigma}
G(x_1,x_2,x_3) &=& \int_{0}^{1} t^a(\lambda) t^b(\lambda) \left(\sigma_{ab}(x(\lambda),x_1) - g_{ab}(x(\lambda))\right) d\lambda\\ &=&  
\int_{0}^{1} t^a(\lambda) t^b(\lambda) t^c(\lambda) t^d(\lambda) R_{acbd}(x(\lambda))d\lambda + \dots = O(|\xi(x_2)|^4+|\xi(x_3)|^4)\,,\notag
\end{eqnarray}
where the absolute value of the normal coordinates $|\xi(x_i)|$ of $x_i$, $i=2,3$ is intended in the Euclidean sense.




\end{proof}


We are now in position to analyse the action of $\rho$ on the distribution $\Asf_\gamma^{(\alphabd)}$ introduced in \eqref{eq:amplitude_sigma_reg}. The distribution $\Asf_\gamma^{(\alphabd)}$ can be written as a sum of homogeneous distributions plus a remainder term with lower sacling degree which can be directly extended towards the total diagonal. Let us present precisely this result in the following proposition.


\begin{proposition}\label{prop:almost_homo}
The distribution $\Asf_\gamma^{(\alphabd)}$ introduced in \eqref{eq:amplitude_sigma_reg} can be written as a sum of homogeneous distributions with respect to scaling towards the total diagonal $d_n$ plus a remainder.
%
\begin{equation*}
\Asf_\gamma^{(\alphabd)} = \sum_k \ \ \Asf_{\gamma,k}^{(\alphabd)} + \left(r_\gamma^{(\alphabd)}\right)_k \ .
\end{equation*}
%
The degrees of homogeneity of the homogeneous distributions $\Asf_{\gamma,k}^{(\alphabd)}$ are contained in the following set
%
\begin{equation*}
\left\{k-\sum_{(i,j)\in\gamma} 2 \ell_{ij}(1+ \alpha_{ij}) \right\} \ ,
\end{equation*}
%
with $k\in \mathbb{N}\cup \{0\}$.
\end{proposition}


\begin{proof}
%%TODO PROOF







We perform this analysis with $\epsilon$ in $\sigma_F$ taken to be strictly positive. We start by applying $\rho$ given in %\eqref{eq:rho}
to $t_\Gamma^{(\alphabd)}$. Thanks to the results stated in Lemma %\ref{le:rho-over-squared}
we have
\[
\rho t_\Gamma^{(\alphabd)}  =    C   t_\Gamma^{(\alphabd)}    + r^{(\alphabd)}_\Gamma,
\]
where the constant $C$ is
\[
C= -\sum_{1\leq i<j\leq n} 2 l_{ij}(1+ \alpha_{ij}).
\]
Furthermore, Lemma %\ref{le:rho-over-squared} 
and in particular %\eqref{eq:remainder-sigma}
implies that the remainder $r^{(\alphabd)}_\Gamma$ has a scaling degree towards $d_n$ which is lower than the one of $t_\Gamma^{(\alphabd)}$ by at least two, 
\begin{equation}\label{eq:sd-tgamma}
\sd(r^{(\alphabd)}_\Gamma) \leq \sd(t_\Gamma^{(\alphabd)})-2 = \sum_{1\leq i<j\leq n} 2 l_{ij}(1+ \Re(\alpha_{ij})) -2\,.
\end{equation}
Proposition %\ref{pr:set}
then implies that the distribution $t_\Gamma^{(\alphabd)}$ can be written as a homogeneous distribution of degree $C$  
plus a remainder with lower scaling degree.

In order to finalise the proof we need to control the recursive application of $\rho$, therefore we discuss the application of $\rho$ on $\rho^nt_\Gamma^{(\alphabd)}$ for an arbitrary $n$. 
Let us start with $n=1$. In this case, we observe that the relevant contribution is the one given by the remainder $\rho r^{(\alphabd)}_\Gamma$, which reads 
\[
r^{(\alphabd)}_\Gamma = \sum_{1\leq i < j \leq n} l_{ij}(1+ \alpha_{ij}) \frac{G(x_1,x_i,x_j)}{\sigma_F(x_i,x_j)}t_\Gamma^{(\alphabd)}.
%    \prod_{1\leq i < j \leq n } \frac{1}{\sigma_F(x_i,x_j)^{l_{ij}(1+ \alpha_{ij})}} 
\]
Note that for every $i<j$, $\sigma_F(x_i,x_j)t_\Gamma^{(\alphabd)}$ has the same structure like $t_\Gamma^{(\alphabd)}$, but the scaling degree $\sd(t_\Gamma^{(\alphabd)}) +2$, whereas $G(x_1,x_i,x_j)$ defined in \eqref{eq:remainder-sigma} is a smooth function whose Taylor expansion for $x_i, x_j$ around $x_1$ starts with components of order $4$. Hence, if we apply $\rho$ to $r^{(\alphabd)}_\Gamma$ we obtain
a constant multiple of $r^{(\alphabd)}_\Gamma$ plus a remainder which has scaling degree lower or equal to $\sd(r^{(\alphabd)}_\Gamma)-1$, where the difference with respect to \eqref{eq:sd-tgamma} stems from the fact that $G$ can be expanded as a polynomial in $\sigma_a(x_i)$ whose lowest components are monomials of degree $4$ multiplied by curvature tensors. These monomials are homogeneous and thus contribute to the degree of homogeneity of $\rho r^{(\alphabd)}_\Gamma$, while the contributions in $G$ with degree higher or equal to five influence the scaling degree of the remainder.
Repeating this analysis for a generic $n$, we find that similar results hold when $\rho$ is applied recursively to the remainder.

Consequently, an iterated application of Proposition %\ref{pr:set}
implies that the distribution $t_\Gamma^{(\alphabd)}$ can be written as a finite sum of homogeneous distributions plus a remainder. Furthermore, since the scaling degree of these distributions is always finite, the degree of homogeneity of these components is finite as well. 







\end{proof}


We can use proposition \ref{prop:almost_homo} in conjunction with the propositions \ref{prop:regularization} and \ref{prop:expose_poles} in order to extend the distributions $\Asf_\gamma^{(\alphabd)}$ in a unique and weakly meromorphic fashion to a normal neighbourhood of the union of all partial diagonal $D_n$, and to compute the relevant pole part of this extension as used in the forest formula, cf. thorem \ref{theo:renorm_t_prod_ms_forest}. 


To this avail, we stress that proposition \ref{prop:almost_homo} holds in particular for any subgraph $\gamma_I$, with $I\subset\{1,\dots,n\}$ of $\gamma$ and the corresponding distribution $\Asf_{\gamma_I}^{(\alphabd)}$ which is obtained by omitting all factors in $\Asf_{\gamma}^{(\alphabd)}$ which correspond to edges not contained in $\gamma_I$. Finally, the recursive structure of the forest formula in theorem \ref{theo:renorm_t_prod_ms_forest} implies that we are not dealing only with 
expressions of the form $\left.\Asf_{\gamma_I}^{(\alphabd)}\right|_{\alpha_{ij}=\alpha_I}, \  \forall i,j\in I$, but also with expressions which are of this form up to a subtraction of their principal part. However, our above analysis and in particular the discussion in the proof of proposition \ref{prop:regularization} implies that the propositions \ref{prop:almost_homo} and \ref{prop:expose_poles} also hold in this case.


%----------------------------------------------------------------------------%
\section{Properties}
%----------------------------------------------------------------------------%


We conclude the general analysis of the regularization scheme introduced in this work by demonstrating that this scheme satisfies (up to one property we shall mention at the end of this section) all axioms of \cite{hollands_local_2001,hollands_existence_2002,hollands_conservation_2005} which, as argued in these works, any physically meaningful scheme to regularize time ordered products should satisfy. We refer to these works for a detailed formulation and discussion of these axioms. In addition to showing these properties of the scheme, we also argue that it preserves invariance under any spacetime isometries present.


\begin{proposition}\label{prop:properties_scheme} 
The time ordered product $\Tcal_n$ defined by means of the relation in theorem \ref{theo:renorm_t_prod_ms_forest}, where the quantities appearing in this formula are defined by means of \eqref{eq:pp_op}, \eqref{eq:kernel_reg}, \eqref{eq:hadamard_rep_reg} have the following properties.
%
\begin{enumerate}
%
\item $\Tcal_n$ is symmetric and satisfies the causal factorisation condition.
%
\item $\Tcal_n$ is unitary.
%
\item $\Tcal_n$ is local and covariant.
%
\item $\Tcal_n$ satisfies the microlocal spectrum condition.
%
\item $\Tcal_n$ is $\phi$ independent.
%
\item $\Tcal_n$ satisfies the Leibniz rule.
%
\item $\Tcal_n$ satisfies the Principle of Perturbative Agreement for perturbations of the generalised mass term $\mu$ in the free Klein Gordon equation
%
\begin{equation*}
\Psf \phi = \left( - \Box + \mu \right) \phi = 0 \ . 
\end{equation*}
%
\item If the spacetime $\Mcal$ has non trivial isometries and if the Feynman propagator $\Hsf_\fsf$ is chosen such as to be invariant under these isometries, then $\Tcal_n$ is invariant under these isometries as well.
%
\end{enumerate}
%
\end{proposition}


\begin{proof}
%%TODO PROOF





$a)$ holds because we constructed the renormalised time--ordered product by means of the forest formula %\eqref{eq:forest-formula}
and because, as implied by Proposition %\ref{pr:regularization}
, all counterterms subtracted in the forest formula are local.

$b)$ Unitarity holds because the operation of extracting the relevant principal part of a regularized amplitude $\tau^{(\alphabd)}_\Gamma$ commutes with complex conjugation (even if $\alphabd$ is not real).

$c)$ The regularized amplitudes $\tau^{(\alphabd)}_\Gamma$ satisfy locality and covariance. Upon setting $\alpha_{ij}=\alpha_I$ for $i,j\in I\subset \{1,\dots,n\}$, $\tau^{(\alphabd)}_\Gamma$ is weakly meromorphic in $\alpha_I$. Thus locality and covariance holds for each term in the corresponding Laurent series and consequently also after subtracting the principal part of this series.

$d)$ As argued in the proof of Proposition %\ref{pr:prod-sigma}
, the distributions $t^{(\alphabd)}_\Gamma$ defined in %\eqref{eq:prod-sigma}
satisfy the microlocal spectrum condition, i.e. they have the correct wave front set. Consequently, the regularized amplitudes $\tau^{(\alphabd)}_\Gamma$ have the correct wave front set as well. As $\tau^{(\alphabd)}_\Gamma$ is weakly meromorphic in the sense recalled in the proof of $c)$, each term in the corresponding Laurent series has a wave front set bounded by the wave front set of $\tau^{(\alphabd)}_\Gamma$. Consequently the microlocal spectrum condition holds after subtracting the principal part and considering the limit of vanishing regularization parameters.

$e)$ This property follows directly from the construction. In particular the subtraction of counterterms is defined in terms of numerical distributions and independent of the field $\phi$.

$f)$ In analogy to $b)$, the Leibniz rule holds because the operation of extracting the relevant principal part of a regularized amplitude $\tau^{(\alphabd)}_\Gamma$ commutes with all partial differential operators.

$g)$ The Principle of Perturbative Agreement for perturbations of the generalised mass term $\mu$ demands essentially that upon setting $\mu=\mu_0 + \mu_1$, the renormalisation of $\mathcal{T}_n$ commutes with the operation of perturbatively expanding quantities in $\mu_1$ around $\mu_0$. A Feynman propagator $H_F$ depends on $\mu$ only via the Hadamard coefficients $v$ and $w$ in %\eqref{eq:hadamard}
. However, in the definition of the analytically regularized $H^{\alpha}_F$ in %\eqref{eq:anal-feynman}
and the corresponding regularized amplitudes $\tau^{(\alphabd)}_\Gamma$ defined in %\eqref{def:regularizedamplitudes}
, these coefficients are not altered but only the $\sigma$--dependent terms multiplying these coefficients are modified. Consequently, the analytic regularization and minimal subtraction scheme we consider commutes with a perturbative expansion in $\mu_1$ around $\mu_0$.

$h)$ As recalled in $g)$ all operations in our analytic regularization and minimal subtraction scheme act directly on quantities defined entirely in terms of the geometric quantity $\sigma$. As $\sigma$ is invariant under any spacetime isometries present, the renormalisation scheme preserves this invariance.





\end{proof}


\bigskip


Note that the Principle of Perturbative Agreement (PPA) as introduced in %\cite{hollands_conservation_2005} 
also poses conditions on $\Tcal_1$, i.e. the regularization of local and covariant Wick polynomials, which we omitted in our analysis. However, given $\Tcal_n$ for $n>1$, $\Tcal_1$ can be adjusted in order to satisfy the PPA for changes of $\mu$ by using e.g. %\cite[Theorem 3.3]{drago_generalised_2015}
. Moreover, the PPA as introduced in %\cite{hollands_conservation_2005} 
further demands that, setting $g = g_0 + g_1$, the regularization also commutes with perturbatively expanding quantities in $g_1$ around an arbitrary but fixed background metric $g_0$. Since $\sigma$ depends on $g$, it is not easy to check whether a perturbative expansion in $g_1$ commutes with our analytic regularization and minimal subtraction scheme and thus it might well be that the regularization scheme discussed in the present work fails to satisfy this part of the PPA. However, if this is the case, 
the scheme can be modified according to the construction in \cite{hollands_conservation_2005}
in order to satisfy also this condition while preserving the other properties in proposition \ref{prop:properties_scheme}, including the invariance under any spacetime isometries present.


\bigskip


We have omitted the explicit dependence of regularized quantities on the mass scale $M$ appearing in the analytically regularized Feynman propagator $\Hsf^{(\alpha)}_\fsf$ \eqref{eq:hadamard_rep_reg}, but our analysis implies that the dependence of these quantities on $M$ is such that all regularized quantities are polynomials of (derivatives of) $\log\left( M^2 \sigma_\fsf(x_i,x_j)\right)$, see also the examples in the next section. Thus, the regularization group flow with respect to changes of $M$ may be computed.

\bigskip

Proposition \ref{prop:expose_poles} and the above analysis imply that our regularization scheme is in fact a particular form of differential regularization. Notwithstanding, the advantage of formulating this scheme in terms of analytic regularization and minimal subtraction is the ability to define the regularization scheme in a closed form at all orders by means of the forest formula (look at theorem \ref{theo:renorm_t_prod_ms_forest}.


%----------------------------------------------------------------------------%
\chapter{Computations in our scheme}
%----------------------------------------------------------------------------%


%----------------------------------------------------------------------------%
\section{Examples on generic curved spacetime}
\label{p:EXOS}
%----------------------------------------------------------------------------%


In this section we illustrate the method developed in the previous chapter to explicitly compute regularized quantities in our scheme by considering first the example of the fish graph and the sunset graph, i.e. $\Delta^n_\fsf$ for $n=2,3$. These pointwise powers of the Feynman propagator are the only ones occurring in renormalisable scalar field theories in four spacetime dimensions. Afterwards we will consider a triangular graph in section \ref{p:COMPLICATED_GRAPH} in order to illustrate the method in the case of more than two vertices. We shall work only on subsets of the spacetime where the geodesic distance is well defined without loss of generality.


In the special case of $\Delta^n_\fsf$, we are dealing with distributions which are already defined on $\Mcal^2\setminus d_2$ and have to be extended to $\Mcal^2$. In order to accomplish this task we shall use \eqref{eq:expose_poles} in order to expose the poles before subtracting them. In this context, we note that  $\Esf^\dagger_1$ given in \eqref{eq:euler_operator} applied to a distribution $t$ whose integral kernel $t(\sigma_\fsf)$ depends on $x,y$  only via $\sigma_\fsf = \sigma(x,y)$, can be further simplified. In particular, introducing $t_1(\sigma_\fsf)$ such that $\nabla^a t_1(\sigma_\fsf) = \sigma^a t(\sigma_\fsf)$, we have
%
\begin{eqnarray}
\Esf_1^\dagger t(\sigma_\fsf) &=& -\left( 4 + \sigma^a\nabla_a \right) t(\sigma) \nonumber \\
&=& - \nabla_a \sigma^a \ t - 2 \sigma^a (\nabla_a \log (u)) \ t(\sigma_\fsf) \nonumber \\ 
&=& - \Box t_1(\sigma_\fsf) - 2 \frac{\nabla_a u}{u} \ \nabla^a t_1(\sigma_\fsf) \ , 
\label{eq:E_simplified}
\end{eqnarray}
%
where $x$ is considered to be arbitrary but fixed and all the covariant derivatives are taken with respect to $y$.


%----------------------------------------------------------------------------%
\subsection{The regularized fish and sunset graphs}
%----------------------------------------------------------------------------%

%----------------------------------------------------------------------------%
\subsubsection{Standard approach}
%----------------------------------------------------------------------------%


We recall that the Feynman propagator $\Delta_\fsf(x,y)$ admit the Hadamard representation $\Hsf_\fsf(x,y)$ introduced in \eqref{eq:hadamard_rep}.


\begin{figure}[ht!]
\begin{center}
%
\begin{tikzpicture}[thick,scale=1] 
\draw[dashed] (0,0) circle (1cm and 0.3cm);
\draw (0,0) circle (1cm and 0.6cm);
\draw[dashed] (-1,0) -- (1,0);
\filldraw (-1,0) circle (2pt) node[left] {$x_1$};
\filldraw (1,0) circle (2pt) node[right] {$x_2$};
\end{tikzpicture}
%
\end{center}
\caption{Graphs with two vertices}
\end{figure}


From \eqref{eq:hadamard_rep} we can infer that, in order to regularize $\Hsf^2_\fsf$ and $\Hsf^3_\fsf$, i.e. in order to extend them from $\Mcal^2 \setminus d_2$ to $\Mcal^2$, we need to regularize the three distributions
%
\begin{equation}
\frac{1}{\sigma_\fsf^2} \ , \qquad \frac{\log \left(M^2 \sigma_\fsf\right)}{\sigma_\fsf^2} \ , \qquad \frac{1}{\sigma_\fsf^3} \ ,
\label{eq:sigma_problematic}
\end{equation}
%
because all other occurring powers of $\sigma_\fsf$, i.e. $\sigma^{-m}_\fsf\log^n(\sigma_\fsf)$ for $m\in\{0,1\}$ and $n\in\{0,1,2,3\}$ have a scaling degree for $y\to x$ smaller than 4, and thus can be uniquely extended to the diagonal. To this avail, we define 
%
\begin{equation*}
\sigma_{a_1\cdots a_n} = \nabla_{a_n} \cdots \nabla_{a_1} \sigma \ , \qquad [B](x) = B(x,x) \ , 
\end{equation*}
%
where the covariant derivatives are taken with respect to $x$ and $B$ is a general bitensor. We recall the following identities satisfied by $\sigma$
%
\begin{equation}
\sigma_a \sigma^a = 2 \sigma \ , \qquad 
\sigma_{ab} \sigma^b = \sigma_a \ , \qquad 
\Box \sigma = 4 - 2 \frac{\sigma^a \nabla_a u}{u} \ .
\label{eq:sigma_identities}
\end{equation}
%
For our purposes, it will prove useful to use the last identity in the form
%
\begin{equation*}
\Box \sigma_\fsf = 4 + f \sigma_\fsf \ , \qquad
\mbox{ with } \qquad f = - 2 \frac{\sigma^a \nabla_a u}{u \ \sigma_\fsf} \ ,
\label{eq:def_f_sigma}
\end{equation*}
%
where $f$ is a distribution, which considered as a distribution in $y$ for fixed $x$, has scaling degree zero for $y \to x$ as can be seen from the covariant Taylor expansion
%
\begin{equation*}
u = [u] + \bigg( [\nabla_a u] - \nabla_a [u] \bigg) \sigma^a + \Rcal_u = 1 + \Rcal_u \ , 
\end{equation*}
%
where the remainder $\Rcal_u$ vanishes towards the diagonal faster than $\sigma_a$  (see e.g. \cite{poisson_motion_2011}).


%\bigskip


As $f$ has vanishing scaling degree for $y \to x$, the pointwise product 
%
\begin{equation*}
f(x,y) t(x,y) 
\end{equation*}
%
with any bidistribution $t$ of scaling degree for $y\to x$ lower than $4$ may be uniquely extended to the diagonal. However, we will also encounter expressions which are naively of the form 
%
\begin{equation}
f(x,y) \delta(x,y) 
\label{eq:f_delta}
\end{equation}
%
and which are a priory ill defined because $f$ is in general divergent for $x$ and $y$ light like related, and thus not continuous on the diagonal. Notwithstanding, the distribution \eqref{eq:f_delta}, which is well defined and identically vanishing outside of the diagonal $x=y$, may be extended to the diagonal. In fact, our scheme, in which expressions of the form \eqref{eq:f_delta} appear as $\alpha \to 0$ limits of particular weakly analytic expressions, provides a unique and non vanishing extension of \eqref{eq:f_delta} to the diagonal by the very analyticity of the aforementioned expressions. In particular our scheme implies the following unique and well defined definitions of distributions on $\Mcal^2$.
%
\begin{equation}
f \ \Box \left( \frac{\log^n (M^2 \sigma_\fsf) }{\sigma_\fsf} \right) = \lim_{\alpha \to 0} \ f \ \Box \left(\frac{\log^n(M^2 \sigma_\fsf)}{\sigma^{1+\alpha}_\fsf} \right) \ , \ \ n \geq 0 \ ,
\label{eq:f_dists}
\end{equation}
%
Hereby uniqueness and weak analyticity of 
%
\begin{equation*}
f \ \Box\left( \frac{ \log^n(M^2 \sigma_\fsf) }{ \sigma^{1+\alpha}_\fsf } \right) 
\end{equation*}
%
follow from arguments used throughout previous chapter.


\bigskip


From proposition \ref{prop:sigma_1}, we know that
%
\begin{equation*}
\frac{1}{\sigma^{n+\alpha}_\fsf} 
\end{equation*}
%
is weakly meromorphic in $\alpha$. In order to compute the Laurent series, we use the above mentioned identities for $\sigma$ and obtain
%
\begin{equation*}
\frac{1}{\sigma^{n+1+\alpha}_\fsf} = \frac{1}{2(n+\alpha)(n-1+\alpha)} \left(\Box+(n+\alpha)f\right) \frac{1}{\sigma^{n+\alpha}_\fsf} \ ,
\end{equation*}
%
in accordance with \eqref{eq:expose_poles} and \eqref{eq:E_simplified}. Using this, we may compute the following Laurent series, where we recall that in $\Hsf^{(\alpha)}_\fsf$ \eqref{eq:hadamard_rep_reg} we use the same (arbitrary) constant $M$ present in the logarithmic term  of \eqref{eq:hadamard_rep} to correct for the change of dimension and a sufficiently regular function $k$ for later purposes,
%
\begin{eqnarray}
\frac{1}{(Mk)^{2\alpha}} \ \frac{1}{\sigma^{2+\alpha}_F} &=& \frac12 \left(\Box+f\right) \left( \frac{1}{\alpha \ \sigma_\fsf} \ - \ \frac{\log\left(M^2 \sigma_\fsf\right)}{\sigma_\fsf}\right) \ - \ \frac{\log(k^2)}{2} \ \left(\Box+f\right) \ \frac{1}{\sigma_\fsf} \nonumber \\
&& - \ \Box \ \frac{1}{2\sigma_\fsf} \ + \ \Ocal(\alpha) \ , \nonumber \\
%
%
\frac{d}{d\alpha} \left( \frac{1}{(Mk)^{2\alpha}} \ \frac{1}{\sigma^{2+\alpha}_\fsf} \right) &=& \frac12 \left(\Box+f\right) \left(-\frac{1}{\alpha^2 \ \sigma_\fsf} \ + \ \frac{\log^2\left(M^2 \sigma_\fsf\right)}{2 \ \sigma_\fsf}\right)  \nonumber \\
&& + \ \Box\left(\frac{\log\left(M^2 \sigma_\fsf\right)+1}{2 \ \sigma_\fsf}\right) \ + \ \log^2(k^2) \ \left(\Box+f\right) \ \frac{1}{4\sigma_\fsf} \nonumber \\
&& + \ \log(k^2) \ \left(\Box\frac{1}{2 \ \sigma_\fsf} \ + \ \left(\Box+f\right) \ \frac{\log\left(M^2 \sigma_\fsf\right)}{2 \ \sigma_\fsf}\right) \ + \ \Ocal(\alpha) \ , \nonumber \\
%
%
\frac{1}{(M h)^{2\alpha}} \ \frac{1}{\sigma^{3+\alpha}_\fsf} &=& \frac18 \left(\Box+2f\right) \ \left(\Box+f\right) \ \left(\frac{1}{\alpha\sigma_\fsf} \ - \ \frac{\log \left(M^2 \sigma_\fsf\right)}{\sigma_\fsf}\right) \nonumber \\
&& - \ \frac{\log(k^2)}{8} \ (\Box+2f) \ (\Box+f) \ \frac{1}{\sigma_\fsf} \nonumber \\
&& - \ \frac{1}{16} \ \bigg( (5\Box+8f) \ (\Box+f) \ - \ 2 (\Box+2f) \ f \bigg) \ \frac{1}{\sigma_\fsf} \ + \ \Ocal(\alpha) \ . \nonumber \\
\label{eq:general_expansion}
\end{eqnarray}
%
Note that by means of lemma \ref{lem:product_identities} one may explicitly check that the pole terms in these Laurent series are local expressions as expected.


Using the Laurent series, the lowest regularized powers of $\sigma_\fsf$ may be defined and computed as\footnote{Note that we use here a definition of the analytic regularization of the logarithm in terms of a direct derivative rather than a limit of differences like in \eqref{eq:hadamard_rep_reg}. While the two definitions differ up to a constant factor in the principal part, they coincide in the constant regular part and thus give the same $(\sigma^{-2}_\fsf \log(M^2 \sigma_\fsf))_\ms$.}. 
%
%%TODO ??
\begin{eqnarray}
\left(\frac{1}{\sigma_\fsf^2}\right)_\ms &=& \lim_{\alpha\to 0} \left(\frac{1}{M^{2\alpha}} \frac{1}{\sigma^{2+\alpha}_\fsf} \ - \ \pp\left(\frac{1}{M^{2\alpha}} \frac{1}{\sigma^{2+\alpha}_\fsf}\right)\right) \nonumber \\
&=& - \frac12 (\Box+f) \ \frac{\log \left(M^2 \sigma_\fsf\right)}{\sigma_\fsf} \ - \ \Box\left(\frac{1}{2\sigma_\fsf}\right) \ , \nonumber \\
%
\left(\frac{\log\left(M^2\sigma_\fsf\right)}{\sigma_\fsf^2}\right)_\ms &=& - \ \lim_{\alpha\to 0} \Bigg( \frac{d}{d\alpha}\left(\frac{1}{M^{2\alpha}} \ \frac{1}{\sigma^{2+\alpha}_\fsf}\right) \ - \ \pp\frac{d}{d\alpha} \left( \frac{1}{M^{2\alpha}} \ \frac{1}{\sigma^{2+\alpha}_\fsf}\right) \Bigg) \nonumber \\ 
&=& - \ \frac14 \left(\Box+f\right) \ \frac{\log^2\left(M^2 \sigma_\fsf\right)}{\sigma_\fsf} \ - \ \Box\left(\frac{\log \left(M^2 \sigma_\fsf\right)+1}{2\sigma_\fsf}\right) \ , \nonumber \\
%
\left(\frac{1}{\sigma_\fsf^3}\right)_\ms &=& \lim_{\alpha\to 0} \left( \frac{1}{M^{2\alpha}} \ \frac{1}{\sigma^{3+\alpha}_\fsf} \ - \ \pp\left(\frac{1}{M^{2\alpha}} \ \frac{1}{\sigma^{3+\alpha}_\fsf} \right)\right) \ , \nonumber \\
&=& - \ \frac18 (\Box+2f) (\Box+f) \ \frac{\log\left(M^2 \sigma_\fsf\right)}{\sigma_\fsf} \nonumber \\ 
&& - \ \frac{1}{16} \bigg( (5\Box+8f)(\Box+f) \ - \ 2(\Box+2f)f \bigg) \ \frac{1}{\sigma_\fsf} \ . \nonumber \\
\label{eq:sigma_ms}
\end{eqnarray}
%
Finally $\left(\Hsf^2_\fsf\right)_\ms$ and $\left(\Hsf^3_\fsf\right)_\ms$ are defined and computed by expanding the unregularized powers $\Hsf^2_\fsf$ and $\Hsf^3_\fsf$ and replacing the three problematic expressions \eqref{eq:sigma_problematic} by their regularized versions \eqref{eq:sigma_ms}.


%----------------------------------------------------------------------------%
\subsubsection{Alternative computation}
%----------------------------------------------------------------------------%


As a preparation towards the application of our regularization scheme to \textbf{QFT} in cosmological spacetimes, we shall now derive an alternative way to compute $\left(\Hsf_\fsf^2\right)_\ms$ and $\left(\Hsf_\fsf^3\right)_\ms$ which is better suited for practical computations. We start by stating and proving a few distributional identities.


\begin{lemma}\label{lem:product_identities}
The following distributional identities hold.
%
\begin{eqnarray*}
%
&1.& \mbox{For any continuous } \ F_0 \ \mbox{ and any twice continuously differentiable } \  F_2 \ , \\
&& \qquad \sigma F_0 \delta = 0 \ , \qquad \sigma_a F_0 \delta = 0 \ , \qquad F_0 \nabla_{\nabla\sigma} \delta = - [F_0 \Box \sigma] \delta \ ,  \\
&& \qquad F_2 \Box \delta \ = \ [\Box F_2]\delta \ + \ \Box [F_2]\delta \ - \ 2\nabla^a[\nabla_aF_2]\delta \ .\\
%
&2.& \qquad (\Box+f)\frac{1}{\sigma_\fsf} \ = \ 8\pi^2i\delta \ , \qquad (\Box+2f)(\Box+f)\frac{1}{\sigma_\fsf} \ = \ 8\pi^2i \left(\Box-\frac R3\right)\delta \ . \\
%
&3.& \mbox{For all } \ n_1, n_2, n_3 \in \Nbb_0 \ \mbox{ and } n_4, n_5, n_6 \in \{0,1\} \ \mbox{ with } \ n_2-n_3+n_4 \geq - 1 \ , \\
&& \qquad \log^{n_1}\left(\sigma_\fsf\right) \ (\sigma^a_\fsf)^{n_4} \ \sigma^{n_2}_\fsf \ \left(\frac{1}{\sigma_\fsf^{n_3}}\right)_\ms \ = \ \log^{n_1}\left(\sigma_\fsf\right) \ (\sigma^a_\fsf)^{n_4} \ \sigma^{n_2-n_3}_\fsf \ , \\
&& \qquad \Box \log(\sigma_\fsf) \ = \ \frac{\Box \sigma -2}{\sigma_\fsf} \ , \\
&& \qquad \nabla_a\left(\frac{\log^{n_5}(\sigma_\fsf)}{\sigma^{n_6}_\fsf}\right) \ = \ \frac{\left(n_5-n_6\log^{n_5}(\sigma_\fsf)\right) \ \nabla_a \sigma}{\sigma^{n_6+1}_\fsf} \ . \\
%
&4.& \qquad \sigma_\fsf\left(\frac{1}{\sigma_\fsf^3}\right)_\ms \ = \ \left(\frac{1}{\sigma_\fsf^2}\right)_\ms \ .
%
\end{eqnarray*}
%
\end{lemma}


\begin{proof}
%%TODO PROOF




$a)$ These identities follow from $B\delta=[B]\delta$ for any continuous bitensor $B$, $[\sigma]=0$, $[\sigma_a]=0$ and the definition of weak derivatives.

$b)$ The first identity holds in Minkowski spacetime because $1/(8\pi^2\sigma_F)$ is the Feynman propagator of the massless vacuum state. In curved spacetimes %\eqref{eq_basicidentities} 
imply that $(\Box+f)1/\sigma_F$ vanishes outside of the origin and thus must be a sum of derivatives of $\delta$ distributions. Because $\sigma$ depends smoothly on the metric, the coefficients in this sum must be smooth functions of the metric with appropriate mass dimension and thus $(\Box+f)1/\sigma_F=c\delta$ with a constant $c$ that can be fixed in Minkowski spacetime. 

In order to prove the second identity we recall Remark %\ref{rem_fdists}
and observe that it is sufficient to compute 
%
$$
t:=\lim_{\alpha\to 0}f(\Box+f)\frac{1}{\sigma^{1+\alpha}_F}
$$
%
This expression has for $y\to x$ a scaling degree $\le 4$, vanishes outside of $x=y$, depends smoothly on the metric, is covariant and has mass dimension $6$. Consequently $t=cR\delta$ where the dimensionless constant $c$ can be computed on any spacetime with $R\neq 0$. Moreover, in any spacetime where $f$ is actually continuous in a neighbourhood of the diagonal we have $t(x,y)=8\pi i f(x,x) \delta(x,y)$. A spacetime which satisfies both properties is (the patch of) de Sitter spacetime defined in conformal coordinates by the metric line element 
%
$$ds^2=\frac{1}{H^2\tau^2}\left(-d\tau^2 + d\vec{x}^2\right)$$
%
on $(-\infty,0)\times\Rbb^3$, where $H$ is a constant. On this spacetime we have $R=12H^2$ and
%
$$\mu^2:=2 H^2 \sigma(\tau_1,\vec{x}_1,\tau_2,\vec{x}_2)=\cos^{-1}\left(\frac{\tau^2_1+\tau^2_2-(\vec{x}_1-\vec{x}_2)^2}{2\tau_1\tau_2}\right),$$
%
see e.g. %\cite{Allen:1985ux}
, where analytic continuation of $\cos^{-1}$ is understood for time--like separations. From this one can infer 
%
$$
\Box \sigma = 1+3 \mu \cot (\mu) \qquad \Rightarrow \qquad f = \frac{\Box \sigma-4}{\sigma} = 6H^2 \frac{\mu \cot (\mu) - 1}{\mu^2}= -\frac{R}{6} + O(\mu^2)
$$
%
which demonstrates that on de Sitter spacetime $f$ is continuous in a neighbourhood of the diagonal with $f(x,x)=-R/6$. 

$c)$ The distributions on both sides of each equation, considered as distributions in $y$ for fixed $x$, have the same scaling degree $<4$ for $y\to x$ and agree outside of the diagonal. Thus they agree also on the diagonal as unique extensions.

$d)$ As in the proof of a) we observe that the potential local correction term on the right hand side must be a sum of derivatives of $\delta$ with coefficients that depend smoothly on the metric because $\sigma$ does. Thus the correction term must be of the form $c\delta$ with a constant $c$ that can be computed in Minkowski spacetime. This computation may be performed by using %\eqref{eq_basicidentities}
, the previous statements of this lemma, and the following identities which are valid in Minkowski spacetime for any function $F$ s.t. $F(\sigma_F)$ is a distribution
$$\sigma_F\Box  F(\sigma_F)=\Box \sigma_F F(\sigma_F) - 4 F(\sigma_F) - 2\nabla_{\nabla\sigma_F}F(\sigma_F)\,,$$
$$\sigma_F\Box^2  F(\sigma_F)=\Box^2 \sigma_F F(\sigma_F) - 4 \Box F(\sigma_F) - 4\Box \nabla_{\nabla\sigma_F}F(\sigma_F)\,,$$
whereby one finds that $c=0$.





\end{proof}



These identities can be used to compute $\left(\Hsf_\fsf^2\right)_\ms$ and $\left(\Hsf_\fsf^3\right)_\ms$ in an alternative way under certain conditions.


\begin{proposition}\label{prop:equivalent_scheme}
Let $\Ncal$ be a normal neighbourhood in $\Mcal$ and let $\Hsf_\fsf$ be a distribution on $\Ncal \times \Ncal$ of Feynman Hadamard form \eqref{eq:hadamard_rep}. Then the following identities hold.
%
\begin{eqnarray*}
%
&& \hspace*{-18pt} \mbox{If } \ \Hsf_\fsf^{\alpha} \ \mbox{ is a well defined distribution which is weakly meromorphic in } \ \alpha \ , \mbox{ then } \\
%
&& 1. \hspace*{8pt} (\Hsf_\fsf^2)_\ms = \lim_{\alpha\to 0} \left( \frac{1}{M^{2\alpha}} \ \Hsf_\fsf^{2+\alpha} - \pp\left(\frac{1}{M^{2\alpha}} \ \Hsf_\fsf^{2+\alpha} \right) \right) + \frac{i\log(8\pi^2)}{16\pi^2} \delta \ , \\
%
&& 2. \hspace*{8pt} \left(\Hsf_\fsf^2 \ \log \left(M^{-2}\Hsf_\fsf\right)\right)_\ms = \lim_{\alpha\to 0} \Bigg(\frac{d}{d\alpha}\left(\frac{1}{M^{2\alpha}}(\Hsf_\fsf)^{2+\alpha}\right) - \pp \ \frac{d}{d\alpha}\left(\frac{1}{M^{2\alpha}}(\Hsf_\fsf)^{2+\alpha}\right)\Bigg) \\
&& \hspace*{145pt} - \ \frac{i\log^2(8\pi^2)}{32\pi^2}\delta \ , \\
%
&& \hspace*{-18pt} \mbox{ and if } \ [v]=0 \ , \ \mbox{then} \\
%
&& 3. \hspace*{12pt} (\Hsf_\fsf^3)_\ms = \lim_{\alpha\to 0}\left(\frac{1}{M^{2\alpha}} \Hsf_\fsf^{3+\alpha} - \pp\left(\frac{1}{M^{2\alpha}} \Hsf_\fsf^{3+\alpha} \right)\right) \\
&& \hspace*{70pt} + \ \frac{i}{48(8\pi^2)^2} \ \bigg((1+2\log(8\pi^2))R+192\pi^2[w]\bigg) \ \delta \ . \\
%
\end{eqnarray*}
%
\end{proposition}


\begin{proof}
%%TODO PROOF



$a)$ Setting $h=8\pi^2\sigma_F \Delta_F$ and $k=\sqrt{8\pi^2/h}$, we obtain
%
$$\frac{1}{M^{2\alpha}}\Delta_F^{2+\alpha}=\frac{h^2}{(8\pi^2)^2}\frac{1}{(Mk)^{2\alpha}}\frac{1}{\sigma_F^{2+\alpha}}\,.$$ 
%
Using %\eqref{eq_generalexpansion}
, $[h^2]=[u^2]=1$ and Lemma %\ref{lem_productidentities}
a), b) \& c) we may compute
%
\begin{align*}&\lim_{\alpha\to 0}\left(\frac{1}{M^{2\alpha}}\Delta_F^{2+\alpha}-\pp\frac{1}{M^{2\alpha}}\Delta_F^{2+\alpha}\right)\\
%
=\quad&\frac{h^2}{(8\pi^2)^2}\lim_{\alpha\to 0}\left(\frac{1}{(Mk)^{2\alpha}}\frac{1}{\sigma_F^{2+\alpha}}-\pp\frac{1}{(Mk)^{2\alpha}}\frac{1}{\sigma_F^{2+\alpha}}\right)\\
%
=\quad&\frac{h^2}{(8\pi^2)^2} \left(\left(\frac{1}{\sigma^2_F}\right)_\ms-\frac{\log (k^2)}{2}\left(\Box + f\right)\frac{1}{\sigma_F}\right)=(\Delta^2_F)_\ms-\frac{i\log(8\pi^2)}{16\pi^2}\delta\,.\end{align*}

$b)$ In analogy to $a)$, we may compute
%
\begin{align*}&\lim_{\alpha\to 0}\left(\frac{d}{d\alpha}\frac{1}{M^{2\alpha}}\Delta_F^{2+\alpha}-\pp\frac{d}{d\alpha}\frac{1}{M^{2\alpha}}\Delta_F^{2+\alpha}\right)\\
%
=\quad&\frac{h^2}{(8\pi^2)^2} \left(-\left(\frac{\log \left(M^2 \sigma_F\right)}{\sigma^2_F}\right)_\ms-\log\left(\frac{8\pi^2}{h^2}\right)\left(\frac{1}{\sigma^2_F}\right)_\ms+\frac{\log^2 \left(\frac{8\pi^2}{h^2}\right)}{4}\left(\Box + f\right)\frac{1}{\sigma_F}\right)\\
%
=\quad &\left(\Delta^2_F\log \left(M^{-2}\Delta_F\right)\right)_\ms+\frac{i\log^2(8\pi^2)}{32\pi^2}\delta\,.\end{align*}
 
$c)$ This can be proven in analogy to a) and b), whereby one also needs Lemma %\ref{lem_productidentities} 
d) and the fact that $[v]=0$ implies by means of the covariant expansion of bitensors near the diagonal (see e.g. %\cite[Section 5]{Poisson:2011nh}
) that 
$$v=[v]+([\nabla_a v]-\nabla_a[v])\sigma^a+R_v=[\nabla_a v]\sigma^a+R_v\,,$$
where the remainder term $R_v$ vanishes towards the diagonal fast than $\sigma_a$. Thus, the assumption $[v]=0$ implies that the term in $\Delta^3_F$ proportional to $\sigma^{-2}_F \log M^2\sigma_F$ does not need to be renormalised, which is crucial for the present proof. The correction term arises from the  $\log h/(8\pi^2)$ term in the expansion of $$\frac{1}{(Mk)^{2\alpha}}\frac{1}{\sigma_F^{3+\alpha}}$$ whose contribution may be computed as
%
$$\frac{h^3 \log\left( \frac{h}{8\pi^2}\right)}{8(8\pi^2)^3}(\Box+2f)(\Box+f)\frac{1}{\sigma_F}=-\frac{i\left(-\log(8\pi^2)\frac{R}{3}-[\Box h^3 \log (h)]\right)\delta}{8(8\pi^2)^2}=$$
%
$$=\frac{i\left(-\log(8\pi^2)\frac{R}{3}+[\Box u + 8\pi^2 w\Box \sigma]\right)\delta}{8(8\pi^2)^2}=\frac{i\left((1+2\log(8\pi^2))R+192\pi^2[w]\right)}{48(8\pi^2)^2}\delta\,,$$
%
where again Lemma %\ref{lem_productidentities}
a) \& b) prove to be  useful.




\end{proof}


%----------------------------------------------------------------------------%
\subsection{A more complicated graph}
\label{p:COMPLICATED_GRAPH}
%----------------------------------------------------------------------------%


\begin{wrapfigure}{r}{0.3\textwidth}
\begin{center}
\begin{tikzpicture}[thick,scale=1] 
\draw (0,0) -- (2,0);
\draw (2,0) -- (1,1.4);
\draw [bend left] (0,0) edge (1,1.4);
\draw [bend left] (1,1.4) edge (0,0);
\filldraw (0,0) circle (2pt) node[left] {$x_1$};
\filldraw (2,0) circle (2pt) node[right] {$x_3$};
\filldraw (1,1.4) circle (2pt) node[above] {$x_2$};
\end{tikzpicture}
\end{center}
\end{wrapfigure}


In order to show how the proposed regularization scheme works for graphs which have more than two vertices we discuss the regularization of the following triangular graph
%
\begin{equation*}
\tsf_\gamma \ = \ \Hsf_{13} \ \Hsf_{23} \ \Hsf_{12}^2 \ , 
\end{equation*}
%
where $\Hsf_{ij} = \Hsf_\fsf(x_i,x_j)$. In order to apply the forest formula from theorem \ref{theo:renorm_t_prod_ms_forest} to regularize this graph, we already know that the forests which correspond to divergent contributions are 
%
\begin{equation*}
\{12\} \ , \quad \{123\} \ , \quad \{12,123\} \ .
\end{equation*} 
%
Then as already said in \eqref{eq:kernel_trig_ms} the regularization of $\tsf_\gamma$ thus reads
%
\begin{equation*}
\left(\tsf_\gamma\right)_\ms = \left(1+\Rsf_{12}+\Rsf_{123}+\Rsf_{123}\Rsf_{12}\right) \tsf^{(\alphabd)}_\gamma = (1+\Rsf_{123})(1+\Rsf_{12}) \tsf^{(\alphabd)}_\gamma \ .
\end{equation*}
%
In order to illustrate the explicit form of the $\Rsf$, we consider only the most singular contribution to $\tsf^{(\alphabd)}_\gamma$, namely
%
\begin{equation}
\left(t_{\gamma}^{(\alphabd)}\right)_0 = \frac{1}{\sigma_{13}^{1+\alpha_{13}}} \frac{1}{\sigma_{12}^{2(1+\alpha_{12})}} \frac{1}{\sigma_{23}^{1+\alpha_{23}}} \ ,
\label{eq:amplitude_sigma_trig_0}
\end{equation}
%
where $\sigma_{ij} = \sigma_\fsf(x_i,x_j)$. Note that, with obvious notation
%
\begin{equation*}
(8\pi^2)^{-4} \ u_{13} \ u^2_{12} \ u_{23} \ \left(t_{\Gamma}\right)_0 
\end{equation*}
%
is in fact the only contribution to $\tsf_\gamma$ which needs to be renormalized. The application  
%
\begin{equation*}
\mbox{of } \ 1 + \Rsf_{12} \ \mbox{ to } \ \eqref{eq:amplitude_sigma_trig_0}
\end{equation*}
% 
has already been discussed in the preceding sections and corresponds to the regularization of the \textbf{fish graph}. Indeed, after setting 
%
\begin{equation*}
\alpha_{12} \ , \ \ \alpha_{23} \ , \ \ \mbox{and } \ \alpha_{13} \ , \ \mbox{to } \ \alpha \ = \ \alpha_I \ \ \mbox{for } \ I=\{1,2,3\} 
\end{equation*}
%
we obtain
%
\begin{eqnarray}
\left(t_{\gamma}^{(\alpha)}\right)_1 &=& \lim_{\alpha_{ij}\to\alpha} (1+\Rsf_{12}) \ \left(t_{\gamma}^{(\alphabd)}\right)_0 \nonumber \\
&=& \left(\left(\frac{1}{\sigma_{12}^2}\right)_\ms + \Ocal(\alpha)\right) \frac{1}{(\sigma_{13})^{1+\alpha}} \frac{1}{(\sigma_{23})^{1+\alpha}} \ .
\label{eq:amplitude_sigma_trig_1}
\end{eqnarray}
%
The distribution 
%
\begin{equation*}
\left(\frac{1}{\sigma^2_{12}}\right)_\ms 
\end{equation*}
%
is a homogeneous distribution of degree $\delta=-4$ under scaling of $x_2$ towards $x_1$, consequently, \eqref{eq:amplitude_sigma_trig_1} has scaling degree $8+4\alpha$. Owing to proposition \ref{prop:almost_homo}, we know that \eqref{eq:amplitude_sigma_trig_1} can be decomposed into the sum of a homogeneous distribution of degree $-(8+4\alpha)$ and a remainder. Hence, in order to expose the poles of \eqref{eq:amplitude_sigma_trig_1}, we can directly apply proposition \ref{prop:expose_poles} with $m=1$ and $c_0 = -4\alpha$. To this end, we set 
%
\begin{equation*}
U_0 = \left(t_{\gamma}^{(\alpha)}\right)_1
\end{equation*}
%
and find 
%
\begin{eqnarray*}
U_1 &=& -4 \alpha \ U_0 - \Esf^\dagger_1 U_0 \\
&=& \left(\left(\frac{1}{\sigma_{12}^{2}} \right)_\ms + \Ocal(\alpha) \right) \frac{1}{(\sigma_{13})^{1+\alpha}} \frac{1}{(\sigma_{23})^{2+\alpha}} \ G \ ,
\end{eqnarray*}
%
where $G=G(x_1,x_2,x_3)$ is the smooth function introduced in lemma \ref{lem:rho_over_squared}. From \eqref{eq:expose_poles} we can infer that the principal part of \eqref{eq:amplitude_sigma_trig_1} is
%
\begin{equation*}
\pp\left(t_{\gamma}^{(\alpha)}\right)_1 = - \frac{1}{4\alpha} \left(E^\dagger_1 + \frac{G}{\sigma_{23}}\right) \left( \left(\frac{1}{\sigma_{12}^2}\right)_\ms \frac{1}{\sigma_{13}} \ \frac{1}{\sigma_{23}} \right) \ ,
\end{equation*}
%
whereas the constant regular part can be easily computed as well. Consequently, the regularized distribution
%
\begin{equation*}
\left.(t_{\gamma})_0\right|_\ms = \lim_{\alpha\to 0} \left( \left(t_{\gamma}^{(\alpha)}\right)_1 - \pp \left(t_{\gamma}^{(\alpha)}\right)_1 \right)
\end{equation*}
%
can be straightforwardly computed in explicit terms.

%----------------------------------------------------------------------------%
\section{Explicit computations in cosmological spacetimes}
%----------------------------------------------------------------------------%


%----------------------------------------------------------------------------%
\subsection{Propagators in Fourier space}
%----------------------------------------------------------------------------%


In comoving coordinates with conformal time, the Klein-Gordon operator reads
%
\begin{equation*}
\Psf \ = \ - \Box + \xi R + m^2 \ = \ \frac{1}{a(\tau)^3} \left(\partial^2_\tau-\vec{\nabla}^2 + \left(\xi-\frac16\right)R a^2+m^2a^2\right) \ a(\tau) \ .
\end{equation*}
%
It is convenient to employ Fourier transformations with respect to the spatial coordinates in order to expand quantities in QFT on FLRW spacetimes in terms of mode solutions of the free Klein-Gordon equation
%
\begin{equation*}
\phi_{\vec{k}}(\tau,\vec{x}) = \frac{\chi_k(\tau) \ \esf^{i\vec{k}\vec{x}}}{(2\pi)^{\frac32} \ a(\tau)} \ , 
\end{equation*}
%
where the temporal modes $\chi_k(\tau)$ satisfy
%
\begin{equation}
\left(\partial^2_\tau + k^2 + m^2 a^2 + \left( \xi - \frac16 \right) R \ a^2 \right) \chi_k(\tau) \ = \ 0
\label{eq:modes}
\end{equation}
%
and the normalisation condition
%
\begin{equation}
\chi_k \ \partial_\tau \overline{\chi_k} \ - \ \overline{\chi_k} \ \partial_\tau{\chi_k} \ = \ i \ .
\label{eq:modes_normal}
\end{equation}
%
Here $k = \abs{\vec{k}}$ and $\overline{\cdot}$ denotes complex conjugation.


In particular we can use the mode expansion in order to give explicit expressions for the various propagators of the free Klein-Gordon quantum field in a pure, Gaussian, homogeneous and isotropic state $\Omega$ (see \cite{pinamonti_initial_2011,zschoche_chaplygin_2014}
for associated technical conditions on the mode functions). To this avail, we define
%
\begin{equation}
\Delta_\sharp(x_1,x_2) = \lim_{\epsilon\downarrow 0} \frac{1}{8\pi^3 \ a(\tau_1) \ a(\tau_2)} \int_{\mathbb{R}^3} \dsf^3k \ \widehat{\Delta_\sharp}(\tau_1,\tau_2,k) \ \esf^{i\vec{k}(\vec{x}_1-\vec{x}_2)-\epsilon k} \ ,
\label{eq:propagators_fourier}
\end{equation}
%
where $\Delta_\sharp$ stands for either $\Delta_+$ (two-point function), $\Delta_{\rsf/\asf}$ (retarded/advanced propagator) or $\Delta_\fsf$ (Feynman propagator). Recall that our regularization scheme preserves invariance under spacetime isometries and thus we know that regularized powers of the Feynman propagator may also be written in the form \eqref{eq:propagators_fourier}.


The Fourier versions of the single propagators read
%
\begin{eqnarray}
&& \widehat{\Delta_+}(\tau_1,\tau_2,k) = \chi_k(\tau_1) \overline{\chi_k(\tau_2)} \ , \nonumber \\
&& \widehat{\Delta_-}(\tau_1,\tau_2,k) = \overline{\widehat{\Delta_+}(\tau_1,\tau_2,k)} \ , \nonumber \\
&& \widehat{\Delta_\fsf}(\tau_1,\tau_2,k) = \Theta(\tau_1-\tau_2) \ \widehat{\Delta_+}(\tau_1,\tau_2,k) + \Theta(\tau_2-\tau_1) \ \widehat{\Delta_-}(\tau_1,\tau_2,k) \ , \nonumber \\
&& \widehat{\Delta_{\rsf/\asf}}(\tau_1,\tau_2,k) = \mp i \Theta\left(\pm(\tau_1-\tau_2)\right) \ \left(\widehat{\Delta_+}(\tau_1,\tau_2,k) - \widehat{\Delta_-}(\tau_1,\tau_2,k)\right) \ , \nonumber \\
\label{eq:propagators_fourier_exp}
\end{eqnarray}
%
whereas by the convolution theorem, we have the following Fourier versions of products and convolutions of multiple propagators, provided those products and convolutions are well defined. Defining
%
\begin{eqnarray}
&& \left[\Delta_{\sharp_1}\ast_4\Delta_{\sharp_2}\right](x,y) = \int_\Mcal \dsf^4x \ \sqrt{\abs{g}} \ \Delta_{\sharp_1}(x_1,x) \ \Delta_{\sharp_2}(x,x_2) \ , \nonumber \\
%
&& \left[\widehat{\Delta_{\sharp_1}}\ast_1\widehat{\Delta_{\sharp_2}}\right](\tau_1,\tau_2,k) = \int_I \dsf\tau \ a(\tau)^2 \ \widehat{\Delta_{\sharp_1}}(\tau_1,\tau,k) \ \widehat{\Delta_{\sharp_2}}(\tau,\tau_2,k) \ , \nonumber \\
%
&& \left[\widehat{\Delta_{\sharp_1}}\ast_3\widehat{\Delta_{\sharp_2}}\right](\tau_1,\tau_2,k) = \int_{\mathbb{R}^3} \dsf^3p \ \widehat{\Delta_{\sharp_1}}(\tau_1,\tau_2,p) \ \widehat{\Delta_{\sharp_2}}\left(\tau_1,\tau_2,|\vec{k}-\vec{p}|\right) \ , \nonumber \\
\label{eq:def_convolutions}
\end{eqnarray}
%
we have
%
\begin{eqnarray}
&& \widehat{\Delta_{\sharp_1}\ast_4\cdots\ast_4\Delta_{\sharp_n}}=\widehat{\Delta_{\sharp_1}}\ast_1\cdots\ast_1 \widehat{\Delta_{\sharp_n}} \ , \nonumber \\
&& \widehat{\prod^n_{i=1}\Delta_{\sharp_i}}(\tau_1,\tau_2,k)=\frac{1}{\left((2\pi)^3 a(\tau_1)^{2}a(\tau_2)^{2}\right)^{n-1}}\left[\widehat{\Delta_{\sharp_1}}\ast_3\cdots\ast_3\widehat{\Delta_{\sharp_n}}\right](\tau_1,\tau_2,k) \ , \nonumber \\
\label{eq:convolution_identities}
\end{eqnarray}
%
Choosing a pure, Gaussian, homogeneous and isotropic state $\Omega$ of the quantized free Klein-Gordon field on a spatially flat FLRW spacetimes amounts to choosing a solution of \eqref{eq:modes} and \eqref{eq:modes_normal} for each $k$. In order for $\Omega$ to be a Hadamard state the temporal modes $\chi_k$ have to satisfy certain conditions in the limit of large $k$ which are difficult to formulate precisely. Heuristically, a necessary but not sufficient condition is that the dominant part of $\chi_k$ for large $k$, when the mass and curvature terms in \eqref{eq:modes} are dominated by $k^2$, is 
%
\begin{equation*}
\frac{1}{\sqrt{2k}} \ \esf^{-ik\tau} \ , 
\end{equation*}
%
i.e. a positive frequency solution. Note that the retarded and advanced propagators are state-independent and thus 
%
\begin{equation*}
\widehat{\Delta_{\rsf/\asf}}(\tau_1,\tau_2,k) 
\end{equation*}
%
is independent of the particular $\chi_k$ chosen for each $k$.


%----------------------------------------------------------------------------%
\subsection{The regularized fish and sunset graphs in Fourier space}
\label{p:SUNSET_FISH_FLWR}
%----------------------------------------------------------------------------%


In perturbative calculations at low orders we encounter (pointwise) powers of $\Delta_\pm$ and $\Delta_\fsf$. While the powers of $\Delta_\pm$ are well-defined if $\Omega$ is a Hadamard state on account of the wave front set properties of these distributions, we need to regularize the powers of $\Delta_\fsf$ by means of the scheme developed in the previous sections. In order to be useful for explicit computations in FLRW spacetimes, we have to develop a spatial Fourier--space version of this scheme. Having in mind the application to $\phi^4$ theory, we shall compute 
%
\begin{equation*}
\widehat{(\Delta_\fsf)^n_\ms}(\tau_1,\tau_2,k) \ \mbox{ for } \ n = 2 , 3 \ . 
\end{equation*}
%
The difficulty in achieving this is that, to our knowledge, despite of the large symmetry of flat \textbf{FLRW} spacetimes, neither $\sigma$ nor the Hadamard coefficients $u$, $v$ and $w$ may written in a tractable form which can be Fourier transformed easily. Our strategy to circumvent this problem is the following section.


%----------------------------------------------------------------------------%
\subsubsection{Strategy}
%----------------------------------------------------------------------------%


\begin{enumerate}
%
%
\item For a general mass $m$ and coupling to the scalar curvature $\xi$ and a general homogeneous and isotropic, pure and Gaussian Hadamard state $\Omega$, split $\Hsf_\fsf$ as
%
\begin{equation}
\Hsf_\fsf = \Hsf_{\fsf,0} + d \ , \qquad d = \Hsf_\fsf - \Hsf_{\fsf,0} \ ,
\label{eq:propagator_split}
\end{equation}
%
where $\Hsf_{\fsf,0}$ must satisfy the following conditions.
%
\begin{itemize}
%
\item $\Hsf_{\fsf,0}$ is explicitly known in position space and Fourier space.
%
\item $\Hsf_{\fsf,0}$ is of the form
%
\begin{equation*}
\Hsf_{\fsf,0} = \frac{1}{8\pi^2} \left( \frac{u_0}{\sigma_\fsf} + v_0 \log\left(M^2\sigma_\fsf\right) \right) + w_0 \ , 
\end{equation*}
%
with $u_0=u$, i.e. it agrees with $\Hsf_\fsf$ in the most singular term but not necessarily in the subleading singularities.
%
\item $[v_0]=0$ and $\Hsf^{\alpha}_{\fsf,0}$ is weakly meromorphic in $\alpha$ such that
%
\begin{equation*}
\left(\Hsf^2_{\fsf,0}\right)_\ms \ , \quad 
\left(\Hsf^2_{\fsf,0} \log\left(M^{-2} \Delta_{\fsf,0}\right) \right)_\ms \ , \quad
\mbox{and} \quad \left(\Hsf^3_{\fsf,0}\right)_\ms 
\end{equation*}
%
may be computed with proposition \ref{prop:equivalent_scheme}. This is crucial for preserving the explicit knowledge of $\Hsf_{\fsf,0}$ in position space in the regularization procedure, so that one may hope to compute the Fourier transforms of the regularized powers.
%
\end{itemize}
%
%
\item With these assumptions on $\Hsf_{\fsf,0}$ it follows that the regularized fish and sunset graphs may be computed as
%
\begin{eqnarray}
&& (\Hsf_\fsf^2)_\ms = (\Hsf_{\fsf,0}^2)_\ms \ + \ 2 \ \Hsf_{\fsf,0} \ d \ + \ d^2 \ , \nonumber \\
&& (\Hsf_\fsf^3)_\ms = (\Hsf_{\fsf,0}^3)_\ms \ + \ 3 \left(\Hsf_{\fsf,0}^2 \ d \right)_\ms \ + \ 3 \Hsf_{\fsf,0} \ d^2 \ + \ d^3 \ , \nonumber \\
\label{eq:fish_sunset_reg}
\end{eqnarray}
%
because the non regularized terms in the above formulae are distributions with scaling degree lower than $4$ for $y \to x$ and thus can be directly and uniquely extended to the diagonal.
%
%
\item $\left(\Hsf_{\fsf,0}^2\right)_\ms$ and  $\left(\Hsf_{\fsf,0}^3\right)_\ms$ may be computed with proposition \ref{prop:equivalent_scheme} as anticipated. In order to compute $\left(\Hsf_{\fsf,0}^2 \ d\right)_\ms$ we further split $d$ as
%
\begin{equation*}%
d = d_1 + d_2 \ , \qquad d_1 = - \frac{[v] \log\left(M^{-2} \ \Hsf_{\fsf,0}\right)}{8\pi^2} \ ,\qquad d_2 = d - d_1 \ .
\label{eq:d_split}
\end{equation*}
%
Because 
%
\begin{equation*}
v = [v] + \Ocal(\sigma_a) \ , 
\end{equation*}
%
$d_1$ contains the leading logarithmic singularity in $d$ (and thus $\Hsf_\fsf$) which is the only logarithmic singularity relevant for the regularization of the sunset graph. Consequently
%
\begin{equation}
\left(\Hsf_{\fsf,0}^2 \ d\right)_\ms \ = \ - \frac{[v]}{8\pi^2} \ \left(\Hsf_{\fsf,0}^2 \ \log\left(M^{-2} \ \Hsf_{\fsf,0} \right) \right)_\ms \ + \ d_2 \ \left(\Hsf_{\fsf,0}^2 \right)_\ms \ ,
\label{eq:fish_sunset_reg_2}
\end{equation}
%
and thus proposition \ref{prop:equivalent_scheme} can be applied again.
%
%
\item Due to the symmetry of \textbf{FLRW} spacetimes and the assumption that the pure and Gaussian Hadamard state $\Omega$ is invariant under this symmetry, $[v]$ and $[w]$ do not depend on the spatial coordinates. Given that one succeeds to compute the spatial Fourier transforms of 
%
\begin{equation*}
\log \left(M^{-2} \ \Hsf_{\fsf,0}\right) \ , \quad
\left(\Hsf_{\fsf,0}^2\right)_\ms \ , \quad
\left(\Hsf_{\fsf,0}^2 \ \log\left(M^{-2} \ \Hsf_{\fsf,0}\right)\right)_\ms \ , \quad  \left(\Hsf_{\fsf,0}^3\right)_\ms \ , 
\end{equation*}
%
and
%
\begin{equation*}
\widehat{(\Hsf_\fsf^n)_\ms}(\tau_1,\tau_2,k)
\end{equation*}
%
may be computed by means of the convolution identities \eqref{eq:convolution_identities}, since the Fourier transforms of $\Hsf_{\fsf,0}$, $d_1$ and $d_2$ are known by construction.
%
%
\end{enumerate}


%----------------------------------------------------------------------------%
\subsubsection{Explicit computations}
%----------------------------------------------------------------------------%


In order to follow the computational strategy outlined above, we first compute $[v]$ and $[w]$.  Indeed, the coinciding point limit of the Hadamard coefficient $v$ reads (see e.g. \cite[Section III.1.2]{hack_backreaction_2010} for details)
%
\begin{equation*}
[v]=\frac{m^2+\left(\xi-\frac16\right)R}{2} \ .
\label{eq:coinciding_v}
\end{equation*}
%
Moreover, using the method of (CITE--SCLEMMER) to compute a spatial Fourier representation of the Hadamard parametrix $\hsf_\fsf$ (here considered as \eqref{eq:hadamard_rep} with $w=0$) in \textbf{FLRW} spacetimes, one can compute 
%
\begin{eqnarray}
[w] &=& \lim_{x\to y} \left(\Hsf_\fsf(x,y) - \Hsf_\fsf(x,y) \right) \nonumber \\
&=& \frac{1}{(2\pi)^3 a^2} \ \int_{\Rbb^3} \dsf^3k \ \abs{\chi_k(\tau)}^2 \ - \ \frac{1}{2\sqrt{k^2+a^2m^2+a^2\left(\xi-\frac16\right)R}} \nonumber \\
%
&& + \frac{1}{16\pi^2} \ \left(m^2+\left(\xi-\frac16\right)R\right)\left(2\gamma-1+\log\left(
\frac{m^2+\left(\xi-\frac16\right)R}{2M^2}\right)\right) \ - \ \frac{R}{36(8\pi^2)} \nonumber \\
\label{eq:coinciding_w} 
\end{eqnarray}
%
where $\gamma$ is the Euler-Mascheroni constant and $\Hsf_\fsf$ is taken with the mass scale $M$ inside of the logarithm of $\sigma$.


Note that one may take instead of the function 
%
\begin{equation*}
F(k) = \frac{1}{(2\sqrt{k^2+a^2m^2+a^2\left(\xi-\frac16\right)R})}
\end{equation*}
%
in \eqref{eq:coinciding_w} any distribution $F^\prime(k)$ such that $F^\prime(k)-F(k)$ is $\Ocal(k^{-5})$ for large $k$ and integrable. By taking e.g. 
%
\begin{equation*}
F^\prime(k) = \frac{1}{2k} - \Theta(k-am)\frac{a^2 m^2 + a^2 \left( \xi - \frac16 \right) R}{4k^3}
\end{equation*}
%
one may cancel the $\log\left(R\right)$ term outside of the integral.


As anticipated we see that $[v]$ and $[w]$ are functions of time only (recall \eqref{eq:rflrw}). Moreover, we see that $[v]=0$ for a conformally coupled ($\xi=\frac16$) massless scalar field. Thus, in order to pursue our computational strategy, we should look for a candidate for $\Hsf_{\fsf,0}$ among the Feynman propagators in suitable states of this theory. In fact, choosing the conformal vacuum state of the massless conformally coupled scalar field does the job. The conformal vacuum is given by choosing the modes
%
\begin{equation*}
\chi_k(\tau) = \frac{\esf^{-ik\tau}}{\sqrt{2k}} \ , 
\end{equation*}
%
and thus the Feynman propagator $\Delta_{\fsf,0}$ in this state is of the form
%
\begin{equation*}
\Delta_{\fsf,0}(x_1,x_2)=\frac{1}{8\pi^2 a(\tau_1)a(\tau_2)}\frac{1}{\sigma_{\fsf,\mathbb{M}}(x_1,x_2)} \ , \qquad 
\widehat{\Delta_{\fsf,0}}(\tau_1,\tau_2,k)=\frac{e^{-ik|\tau_1-\tau_2|}}{2k}\,.
\end{equation*}
%
Here, and in the following, the index $_{\Mbb}$ indicates quantities in Minkowski spacetime, in particular 
%
\begin{equation*}
\sigma_{\Mbb}(x_1,x_2) = \frac12 \left(\vec{x}_1-\vec{x_2}\right)^2 - \frac12 \left(\tau_1-\tau_2\right)^2 \ . 
\end{equation*}
%
$\Delta^\alpha_{\fsf,0}$ is weakly meromorphic in $\alpha$ because the massless vacuum Feynman propagator in Minkowski spacetime has this property and the conformal rescaling by $a$ does not violate it. Thus, we may follow our computational strategy and compute
%
\begin{equation*}
\left(\Delta^2_{F,0}\right)_\ms \ , \quad \left(\Delta^2_{F,0}\log\left(M^{-2}\Delta_{F,0}\right)\right)_\ms \ , \ \mbox{and} \ \left(\Delta^3_{F,0}\right)_\ms \ ,
\end{equation*}
%
by means of proposition \ref{prop:equivalent_scheme}. This is easily done using \eqref{eq:general_expansion} for $\sigma_{\fsf,\Mbb}$ rather than $\sigma_\fsf$ and
%
\begin{equation*}
h = \sqrt{8 \pi^2 \ a(\tau_1)a(\tau_2)} = \sqrt{8\pi^2 \  a\otimes a} \ . 
\end{equation*}
%
The results are
%
\begin{eqnarray*}
(\Delta_{\fsf,0}^2)_\ms &=& \lim_{\alpha\to 0} \left( \frac{1}{M^{2\alpha}}(\Delta_{\fsf,0})^{2+\alpha} - \pp\frac{1}{M^{2\alpha}}(\Delta_{\fsf,0})^{2+\alpha} \right) + \frac{i\log(8\pi^2)}{16\pi^2} \delta \\
&=& \lim_{\alpha\to 0} \frac{1}{(8\pi^2)^2 a^2\otimes a^2} \left(\frac{1}{(M\sqrt{8\pi^2 a\otimes a})^{2\alpha}}\frac{1}{\sigma_{\fsf,\mathbb{M}}^{2+\alpha}}-\pp \frac{1}{(M\sqrt{8\pi^2 a\otimes a})^{2\alpha}}\frac{1}{\sigma_{\fsf,\mathbb{M}}^{2+\alpha}}\right) \\
&& + \ \frac{i\log(8\pi^2)}{16\pi^2} \ \delta \\
&=& - \frac{1+2\log (a)}{16\pi^2 a^4} \ i \ \delta_\mathbb{M} \ - \ \frac{1}{2(8\pi^2)^2 a^2\otimes a^2} \ \Box_{\mathbb{M}}\left(\frac{\log\left(M^2\sigma_{\fsf,\mathbb{M}}\right)}{\sigma_{\fsf,\mathbb{M}}}\right) \ ,
\end{eqnarray*}
%
\begin{eqnarray*}
&& \left(\Delta^2_{\fsf,0} \log\left(M^{-2} \Delta_{\fsf,0}\right)\right)_\ms \\
&=& \lim_{\alpha\to 0}\left(\frac{d}{d\alpha}\frac{1}{M^{2\alpha}}(\Delta_{\fsf,0})^{2+\alpha}-\pp\frac{d}{d\alpha}\frac{1}{M^{2\alpha}}(\Delta_{\fsf,0})^{2+\alpha}\right) \ - \ \frac{i\log^2(8\pi^2)}{32\pi^2}\delta \\
%
&=& \frac{2+2\log (a^2 8\pi^2)+\log^2 (a^2)}{32\pi^2 a^4}i\delta_\mathbb{M}+\frac{1}{4(8\pi^2)^2 a^2\otimes a^2}\Box_{\mathbb{M}}\frac{\log^2\left(M^2\sigma_{\fsf,\mathbb{M}}\right)}{\sigma_{\fsf,\mathbb{M}}}\\
%
&& + \ \frac{1+\log(8\pi^2) a\otimes a}{2(8\pi^2)^2 a^2\otimes a^2}\Box_{\mathbb{M}}\frac{\log\left(M^2\sigma_{\fsf,\mathbb{M}}\right)}{\sigma_{\fsf,\mathbb{M}}} \ ,
\end{eqnarray*}
%
and
%
\begin{eqnarray*}
&& (\Delta_{\fsf,0})^3_\ms \\ 
&=& \lim_{\alpha\to 0}\left(\frac{1}{M^{2\alpha}}(\Delta_{\fsf,0})^{3+\alpha}-\pp\frac{1}{M^{2\alpha}}(\Delta_{\fsf,0})^{3+\alpha}\right)+\frac{i\left((1+2\log(8\pi^2))R+192\pi^2[w]\right)}{48(8\pi^2)^2}\delta\\
%
&=&-\frac{(15+12\log (a))\Box_\mathbb{M}+6(\Box_\mathbb{M} \log (a))+2(\partial^2_\tau a)/a}{48(8\pi^2)^2a^6}i\delta_\mathbb{M} \\ 
&& - \frac{1}{8(8\pi^2)^3 a^3\otimes a^3}\Box^2_{\mathbb{M}}\frac{\log\left(M^2\sigma_{\fsf,\mathbb{M}}\right)}{\sigma_{\fsf,\mathbb{M}}}\,.
\end{eqnarray*}
%
where we have used $\delta = \dfrac{\delta_{\Mbb}}{a^4}$, $f_{\Mbb}=0$, and the fact that by \eqref{eq:coinciding_w} $8\pi^2[w_0]=-\dfrac{R}{36}$ for the conformal vacuum state of the massless, conformally coupled scalar field. 


We can finally obtain the Fourier versions of the regularized powers of $\Delta_{F,0}$. For instance, we find for 
%
\begin{eqnarray}
&& \widehat{\left(\Delta^2_{F,0}\right)_\text{ms}}(\tau_1,\tau_2,k) \ = \ - \ \frac{1+2\log (a(\tau_1))}{16a(\tau_1)^2\pi^2}\delta(\tau_1-\tau_2)- \nonumber \\
&& - \ \frac{1}{16\pi^3 a(\tau_1)a(\tau_2)}(\partial^2_{\tau_1}+k^2)\bigint_{\mathbb{R}^3}d^3p\,\left(\frac12\left(\frac{1}{p^3}\right)_{\text{ren},M}  + \ \frac{i|\tau_1-\tau_2|}{2p^2}\right) \nonumber \\
&& \cdot \ \frac{1}{2|\vec{k}-\vec{p}|}e^{-i(p+|\vec{k}-\vec{p}|)|\tau_1-\tau_2|}
\label{eq:fourier_square}
\end{eqnarray}
%
where the appearing regularization of $\dfrac{1}{p^3}$ is defined in \eqref{eq:regk3}. Note that the $\vec{p}$ integral has no convergence problems for large $p$ because one may write the potentially dangerous contribution 
%
\begin{equation*}
\frac{-i|\tau_1-\tau_2|e^{-2ip|\tau_1-\tau_2|}}{p}
\quad
\mbox{as}
\quad
\partial_p\left( \dfrac{e^{-2ip|\tau_1-\tau_2|}}{2p^2}\right) 
\end{equation*}
%
plus an $\Ocal(p^{-3})$ term. Regarding the convergence for small $p$ we observe that the integral is manifestly convergent if $k\neq0$, thus yielding a well defined distribution in $\vec{k}$ on $\mathbb{R}^3\setminus\{0\}$. The scaling degree of this distribution is easily seen to be $1<3$ and thus a unique extension towards the origin exists. In practical terms this means that the integral for $k=0$ may be computed as a limit $k\to0$ of the integral with nonvanishing $k$ without any regularization. 


%----------------------------------------------------------------------------%
\subsubsection{Fourier transform on Minkowski spacetime}
%----------------------------------------------------------------------------%

In order to compute the Fourier transform of $\log\left( M^2 \sigma_{\fsf,\Mbb}\right)$, we recall that the Feynman propagator of the Klein-Gordon field with mass $m$ in the Minkowski vacuum is given by
%
\begin{eqnarray*}
\Delta_{\fsf,m,\Mbb}&=&\lim_{\epsilon\downarrow 0} \frac{1}{(2\pi)^3} \int_{\Rbb^3} \dsf^3k \ \frac{1}{2\sqrt{k^2+m^2}} \ \esf^{-i\sqrt{k^2+m^2}|\tau_1-\tau_2|} \ \esf^{i\vec{k}\left(\vec{x}-\vec{y}\right)} \ \esf^{-\epsilon k}\\
&=&\frac{1}{8\pi^2}\sqrt{\frac{2m^2}{\sigma_{F,\Mbb}}}K_1\left(\sqrt{2 m^2\sigma_{F,\Mbb}}\right)\\
&=&\frac{1}{8\pi^2} \bigg( \frac{1}{\sigma_{\fsf,\Mbb}} + \frac{m^2}{2} \left(1+\frac{m^2 \sigma_{\fsf,\Mbb}}{4} \right) \log\left(\frac{\esf^{2\gamma}m^2 \sigma_{\fsf,\Mbb}}{2}\right) \\
&& - \ \frac{m^2}{2} \left(1+\frac{5m^2\sigma_{\fsf,\Mbb}}{8}\right)\bigg) + \Ocal(m^4),
\end{eqnarray*}
%
where $K_1$  is a modified Bessel function and $\gamma$ is the Euler-Mascheroni constant. Using this, we find
%
\begin{eqnarray*}
&&\log \left(M^2 \sigma_{\fsf,\Mbb}\right) \\
%
&=&\lim_{m\to 0}\left(16\pi^2 \frac{d \ \Delta_{\fsf,m,\Mbb}}{d \,m^2} - 
\log\left(\frac{\esf^{2\gamma}m^2}{2 M^2}\right)\right)\\
%
&=& - \ \lim_{m\to 0}\bigg(\lim_{\epsilon\downarrow 0} \frac{1}{\pi} \int_{\Rbb^3} \dsf^3 k \frac{1+i\sqrt{k^2+m^2}|\tau_1-\tau_2|}{2(k^2+m^2)^{\frac32}} \ \esf^{-i\sqrt{k^2+m^2}|\tau_1-\tau_2|} \ \esf^{i\vec{k}\left(\vec{x}-\vec{y}\right)} \ \esf^{-\epsilon k} \\
&& + \ \log\left(\frac{e^{2\gamma}m^2}{2 M^2}\right)\bigg)\\
%
&=& - \ \lim_{\epsilon\downarrow 0}\frac{1}{\pi}\int_{\Rbb^3}d^3 k\left(\lim_{m\to 0}\left(\frac{1}{2(k^2+m^2)^{\frac32}}+\pi\log\left(\frac{e^{2\gamma}m^2}{2 M^2}\right)\delta(\vec{k})\right)\right.\\
%
&& \left.+ \ \frac{i|\tau_1-\tau_2|}{2k^2}\right)e^{-ik|\tau_1-\tau_2|}e^{i\vec{k}\left(\vec{x}-\vec{y}\right)}\,e^{-\epsilon k}\\
&=&-\lim_{\epsilon\downarrow 0}\frac{1}{\pi}\int_{\Rbb^3}d^3 k\left(\frac{1}{2}\left(\frac{1}{k^3}\right)_{\text{ren},M}+\frac{i|\tau_1-\tau_2|}{2k^2}\right)e^{-ik|\tau_1-\tau_2|}e^{i\vec{k}\left(\vec{x}-\vec{y}\right)}\,e^{-\epsilon k}\\
&=&\frac{1}{(2\pi)^{\frac32}}\lim_{\epsilon\downarrow 0}\int_{\Rbb^3}d^3 k \;
\text{flog}(\tau_1-\tau_2,k)\,e^{i\vec{k}\left(\vec{x}-\vec{y}\right)}\,e^{-\epsilon k}
\end{eqnarray*}
%
where the appearing regularization of the (tempered) distribution $\dfrac{1}{k^3}$ is 
%
\begin{equation}
\left(\frac{1}{k^3}\right)_{\text{ren},M}:=\lim_{m\to 0}\left(\frac{1}{(k^2+m^2)^{\frac32}}+\pi\log\left(\frac{e^{4\gamma}m^4}{4 M^4}\right)\delta(\vec{k})\right)
\label{eq:regk3}
\end{equation}
and
\begin{equation}
\text{flog}(\tau_1-\tau_2,k):=-\sqrt{8\pi}\left(\frac{1}{2}\left(\frac{1}{k^3}\right)_{\text{ren},M}+\frac{i|\tau_1-\tau_2|}{2k^2}\right)e^{-ik|\tau_1-\tau_2|}
\label{eq:flog}
\end{equation}
%
is the sought-for spatial Fourier transform of $\log \left(M^2 \sigma_{F,\Mbb}\right)$. 

%----------------------------------------------------------------------------%
\subsection{Two point function for a quartic potential up to second order}
%----------------------------------------------------------------------------%


In order to compute the analytic expressions corresponding to the graphs in figure \ref{fig:2pf}, we may use the Fourier versions of the appearing propagators \eqref{eq:propagators_fourier_exp}, \eqref{eq:fourier_square}, and the analogous expressions for 
%
\begin{equation*}
\widehat{\left(\Delta^2_{\fsf,0}\log\left(M^{-2}\Delta^2_{\fsf,0}\right) \right)_\ms}(\tau_1,\tau_2,k) \quad \mbox{and } \quad \widehat{\left(\Delta^3_{\fsf,0}\right)_\text{ms}}(\tau_1,\tau_2,k)
\end{equation*}
%
the explicit form of $\mu(x)=3\lambda w(x,x)$ in \eqref{eq:coinciding_w}, as well as \eqref{eq:fish_sunset_reg}, \eqref{eq:fish_sunset_reg_2} and the identities for products and convolutions \eqref{eq:def_convolutions}, \eqref{eq:convolution_identities}. Note that $\mu(x)$ is in fact only time-dependent because $\Omega$ was chosen homogeneous and isotropic. Thus the integrals with $\mu$-vertices can be computed partly with the above mentioned identities by means of 
%
\begin{equation*}
\widehat{(1\otimes \mu) \Delta_{\sharp}}(\tau_1,\tau_2,k)=\mu(\tau_2)\widehat{\Delta_{\sharp}}(\tau_1,\tau_2,k) \ , \quad\widehat{(\mu\otimes 1) \Delta_{\sharp}}(\tau_1,\tau_2,k)=\mu(\tau_1)\widehat{\Delta_{\sharp}}(\tau_1,\tau_2,k) \ . 
\end{equation*}
%
Similarly, the bubbles in the third line of Figure \ref{fig:2pf} contribute only time dependent vertex factors which can be computed as 
%
\begin{equation*}
h_\sharp(\tau):=\int_\Mcal d\tau_1 d^3x_1\; a(\tau_1)^4\mu(\tau_1)\Delta_\sharp(\tau,\tau_1,\vec{x}-\vec{x}_1)=\frac{1}{a(\tau)}\int_I d\tau_1\;a(\tau_1)^3 \mu(\tau_1)\widehat{\Delta_\sharp}(\tau,\tau_1,0) 
\end{equation*}
%
where $\Delta_\sharp$ is either $\Delta^2_+$ or $\left(\Delta^2_\fsf\right)_\ms$.


With these preparations, we can compute e.g. the first graphs of the fourth and fifth line in figure \ref{fig:2pf} in Fourier space as
%
\begin{eqnarray*}
&& \widehat{\Delta_R\ast_4((h_\fsf \otimes 1) \Delta_+)} \ = \ \widehat{\Delta_R}\ast_1\widehat{\left((h_F\otimes 1)\Delta_+)\right)} \\
&=& \int_{I^2} d\tau_3\,d\tau_4\; a(\tau_3)a(\tau_4)^3\mu(\tau_4)\widehat{\Delta_R}(\tau_1,\tau_3,k)\widehat{\Delta_+}(\tau_3,\tau_2,k)\widehat{(\Delta^2_F)_\text{ms}}(\tau_3,\tau_4,0)
\end{eqnarray*}
%
and
%
\begin{eqnarray*}
&& \widehat{\Delta_R\ast_4(\Delta_F)^3_\ms\ast_4\Delta_+} \ = \ \widehat{\Delta_R}\ast_1\widehat{(\Delta_F)^3_\text{ms}}\ast_1\widehat{\Delta_+} \\
&=& \int_{I^2} d\tau_3\,d\tau_4\; a(\tau_3)^2a(\tau_4)^2\widehat{\Delta_R}(\tau_1,\tau_3,k)\widehat{(\Delta^3_F)_\text{ms}}(\tau_3,\tau_4,k)\widehat{\Delta_+}(\tau_4,\tau_2,k) \ .
\end{eqnarray*}


%----------------------------------------------------------------------------%
\subsection{More complicated graphs on cosmological spacetimes}
%----------------------------------------------------------------------------%


In order to compute the Fourier transforms of more complicated graphs on FLRW spacetimes, one can use a strategy generalising the one employed in section \ref{p:SUNSET_FISH_FLWR}. Namely, one again decomposes the Feynman propagator $\Delta_\fsf$ into several pieces which capture the relevant singularities and can be expressed in terms of the conformal vacuum Feynman propagator $\Delta_{\fsf,0}$ whose explicit form in position and Fourier space is well known in contrast to the form of $\sigma$ itself. The corresponding decomposition of general Feynman amplitudes $\tsf_\gamma$ is straightforward. The only non--trivial step is to generalise proposition \ref{prop:equivalent_scheme} to the case of general amplitudes, i.e. to compute the difference between the minimal subtraction scheme used in conjunction with either analytically regularizing powers of $\sigma$ directly or analytically regularizing powers of the full propagator $\Delta_{\fsf,0}$. However, we do not foresee any problems in obtaining such a generalisation by proving versions of lemma \ref{lem:rho_over_squared} and proposition \ref{prop:almost_homo} for $\Delta_{\fsf,0}$ rather than $\sigma$.


In fact, one can also skip this last step by taking a rather pragmatic approach and working directly with the regularization scheme consisting of decomposition in $\Delta_{\fsf,0}$, analytic regularization of powers of this propagator and minimal subtraction of the principal parts. This scheme, clearly applicable only to conformally flat spacetimes, satisfies all properties proved in proposition \ref{prop:properties_scheme}, with two exceptions. It is not obvious whether the Principle of Perturbative agreement with respect to generalised mass perturbations holds for this scheme, whereas locality and covariance of course only hold in the sense restricted to conformally flat spacetimes. In this respect it is essential that the Feynman propagator of the conformal vacuum $\Delta_{\fsf,0}$ on conformally flat spacetimes is manifestly ``geometric'', because the corresponding propagator of the massless Minkowski vacuum has this property.


%----------------------------------------------------------------------------%
\section{Stress energy tensor}
%----------------------------------------------------------------------------%


(blablabla)


%----------------------------------------------------------------------------%


\chapter*{Conclusion}


\addcontentsline{toc}{chapter}{Conclusion}


%----------------------------------------------------------------------------%


(blablabla)


%----------------------------------------------------------------------------%


\newpage


\vspace*{100pt}


\thispagestyle{empty}


\chapter*{Acknowledgements}


\addcontentsline{toc}{chapter}{Acknowledgements}


%----------------------------------------------------------------------------%


(blablabla)


%Nicola
%Thomas
%Jean-Christophe
%Pierre
%Daniel
%Nicolò


%----------------------------------------------------------------------------%


\nocite{*}


\bibliographystyle{abbrv}


\bibliography{biblio}


\addcontentsline{toc}{chapter}{Bibliography}


%----------------------------------------------------------------------------%


\printindex


%============================================================================%
\end{document}
%============================================================================%