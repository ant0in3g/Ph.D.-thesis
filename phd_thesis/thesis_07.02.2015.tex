%============================================================================%
% Antoine Gé́ré (gereantoine@gmail.com).
%============================================================================%


% LaTeX environment used : Kile, available at : http://kile.sourceforge.net/
%
% Package used : available at http://www.ctan.org/
%
% The comprehensive latex symbole list : available at http://www.ctan.org/tex-archive/info/symbols/comprehensive/
% and Detexif, an attempt to simplify to search in the list : available at http://detexify.kirelabs.org/classify.html
%
% Bibliography done with Zotero , available at : https://www.zotero.org/
% Bibtex style ????? .


%----------------------------------------------------------------------------%

\documentclass[10pt]{book}

%----------------------------------------------------------------------------%

%a
\usepackage{amscd}
\usepackage{amsmath}
\usepackage{amsfonts}
\usepackage{amssymb}
\usepackage{amsxtra}
\usepackage{array} 
%b
\usepackage[english]{babel}
%c
\usepackage{cite}
\usepackage{color}
%e
\usepackage{enumitem}
%f
\usepackage{fancyhdr}
\usepackage{filecontents}
\usepackage[T1]{fontenc}
%g
\usepackage{geometry}
%h
\usepackage{hyperref}
%i
\usepackage[totoc]{idxlayout}
\usepackage[utf8]{inputenc}
%m
\usepackage{makeidx}
%n
\usepackage[thmmarks]{ntheorem}
%q
\usepackage[sfdefault]{quattrocento}
%t
\usepackage{tikz}
%u
\usepackage{upgreek}

%----------------------------------------------------------------------------%

\begin{filecontents}{biblio.bib}
%
%
@article{Gere:2015qsa,
      author         = "Géré, Antoine and Hack, Thomas-Paul and Pinamonti, Nicola",
      title          = "An analytic regularisation scheme on curved spacetimes with applications to cosmological spacetimes",
      year           = "2015",
      eprint         = "1505.00286",
      archivePrefix  = "arXiv",
      primaryClass   = "math-ph",
      SLACcitation   = "%%CITATION = ARXIV:1505.00286;%%",
}
%
%
@article{Duetsch:2002yp,
      author         = "Duetsch, Michael and Fredenhagen, Klaus",
      title          = "The Master Ward Identity and generalized Schwinger-Dyson equation in classical field theory",
      journal        = "Commun.Math.Phys.",
      volume         = "243",
      pages          = "275-314",
      doi            = "10.1007/s00220-003-0968-4",
      year           = "2003",
      eprint         = "hep-th/0211242",
      archivePrefix  = "arXiv",
      primaryClass   = "hep-th",
      reportNumber   = "DESY-02-211",
      SLACcitation   = "%%CITATION = HEP-TH/0211242;%%",
}
%
%
@article{Brunetti:2012ar,
      author         = "Brunetti, Romeo and Fredenhagen, Klaus and Ribeiro, Pedro
                        Lauridsen",
      title          = "Algebraic Structure of Classical Field Theory I: Kinematics and Linearized Dynamics for Real Scalar Fields",
      year           = "2012",
      eprint         = "1209.2148",
      archivePrefix  = "arXiv",
      primaryClass   = "math-ph",
      SLACcitation   = "%%CITATION = ARXIV:1209.2148;%%",
}
%
%
@article{Bar:2007zz,
      author         = "Bar, Christian and Ginoux, Nicolas and Pfaffle, Frank",
      title          = "Wave equations on Lorenzian manifolds and quantization",
      journal        = "ESI Lectures in Mathematical Physics. Zürich: European Mathematical Society",
      pages          = "1-199",
      doi            = "10.4171/037",
      year           = "2007",
      eprint         = "0806.1036",
      archivePrefix  = "arXiv",
      primaryClass   = "math.DG",
      SLACcitation   = "%%CITATION = INSPIRE-773698;%%",
}
%
%
@book{waldGR,
      author         = "Wald, R.M.",
      title          = "General Relativity",
      publisher      = "University of Chicago Press",
      year           = "2010",
      isbn           = "9780226870373",
      url            =  "http://www.worldcat.org/isbn/0226870332",
}
%
%
@article{Khavkine:2014mta,
      author         = "Khavkine, Igor and Moretti, Valter",
      title          = "Algebraic QFT in Curved Spacetime and quasifree Hadamard states: an introduction",
      year           = "2014",
      eprint         = "1412.5945",
      archivePrefix  = "arXiv",
      primaryClass   = "math-ph",
      SLACcitation   = "%%CITATION = ARXIV:1412.5945;%%",
}
%
%
@article{Hack:2010iw,
      author         = "Hack, Thomas-Paul",
      title          = "On the Backreaction of Scalar and Spinor Quantum Fields in Curved Spacetimes",
      year           = "2010",
      eprint         = "1008.1776",
      archivePrefix  = "arXiv",
      primaryClass   = "gr-qc",
      reportNumber   = "DESY-THESIS-2010-042",
      SLACcitation   = "%%CITATION = ARXIV:1008.1776;%%",
}
%
%
\end{filecontents}

%----------------------------------------------------------------------------%

\geometry{
a4paper,
left=20mm,
right=20mm,
top=20mm,
bottom=20mm,
}

%----------------------------------------------------------------------------%

\setlength\parindent{0pt}

%----------------------------------------------------------------------------%

\makeindex

%----------------------------------------------------------------------------%

\renewcommand{\headrulewidth}{0pt}
\renewcommand{\footrulewidth}{0pt}

\setlength{\headheight}{22pt} 

\pagestyle{fancy}
%
\renewcommand{\chaptermark}[1]{ \markboth{#1}{} }
\renewcommand{\sectionmark}[1]{ \markright{#1} }
%
\fancyhf{}
\fancyhead[LE,RO]{\thepage}
\fancyhead[RE,CE]{}
\fancyhead[LO,CO]{}

\fancypagestyle{plain}{ %
\fancyhf{}
}

%----------------------------------------------------------------------------%

\newcommand{\supp}{\mathsf{supp}}
\newcommand{\WF}{\mathsf{WF}}

\newcommand{\abs}[1]{\left|#1\right|}
\newcommand{\sm}[1]{\left\langle#1\right\rangle}

\renewcommand{\det}{\mathsf{det}}

%----------------------------------------------------------------------------%

\newcommand{\Acal}{\mathcal{A}}
\newcommand{\Bcal}{\mathcal{B}}
\newcommand{\Ccal}{\mathcal{C}}
\newcommand{\Dcal}{\mathcal{D}}
\newcommand{\Ecal}{\mathcal{E}}
\newcommand{\Fcal}{\mathcal{F}}
\newcommand{\Gcal}{\mathcal{G}}
\newcommand{\Hcal}{\mathcal{H}}
\newcommand{\Ical}{\mathcal{I}}
\newcommand{\Jcal}{\mathcal{J}}
\newcommand{\Kcal}{\mathcal{K}}
\newcommand{\Lcal}{\mathcal{L}}
\newcommand{\Mcal}{\mathcal{M}}
\newcommand{\Ncal}{\mathcal{N}}
\newcommand{\Ocal}{\mathcal{O}}
\newcommand{\Pcal}{\mathcal{P}}
\newcommand{\Qcal}{\mathcal{Q}}
\newcommand{\Rcal}{\mathcal{R}}
\newcommand{\Scal}{\mathcal{S}}
\newcommand{\Tcal}{\mathcal{T}}
\newcommand{\Ucal}{\mathcal{U}}
\newcommand{\Vcal}{\mathcal{V}}
\newcommand{\Wcal}{\mathcal{W}}
\newcommand{\Xcal}{\mathcal{X}}
\newcommand{\Ycal}{\mathcal{Y}}
\newcommand{\Zcal}{\mathcal{Z}}

%----------------------------------------------------------------------------%

\newcommand{\Abb}{\mathbb{A}}
\newcommand{\Bmbb}{\mathbb{B}}
\newcommand{\Cbb}{\mathbb{C}}
\newcommand{\Dbb}{\mathbb{D}}
\newcommand{\Ebb}{\mathbb{E}}
\newcommand{\Fbb}{\mathbb{F}}
\newcommand{\Gbb}{\mathbb{G}}
\newcommand{\Hbb}{\mathbb{H}}
\newcommand{\Ibb}{\mathbb{I}}
\newcommand{\Jbb}{\mathbb{J}}
\newcommand{\Kbb}{\mathbb{K}}
\newcommand{\Lbb}{\mathbb{L}}
\newcommand{\Mbb}{\mathbb{M}}
\newcommand{\Nbb}{\mathbb{N}}
\newcommand{\Obb}{\mathbb{O}}
\newcommand{\Pbb}{\mathbb{P}}
\newcommand{\Qbb}{\mathbb{Q}}
\newcommand{\Rbb}{\mathbb{R}}
\newcommand{\Sbb}{\mathbb{S}}
\newcommand{\Tbb}{\mathbb{T}}
\newcommand{\Ubb}{\mathbb{U}}
\newcommand{\Vbb}{\mathbb{V}}
\newcommand{\Wbb}{\mathbb{W}}
\newcommand{\Xbb}{\mathbb{X}}
\newcommand{\Ybb}{\mathbb{Y}}
\newcommand{\Zbb}{\mathbb{Z}}

%----------------------------------------------------------------------------%

\newcommand{\Arak}{\mathfrak{A}}
\newcommand{\Brak}{\mathfrak{B}}
\newcommand{\Crak}{\mathfrak{C}}
\newcommand{\Drak}{\mathfrak{D}}
\newcommand{\Erak}{\mathfrak{E}}
\newcommand{\Frak}{\mathfrak{F}}
\newcommand{\Grak}{\mathfrak{G}}
\newcommand{\Hrak}{\mathfrak{H}}
\newcommand{\Irak}{\mathfrak{I}}
\newcommand{\Jrak}{\mathfrak{J}}
\newcommand{\Krak}{\mathfrak{K}}
\newcommand{\Lrak}{\mathfrak{L}}
\newcommand{\Mrak}{\mathfrak{M}}
\newcommand{\Nrak}{\mathfrak{N}}
\newcommand{\Orak}{\mathfrak{O}}
\newcommand{\Prak}{\mathfrak{P}}
\newcommand{\Qrak}{\mathfrak{Q}}
\newcommand{\Rrak}{\mathfrak{R}}
\newcommand{\Srak}{\mathfrak{S}}
\newcommand{\Trak}{\mathfrak{T}}
\newcommand{\Urak}{\mathfrak{U}}
\newcommand{\Vrak}{\mathfrak{V}}
\newcommand{\Wrak}{\mathfrak{W}}
\newcommand{\Xrak}{\mathfrak{X}}
\newcommand{\Yrak}{\mathfrak{Y}}
\newcommand{\Zrak}{\mathfrak{Z}}

%----------------------------------------------------------------------------%

\newcommand{\Asf}{\mathsf{A}}
\newcommand{\Bsf}{\mathsf{B}}
\newcommand{\Csf}{\mathsf{C}}
\newcommand{\Dsf}{\mathsf{D}}
\newcommand{\Esf}{\mathsf{E}}
\newcommand{\Fsf}{\mathsf{F}}
\newcommand{\Gsf}{\mathsf{G}}
\newcommand{\Hsf}{\mathsf{H}}
\newcommand{\Isf}{\mathsf{I}}
\newcommand{\Jsf}{\mathsf{J}}
\newcommand{\Ksf}{\mathsf{K}}
\newcommand{\Lsf}{\mathsf{L}}
\newcommand{\Msf}{\mathsf{M}}
\newcommand{\Nsf}{\mathsf{N}}
\newcommand{\Osf}{\mathsf{O}}
\newcommand{\Psf}{\mathsf{P}}
\newcommand{\Qsf}{\mathsf{Q}}
\newcommand{\Rsf}{\mathsf{R}}
\newcommand{\Ssf}{\mathsf{S}}
\newcommand{\Tsf}{\mathsf{T}}
\newcommand{\Usf}{\mathsf{U}}
\newcommand{\Vsf}{\mathsf{V}}
\newcommand{\Wsf}{\mathsf{W}}
\newcommand{\Xsf}{\mathsf{X}}
\newcommand{\Ysf}{\mathsf{Y}}
\newcommand{\Zsf}{\mathsf{Z}}

\newcommand{\asf}{\mathsf{a}}
\newcommand{\bsf}{\mathsf{b}}
\newcommand{\csf}{\mathsf{c}}
\newcommand{\dsf}{\mathsf{d}}
\newcommand{\esf}{\mathsf{e}}
\newcommand{\fsf}{\mathsf{f}}
\newcommand{\gsf}{\mathsf{g}}
\newcommand{\hsf}{\mathsf{h}}
\newcommand{\isf}{\mathsf{i}}
\newcommand{\jsf}{\mathsf{j}}
\newcommand{\ksf}{\mathsf{k}}
\newcommand{\lsf}{\mathsf{l}}
\newcommand{\msf}{\mathsf{m}}
\newcommand{\nsf}{\mathsf{n}}
\newcommand{\osf}{\mathsf{o}}
\newcommand{\psf}{\mathsf{p}}
\newcommand{\qsf}{\mathsf{q}}
\newcommand{\rsf}{\mathsf{r}}
\newcommand{\ssf}{\mathsf{s}}
\newcommand{\tsf}{\mathsf{t}}
\newcommand{\usf}{\mathsf{u}}
\newcommand{\vsf}{\mathsf{v}}
\newcommand{\wsf}{\mathsf{w}}
\newcommand{\xsf}{\mathsf{x}}
\newcommand{\ysf}{\mathsf{y}}
\newcommand{\zsf}{\mathsf{z}}

%----------------------------------------------------------------------------%

\newcommand*{\makepagetitle}{%
%
\thispagestyle{empty}
%
{{\raggedright% 
%
\vspace*{44pt}%
%
{\LARGE Antoine Géré}\\[\baselineskip]% 
%
\vspace*{100pt}%
%
{\Huge\bfseries Algebraic and Noncommutative \\[8pt] approaches to Quantum Field Theory}\\[\baselineskip]%
%
\vspace*{22pt}%
%
{\LARGE Ph.D. thesis}\\[\baselineskip]% 
%
\vspace*{44pt}%
%
{\LARGE Dipartimento di Matematica}\\[\baselineskip]% 
%
{\LARGE Università degli Studi di Genova}\\[\baselineskip]% 
%
\vfill% 
%
\newpage%
%
\thispagestyle{empty}%
%
\ \vfill%
%
\textbf{Algebraic and Noncommutative approaches to Quantum Field Theory} \\
Ph.D. thesis submitted by \href{mailto:gere@dima.unige.it}{Antoine Géré} \\
Genova, ???? 2016 \\[8pt]
%
Dipartimento di Matematica \\
Università degli Studi di Genova \\[8pt]
%
Supervisor: \href{mailto:pinamont@dima.unige.it}{Prof. Dr. Nicola Pinamonti} \\
Examiner: ????
%
}}%
%
}%

%----------------------------------------------------------------------------%

\theoremclass{LaTeX}
\theoremstyle{break}
\theoremheaderfont{\normalfont\bfseries}
\theorembodyfont{\normalfont}
\theoremseparator{}
\theoremsymbol{\ensuremath{\blacktriangleright}}
\newtheorem{theorem}{Theorem}
\newtheorem{proposition}{Proposition}
\newtheorem{lemma}{Lemma}
\newtheorem{corollary}{Corollary}
\theoremsymbol{\ensuremath{\blacklozenge}}
\newtheorem{example}{Example}
\newtheorem{remark}{Remark}
\newtheorem{definition}{Definition}
\theoremsymbol{\ensuremath{\blacksquare}}
\newtheorem{proof}{Proof}
\qedsymbol{\ensuremath{_\blacksquare}}

%----------------------------------------------------------------------------%

\definecolor{hypercolor}{rgb}{0.1,0.2,0.6}

\hypersetup{     
 unicode=false,      
 pdftoolbar=true,    
 pdfmenubar=true,    
 pdffitwindow=true,  
 pdfstartview={FitH},
 pdftitle={PhD thesis},    
 pdfauthor={Antoine Géré},     
 pdfsubject={Mathematical Physics},
 pdfcreator={LaTeX},  
 pdfproducer={pdfTex},
 pdfkeywords={Algebraic Quantum Field Theory; Noncommutative Field Theory.},  
 pdfnewwindow=true,  
 colorlinks=true, 
 linkcolor=hypercolor, 
 urlcolor=hypercolor, 
 citecolor=hypercolor,
 filecolor=hypercolor,         
}

%============================================================================%
\begin{document}
%============================================================================%

\pagenumbering{Roman}

\makepagetitle

\newpage

%----------------------------------------------------------------------------%

\ \vfill

\begin{flushright}
to (blablabla) 
\end{flushright}

\vfill

%----------------------------------------------------------------------------%

\newpage

\ \vfill

\begin{flushright}
citation
\end{flushright}

\vfill

%----------------------------------------------------------------------------%

\newpage

\vspace*{100pt}

\thispagestyle{empty}

\section*{Abstract}

(blablabla)

%----------------------------------------------------------------------------%

\tableofcontents

%----------------------------------------------------------------------------%

\part*{Introduction}
\addcontentsline{toc}{part}{Introduction}
\pagenumbering{arabic}

%----------------------------------------------------------------------------%

(blablabla)

%----------------------------------------------------------------------------%
\part{Algebraic approach to quantum field theory}
%----------------------------------------------------------------------------%

%----------------------------------------------------------------------------%
\chapter{Spacetime}
%----------------------------------------------------------------------------%


The starting block for a physical theory is the notion of spacetime. Looked at space and time as an unique entity has been an important turning point in our understanding of ``the laws of nature''. Newton's physics treated them separately, but at the beginning of the last century Einstein introduced a completely new point of view of these two entities. In his theory of gravitation, the physical background (i.e. the spacetime) is an ``active actor''. Indeed gravitation roughly speaking can be viewed as a deformation of the spacetime. Therefore we shall introduce the notion of spacetime starting from the very beginning.


%----------------------------------------------------------------------------%
\section{From topology to manifold}
%----------------------------------------------------------------------------%

The most fundamental way to define a space is to use the notion of topology. It is concerned with the intrinsec properties of spaces.

\begin{definition}[Topological space] 
Let $\Xsf$ be a set. A topology on $\Xsf$ is a collection $\Tcal$ of subsets satisfying the three following axioms%
%
\begin{itemize}
\vspace*{-4pt}
\setlength{\itemsep}{-1pt}
\item \textbf{conventions on empty set} : $\emptyset , \ \Xsf \in \Tcal$ ;
\item \textbf{arbitrary union} : $U_i \in \Tcal \mbox{ for } i \in I \Longrightarrow \bigcup_{i\in I} U_i \in \Tcal$, where $I$ is an arbitrary index set ;
\item \textbf{finite intersection} : $U_1 , \dots , U_n \in \Tcal \Longrightarrow U_1 \cap \dots \cap U_n \in \Tcal$ .
\end{itemize}
%
\end{definition}


The pair $(\Xsf,\Tcal)$ is called a topological space. The element of $\Tcal$ are the open sets of $\Xsf$. We shall often omit to precise the topology $\Tcal$, and simply say that $\Xsf$ is a topological space. It is still general, it does not implement much structure. Nonetheless it already characterizes maps between different topological spaces.\par%


\begin{definition}[Continous maps and homeomorphism]
Let $\Xsf$ and $\Ysf$ be topological spaces. We consider a map $f : \Xsf \to \Ysf$. We say
%
\begin{itemize}
\vspace*{-4pt}
\setlength{\itemsep}{-1pt}
\item $f$ is \textbf{continous} if $f^{-1}(U) \subset X$ is open for every open $U \subset\Ysf$ ;
\item $f$ is a \textbf{homeomorphism} if $f$ is bijective and both $f$ and $f^{-1}$ are continous.
\end{itemize}
%
\end{definition}


In the purpose of implementing ``some physics'' on these spaces we need more than only the topological structure. A first step is to give meaning of notion of distance, also called metric.


\begin{definition}[Metric]
Let $\Xcal$ be a set. A metric on $\Xcal$ is a map $g : \Xcal \times \Xcal \to \Rbb_+$ such that for all $x, y, z \in \Xcal$,%
%
\begin{itemize}
\vspace*{-4pt}
\setlength{\itemsep}{-1pt}
\item \textbf{separation} : $g(x,y) = 0 \Leftrightarrow x=y$ ; 
\item \textbf{symetry} : $g(x,y) = g(y,x)$ ;
\item \textbf{triangle inequality} : $g(x,y) \leq g(x,z) + g(z,y)$ .
\end{itemize}
%
\end{definition}


A set $\Xcal$ endowed with a metric $g$ is called a metric space, and denoted $(\Xcal,g)$. We can show that a metric space is also a topological space for which the topology is induced by the metric. Let us detail this in the following lemma.


\begin{lemma}[Topological metric space]
Every metric space $(\Xcal,g)$ is also a topological space for which the topology is induced by the collection of open sets in $\Xcal(=\Xsf)$. We denote it by $(\Xsf,\Tcal_g)$.
\end{lemma}


\begin{proof}
%%TODO
Decide if i show the proof or not.
%We shall consider the following collection of open subsets in $\Xcal$.
%
%\begin{equation*}
%\Tcal_g = \left\{ B_i \right\}_{i \in I} = \left\{B_i = B(x_i,\epsilon) , i \in I \ \bigg| \ \forall x_i \in \Xcal , \ \exists \epsilon > 0, \ B(x_i,\epsilon) = \left\{ y \in \Xcal , g(x_i,y) < \epsilon \right\} \subset \Xcal \right\}
%\end{equation*}
%
%First observation is that $\emptyset$ and $\Xcal$ are open and contained in $\Tcal_g$. Indeed there are no points in the empty set, therefore it is true that $x \in \emptyset$ whenever $g(x,e) < 1$, $\forall e \in \emptyset$, and conservely every points $x$ belongs to $\Xcal$, therefore it is also true that $x \in \Xcal$ whenever $g(x,e) < 1$,  $\forall e \in \Xcal$.\par%
%
%Second observation is because $B_i$ is open $\forall i \in I$, then $\bigcup_{i\in I} B_i$ is open. Indeed if $e \in \bigcup_{i\in I} B_i$, then we can find a particular $i_1 \in I$ with $e \in V_{i_1}$. Since $B_{i_1}$ is open we can find $\epsilon > 0$ such that $x \in B_{i_1}$ whenever $g(x,e) < \epsilon$. And since $B_{i_1} \subseteq \bigcup_{i\in I} U_i$ we have $x \in \bigcup_{i\in I} B_i$ whenever $g(x,e) <e$. Thus $\bigcup_{i\in I} B_i$ is open and contained in $\Tcal_g$.\par%
%
%Third and last observation is that if $B_j$ is open $\forall j \in J$, then $\bigcap_{j\in J} B_j$ is open. Indeed if $e \in \bigcap_{j \in J} B_j$, then $e \in B_j$, $\forall j \in J$. Since $B_j$ is open we can find a $\epsilon_j > 0$ such that $x \in B_j$ whenever $g(x,e) < \epsilon_j$. We set $\epsilon = min_{j\in J} \epsilon_j (>0)$, then we have $x \in B_j$ whenever $d(x,e) < \epsilon$, $\forall j \in J$. Thus $x \in \bigcap_{j\in J} B_j$ whenever $d(x,e) < \epsilon$, and therefore $\bigcap_{j\in J} B_j$ is open and contained $\Tcal_g$.\par%
\end{proof}


Let us give some generic definition which shall appear to be useful later on. $\Xsf$ shall always denote a topological space with topology $\Tcal$, and $\Zsf$ a subspace of $\Xsf$.%


\begin{enumerate}
%
\vspace*{-4pt}
\setlength{\itemsep}{-1pt}%
%
\item $\overline{\Zsf}$, the \textbf{closure} of $\Zsf \subset \Xsf$, is the intersection of all closed sets\footnote{A set $\Csf \subset \Xsf$ is \textbf{closed} if $\Xsf \backslash \Csf$ is open.} containing $\Zsf$.% 
%
\item $\Zsf$ is \textbf{dense} in $\Xsf$ if $\overline{\Zsf} = \Xsf$.%
%
\item $\Bcal \subset \Tcal$ is \textbf{topological basis} if every elements in $\Tcal$ can be written as the union of elements of the bais $\Bcal$.%
%
\item $\Xsf$ is \textbf{second countable} if it has a countable\footnote{A set is said to be \textbf{countable} if there exists a one to one correspondence between the set considered and the set of natural numbers.} topological basis.%
%
\item $\Xsf$ is \textbf{separable} if there exists a dense and countable subset.%
%
\item $\Ksf \subset \Xsf$ is \textbf{compact} if any covering of it admit a finite subcovering.%
%
\item $\Xsf$ is \textbf{locally compact} if every point in $\Xsf$ admit a neighborhood which has compact closure.%
%
\item $\Xsf$ is \textbf{connected} if it cannot be written as a disjoint union of two nonempty open subsets.%
%
\item $\Xsf$ is \textbf{Hausdorff} if every pair of points have disjoint neighborhood.%
%
\item $\Xsf$ is \textbf{paracompact} if every open cover has a refinement covering that is locally finite\footnote{$\left\{\Usf_\alpha\right\}$, a cover of $\Xsf$, is \textbf{locally finitte} if every points in $\Xsf$ has a neighborhood which has a nonempty intersection with a finite numbers of $\Usf_\alpha$.}.%
%
\item $\Xsf$ is said to be \textbf{metrizable} if there is a metric on $\Xsf$ for which the induced topology is $\Tcal$.
%
\end{enumerate}


We shall later one consider only Hausdorff topological spaces which shall also be metric spaces. Therefore it is interesting to consider the following lemma.


\begin{lemma}[Hausdorff metric space]
Every metric space is Hausdorff. 
\end{lemma}


\begin{proof}
%%TODO
Decide if i show the proof or not.
%Let $x$ and $y$ be two points of a topological space $\Xsf$, and $U_x$ (respectively $U_y$) a neighborhood of $x$ (respectively of $y$). We set the radius of the two neighborhoods as $g(x,y)/2$. We suppose now that $U_x$ and $U_y$ are not disjoint, then we can always find a point $z$ which belong in both neighborhood. Using the triangle inequality an the fact that $g(x,z) < g(x,y)/2$ and $g(x,z) < g(x,y)/2$, we get that $g(x,y) < g(x,y)$ which is a contradiction. Therefore the two neighborhoods are disjoint and every metric space is Hausdorff.
\end{proof}


For later puposes we give an equivalence lemma between Hausdorff and secound countable, and metrizable and separable spaces.

\begin{lemma}
Let $\Xsf$ be a topological spce of dimension $n$ in which all points admit a neighborhood homeomorphic to an open set in $\Rbb^n$. Then the following two properties are equivalent
%
\begin{enumerate}
\vspace*{-4pt}
\setlength{\itemsep}{-1pt}
\item $\Xsf$ is Hausdorff and second countable ;
\item $\Xsf$ is metrizable and separable.
\end{enumerate}
%
\end{lemma}


\begin{proof}
%%TODO
Decide if i show the proof or not.
\end{proof}


We have now enough background to introduce the fundamental notion of manifold. We start with topological manifolds and will implement differential structure in the next section.


\begin{definition}[Topological manifold]
A topological manifold $\Mcal$ of dimension $n$ is a Hausdorff and second countable topological space in which every points admit an open neighborhood homeomorphic to a subset of $\Rbb^n$.
\end{definition}

Due to the previous lemma, instead of asking to $\Mcal$ to be Hausdorff and second countable, we can ask to be separable and metrizable. But these properties are global, what is important is that $\Mcal$ admit locally the same topological properties as $\Rbb^n$.\par%


\bigskip


An important tool to pass from a local to a global point of view is the \textbf{partition of unity} of $\Xsf$. It is a collection of continous maps $(\phi_i)_{i \in I}$, $\phi_i : \Xsf \to [0,1]$, such that the collection of their supports is locally finite, and that $\sum_{i\in I} \phi_i = 1$. We call, $(\phi_i)_{i \in I}$, partition of unity subordinate to an open cover $U=(U_i)_{i \in I}$, if for all $i \in I$, the support of $\phi_i$ is contained in $U_i$.


\begin{lemma}
A topological manifold $\Mcal$ is paracompact and admit a partition of unity subordinate to every open cover of $\Mcal$.
\end{lemma}


\begin{proof}
%%TODO
Decide if i show the proof or not.
\end{proof}


%----------------------------------------------------------------------------%
\section{Differential structure}
%----------------------------------------------------------------------------%

We gave the definition of a topological manifold in the previous section. We want now to implement differential structure in order to be able to define what is a smooth manifold.

\vskip4pt

As we know a topological manifold $\Mcal$ of dimension $n$ is a Hausdorff and second countable topological space or equivalently metrizable and separable, and furthermore we require that every points in $\Mcal$ admit an open neighborhood homeomorphic to a subset of $\Rbb^n$. 

\vskip4pt

We provide every points $x$ of $\Mcal$ with a chart $(U,\phi)$. $U \subset \Mcal$ is called the coordinate neighborhood, and $\phi$ is a homeomorphism from $U$ to a subset of $\Rbb^n$ and is represented by $n$ functions $(x_1,\dots,x_n)$. A set of chat $\left\{ (U, \phi) \right\}$ which covers $\Mcal$ is called an atlas on $\Mcal$. In the case of two overlapping coordinate neighborhoods $U$ and $V$, if the transition function $\phi \circ \psi^{-1}$ is smooth, we said that the atlas is smooth. A smooth atlas $A$ is called maximal if it is not properly contained in a lager smooth atlas of $\Mcal$.

%%TODO
\bigskip
\begin{tikzpicture}[thick,scale=0.7] 
\draw[color=black] (0,0) to [out=50,in=190] (6,3);
\draw[color=black] (6,3) to [out=10,in=90] (10,0);
\draw[color=black] (10,0) to [out=170,in=30] (3,-3);
\draw[color=black] (3,-3) to [out=90,in=10] (0,0);
\filldraw (3.5,-2) circle (0pt) node[above] {$\Mcal$}; 
\filldraw (5,1) circle (2pt) node[above] {$x$};
%
\draw[->] (5.5,1.5) to [out=50,in=150] (17.5,1.5);
\filldraw (11,3.7) circle (0pt) node[above] {$f$};
%
\draw[color=black] (14,0) to [out=50,in=190] (19,3);
\draw[color=black] (19,3) to [out=10,in=170] (23,0);
\draw[color=black] (23,0) to [out=170,in=30] (16,-3);
\draw[color=black] (16,-3) to [out=70,in=10] (14,0);
\filldraw (17,-2) circle (0pt) node[above] {$\Ncal$}; 
\filldraw (18,1) circle (2pt) node[above] {$y$};
%
\end{tikzpicture}
\bigskip

\begin{lemma}[Uniquenes of a maximal atlas]
For a topological space $\Mcal$, every smooth atlas $A$ of $\Mcal$ is containes in an unique smooth maximal atlas.
\end{lemma}

\begin{proof}
%%TODO
Decide to show the proof or not.
\end{proof}

We can give now the definition of a smooth manifold.

.\begin{definition}[Smooth manifold]
$\Mcal$ is a smmooth manifold if $\Mcal$ is a topological manifold with a smooth maximal atlas $\left\{(U,\phi)\right\}$.
\end{definition}


\begin{enumerate}
\setlength{\itemsep}{-4pt}
\item topological manifold $\to$ VA BENE
\item (smooth) coordinate chart $\to$ VA BENE
\item (smooth) atlas $\to$ VA BENE
\item smooth manifold $\to$ VA BENE
\item smooth maps
\item diffeomorphism
\item fiber bundle
\item vector bundle
\item tangent space
\item tangent bundle
\item section
\item connection
\item curvature
\end{enumerate}


%----------------------------------------------------------------------------%
\section{Causality}
%----------------------------------------------------------------------------%




The invariant volume measure will be respectively denoted by  

\begin{equation*} 
 \dsf\mu x \ \doteq \ \sqrt{\abs{\det(\gsf)}} \ dx \ .
\end{equation*}



In general relativity the local causal structure is like the causal structure in flat spacetime of special relativity. It means that locally, to each point (event) $x\in\Mcal$, we can ``associate'' a light cone. We precise that we will refer to the light cone passing through the origin of $T_x\Mcal$ as the light cone of $x$, it is important to keep in mind that for us the light cone of an event is a subset of the tangent space of this event.
As usual we assign to one half of the cone the label ``future'' and to the other half ``past''. \par

In our definition of CST, we required a time orientable spacetime, the reason is that we want to be able to make a continous designation of ``future'' and ``past '' as $x$ varies over $\Mcal$. \par

The metric tensor field $\gsf$ at $x$ evaluated on a vector $v \in T_x\Mcal$ can be
\begin{description}
 \item $\gsf_x(v,v) > 0$, then $v$ is a timelike vector,
 \item $\gsf_x(v,v) = 0$, then $v$ is a null vector,
 \item $\gsf_x(v,v) < 0$, then $v$ is a spacelike vector.
\end{description}
A time like or null vector lying in the ``futur half'' of the light cone will be called future directed.

\begin{lemma}[]
Let $\Mcal$ be a time orientable spacetime. Then there exists a nonunique smooth nonvanishing timelike vector field on $\Mcal$.
\end{lemma}

\begin{corollary}[]
If a continous timelike vector field can be chosen on a spacetime $\Mcal$, then $\Mcal$ is time orientable.
\end{corollary}

The set of all timelike vectors is called the open lightcone 

\begin{equation*}
 \Vcal=\Vcal^{+} \ \dot{\cup} \ \Vcal^{-} \ . 
\end{equation*}

It is the disjoint union of two connected components, which we refer to as the forward and backward lightcones, 

\begin{equation*}
\Vcal^{\pm}=\left\{ x\in\Mcal \ | \ x^{2}>0, \ \pm x^{0}>0 \right\} \ . 
\end{equation*}

We denote by $\overline{\Vcal^{\pm}}$ and $\partial\Vcal^{\pm}$ the closure and boundary of these sets, respectively. \par

\begin{definition}[]
A vector $v \in T_x\Mcal$ is \textbf{future} (respectively \textbf{past}) \textbf{directed} if it is timelike or lightlike and $v \in \Vcal^+$ (respectively $v \in \Vcal^-)$. \\[3pt]
A differentiable curve $\gamma(\lambda)$ is said to be 

\begin{description}
\item a \textbf{future} (respectively \textbf{past}) \textbf{directed timelike curve} if at each point $x(\lambda) \in \gamma$ the tangent vector $v$ is a future (respectively past) directed timelike vector ;
\item a \textbf{future} (respectively \textbf{past}) \textbf{directed causal curve} if at each point $x(\lambda) \in \gamma$ the tangent vector $v$ is either a future (respectively past) directed timelike or null vector. 
\end{description} 

\end{definition}

\begin{definition}[Chronological future/past]
The \textbf{chronological future} of $p \in M$, denoted by $I^{+}(p)$ is defined as the sets of events that can be reached by a future directed timelike curve starting from $p$,

\begin{equation*}
I^{+}(p) = \left\{ q \in M \; \bigg| \; \begin{array}{l} \text{There exists a future directed timelike curve $\lambda(t)$,} \\ \text{with $\lambda(0)=p$ and $\lambda(1)=q$} \end{array} \; \right\},
\end{equation*}

for any subset $S \subset M$, we define $I^{+}(S)$, by,

\begin{equation*}
I^{+}(S) \; = \; \bigcup_{p \in S} I^{+}(p). 
\end{equation*}

The \textbf{chronological past} of $p \in M$, denoted by $I^{-}(p)$ is defined as the sets of events that can be reached by a past directed timelike curve starting from $p$,

\begin{equation*}
I^{-}(p) = \left\{ q \in M \; \bigg| \; \begin{array}{l} \text{There exists a past directed timelike curve $\lambda(t)$,} \\ \text{with $\lambda(0)=p$ and $\lambda(1)=q$} \end{array} \; \right\},
\end{equation*}

we define $I^{-}(S)$, by,

\begin{equation*}
I^{-}(S) \; = \; \bigcup_{p \in S} I^{-}(p). 
\end{equation*}

for any subset $S \subset M$. 

\end{definition}

The causal future/past of an event of the spacetime is defined in the same way as the chronological future/past of this event.

\begin{definition}[Causal future/past] 
The \textbf{causal future} of $p \in M$, denoted by $J^{+}(p)$, is defined as the sets of events that can be reached a future directed causal curve starting from $p$,
\begin{equation*}
J^{+}(p) = \left\{ q \in M \; \bigg| \; \begin{array}{l} \text{There exists a future directed causal curve $\lambda(t)$,} \\ \text{with $\lambda(0)=p$ and $\lambda(1)=q$} \end{array} \; \right\},
\end{equation*}
for any subset $S \subset M$, we define $J^{+}(S)$, by,
\begin{equation*}
J^{+}(S) \; = \; \bigcup_{p \in S} J^{+}(p). 
\end{equation*}
The \textbf{causal past} of $p \in M$, denoted by $J^{-}(p)$, is defined as the sets of events that can be reached a past directed causal curve starting from $p$,
\begin{equation*}
J^{-}(p) = \left\{ q \in M \; \bigg| \; \begin{array}{l} \text{There exists a past directed causal curve $\lambda(t)$,} \\ \text{with $\lambda(0)=p$ and $\lambda(1)=q$} \end{array} \; \right\},
\end{equation*}
we define $J^{-}(S)$, by,
\begin{equation*}
J^{-}(S) \; = \; \bigcup_{p \in S} J^{-}(p). 
\end{equation*}
for any subset $S \subset M$. 
\end{definition}

We denote $\partial I^{+}$ the boundary of $I^+$, and in the same way we defined $\partial J^+$.

\begin{definition}[Closed achronal set]
A subset $S \subset M$ is said to be achronal if there do not exist $p, q \in S$ such that $q \in I^{+}(p)$, i.e., if $I^{+}(S) \bigcup S = \emptyset$. 
\end{definition}

\begin{definition}[Domains of Dependance]
We define the \textbf{future domain of dependence} of $S$, denoted by $D^{+}(S)$, by
\begin{equation*}
 D^{+}(S) = \left\{ p \in M \; \bigg| \; \begin{array}{l} \text{Every past inextendible causal curve} \\ \text{through p intersects $S$} \end{array} \; \right\}.
\end{equation*}
We define the \textbf{past domain of dependence} of $S$, denoted by $D^{-}(S)$, by
\begin{equation*}
 D^{-}(S) = \left\{ p \in M \; \bigg| \; \begin{array}{l} \text{Every future inextendible causal curve} \\ \text{through p intersects $S$} \end{array} \; \right\}.
\end{equation*}
The (full) \textbf{domain of dependence} of $S$, denoted by $D(S)$, is defined as,
\begin{equation*}
D(S) \; = \; D^{+}(S) \; \cup \; D^{-}(S).
\end{equation*}
The set $S$ is a closed, achronal set (possibly with edge). 
\end{definition}

\begin{definition}[Cauchy surface]
A closed achronal set $\Sigma$ for which $D(\Sigma) = M$ is called a Cauchy surface. 
\end{definition}

A spacetime $(\Mcal,\gsf)$ which possesses Cauchy surface is said to be globally hyperbolic. \par


We have enough background to define a curved spacetime.

\begin{definition}[Curved spacetime]
A pair $(\Mcal,g)$ is a curved space time if $\Mcal$ is a $n$\footnote{$n>2$} dimensional, smmooth, and connected manifold, endowed with a Lorentzian metric of signature $( - + \dots +)$. The spacetime is required to be orientable, time orientable, and global hyperbolic. 
\end{definition}


%----------------------------------------------------------------------------%
\chapter{Free theory}
%----------------------------------------------------------------------------%


%%TODO
You should say something about the theory you want to analyze.
Namely you should day that you want to describe quantum field theories on curved backgrounds. You will restrict your attention to scalar field only.  Later you will treat interaction perturbatively.


%----------------------------------------------------------------------------%
\section{Functional approach}
%----------------------------------------------------------------------------%

%----------------------------------------------------------------------------%
\subsection{Off shell configuration space}
%----------------------------------------------------------------------------%

We shall now describe the mathematical elements necessary to describe quantum field theories on a curved spacetime $\Mcal$. % 

%
One starts to define the off shell space of configurations $\Ecal(\Mcal)$ as the set of all smooth real sections on a fiber $E$, $\phi \in \Ccal^\infty\left(\Mcal\right)$. For now we do not implement any dynamic, therefore we shall not put any further restriction on the field configurations. %
%
For later purposes we shall introduce the space of compactly supported smooth sections in $E$, $\Dcal(\Mcal)$.\par% 
%
\begin{definition}[Off shell configuration space]
The off shell configuration space over $\Mcal$ is 
%
\begin{equation*}
\Ecal(\Mcal) = \left\{ \phi \in \Ccal^\infty, \  \phi : \Mcal \to E \right\} \ .
\end{equation*}
%
For $\phi$ smooth and compactly supported, the configuration space is denoted by $\Dcal(\Mcal) \subset \Ecal(\Mcal)$.
\end{definition}


%----------------------------------------------------------------------------%
\subsection{Functional observables}
%----------------------------------------------------------------------------%

Therefore we need a way to measure these physical properties, it is done by introducing the notion of observable. We call observable a functional which maps the fieds to complex numbers%
%
\begin{equation*}
\Fsf : \left\{
\begin{array}{ccc}
\Ecal(\Mcal) & \to     & \Cbb \\
\phi  & \mapsto & \Fsf(\phi)
\end{array}
\right. \ .
\end{equation*}
%
We shall work with particular sets of functionals, denoted $\Fcal_\sharp$, which will be defined using regularity and support properties. \par%


%%TODO
what is regularity good for?  what are the support properties useful for? 


\bigskip

Let us first of all characterize the regularity of a functional. We would like to chose $\Fcal_\sharp(\Mcal)$ such that it gives us ``smooth functionals''. Therefore we need a careful definition of differentiability.%
%
\begin{definition}[Functional derivatives] \label{def:func-deriv}
The $n$-th functional derivative of $\Fsf$ at $\phi\in\Ecal(\Mcal)$ with respect to the directions $\psi_1, \dots, \psi_n \in\Ecal(\Mcal)$ is defined as%
%
\begin{equation*}%
\Fsf(\phi)^{(n)}[\psi_1,\dots ,\psi_n] = \lim_{t \to 0} \ \frac{1}{t} \bigg( \Fsf(\phi_n + t \psi)^{(n-1)}[\psi_1,\dots ,\psi_{n-1}] - \Fsf(\phi)^{(n-1)}[\psi_1,\dots ,\psi_{n-1}] \bigg) \ ,
\end{equation*}
%
whenever the limit exists. We call it smooth if $\Fsf(\phi)^{(n)}[\psi_1,\dots ,\psi_n]$ exists as jointly continuous map from $\Ecal(\Mcal) \times \Ecal(\Mcal)^{\otimes n}$ to $\Cbb$, for every $n$. We denote by $\Fcal^\infty(\Mcal)$ the space of smooth functionals.

%%TODO
- what does it mean jointly continuous? say it below \\
- is it the Gateaux functional derivative?

\end{definition}
%
A direct consequence from the definition of smooth functionals given in \ref{def:func-deriv} is that $\Fsf(\phi)^{(n)}$ is a distribution of compact support on $\Mcal^n$.  Let us illustrate this definition via a simple example.%
\begin{example}
%
Here is the first two derivatives of a ``functional potential'' $\phi^4$. 
%
\begin{eqnarray*}
&& \Vsf(\phi) = \int \dsf x \ \sqrt{\abs{\det(\gsf)}} \ \frac{\lambda(x)}{4!} \phi(x)^4 \ ,\\
%
&& \Vsf(\phi)^{(1)}(x) = \frac{\lambda(x)}{3!} \phi(x)^3 \ , \qquad
%
\Vsf(\phi)^{(2)}(x,y) = \frac{\lambda(x)}{2!} \phi(x)^2 \delta(x,y) \ .
\end{eqnarray*}
%
\end{example}
%
Let us mention the following property.
%
\begin{proposition}[Leibniz formula]
The Leibniz formula still holds in this functional approach.
%
\begin{equation*}
\left(\Fsf \cdot \Gsf\right)(\phi)^{(n)}[\psi_1, \dots ,\phi_n] = \sum_{k=0}^{n} \binom{n}{k} \Fsf(\phi)^{(k)}[\psi_1, \dots , \psi_k] \Gsf(\phi)^{(n-k)}[\psi_1, \dots , \psi_{n-k}] \ .
\end{equation*}
%
\end{proposition}
%
%
We have already noticed that a derivative of a regular functional, e.g. $\Fsf(\phi)^{(1)}$, is a distribution. Thus to characterize its regularity we will use the notion of wave front set of a distribution, introduced in the first chapter.%
%%TODO 
I don’t understand the logic (pour le wave front set)

\bigskip

Let us now discuss the localization properties of an observable in a particular region of the spacetime $\Mcal$. In order to properly discuss this extent let us discuss the support of an observable. We define for that the spacetime support of an observable. It is the set of points $x \in \Mcal$, such that for all neighborhood $U_x$ of $x$, there is two fields $\phi$ and $\psi$, with the property $\supp\left(\psi\right) \subset U_x$, and $\Fsf(\phi+\psi) \neq \Fsf(\phi)$.
%
\begin{definition}[Spacetime support] \label{def:spacetime-supp}
The spacetime support of an observable $\Fsf \in \Fcal_\sharp$ is
%
\begin{equation*}
\supp(\Fsf) \doteq \left\{ x \in \Mcal \bigg| 
\begin{array}{l} 
\forall \ \mbox{neighborhood } U_x \mbox{ of } x, \ \exists \ \phi, \psi \in \Ecal(\Mcal), \\
\supp(\psi) \subset U_x, \mbox{ such that } \Fsf(\phi + \psi) \neq \Fsf(\phi).
\end{array}
\right\} \ .
\end{equation*}
%
\end{definition}
%
In other words a functional do not ``feel'' the fields which have support outside its own support.

We denote by $\Fcal_0(\Mcal)$ the functionals with compact spacetime support over $\Mcal$. We follow
\cite{Brunetti:2012ar} and endow $\Fcal_0(\Mcal)$ with the following algebraic structure.
%
\begin{itemize}
\item Sum : $(\Fsf+\Gsf)(\phi) = \Fsf(\phi) + \Gsf(\phi)$ ;
\item Multiplication by a scalar $z\in\Cbb$ : $(z \cdot \Fsf)(\phi) = z \Fsf(\phi)$ ;
\item Pointwise product : $(\Fsf \cdot \Gsf)(\phi) = \Fsf(\phi) \cdot \Gsf(\phi)$ ;
\item Involution : $\Fsf^\ast(\phi) = \overline{\Fsf(\phi)}$ ;
\item Unit : $\Ibb = \Fsf(\phi) = 1$.
\end{itemize}
%
A direct consequence is that $\Fcal_0(\Mcal)$ is a commutative unital $\ast$-algebra. And we can check that these algebraic operations do not modify the spacetime support \cite[Lemma 2.3.3]{Brunetti:2012ar}.%
%
\begin{lemma}[``Rigidity'' of the spacetime support] \label{lem:spacetime}
The above algebraic relations do preserve the spacetime support of a functional. In particular we have
%
\begin{itemize}
\item Sum : $\supp(\Fsf + \Gsf) \subseteq \supp(\Fsf) \cup \supp(\Gsf)$ ;
\item Pointwise product :  $\supp(\Fsf \cdot \Gsf) \subseteq \supp(\Fsf) \cap \supp(\Gsf)$ .
\end{itemize}
%
\end{lemma}
%
%
\begin{proof}
(blablabla)
\end{proof}
%
It has been proved in \cite[Lemma 2.3.8]{Brunetti:2012ar} that the spacetime support of a functional can be described by its first derivatives.%
%
\begin{lemma}[``Characterization'' of the spacetime support]
If the first derivative of $\Fsf\in\Fcal_0(\Mcal)$ exists, then
%
\begin{equation*}
\supp\left(\Fsf\right) = \overline{\bigcup_{\phi\in\Ecal(\Mcal)} \supp\left(\Fsf^{(1)}(\phi)\right)} \ ,
\end{equation*}
%
with $\supp\left(\Fsf^{(1)}(\phi)\right)$ the usual support of the distribution $\Fsf^{(1)}(\phi)$.
\end{lemma}
%
\begin{proof}
(blablabla) 
\end{proof}


\bigskip


We now have all the tools to carefully identify the space of functionals which have ``good'' working property. %
The simplest space is the regular space $\mathcal{F}_\mathsf{reg}(\Mcal)$, it is the space of all smooth functionals, with compactly sumported derivatives and having an empty wave front set. %
%
\begin{definition}[Space of regular functionals]
We define the space of regular functional as follow
%
\begin{equation*}
\Fcal_{\mathsf{reg}}(\Mcal) = \left\{ \Fsf(\phi) \ \bigg| \ \Fsf(\phi) \in \Fcal^\infty(\Mcal), \ \Fsf(\phi)^{(n)} \in \Ecal^\prime(\Mcal^{\otimes n}), \mbox{ and } \ \WF(\Fsf(\phi)^{(n)}) = \emptyset \right\} \ ,
\end{equation*}
%
with $\phi$ a test function, i.e. element of $\Ecal(\Mcal)$. 
\end{definition}
%
However it does not contain the interaction functionals, those functionals that we would like to work with. Therefore we have to impose a less restrictive condition on the wave front set, we set that the wave front set of $F^{(n)}$ does not intersect the set $\mathcal{M} \times (\overline{V^n_+} \cup \overline{V^n_-})$ where $\overline{V_\pm}$ denotes the closed forward and backward light cone, respectively. It forms the space of microcausal functional $\mathcal{F}_\mathsf{\mu c}(\Mcal)$.%
%
\begin{definition}[Space of microcausal functional]
We define the space of microcausal functional as follow
%
\begin{equation*}
\Fcal_{\mu\csf}(\Mcal) = \left\{ 
\Fsf(\phi) \ \bigg| \ 
\begin{array}{l}
\Fsf(\phi) \in \Fcal^\infty(\Mcal), \ \Fsf(\phi)^{(n)} \in \Ecal^\prime(\Mcal^{\otimes n}) \\
\mbox{ and } \ \WF(\Fsf^{(n)}(\phi)) \cap \left( \Mcal^n \times ( \overline{V^{n}_{+}} \cup \overline{V^{n}_{-}} ) \right)  = \emptyset 
\end{array}
\right\} \ .
\end{equation*}
%
\end{definition}
%
This space contains the interactions functionals but not only. For instance the regular functionals are still contained in it. The space which contains only the interaction functionals is called the local space $\mathcal{F}_\mathsf{loc}$. We define it as the space of microcausal functionals having as support for their derivatives the small diagonal, $d_n = \left\{ (x,\dots,x) \subset \Mcal^n \right\}$.%
%
\begin{definition}[Space of local functional]
The local functionals are a subspace of microcausal functionals $\Fcal_{\mathsf{\mu c}}(\Mcal)$ defined as follow
%
\begin{equation*}
\Fcal_{\mathsf{loc}}(\Mcal) = \left\{ \Fsf(\phi) \in \Fcal_{\mu\csf}(\phi) \ \bigg| \ \supp\left(\Fsf(\phi)^{(n)}\right) \subset d_n = \left\{ (x,\dots,x) \subset \Mcal^n \right\} \right\} \subset \Fcal_{\mu\csf}(\Mcal) \ .
\end{equation*}
%
\end{definition}
%
We can define $\Fcal_{\mathsf{loc}}(\Mcal)$ by imposing the additivity property. And in this case the definition \ref{def:spacetime-supp} becomes natural.
%
\begin{definition}[Additivity]
A functional $\Fsf(\phi) \in \Fcal_0(\Mcal)$ is said to be additive if for all $\phi, \psi, \chi \in \Ecal(\Mcal)$ and $\supp(\phi) \cap \supp(\chi) = \emptyset$ we have 
%
\begin{equation*}
\Fsf(\phi + \psi + \chi) = \Fsf(\phi + \psi) - \Fsf(\psi) + \Fsf(\psi + \chi) \ . 
\end{equation*}
%
\end{definition}
%
%
From this definition it follows
%
\begin{lemma}[Locality via the additivity condition]
If $\Fsf\phi)$ is additive, then
\begin{equation*}
\Fsf(\phi + \psi + \chi)^{(n)}[\gamma_1,...,\gamma_n] = \Fsf(\phi + \psi)^{(n)}[\gamma_1,...,\gamma_n] - \Fsf(\psi)^{(n)} + \Fsf(\psi + \chi)^{(n)}[\gamma_1,...,\gamma_n] \ . 
\end{equation*}
and in particular if furthermore $\WF\left(\Fsf(\phi)^{(n)}\right) \perp Td_n$, we have that the derivatives $F(\phi)^{(n)}$ have support on the small diagonal $d_n$. 
\end{lemma}
%
\begin{proof}
(blablabla)
\end{proof}

%
An interesting property for additive functional is the following one \cite[Lemma 2.3.5]{Brunetti:2012ar}.
%
\begin{lemma}[Decomposition of additive functionals]
Any additive functional $\Fsf(\phi)$ can be decomposed as a finite sum of additive functionals with arbitrarily small spacetime support.
\end{lemma}
%
\begin{proof}
proof
\end{proof}
%
The study of these additive functionals is motivated by the fact that the renormalization freedom will correspond to this type of term.


%----------------------------------------------------------------------------%
\section{Classical field theory}
%----------------------------------------------------------------------------%

\begin{itemize}
\item actions
\item euler lagrange
\item klein gordon equation
\item adv - ret fund. sol.
\item cauchy problem
\item propagator
\item poisson algebra
\end{itemize}

\vspace*{88pt}


After having introduce the fucntional approach which will be used here, we formulate the clasical field theory. We work with scalar fields on curved spacetime, therefore we have as equation of motion the generalised Klein Gordon eqation.%
%
\begin{equation}
\Psf \phi = \left( \Box + \xi \Rsf + m^2 \right) \phi = 0 \ , 
\label{eq:klein-gordon}
\end{equation}
%
with $m$ the (positive real) mass of the theory, $\xi \in \Rbb$, and $\Rsf$ the scalar curvature. We required in the case of vanishing curvature eqref{eq:klein-gordon} reduces to the Klein Gordon equation of the free scalar field theory on Minkowski spacetime. The case $\xi=0$ is called minimal coupling, and $\xi=\frac16$ the conformally coupling \cite[Appendix D]{waldGR}.\par%


\bigskip


The spacetime $\Mcal$ we considere is globally hyperbolic therefore the differential equation eqref{eq:klein-gordon} admit unique solution once we give sufficient data condition. It has been shown in \cite[section 3]{Bar:2007zz} that the operator $\Psf$ has unique retarded and advanced fundamental solutions. We will denote by $\Hsf_\asf$ (respectively $\Hsf_\rsf$) the fundamental advanced solution (respectively the retarded solution). 
%
\begin{equation*}
\supp\left( \Hsf_{\asf/\rsf} f \right) \subset J^{\pm}\left(\supp\left(f\right),\Mcal\right) \ , \ \ f \in \Ccal^\infty_0(\Mcal) \ . 
\end{equation*}
%



\begin{definition}[Action]
A map $\Scal$ such that
%
\begin{equation*}
\Scal : \Dcal(\Mcal) \to \Fcal_{\mathsf{loc}}(\Mcal) 
\end{equation*}
%
is an action if it fulfills the following rquirements.
%
\begin{itemize}
\item $f \mapsto S[f]$ is linear ;
\item $S[f]$ is real ;
\item $S[f]^\ast = S[f^\ast]$ ;
\item $\supp\left( S[f] \right) \subset \supp\left( f \right)$ .
\end{itemize}
%
\end{definition}

Two action $\Scal_1$ and $\Scal_2$ will be called equivalent when
%
\begin{equation*}
\supp\left( S_1[f] - S_2[f] \right) \subset \supp\left( \dsf f \right) 
\end{equation*}

%----------------------------------------------------------------------------%
\section{Quantization via formal deformation}
%----------------------------------------------------------------------------%

\begin{itemize}
\item definition (formal power series of functionals) $\Fcal_\sharp[[\hbar]]$
\item the noncommutative algebra
\item definition (Hadamard two point functions)
\item definition ($\star$ product)
\item definition ($\star$ algebra of off shell observables)
\item equivalent $\star$ product / algebra
\end{itemize}



%----------------------------------------------------------------------------%
\chapter{Interacting quantum field theory}
%----------------------------------------------------------------------------%

(blablabla)

%----------------------------------------------------------------------------%
\section{(blablabla)}
%----------------------------------------------------------------------------%

(blablabla)

%----------------------------------------------------------------------------%
\chapter{A regularisation sheme}
%----------------------------------------------------------------------------%

(blablabla)

%----------------------------------------------------------------------------%
\section{(blablabla)}
%----------------------------------------------------------------------------%

(blablabla)

%----------------------------------------------------------------------------%
\part{Noncommutative approach to field theory}
%----------------------------------------------------------------------------%

%----------------------------------------------------------------------------%
\chapter{(blablabla)}
%----------------------------------------------------------------------------%

(blablabla)

%----------------------------------------------------------------------------%
\section{(blablabla)}
%----------------------------------------------------------------------------%

(blablabla)

%----------------------------------------------------------------------------%

\part*{Conclusion}
\addcontentsline{toc}{part}{Conclusion}

%----------------------------------------------------------------------------%

\newpage
\vspace*{100pt}
\thispagestyle{empty}
\section*{Acknowledgements}

%----------------------------------------------------------------------------%

(blablabla) \index{(blablabla)}

%----------------------------------------------------------------------------%

\nocite{*}
\bibliographystyle{abbrv}
\bibliography{biblio}
\addcontentsline{toc}{chapter}{Bibliography}

%----------------------------------------------------------------------------%

\printindex

%============================================================================%
\end{document}
%============================================================================%