%============================================================================%
% Antoine Géré (gereantoine@gmail.com).
% Ph.D. thesis. (2013--2015)
% Analytic regularization of quantum field theories on curved backgrounds.
% --
% Università degli Studi di Genova.
% http://www.unige.it
% --
% Dipartimento di Matematica,
% Via Dodecaneso 35, 16146 Genova, Italia.
%============================================================================%
% LaTeX environment, Kile : available at http://kile.sourceforge.net/
% --
% Package : available at http://www.ctan.org/
% --
% The comprehensive latex symbole list : available at http://www.ctan.org/tex-archive/info/symbols/comprehensive/
% Detexif (an attempt to simplify the search in the latex symbole list) : available at http://detexify.kirelabs.org/classify.html
% --
% Bibliography with Zotero : available at http://www.zotero.org/
% --
% Latex font catalogue : available at http://www.tug.dk/FontCatalogue/
%============================================================================%

\documentclass[11pt]{book}

%----------------------------------------------------------------------------%

\usepackage{amscd}
\usepackage{amsmath}
\usepackage{amsfonts}
\usepackage{amssymb}
\usepackage{amsxtra}
\usepackage{array} 
\usepackage[english]{babel}
\usepackage{calligra}
\usepackage{cite}
\usepackage{color}
\usepackage{cutwin}
\usepackage{enumitem}
\usepackage{fancyhdr}
\usepackage{filecontents}
\usepackage[T1]{fontenc}
\usepackage{geometry}
\usepackage{hyperref}
\usepackage[utf8]{inputenc}
\usepackage[amsmath,amsthm,thmmarks]{ntheorem}
\usepackage{palatino}
\usepackage{tikz}
\usepackage{upgreek}
\usepackage{wrapfig}

%----------------------------------------------------------------------------%

%%%%%%%%%%%%%%%%%%%%%%%%%%%%%%%COMMENTS%%%%%%%%%%%%%%%%%%%%%%%%%%%%%%
\usepackage[normalem]{ulem}
\def\bcom{{\color{red}\bf}\def\ecom{}}
\newcommand{\com}[1]{{\color{red}\bf #1}}
\newcommand{\sbar}[1]{\sout{\color{red} #1}}
\newcommand{\hilight}[1]{\colorbox{yellow!80!black}{#1}}
%%%%%%%%%%%%%%%%%%%%%%%%%%%%%%%%%%%%%%%%%%%%%%%%%%%%%%%%%%%%%%%%%%%%%

%----------------------------------------------------------------------------%

%\begin{filecontents}{bibstyle.bst}
% 
%\end{filecontents}

%----------------------------------------------------------------------------%

\begin{filecontents}{biblio.bib}
@book{FR_2014,
	title = {Perturbative algebraic quantum field theory},
	author = {Fredenhagen, K. and Rejzner, K.},
	year = {2014},
	archivePrefix = {arXiv},
        eprint = {1208.1428},
        primaryClass  = {math-ph}
}

@book{PS_1995,
	title = {An Introduction To Quantum Field Theory},
	author = {Peskin, M. E. and Schroeder, D. V.},
        year = {1995},	
        publisher = {Westview Press},
}

@book{IZ_1980,
	title = {Quantum field theory},
        author = {Itzykson, C. and Zuber, J.-B.},	
	year = {1980},
        publisher = {McGraw-Hill International Book Co.}
}

@book{STEINMANN_1971,
	title = {Perturbation Expansions in Axiomatic Field Theory},
	author = {Steinmann, O.},
	year = {1971},
	publisher = {Berlin, New York, Springer-Verlag}
}

@book{MORETTI_2013,
	title = {Spectral Theory and Quantum Mechanics},
	author = {Moretti, V.},
	year = {2013},
	publisher = {Milan : Springer Verlag},
        doi = {10.1007/978-88-470-2835-7}
}

@book{HACK_2015,
	title = {Cosmological Applications of Algebraic Quantum Field Theory in Curved Spacetimes},
	author = {Hack, T.-P.},
	year = {2015-06-05},
	archivePrefix = {arXiv},
        eprint = {1506.01869},
        primaryClass  = {gr-qc},	
	publisher = {Springer Verlag}
}

@book{BOURBAKI_1987,
	title = {Topological Vector Spaces},
	author = {Bourbaki, N.},
	year = {1987},
	publisher = {Springer Berlin Heidelberg}
}

@book{BGP_2007,
	title = {Wave Equations on Lorentzian Manifolds and Quantization},
	author = {Baer, C. and Ginoux, N. and Pfäffle, F.},
	year = {2007},
	archivePrefix = {arxiv},
	eprint = {0806.1036},
	primaryClass ={math},
        %publisher = {Zurich, Switzerland : European Mathematical Society}
}

@book{WALD_1984,
	title = {General Relativity},
	author = {Wald, R. M.},
        year = {1984},
        publisher = {University of Chicago Press}
}

@book{HORMANDER_1990,
	title = {The Analysis of Linear Partial Differential Operators I},
	author = {Hörmander, L.},
	year = {1990},
	publisher = {Springer Berlin Heidelberg}
}

@book{CONNES_1994,
	title = {Noncommutative Geometry},
	author = {Connes, A.},
	year = {1994},
	publisher = {San Diego, Academic Press}
}

@book{ONEIL_1983,
	title = {Semi-Riemannian geometry: with applications to relativity},
	author = {O'Neill, B.},
	year = {1983},
        publisher = {Academic Press}
}

@book{MADORE_1999,
	title = {An introduction to noncommutative differential geometry and its physical applications},
	author = {Madore, J.},
	year = {1999},
        publisher = {Cambridge University Press}
}

@thesis{DEGNER_2013,
	title = {Properties of States of Low Energy on Cosmological Spacetimes},
	author = {Degner, A.},
	year = {2013},
	institution = {Universität Hamburg},
	type = {phdthesis}
}

@thesis{SCHLEMMER_2010,
	title = {Local Thermal Equilibrium on Cosmological Spacetimes},
	author = {Schlemmer, J.},
	year = {2010},
	institution = {Universität Leipzig},
	type = {phdthesis}
}

@thesis{BLASCHKE_2008,
	title = {Towards consistent non-commutative Gauge Theories},
	author = {Blaschke, D.},
	year = {2008},
        institution = {Universität Wien Fakultät für Physik},
	type = {phdthesis}
}

@thesis{LINDNER_2013,
	title = {Perturbative Algebraic Quantum Field Theory at Finite Temperature},
	author = {Lindner, F.},
        year = {2013},
	institution = {Universität Hamburg},
	type = {phdthesis}
}

@thesis{SIEMSSEN_2015,
	title = {The Semiclassical Einstein Equation on Cosmological Spacetimes},
	author = {Siemssen, D.},
	year = {2015},
	institution = {Università degli studi di Genova},
        type = {phdthesis},	
        archivePrefix = {arXiv},
        eprint = {1503.01826},
        primaryClass  = {math-ph}
}

@thesis{DANG_2013,
	title = {Renormalization of quantum field theory on curved space-times: a causal approach.},
	author = {Dang, N. V.},
	year = {2013},
	institution = {Université Paris-Diderot - Paris VII},
	type = {phdthesis},
	archivePrefix = {arXiv},
        eprint = {1312.5674},
        primaryClass  = {math-ph}
}

@thesis{HACK_2010,
	title = {On the Backreaction of Scalar and Spinor Quantum Fields in Curved Spacetimes - From the Basic Foundations to Cosmological Applications},
	author = {Hack, T.-P.},
	year = {2010},
	institution = {Universität Hamburg},
	type = {phdthesis},
	archivePrefix = {arXiv},
        eprint = {1008.1776},
        primaryClass  = {gr-qc}	
}

@thesis{KELLER_2010,
	title = {Dimensional Regularization in Position Space and a Forest Formula for Regularized Epstein-Glaser Renormalization},
	author = {Keller, K. J.},
	year = {2010},
	institution = {Universität Hamburg},
	type = {phdthesis},
	archivePrefix = {arXiv},
	eprint = {1006.2148},
	primaryClass  = {math-ph}
}

@article{GHP_2015,
	title = {An analytic regularisation scheme on curved spacetimes with applications to cosmological spacetimes},
	author = {Géré, A. and Hack, T.-P. and Pinamonti, N.},
	year = {2015},
	archivePrefix = {arxiv},
	eprint = {1505.00286},
	primaryClass = {math-ph}
}

@article{GW_2015,
	title = {Spectral theorem in noncommutative field theories: Jacobi dynamics},
	author = {Géré, A. and Wallet, J.-C.},	
	year = {2014},
	archivePrefix = {arxiv},
	eprint = {1402.6976},
        primaryClass = {math-ph},
        journal = {J. Phys. Conf. Ser.},
        volume = {634},
        pages = {012006},
        doi = {10.1088/1742-6596/634/1/012006}
}

@article{DFR_1994,
	title = {Spacetime quantization induced by classical gravity},
	author = {Doplicher, S. and Fredenhagen, K. and Roberts, J. E.},
	year = {1994}
	journal = {Physics Letters B},
	volume = {331},
	pages = {39--44},
	doi = {10.1016/0370-2693(94)90940-7}
}

@article{GJW_2015,
	title = {Noncommutative gauge theories on $\mathbb{R}^3_\lambda$: Perturbatively finite models},
	author = {Géré, A. and Jurić, T. and Wallet, J.-C.},
	year = {2015},
	archivePrefix = {arxiv},
	eprint = {1507.08086},
	primaryClass = {hep-th}
}

@article{VW_2013,
	title = {Noncommutative field theories on $\mathbb{R}^3_\lambda$: Towards UV/IR mixing freedom},
	author = {Vitale, P. and Wallet, J.-C.},
	year = {2013},
	archivePrefix = {arxiv},
	eprint = {1212.5131},
        primaryClass = {hep-th},
        journal = {Journal of High Energy Physics},	
	volume = {2013},
        pages = {115},
	doi = {10.1007/JHEP04(2013)115}
}

@article{WALLET_2009,
	title = {Derivations of the Moyal Algebra and Noncommutative Gauge Theories},
	author = {Wallet, J.-C.},
	year = {2009},
	archivePrefix = {arxiv},
	eprint = {0811.3850},
	primaryClass = {math-ph},
	journal = {Symmetry, Integrability and Geometry: Methods and Applications},
        volume = {5},
        pages = {013},
	doi = {10.3842/SIGMA.2009.013}
}

@article{DV_1999,
	title = {Lectures on graded differential algebras and noncommutative geometry},
	author = {Dubois Violette, M.},
	year = {1999},
	archivePrefix = {arxiv},
	eprint = {9912017},
	primaryClass = {math},
}

@article{VITALE_2014,
	title = {Noncommutative field theory on $\mathbb{R}^3_\lambda$},
	author = {Vitale, P.},
	year = {2014},	
	archivePrefix = {arxiv},
	eprint = {1406.1372},
	primaryClass = {hep-th},
	journal = {Fortschritte der Physik},
        volume = {62},
	pages = {825--834},
	doi = {10.1002/prop.201400037},
}

@article{HLSJ_2002,
	title = {Coherent State Induced Star-Product on $\mathbb{R}^3_\lambda$ and the Fuzzy Sphere},
	author = {Hammou, A. B. and Lagraa, M. and Sheikh-Jabbari, M. M.},
        year = {2002},
	archivePrefix = {arxiv},
	eprint = {0110291},
	primaryClass = {hep-th},
	journal = {Physical Review D},
	volume = {66},
        pages = {025025},
	doi = {10.1103/PhysRevD.66.025025}
}

@article{MVRS_2000,
	title = {Noncommutative perturbative dynamics},
	author = {Minwalla, S. and Raamsdonk, M. V. and Seiberg, N.},
	year = {2000},
	journal = {Journal of High Energy Physics},
	volume = {2},
	pages = {020},
	doi = {10.1088/1126-6708/2000/02/020}
}

@article{GBLMV_2002,
	title = {Infinitely many star products to play with},
	author = {Gracia-Bondia, J. M. and Lizzi, F. and Marmo, G. and Vitale, P.},
	year = {2002},
	archivePrefix = {arxiv},
	eprint = {0112092},
	primaryClass = {hep-th},
	journal = {Journal of High Energy Physics},
	volume = {4},
	pages = {026--026},
	doi = {10.1088/1126-6708/2002/04/026}
}

@article{GVW_2014,
	title = {Quantum gauge theories on noncommutative 3-d space},
	author = {Géré, A. and Vitale, P. and Wallet, J.-C.},
	year = {2014},
	archivePrefix = {arxiv},
	eprint = {1312.6145},
	primaryClass = {hep-th},
	journal = {Physical Review D},
	volume = {90},
        pages = {045019},
	doi = {10.1103/PhysRevD.90.045019}
}

@article{DF_2004,
	title = {Causal perturbation theory in terms of retarded products, and a proof of the Action Ward Identity},
	author = {Duetsch, M. and Fredenhagen, K.},
	year = {2004},
	archivePrefix = {arxiv},
	eprint = {0403213},
	primaryClass = {hep-th},
	journal = {Reviews in Mathematical Physics},
	volume = {16},
	pages = {1291--1348},
	doi = {10.1142/S0129055X04002266},
}

@article{DFKR_2014,
	title = {Dimensional Regularization in Position Space, and a Forest Formula for Epstein-Glaser Renormalization},
	author = {Duetsch, M. and Fredenhagen, K. and Keller, K. J. and Rejzner, K.},
	year = {2014},
	archivePrefix = {arxiv},
	eprint = {1311.5424},
	primaryClass = {hep-th},
	journal = {Journal of Mathematical Physics},
	volume = {55},
	pages = {122303},
	doi = {10.1063/1.4902380}
}

@article{LUSCHER_1982,
	title = {Dimensional regularisation in the presence of large background fields},
	author = {Lüscher, M},
	year = {1982},
	journal = {Annals of Physics},
	volume = {142},
	pages = {359--392},
	doi = {10.1016/0003-4916(82)90076-8}
}

@article{BUNCH_1981,
	title = {BPHZ renormalization of $\lambda\phi^4$ field theory in curved spacetime},
	author = {Bunch, T. S.},
	year = {1981},
	journal = {Annals of Physics},
	volume = {131},
	pages = {118--148},
	doi = {10.1016/0003-4916(81)90187-1},
}

@article{ALLEN_1985,
	title = {Vacuum states in de Sitter space},
	author = {Allen, B.},
	year = {1985},
	journal = {Physical Review D},
	volume = {32},
	pages = {3136--3149},	
	doi = {10.1103/PhysRevD.32.3136}
}

@article{PS_2014,
	title = {Global Existence of Solutions of the Semiclassical Einstein Equation for Cosmological Spacetimes},
	author = {Pinamonti, N. and Siemssen, D.},
	year = {2014},
	archivePrefix = {arxiv},
	eprint = {1309.6303},
        primaryClass = {math-ph},
	journal = {Communications in Mathematical Physics},
	volume = {334},
	pages = {171--191},
	doi = {10.1007/s00220-014-2099-5}
}

@article{BFLR_2012,
	title = {Algebraic Structure of Classical Field Theory I: Kinematics and Linearized Dynamics for Real Scalar Fields},
	author = {Brunetti, R. and Fredenhagen, K. and Ribeiro, P. L.},
	year = {2012},
	archivePrefix = {arxiv},
	eprint = {1209.2148},
	primaryClass = {math-ph}
}

@article{WITTEN_1986,
	title = {Non-commutative geometry and string field theory},
	author = {Witten, E.},
	year = {1986},
	journal = {Nuclear Physics B},
	volume = {268},
	pages = {253--294},
	doi = {10.1016/0550-3213(86)90155-0}
}

@article{BDF_2009,
	title = {Perturbative Algebraic Quantum Field Theory and the Renormalization Groups},
	author = {Brunetti, R. and Duetsch, M. and Fredenhagen, K.},
	year = {2009},
	archivePrefix = {arxiv},
	eprint = {0901.2038},
	primaryClass = {math-ph},
	journal = {Adv. Theor. Math. Phys.},
	volume = {2009},
	pages = {1541--1599},
	doi = {10.4310/ATMP.2009.v13.n5.a7}
}

%@article{KAY_2006,
%	title = {Quantum field theory in curved spacetime},
%	author = {Kay, B. S.},
%	year = {2006},
%	archivePrefix = {arxiv},
%	eprint = {0601008}
%	primaryClass = {gr-qc},
%}

@article{HW_2001,
	title = {Local Wick Polynomials and Time Ordered Products of Quantum Fields in Curved Spacetime},
	author = {Hollands, S. and Wald, R. M.},
	year = {2001},
        archivePrefix = {arxiv},
	eprint = {0103074},
        primaryClass = {gr-qc},
	journal = {Communications in Mathematical Physics},
	volume = {223},
	pages = {289--326},
	doi = {10.1007/s002200100540}
}

@article{BF_1997,
	title = {Interacting Quantum Fields on a Curved Background},
	author = {Brunetti, R. and Fredenhagen, K.},
	year = {1997},
	archivePrefix = {arxiv},
	eprint = {9709011},
	primaryClass = {hep-th}
}

@article{HV_2012,
	title = {On the Stress-Energy Tensor of Quantum Fields in Curved Spacetimes - Comparison of Different Regularization Schemes and Symmetry of the Hadamard/Seeley - DeWitt Coefficients},
	author = {Hack, T.-P. and Moretti, V.},
	year = {2012},
	archivePrefix = {arxiv},
	eprint = {1202.5107},
	primaryClass = {gr-qc},
	journal = {Journal of Physics A: Mathematical and Theoretical},
	volume = {45},
	pages = {374019},
	doi = {10.1088/1751-8113/45/37/374019}
}

@article{PINAMONTI_2011,
	title = {On the initial conditions and solutions of the semiclassical Einstein equations in a cosmological scenario},
        author = {Pinamonti, N.},
	year = {2011},
	archivePrefix = {arxiv},
	eprint = {1001.0864},
	primaryClass = {gr-qc},
        journal = {Communications in Mathematical Physics},
	volume = {305},
	pages = {563--604},
	doi = {10.1007/s00220-011-1268-z}
}

@article{HW_2005,
	title = {Conservation of the stress tensor in perturbative interacting quantum field theory in curved spacetimes},
	author = {Hollands, S. and Wald, R. M.},
	year = {2005},
	archivePrefix = {arxiv},
	eprint = {0404074},
	primaryClass = {gr-qc},
	journal = {Reviews in Mathematical Physics},
	volume = {17},
	pages = {227--311},
	doi = {10.1142/S0129055X05002340}
}

@article{EG_1973,
	title = {The role of locality in perturbation theory},
	author = {Epstein, H. and Glaser, V.},
	year = {1973},
	journal = {Annales de l'institut Henri Poincaré (A) Physique théorique},
	volume = {19},
	pages = {211--295}
}

@article{HW_2003,
	title = {On the Renormalization Group in Curved Spacetime},
	author = {Hollands, S. and Wald, R. M.},
	year = {2003},
	archivePrefix = {arxiv},
	eprint = {0209029},
	primaryClass = {gr-qc},
	journal = {Communications in Mathematical Physics},
	volume = {237},
	pages = {123--160},
	doi = {10.1007/s00220-003-0837-1}
}

@article{RADZIKOWSKI_1996,
	title = {Micro-local approach to the Hadamard condition in quantum field theory on curved space-time},
	author = {Radzikowski, M. J.},
	year = {1996},
	journal = {Communications in Mathematical Physics},
	volume = {179},
	pages = {529--553},
	doi = {10.1007/BF02100096},
}

@article{BILAL_2013,
	title = {Multi-Loop Zeta Function Regularization and Spectral Cutoff in Curved Spacetime},
	author = {Bilal, A. and Ferrari, F.},
	year = {2013},
	archivePrefix = {arxiv},
	eprint = {1307.1689},
	primaryClass = {hep-th},
	journal = {Nuclear Physics B},
	volume = {877},
        pages = {956--1027},
	doi = {10.1016/j.nuclphysb.2013.10.003}
}

@article{TOMS_1982,
	title = {Renormalization of interacting scalar field theories in curved space-time},
	author = {Toms, D. J.},
	year = {1982},
	journal = {Physical Review D},
	volume = {26},
	pages = {2713--2729},
	doi = {10.1103/PhysRevD.26.2713},
}

@article{BF_2000,
	title = {Microlocal Analysis and Interacting Quantum Field Theories: Renormalization on Physical Backgrounds},
	author = {Brunetti, R. and Fredenhagen, K.},
	year = {2000},
	archivePrefix = {arxiv},
	eprint = {9903028},
	primaryClass = {math-ph},
	journal = {Communications in Mathematical Physics},
	volume = {208},
	pages = {623--661},
	doi = {10.1007/s002200050004}
}

@article{BFK_1996,
	title = {The microlocal spectrum condition and Wick polynomials of free fields on curved spacetimes},
	author = {Brunetti, R. and Fredenhagen, K. and Koehler, M.},
	year = {1996},
	archivePrefix = {arxiv},
	eprint = {9510056},
	primaryClass = {gr-qc},
	journal = {Communications in Mathematical Physics},
	volume = {180},
	pages = {633--652},
	doi = {10.1007/BF02099626}
}

@article{BFV_2003,
	title = {The generally covariant locality principle -- A new paradigm for local quantum physics},
        author = {Brunetti, R. and Fredenhagen, K. and Verch, R.},
	year = {2003},
        archivePrefix = {arxiv},
	eprint = {0112041},	
        primaryClass = {math-ph},
	journal = {Communications in Mathematical Physics},
        volume = {237},
        pages = {31--68},
	doi = {10.1007/s00220-003-0815-7}
}

@article{DANG_2015,
	title = {Complex powers of analytic functions and meromorphic renormalization in QFT},
	author = {Dang, N. V.},
	year = {2015},
        archivePrefix = {arxiv},
	eprint = {1503.00995},	
        primaryClass = {math-ph}
}

@article{DHP_2015,
	title = {The generalised principle of perturbative agreement and the thermal mass},
	author = {Drago, N. and Hack, T.-P. and Pinamonti, N.},
	year = {2015},
	archivePrefix = {arxiv},
	eprint = {1502.02705},
	primaryClass = {math-ph}
}

@article{DF_2001,
	title = {Algebraic Quantum Field Theory, Perturbation Theory, and the Loop Expansion},
	author = {Duetsch, M. and Fredenhagen, K.},
	year = {2001},
	archivePrefix = {arxiv},
	eprint = {0001129},
	primaryClass = {hep-th},
	journal = {Communications in Mathematical Physics},
	volume = {219},
	pages = {5--30},
	doi = {10.1007/PL00005563}
}

@article{BW_2014,
	title = {On-shell extension of distributions},
	author = {Bahns, D. and Wrochna, M.},
	year = {2014},
	archivePrefix = {arxiv},
	eprint = {1210.5448},
	primaryClass = {math-ph},
	journal = {Annales Henri Poincaré},
	volume = {15},
	pages = {2045--2067},
	doi = {10.1007/s00023-013-0288-y}
}

@article{PV_1949,
	title = {On the Invariant Regularization in Relativistic Quantum Theory},
	author = {Pauli, W. and Villars, F.},
	year = {1949},
	journal = {Reviews of Modern Physics},
	volume = {21},
	pages = {434--444},
	doi = {10.1103/RevModPhys.21.434},
}

@article{FL_2014,
	title = {Construction of KMS States in Perturbative QFT and Renormalized Hamiltonian Dynamics},
	author = {Fredenhagen, K. and Lindner, F.},
	year = {2014},
	archivePrefix = {arxiv},
	eprint = {1306.6519},
	primaryClass = {math-ph},
	journal = {Communications in Mathematical Physics},
	volume = {332},
	pages = {895--932},
	doi = {10.1007/s00220-014-2141-7}
}

@article{PS_2015,
	title = {Scale-Invariant Curvature Fluctuations from an Extended Semiclassical Gravity},
	author = {Pinamonti, N. and Siemssen, D.},
	year = {2015},
	archivePrefix = {arxiv},
	eprint = {1303.3241},
	primaryClass = {gr-qc},
	journal = {Journal of Mathematical Physics},
	volume = {56},
	pages = {022303},
	doi = {10.1063/1.4908127}
}

@article{FR_2013,
	title = {Batalin-Vilkovisky formalism in perturbative algebraic quantum field theory},
	author = {Fredenhagen, K. and Rejzner, K.},
	year = {2013},
	archivePrefix = {arxiv},
	eprint = {1110.5232},
	primaryClass = {math-ph},
	journal = {Communications in Mathematical Physics},
	volume = {317},
	pages = {697--725},
	doi = {10.1007/s00220-012-1601-1}
}

@article{FR_2014,
	title = {QFT on curved spacetimes: axiomatic framework and examples},
	author = {Fredenhagen, K. and Rejzner, K.},
	year = {2014},
	archivePrefix = {arxiv},
	eprint = {1412.5125},
	primaryClass = {math-ph}
}

@article{BCK_2010,
	title = {Coupled scalar fields in a flat {FRW} universe: renormalisation},
	author = {Baacke, J. and Covi, L. and Kevlishvili, N.},
	year = {2010},
	archivePrefix = {arxiv},
	eprint = {1006.2376},
	primaryClass = {hep-ph},
	journal = {Journal of Cosmology and Astroparticle Physics},
	volume = {1008},
	pages = {026},
	doi = {10.1088/1475-7516/2010/08/026}
}

@article{HOLLANDS_2010,
	title = {Correlators, Feynman diagrams, and quantum no-hair in deSitter spacetime},
	author = {Hollands, S.},
	year = {2010},
	archivePrefix = {arxiv},
	eprint = {1010.5367},
	primaryClass = {gr-qc},
	journal = {Commun. Math. Phys.},
	volume = {319},
	pages = {1--68},
	doi = {10.1007/s00220-012-1653-2}
}

@article{HW_2002,
	title = {Existence of Local Covariant Time Ordered Products of Quantum Fields in Curved Spacetime},
	author = {Hollands, S. and Wald, R. M.},
	year = {2002},
	archivePrefix = {arxiv},
	eprint = {0111108}
	primaryClass = {gr-qc},
	journal = {Communications in Mathematical Physics},
	volume = {231},
	pages = {309--345},
	doi = {10.1007/s00220-002-0719-y}
}

@article{CHL_1995,
	title = {A Generic Renormalization Method in Curved Spaces and at Finite Temperature},
        author = {Comellas, J. and Haagensen, P. E. and Latorre, J. I.},
	year = {1995},
	archivePrefix = {arxiv},
	eprint = {9404080},	
        primaryClass = {hep-th},
	journal = {International Journal of Modern Physics A},
        volume = {10},
	pages = {2819--2839},
	doi = {10.1142/S0217751X95001339}
}

@article{MORETTI_2000,
	title = {Proof of the symmetry of the off-diagonal Hadamard/Seeley-deWitt's coefficients in $C^\infty$ Lorentzian manifolds by a local Wick rotation},
	author = {Moretti, V.},
	year = {2000},
	archivePrefix = {arxiv},
	eprint = {9908068},
	primaryClass = {gr-qc},
	journal = {Communications in Mathematical Physics},
	volume = {212},
	pages = {165--189},
	doi = {10.1007/s002200000202}
}

@article{MORETTI_2003,
	title = {Comments on the Stress-Energy Tensor Operator in Curved Spacetime},
	author = {Moretti, V.},
	year = {2003},
	archivePrefix = {arxiv},
	eprint = {0109048},
	primaryClass = {gr-qc},
	journal = {Communications in Mathematical Physics},
	volume = {232},
	pages = {189--221},
	doi = {10.1007/s00220-002-0702-7}
}

@article{PPV_2011,
	title = {The motion of point particles in curved spacetime},
	author = {Poisson, E. and Pound, A. and Vega, I.},
	year = {2011},
	archivePrefix = {arxiv},
	eprint = {1102.0529},
	primaryClass = {gr-qc},
	journal = {Living Reviews in Relativity},
	volume = {14},
        pages = {7},
	doi = {10.12942/lrr-2011-7}
}

@article{PRANGE_1999,
	title = {Epstein-Glaser renormalization and differential renormalization},
	author = {Prange, D.},
	year = {1999},
	journal = {Journal of Physics A: Mathematical and General},
	pages = {2225},
	volume = {32},
        pages = {2225--2238},
	doi = {10.1088/0305-4470/32/11/015}
}

@article{ZSCHOCHE_2014,
	title = {The Chaplygin Gas Equation of State for the Quantized Free Scalar Field on Cosmological Spacetimes},
	author = {Zschoche, J.},
	year = {2014},
	archivePrefix = {arxiv},
	eprint = {1303.4992},
	primaryClass = {gr-qc},
	journal = {Annales Henri Poincaré},
	volume = {15},
	pages = {1285--1325},
	doi = {10.1007/s00023-013-0281-5}
}

@article{BDH_2014,
	title = {A smooth introduction to the wavefront set},
	author = {Brouder, C. and Dang, N. V. and Hélein, F.},
	year = {2014},
	archivePrefix = {arxiv},
	eprint = {1404.1778},
	primaryClass = {math-ph},
	journal = {Journal of Physics A: Mathematical and Theoretical},
	volume = {47},
	pages = {443001},
	doi = {10.1088/1751-8113/47/44/443001}
}

@article{THV_1972,
	title = {Regularization and renormalization of gauge fields},
	author = {'t Hooft, G. and Veltman, M.},
	year = {1972},
        journal = {Nuclear Physics B},
	volume = {44},
	pages = {189--213}
}

@article{HK_1964,
	title = {An Algebraic Approach to Quantum Field Theory},
	author = {Haag, R. and Kastler, D.},
        year = {1964},
	journal = {Journal of Mathematical Physics},	
	volume = {5},
	pages = {848--861},
	doi = {10.1063/1.1704187}
}

@article{BG_1972,
	title = {Dimensional renorinalization : The number of dimensions as a regularizing parameter},
        author = {Bollini, C. G. and Giambiagi, J. J.},
	year = {1972},	
        journal = {Il Nuovo Cimento B},
        volume = {12},
	pages = {20--26},	
	doi = {10.1007/BF02895558}	
}

@article{DF_2006,
	title = {Off-diagonal coefficients of the DeWitt-Schwinger and Hadamard representations of the Feynman propagator},
        author = {Décanini, Y. and Folacci, A.},
	year = {2006},
	archivePrefix = {arxiv},
	eprint = {0511115},	
	primaryClass = {gr-qc},
	journal = {Physical Review D},
        volume = {73},
        pages = {044027},
	doi = {10.1103/PhysRevD.73.044027}
}

@article{SPEER_1974,
	title = {Analytic renormalization using many space-time dimensions},
	author = {Speer, E. R.},
	year = {1974},
	journal = {Communications in Mathematical Physics},
	volume = {37},
	pages = {83--92}
}

@article{CHRISTENSEN_1976,
	title = {Vacuum expectation value of the stress tensor in an arbitrary curved background: The covariant point-separation method},
	author = {Christensen, S. M.},
	year = {1976},
	journal = {Physical Review D},
	volume = {14},
	pages = {2490--2501},
	doi = {10.1103/PhysRevD.14.2490}
}

@article{SPEER_1968,
	title = {Analytic Renormalization},
	author = {Speer, E. R.},
	year = {1968},
	journal = {Journal of Mathematical Physics},
	volume = {9},
	pages = {1404--1410},
	doi = {10.1063/1.1664729}
}

@article{WALD_1978,
	title = {Trace anomaly of a conformally invariant quantum field in curved spacetime},
        author = {Wald, R. M.},
	year = {1978},	
        journal = {Physical Review D},
        volume = {17},
	pages = {1477--1484},
	doi = {10.1103/PhysRevD.17.1477}	
}

@article{KAY_1991,
	title = {Theorems on the uniqueness and thermal properties of stationary, nonsingular, quasifree states on spacetimes with a bifurcate killing horizon},
	author = {Kay, B. S. and Wald, R. M.},
	year = {1991},
	journal = {Physics Reports},
	volume = {207},
	pages = {49--136},
	doi = {10.1016/0370-1573(91)90015-E}
}
\end{filecontents}

%----------------------------------------------------------------------------%

\geometry{
a4paper,
left=27mm,
right=27mm,
top=24mm,
bottom=24mm,
}

%----------------------------------------------------------------------------%

\abovedisplayskip=12pt plus 3pt minus 9pt
\abovedisplayshortskip=0pt plus 3pt
\belowdisplayskip=12pt plus 3pt minus 9pt
\belowdisplayshortskip=7pt plus 3pt minus 4pt
 
%----------------------------------------------------------------------------%

\setcounter{tocdepth}{8}  
\setcounter{secnumdepth}{8}
\setlength\parindent{8pt}
\RequirePackage{filecontents}
\pdfoptionpdfminorversion=6
\renewcommand{\headrulewidth}{0pt}
\renewcommand{\footrulewidth}{0pt}
\setlength{\headheight}{22pt} 
\pagestyle{fancy}
\renewcommand{\chaptermark}[1]{\markboth{#1}{}}
\renewcommand{\sectionmark}[1]{\markright{#1}}
\fancyhf{}
\fancyhead[LE,RO]{\thepage}
\fancyhead[RE,CE]{}
\fancyhead[LO,CO]{}
\fancypagestyle{plain}{\fancyhf{}}
\DeclareMathAlphabet{\mathcalligra}{T1}{calligra}{m}{n}

%----------------------------------------------------------------------------%

\newcommand{\supp}{\mathsf{supp}}
\newcommand{\singsupp}{\mathsf{singsupp}}
\newcommand{\WF}{\mathsf{WF}}
\newcommand{\id}{\mathsf{id}}
\newcommand{\loc}{\mathsf{loc}}
\newcommand{\reg}{\mathsf{reg}}
\newcommand{\pp}{\mathsf{pp}}
\newcommand{\ms}{\mathsf{ms}}
\newcommand{\sd}{\mathsf{sd}}
\newcommand{\vol}{\mathsf{vol}}
\newcommand{\tr}{\mathsf{tr}}
\renewcommand{\Re}{\mathsf{Re}}
\renewcommand{\Im}{\mathsf{Im}}
\newcommand{\MS}{\textbf{MS}}
\renewcommand{\det}{\mathsf{det}}
\renewcommand{\sup}{\mathsf{sup}}
\renewcommand{\inf}{\mathsf{inf}}
\renewcommand{\exp}{\mathsf{exp}}
\newcommand{\alphabd}{\boldsymbol{\alpha}}
\newcommand{\betabd}{\boldsymbol{\beta}}
\newcommand{\xibd}{\boldsymbol{\xi}}
\renewcommand{\log}{\mathsf{log}}
\newcommand{\bigint}{\displaystyle\int}
\newcommand{\muc}{\mu\csf}

\newcommand{\abs}[1]{\left|#1\right|}
\newcommand{\norm}[1]{\left|\left|#1\right|\right|}
\newcommand{\sm}[1]{\left\langle#1\right\rangle}
\newcommand{\wick}[1]{:\!{#1}\!:}
\newcommand{\exte}[1]{\overset{\circ}{#1}}

%----------------------------------------------------------------------------%

\newcommand{\Acal}{\mathcal{A}}
\newcommand{\Bcal}{\mathcal{B}}
\newcommand{\Ccal}{\mathcal{C}}
\newcommand{\Dcal}{\mathcal{D}}
\newcommand{\Ecal}{\mathcal{E}}
\newcommand{\Fcal}{\mathcal{F}}
\newcommand{\Gcal}{\mathcal{G}}
\newcommand{\Hcal}{\mathcal{H}}
\newcommand{\Ical}{\mathcal{I}}
\newcommand{\Jcal}{\mathcal{J}}
\newcommand{\Kcal}{\mathcal{K}}
\newcommand{\Lcal}{\mathcal{L}}
\newcommand{\Mcal}{\mathcal{M}}
\newcommand{\Ncal}{\mathcal{N}}
\newcommand{\Ocal}{\mathcal{O}}
\newcommand{\Pcal}{\mathcal{P}}
\newcommand{\Qcal}{\mathcal{Q}}
\newcommand{\Rcal}{\mathcal{R}}
\newcommand{\Scal}{\mathcal{S}}
\newcommand{\Tcal}{\mathcal{T}}
\newcommand{\Ucal}{\mathcal{U}}
\newcommand{\Vcal}{\mathcal{V}}
\newcommand{\Wcal}{\mathcal{W}}
\newcommand{\Xcal}{\mathcal{X}}
\newcommand{\Ycal}{\mathcal{Y}}
\newcommand{\Zcal}{\mathcal{Z}}

%----------------------------------------------------------------------------%

\newcommand{\Abb}{\mathbb{A}}
\renewcommand{\Bbb}{\mathbb{B}}
\newcommand{\Cbb}{\mathbb{C}}
\newcommand{\Dbb}{\mathbb{D}}
\newcommand{\Ebb}{\mathbb{E}}
\newcommand{\Fbb}{\mathbb{F}}
\newcommand{\Gbb}{\mathbb{G}}
\newcommand{\Hbb}{\mathbb{H}}
\newcommand{\Ibb}{\mathbb{I}}
\newcommand{\Jbb}{\mathbb{J}}
\newcommand{\Kbb}{\mathbb{K}}
\newcommand{\Lbb}{\mathbb{L}}
\newcommand{\Mbb}{\mathbb{M}}
\newcommand{\Nbb}{\mathbb{N}}
\newcommand{\Obb}{\mathbb{O}}
\newcommand{\Pbb}{\mathbb{P}}
\newcommand{\Qbb}{\mathbb{Q}}
\newcommand{\Rbb}{\mathbb{R}}
\newcommand{\Sbb}{\mathbb{S}}
\newcommand{\Tbb}{\mathbb{T}}
\newcommand{\Ubb}{\mathbb{U}}
\newcommand{\Vbb}{\mathbb{V}}
\newcommand{\Wbb}{\mathbb{W}}
\newcommand{\Xbb}{\mathbb{X}}
\newcommand{\Ybb}{\mathbb{Y}}
\newcommand{\Zbb}{\mathbb{Z}}

%----------------------------------------------------------------------------%

\newcommand{\Arak}{\mathfrak{A}}
\newcommand{\Brak}{\mathfrak{B}}
\newcommand{\Crak}{\mathfrak{C}}
\newcommand{\Drak}{\mathfrak{D}}
\newcommand{\Erak}{\mathfrak{E}}
\newcommand{\Frak}{\mathfrak{F}}
\newcommand{\Grak}{\mathfrak{G}}
\newcommand{\Hrak}{\mathfrak{H}}
\newcommand{\Irak}{\mathfrak{I}}
\newcommand{\Jrak}{\mathfrak{J}}
\newcommand{\Krak}{\mathfrak{K}}
\newcommand{\Lrak}{\mathfrak{L}}
\newcommand{\Mrak}{\mathfrak{M}}
\newcommand{\Nrak}{\mathfrak{N}}
\newcommand{\Orak}{\mathfrak{O}}
\newcommand{\Prak}{\mathfrak{P}}
\newcommand{\Qrak}{\mathfrak{Q}}
\newcommand{\Rrak}{\mathfrak{R}}
\newcommand{\Srak}{\mathfrak{S}}
\newcommand{\Trak}{\mathfrak{T}}
\newcommand{\Urak}{\mathfrak{U}}
\newcommand{\Vrak}{\mathfrak{V}}
\newcommand{\Wrak}{\mathfrak{W}}
\newcommand{\Xrak}{\mathfrak{X}}
\newcommand{\Yrak}{\mathfrak{Y}}
\newcommand{\Zrak}{\mathfrak{Z}}

\newcommand{\arak}{\mathfrak{a}}
\newcommand{\brak}{\mathfrak{b}}
\newcommand{\crak}{\mathfrak{c}}
\newcommand{\drak}{\mathfrak{d}}
\newcommand{\erak}{\mathfrak{e}}
\renewcommand{\frak}{\mathfrak{f}}
\newcommand{\grak}{\mathfrak{g}}
\newcommand{\hrak}{\mathfrak{h}}
\newcommand{\irak}{\mathfrak{i}}
\newcommand{\jrak}{\mathfrak{j}}
\newcommand{\krak}{\mathfrak{k}}
\newcommand{\lrak}{\mathfrak{l}}
\newcommand{\mrak}{\mathfrak{m}}
\newcommand{\nrak}{\mathfrak{n}}
\newcommand{\orak}{\mathfrak{o}}
\newcommand{\prak}{\mathfrak{p}}
\newcommand{\qrak}{\mathfrak{q}}
\newcommand{\rrak}{\mathfrak{r}}
\newcommand{\srak}{\mathfrak{s}}
\newcommand{\trak}{\mathfrak{t}}
\newcommand{\urak}{\mathfrak{u}}
\newcommand{\vrak}{\mathfrak{v}}
\newcommand{\wrak}{\mathfrak{w}}
\newcommand{\xrak}{\mathfrak{x}}
\newcommand{\yrak}{\mathfrak{y}}
\newcommand{\zrak}{\mathfrak{z}}

%----------------------------------------------------------------------------%

\newcommand{\Asf}{\mathsf{A}}
\newcommand{\Bsf}{\mathsf{B}}
\newcommand{\Csf}{\mathsf{C}}
\newcommand{\Dsf}{\mathsf{D}}
\newcommand{\Esf}{\mathsf{E}}
\newcommand{\Fsf}{\mathsf{F}}
\newcommand{\Gsf}{\mathsf{G}}
\newcommand{\Hsf}{\mathsf{H}}
\newcommand{\Isf}{\mathsf{I}}
\newcommand{\Jsf}{\mathsf{J}}
\newcommand{\Ksf}{\mathsf{K}}
\newcommand{\Lsf}{\mathsf{L}}
\newcommand{\Msf}{\mathsf{M}}
\newcommand{\Nsf}{\mathsf{N}}
\newcommand{\Osf}{\mathsf{O}}
\newcommand{\Psf}{\mathsf{P}}
\newcommand{\Qsf}{\mathsf{Q}}
\newcommand{\Rsf}{\mathsf{R}}
\newcommand{\Ssf}{\mathsf{S}}
\newcommand{\Tsf}{\mathsf{T}}
\newcommand{\Usf}{\mathsf{U}}
\newcommand{\Vsf}{\mathsf{V}}
\newcommand{\Wsf}{\mathsf{W}}
\newcommand{\Xsf}{\mathsf{X}}
\newcommand{\Ysf}{\mathsf{Y}}
\newcommand{\Zsf}{\mathsf{Z}}

\newcommand{\asf}{\mathsf{a}}
\newcommand{\bsf}{\mathsf{b}}
\newcommand{\csf}{\mathsf{c}}
\newcommand{\dsf}{\mathsf{d}}
\newcommand{\esf}{\mathsf{e}}
\newcommand{\fsf}{\mathsf{f}}
\newcommand{\gsf}{\mathsf{g}}
\newcommand{\hsf}{\mathsf{h}}
\newcommand{\isf}{\mathsf{i}}
\newcommand{\jsf}{\mathsf{j}}
\newcommand{\ksf}{\mathsf{k}}
\newcommand{\lsf}{\mathsf{l}}
\newcommand{\msf}{\mathsf{m}}
\newcommand{\nsf}{\mathsf{n}}
\newcommand{\osf}{\mathsf{o}}
\newcommand{\psf}{\mathsf{p}}
\newcommand{\qsf}{\mathsf{q}}
\newcommand{\rsf}{\mathsf{r}}
\newcommand{\ssf}{\mathsf{s}}
\newcommand{\tsf}{\mathsf{t}}
\newcommand{\usf}{\mathsf{u}}
\newcommand{\vsf}{\mathsf{v}}
\newcommand{\wsf}{\mathsf{w}}
\newcommand{\xsf}{\mathsf{x}}
\newcommand{\ysf}{\mathsf{y}}
\newcommand{\zsf}{\mathsf{z}}

%----------------------------------------------------------------------------%

\newcommand*{\makepagetitle}{%
%
{\raggedright% 
%
%
%
%
\thispagestyle{empty}%
%
\vspace*{50pt}
%
{\Large Antoine Géré}\\% 
%
\vspace*{120pt}%
%
{\Huge\bfseries Analytic regularization of quantum field theories on curved backgrounds}\\[\baselineskip]%
%
\vspace*{60pt}%
%
{\Large Ph.D. thesis}\\[\baselineskip]% 
%
\vspace*{80pt}%
%
{\Large Dipartimento di Matematica}\\[\baselineskip]% 
%
\vspace*{1pt}
%
{\Large Università degli Studi di Genova}\\[\baselineskip]% 
%
\vfill% 
%
%
\newpage%
%
\thispagestyle{empty}%
%
\ \vfill%
%
%
\textbf{Analytic regularization of quantum \\ field theories on curved backgrounds} \\[2pt]
Ph.D. thesis submitted by \href{mailto:gere@dima.unige.it}{Antoine Géré} \\[1pt]
\href{http://www.comune.genova.it/}{Genova}, November 2015. \\[10pt]
%
%
\begin{minipage}{0.1\linewidth}
\includegraphics[scale=1]{unige.pdf}
% unige.pdf: 29x39 pixel, 72dpi, 1.02x1.38 cm, bb=0 0 29 39
\end{minipage}
%
\begin{minipage}{0.85\linewidth}
\href{http://www.dima.unige.it/}{Dipartimento di Matematica} \\[1pt]
\href{http://www.unige.it/}{Università degli Studi di Genova}
\end{minipage}
%
%
\vspace*{10pt} \\
Supervisor: \href{mailto:pinamont@dima.unige.it}{Prof. Dr. Nicola Pinamonti} \\[1pt]
%
Examiner: \href{mailto:claudio.dappiaggi@unipv.it}{Prof. Dr. Claudio Dappiaggi}
%
%
%
%
}%
%
}%

%----------------------------------------------------------------------------%

\theoremclass{LaTeX}
\theoremstyle{break}
\theoremheaderfont{\normalfont\bfseries}
\theorembodyfont{\normalfont}
\theoremseparator{}
\theoremsymbol{\ensuremath{\blacktriangleright}}
\newtheorem{theorem}{Theorem}[chapter]
\newtheorem{proposition}{Proposition}[chapter]
\newtheorem{lemma}{Lemma}[chapter]
\newtheorem{corollary}{Corollary}[chapter]
\theoremsymbol{\ensuremath{\blacklozenge}}
\newtheorem{example}{Example}[chapter]
\newtheorem{examples}{Examples}[chapter]
\newtheorem{remark}{Remark}[chapter]
\newtheorem{definition}{Definition}[chapter]
\theoremsymbol{\ensuremath{\blacksquare}}
\renewtheorem*{proof}{Proof}[chapter]
\qedsymbol{\ensuremath{_\blacksquare}}
\newtheorem*{sketch}{Sketch of the proof}[chapter]
\theoremsymbol{\ensuremath{_\blacksquare}}

%----------------------------------------------------------------------------%

\definecolor{hypercolor}{rgb}{0.1,0.2,0.6}

\hypersetup{     
pdftoolbar=true,    
pdfmenubar=true,    
pdffitwindow=true,  
pdfstartview={FitH},
pdftitle={PhD thesis},    
pdfauthor={Antoine Géré},     
pdfsubject={Mathematical Physics},
pdfcreator={LaTeX},  
pdfproducer={pdfTex},
pdfkeywords={Quantum Field Theory},  
pdfnewwindow=true,  
colorlinks=true, 
linkcolor=hypercolor, 
urlcolor=hypercolor, 
citecolor=hypercolor,
filecolor=hypercolor,         
}

%----------------------------------------------------------------------------%

\usetikzlibrary{decorations.markings,shapes.misc}

\tikzset{
GFleche/.style={
 draw=black, 
 postaction={decorate}, 
 decoration={markings,mark=at position 0.5 with {\arrow{<}}}
 },
DFleche/.style={
 draw=black, 
 postaction={decorate}, 
 decoration={markings,mark=at position 0.5 with {\arrow{>}}}
 },
GDoubleFleche/.style={
 draw=black, 
 postaction={decorate}, 
 decoration={markings,mark=at position 0.55 with {\arrow{<}}},
 decoration={markings,mark=at position 0.45 with {\arrow{<}}}
 },
DDoubleFleche/.style={
 draw=black, 
 postaction={decorate}, 
 decoration={markings,mark=at position 0.55 with {\arrow{>}}},
 decoration={markings,mark=at position 0.45 with {\arrow{>}}}
 },
CFleche/.style={
 draw=black, 
 postaction={decorate}, 
 decoration={markings,mark=at position 0.53 with {\arrow{<}}},
 decoration={markings,mark=at position 1 with {\arrow{>}}}
 },
EGleche/.style={
 draw=black, 
 postaction={decorate}, 
 decoration={markings,mark=at position 0.28 with {\arrow{>}}},
 decoration={markings,mark=at position 0.76 with {\arrow{<}}}
 },
EDleche/.style={
 draw=black, 
 postaction={decorate}, 
 decoration={markings,mark=at position 0.28 with {\arrow{<}}},
 decoration={markings,mark=at position 0.76 with {\arrow{>}}}
 }, 
cross/.style={
 cross out, 
 draw=black, 
 minimum size=#1-\pgflinewidth, 
 inner sep=0pt, 
 outer sep=0pt
 },
cross/.default={4pt} 
}

%---------------------------------
%---------------------------------
%---------------------------------
%---------------------------------
\newcommand{\manifold}{\begin{tikzpicture}[thick,scale=0.7] 
\draw[color=black] (0,0) to [out=50,in=190] (6,3);
\draw[color=black] (6,3) to [out=10,in=90] (10,0);
\draw[color=black] (10,0) to [out=170,in=30] (3,-3);
\draw[color=black] (3,-3) to [out=90,in=10] (0,0);
\filldraw (6.5,2.2) circle (0pt) node[above] {$\Mcal$}; 
\def\firstellipse{(4,0.5) ellipse (1.6 and 0.6)};
\def\secondellipse{(6,1) ellipse (1.6 and 0.6)};
\draw[color=black] \firstellipse \secondellipse;
\filldraw (3.5,1) circle (0pt) node[above] {$U$}; 
\filldraw (8,01) circle (0pt) node[above] {$V$}; 
\begin{scope}
\clip \firstellipse;
\fill[white!40!black] \secondellipse;
\end{scope}
\draw[dashed] (-3,-7) -- (-1,-5) -- (3,-5) -- (1,-7) -- (-3,-7);
\filldraw (1,-7) circle (0pt) node[below] {$\Rbb^n$}; 
\draw[color=black] (0,-6) ellipse (1.6 and 0.6);
\filldraw (-1.8,-7.3) circle (0pt) node[above] {$\Phi(U)$}; 
\draw[dashed] (7,-7) -- (9,-5) -- (13,-5) -- (11,-7) -- (7,-7);
\filldraw (11,-7) circle (0pt) node[below] {$\Rbb^n$}; 
\draw[color=black] (10,-6) ellipse (1.6 and 0.6); 
\filldraw (8.2,-7.3) circle (0pt) node[above] {$\Psi(V)$};
\draw[->,dashed,color=black] (1.5,-1.5) -- (0.5,-4.5);
\filldraw (1,-3.2) circle (0pt) node[right] {$\Phi$}; 
\draw[->,dashed,color=black] (7.5,-1) -- (9.5,-4.5);
\filldraw (8.9,-3.2) circle (0pt) node[right] {$\Psi$}; 
\draw[->,dashed,color=black] (3,-6) -- (7,-6);
\filldraw (5,-6) circle (0pt) node[above] {$\Psi \circ \Phi^{-1}$}; 
\end{tikzpicture}}
%---------------------------------
%---------------------------------
%---------------------------------
%---------------------------------
\newcommand{\manifoldMaps}{\begin{tikzpicture}[thick,scale=0.5] 
\draw[color=black] (0,0) to [out=50,in=190] (6,3);
\draw[color=black] (6,3) to [out=10,in=90] (10,0);
\draw[color=black] (10,0) to [out=170,in=30] (3,-3);
\draw[color=black] (3,-3) to [out=90,in=10] (0,0);
\draw[color=black] (5.2,1) ellipse (2 and 1);
\filldraw (10,0) circle (0pt) node[right] {$\Mcal$}; 
\filldraw (3,0) circle (0pt) node[above] {$U$};
\filldraw (5,1) circle (2pt) node[above] {$x$};
\draw[color=black] (15,0) to [out=50,in=190] (21,3);
\draw[color=black] (21,3) to [out=10,in=90] (25,0);
\draw[color=black] (25,0) to [out=170,in=30] (18,-3);
\draw[color=black] (18,-3) to [out=90,in=10] (15,0);
\draw[color=black] (20.2,1) ellipse (2 and 1);
\filldraw (25,0) circle (0pt) node[right] {$\Ncal$}; 
\filldraw (18,0) circle (0pt) node[above] {$V$}; 
\filldraw (20,1) circle (2pt) node[above] {$f(x)$};
\draw[->,color=black,dashed] (5,-2.5) -- (5,-4.5);
\filldraw (5,-3.5) circle (0pt) node[right] {$\Phi$}; 
\draw[dashed] (1,-7) -- (3,-5) -- (9,-5) -- (7,-7) -- (1,-7);
\draw[color=black] (5,-6) ellipse (1.6 and 0.6);
\filldraw (9,-5) circle (0pt) node[right] {$\Rbb^n$}; 
\filldraw (2.5,-7) circle (0pt) node[above] {$\Phi(U)$}; 
\draw[->,color=black,dashed] (20,-2.5) -- (20,-4.5);
\filldraw (20,-3.5) circle (0pt) node[right] {$\Psi$}; 
\draw[dashed] (16,-7) -- (18,-5) -- (24,-5) -- (22,-7) -- (16,-7);
\draw[color=black] (20,-6) ellipse (1.6 and 0.6);
\filldraw (24,-5) circle (0pt) node[right] {$\Rbb^n$}; 
\filldraw (17.5,-7) circle (0pt) node[above] {$\Psi(V)$}; 
\draw[->,color=black,dashed] (9,-6) -- (16,-6);
\filldraw (12.5,-6) circle (0pt) node[above] {$\Psi \circ f \circ \Phi^{-1}$}; 
\draw[->,color=black,dashed] (10,2) -- (16,2);
\filldraw (13,2) circle (0pt) node[above] {$f$}; 
\end{tikzpicture}}
%---------------------------------
%---------------------------------
%---------------------------------
%---------------------------------
\newcommand{\manifoldFunction}{\begin{tikzpicture}[thick,scale=0.7] 
\draw[color=black] (0,0) to [out=50,in=190] (6,3);
\draw[color=black] (6,3) to [out=10,in=90] (10,0);
\draw[color=black] (10,0) to [out=170,in=30] (3,-3);
\draw[color=black] (3,-3) to [out=90,in=10] (0,0);
\draw[color=black] (5.2,1) ellipse (2 and 1);
\filldraw (10,0) circle (0pt) node[right] {$\Mcal$}; 
\filldraw (3,0) circle (0pt) node[above] {$U$}; 
\draw[->,color=black,dashed] (5,-2.5) -- (5,-4.5);
\filldraw (5,-3.5) circle (0pt) node[right] {$\phi$}; 
\draw[dashed] (1,-7) -- (3,-5) -- (9,-5) -- (7,-7) -- (1,-7);
\draw[color=black] (5,-6) ellipse (1.6 and 0.6);
\filldraw (9,-5) circle (0pt) node[right] {$\Rbb^n$}; 
\filldraw (3,-7) circle (0pt) node[above] {$\Phi(U)$}; 
\draw[->,color=black,dashed] (5,-7.5) -- (5,-9.5);
\filldraw (5,-8.5) circle (0pt) node[right] {$f \circ \Phi^{-1}$}; 
\draw[line width=0.8mm,color=black] (3,-10) -- (7,-10);
\draw[color=black] (0,-10) -- (10,-10);
\filldraw (10,-10) circle (0pt) node[right] {$\Rbb$};
\filldraw (3,-10) circle (0pt) node[above] {$f(U)$}; 
\draw[->,color=black,dashed] (0,-0.5) -- (0,-9.5);
\filldraw (0,-4.5) circle (0pt) node[right] {$f$}; 
\end{tikzpicture}}
%---------------------------------
%---------------------------------
%---------------------------------
%---------------------------------
\newcommand{\tangentSpace}{\begin{tikzpicture}[scale=0.7] 
\draw[color=black] (0,0) to [out=50,in=190] (6,3);
\draw[color=black] (6,3) to [out=10,in=90] (10,0);
\draw[color=black] (10,0) to [out=170,in=30] (3,-3);
\draw[color=black] (3,-3) to [out=90,in=10] (0,0);
\filldraw (10,0) circle (0pt) node[right] {$\Mcal$};  
\draw[color=black] (2,0) -- (4,2) -- (8,2) -- (6,0) -- (2,0);
\filldraw (7.3,1) circle (0pt) node[right] {$T_x\Mcal$};
\draw [color=black] plot [smooth,tension=1] coordinates {(3,0.5) (4.2,1.4) (5.4,0.2) (7,1.5)};
\filldraw (4.2,1.4) circle (1pt) node[below] {$x$};
\end{tikzpicture}}
%---------------------------------
%---------------------------------
%---------------------------------
%---------------------------------
\newcommand{\bundle}{\begin{tikzpicture}[scale=1]
\draw[->,color=black] (0,0) -- (3,0);
\filldraw (1.5,0) circle (0pt) node[above] {$\psi$};
\draw[->,color=black] (0,-0.2) -- (1.3,-1.5);
\filldraw (0.5,-0.8) circle (0pt) node[left] {$\pi$};
\draw[->,color=black] (3,-0.2) -- (1.7,-1.5);
\filldraw (2.5,-0.8) circle (0pt) node[right] {$\psf\rsf_1$};
%
\filldraw (0,0) circle (0pt) node[left] {$\pi^{-1}(U)$};
\filldraw (3,0) circle (0pt) node[right] {$U \times \{ E_x \}$};
\filldraw (1.5,-1.5) circle (0pt) node[below] {$U$};
\end{tikzpicture}}
%---------------------------------
%---------------------------------
%---------------------------------
%---------------------------------
\newcommand{\expoMap}{\begin{tikzpicture}[thick,scale=0.8] 
\draw[color=black] (0,0) to [out=50,in=190] (6,3);
\draw[color=black] (6,3) to [out=10,in=90] (10,0);
\draw[color=black] (10,0) to [out=170,in=30] (3,-3);
\draw[color=black] (3,-3) to [out=90,in=10] (0,0);
\draw[semithick] (5,1) ellipse (1.6 and 0.6); 
\draw[dashed] (2,0) -- (4,2) -- (8,2) -- (6,0) -- (2,0);
\draw[color=black,->] (2,0.2) to [out=120,in=190] (2,4.2);
\draw[semithick] (5,5) ellipse (1.3 and 0.6); 
\draw[dashed] (2,4) -- (4,6) -- (8,6) -- (6,4) -- (2,4);
\filldraw (3.5,-2) circle (0pt) node[above] {$\Mcal$}; 
\filldraw (5,1) circle (2pt) node[above] {$x$};
\filldraw (6.5,1.2) circle (0pt) node[above] {$\Ocal_x$};
\filldraw (6.4,5.1) circle (0pt) node[above] {$\Ocal^\prime_x$};
\filldraw (2.5,4.8) circle (0pt) node[above] {$T_x\Mcal$};
\filldraw (0.5,2.2) circle (0pt) node[above] {$\mathsf{exp}_x$};
\end{tikzpicture}}
%---------------------------------
%---------------------------------
%---------------------------------
%---------------------------------
\newcommand{\geodesic}{\begin{tikzpicture}[thick,scale=1] 
\draw[color=black] (-2,-1) to [out=50,in=190] (8,2);
\draw[color=black] (8,2) to [out=10,in=90] (9,-1);
\draw[color=black] (9,-1) to [out=170,in=30] (0,-3);
\draw[color=black] (0,-3) to [out=90,in=10] (-2,-1); 
\draw[color=black] (1.5,-0.5) to [out=70,in=-60] (7,1);
\filldraw (1.5,-0.5) circle (1pt) node[below] {$x$};
\filldraw (7,1) circle (1pt) node[above] {$x^\prime$}; 
\filldraw (3,0.36) circle (1pt) node[below] {$z(t)$};
\draw[->] (3,0.36) -- (5,0.58) node[above] {$\dot{z}(t)$};
\filldraw (6,0.2) circle (0pt) node[above] {$\gamma$};
\end{tikzpicture}}
%---------------------------------
%---------------------------------
%---------------------------------
%---------------------------------
\newcommand{\lightcone}{\begin{tikzpicture}[scale=0.8]
\fill[left color=gray!50!black,right color=gray!50!black,middle color=gray!50,shading=axis,opacity=0.25] (2,6) -- (0,3) -- (-2,6) arc (180:360:2cm and 0.5cm);
\draw (-2,6) arc (180:360:2cm and 0.5cm) -- (0,3) -- cycle;
\draw[densely dashed] (-2,6) arc (180:0:2cm and 0.5cm);
\fill[left color=gray!50!black,right color=gray!50!black,middle color=gray!50,shading=axis,opacity=0.25] (2,0) -- (0,3) -- (-2,0) arc (180:360:2cm and 0.5cm);
\draw (-2,0) arc (180:360:2cm and 0.5cm) -- (0,3) -- cycle;
\draw[densely dashed] (-2,0) arc (180:0:2cm and 0.5cm);
\filldraw[black] (0,1.5) circle (0pt) node[left] {$\Vcal^{-}$};
\filldraw[black] (0,4.5) circle (0pt) node[left] {$\Vcal^{+}$};
\end{tikzpicture}}
%---------------------------------
%---------------------------------
%---------------------------------
%---------------------------------
\newcommand{\supportPropagator}{
\begin{tikzpicture}[thick,scale=1]
\fill[color=white!80!black] (0,2) ellipse (2 and 0.4);
\fill[color=white!80!black] (0,0) ellipse (1 and 0.2);
\fill[color=white!80!black] (0,-2) ellipse (2 and 0.4);
\fill[color=white!80!black] (2, 2) -- (1, 0) -- (-1,0) -- (-2,2) -- (2,2);
\fill[color=white!80!black] (2, -2) -- (1, 0) -- (-1,0) -- (-2,-2) -- (2,-2);
\draw[semithick] (0,2) ellipse (2 and 0.4);
\draw (2,2) -- (1,0);
\draw (-2,2) -- (-1,0);
\draw[semithick] (0,0) ellipse (1 and 0.2);
\draw (-2,-2) -- (-1,0);
\draw (2,-2) -- (1,0);
\draw[semithick] (0,-2) ellipse (2 and 0.4);
\filldraw[black] (-1.5,1) circle (0pt) node[right] {$J^+\left(\supp(f),\Mcal\right)$};
\filldraw[black] (1.4,0) circle (0pt) node[right] {$\supp(f)$};
\filldraw[black] (-1.5,-1) circle (0pt) node[right] {$J^-\left(\supp(f),\Mcal\right)$};
\draw[->] (1.4,0) to [bend right=45] (0,0);
\end{tikzpicture}}
%---------------------------------
%---------------------------------
%---------------------------------
%---------------------------------
\newcommand{\DeltaF}{\begin{tikzpicture}[thick,scale=1.5]
\useasboundingbox (-0.3,-0.07) rectangle (1,0.4);
\filldraw (0,0) circle (1pt) node[left] {$x$};
\filldraw (0.7,0) circle (1pt) node[right] {$y$};
\draw (0,0) -- (0.7,0);
\end{tikzpicture} }
%---------------------------------
\newcommand{\DeltaPM}{\begin{tikzpicture}[thick,scale=1.5]
\useasboundingbox (-0.3,-0.07) rectangle (1,0.4);
\filldraw (0,0) circle (1pt) node[left] {$x$};
\filldraw (0.7,0) circle (1pt) node[right] {$y$};
\draw[GFleche] (0,0) -- (0.7,0);
\end{tikzpicture} }
%---------------------------------
\newcommand{\DeltaR}{\begin{tikzpicture}[thick,scale=1.5]
\useasboundingbox (-0.3,-0.07) rectangle (1,0.4);
\filldraw (0,0) circle (1pt) node[left] {$x$};
\filldraw (0.7,0) circle (1pt) node[right] {$y$};
\draw[GDoubleFleche] (0,0) -- (0.7,0);
\end{tikzpicture} }
%---------------------------------
\newcommand{\LineCross}{\begin{tikzpicture}[thick,scale=1.5]
\useasboundingbox (-0.3,-0.07) rectangle (1,0.4);
\draw (0,0) -- (0.7,0);
\filldraw (0.35,0) node[cross] {};
\end{tikzpicture} }
%---------------------------------
\newcommand{\CrossEdges}{\begin{tikzpicture}[thick,scale=1.5]
\useasboundingbox (-0.3,-0.07) rectangle (1,0.4);
\draw (0,-0.2) -- (0.7,0.2);
\draw (0,0.2) -- (0.7,-0.2);
\end{tikzpicture} }
%---------------------------------
%---------------------------------
%---------------------------------
%---------------------------------
\newcommand{\Uno}{\begin{tikzpicture}[thick,scale=1.5]
\useasboundingbox (-0.3,-0.07) rectangle (1.5,0.8);
\draw[GFleche] (0,0) -- (0.5,0);
\filldraw (0,0) circle (1pt) node[left] {$x$};
\filldraw (0.5,0) circle (1pt) node[right] {$y$};
\end{tikzpicture} }
%---------------------------------
\newcommand{\Due}{\begin{tikzpicture}[thick,scale=1.5]
\useasboundingbox (-0.3,-0.07) rectangle (1.3,0.4);
\draw[GDoubleFleche] (0,0) -- (0.5,0);
\draw[GFleche] (0.5,0) -- (1,0);
\filldraw (0,0) circle (1pt) node[left] {$x$};
\filldraw (0.5,0) node[cross] {};
\filldraw (1,0) circle (1pt) node[right] {$y$};
\end{tikzpicture} }
%---------------------------------
\newcommand{\Tre}{\begin{tikzpicture}[thick,scale=1.5]
\useasboundingbox (-0.3,-0.07) rectangle (1.5,0.4);
\draw[GFleche] (0,0) -- (0.5,0);
\draw[DDoubleFleche] (0.5,0) -- (1,0);
\filldraw (0,0) circle (1pt) node[left] {$x$};
\filldraw (0.5,0) node[cross] {};
\filldraw (1,0) circle (1pt) node[right] {$y$};
\end{tikzpicture} }
%---------------------------------
\newcommand{\Quattro}{\begin{tikzpicture}[thick,scale=1.5]
\useasboundingbox (-0.3,-0.07) rectangle (1.8,0.4);
\draw[GDoubleFleche] (0,0) -- (0.5,0);
\draw[GDoubleFleche] (0.5,0) -- (1,0);
\draw[GFleche] (1,0) -- (1.5,0);
\filldraw (0,0) circle (1pt) node[left] {$x$};
\filldraw (0.5,0) node[cross] {};
\filldraw (1,0) node[cross] {};
\filldraw (1.5,0) circle (1pt) node[right] {$y$};
\end{tikzpicture} }
%---------------------------------
\newcommand{\Cinque}{\begin{tikzpicture}[thick,scale=1.5]
\useasboundingbox (-0.3,-0.07) rectangle (1.8,0.4);
\draw[GFleche] (0,0) -- (0.5,0);
\draw[DDoubleFleche] (0.5,0) -- (1,0);
\draw[DDoubleFleche] (1,0) -- (1.5,0);
\filldraw (0,0) circle (1pt) node[left] {$x$};
\filldraw (0.5,0) node[cross] {};
\filldraw (1,0) node[cross] {};
\filldraw (1.5,0) circle (1pt) node[right] {$y$};
\end{tikzpicture} }
%---------------------------------
\newcommand{\Sei}{\begin{tikzpicture}[thick,scale=1.5]
\useasboundingbox (-0.3,-0.07) rectangle (1.8,0.4);
\draw[GDoubleFleche] (0,0) -- (0.5,0);
\draw[GFleche] (0.5,0) -- (1,0);
\draw[DDoubleFleche] (1,0) -- (1.5,0);
\filldraw (0,0) circle (1pt) node[left] {$x$};
\filldraw (0.5,0) node[cross] {};
\filldraw (1,0) node[cross] {};
\filldraw (1.5,0) circle (1pt) node[right] {$y$};
\end{tikzpicture} }
%---------------------------------
\newcommand{\Sette}{\begin{tikzpicture}[thick,scale=1.5]
\useasboundingbox (-0.3,-0.07) rectangle (1.3,0.4);
\draw[GDoubleFleche] (0,0) -- (0.5,0);
\draw[GFleche] (0.5,0) -- (1,0);
\draw (0.5,0.2) circle (0.2cm);
\filldraw (0,0) circle (1pt) node[left] {$x$};
\filldraw (0.5,0.4) node[cross] {};
\filldraw (1,0) circle (1pt) node[right] {$y$};
\end{tikzpicture} }
%---------------------------------
\newcommand{\Otto}{\begin{tikzpicture}[thick,scale=1.5]
\useasboundingbox (-0.3,-0.07) rectangle (1.3,0.4);
\draw[GDoubleFleche] (0,0) -- (0.5,0);
\draw[GFleche] (0.5,0) -- (1,0);
\draw[CFleche] (0.5,0.2) circle (0.2cm);
\filldraw (0,0) circle (1pt) node[left] {$x$};
\filldraw (0.5,0.4) node[cross] {};
\filldraw (1,0) circle (1pt) node[right] {$y$};
\end{tikzpicture} }
%---------------------------------
\newcommand{\Nove}{\begin{tikzpicture}[thick,scale=1.5]
\useasboundingbox (-0.3,-0.07) rectangle (1.3,0.4);
\draw[GFleche] (0,0) -- (0.5,0);
\draw[DDoubleFleche] (0.5,0) -- (1,0);
\draw (0.5,0.2) circle (0.2cm);
\filldraw (0,0) circle (1pt) node[left] {$x$};
\filldraw (0.5,0.4) node[cross] {};
\filldraw (1,0) circle (1pt) node[right] {$y$};
\end{tikzpicture} }
%---------------------------------
\newcommand{\Diece}{\begin{tikzpicture}[thick,scale=1.5]
\useasboundingbox (-0.3,-0.07) rectangle (1.3,0.4);
\draw[GFleche] (0,0) -- (0.5,0);
\draw[DDoubleFleche] (0.5,0) -- (1,0);
\draw[CFleche] (0.5,0.2) circle (0.2cm);
\filldraw (0,0) circle (1pt) node[left] {$x$};
\filldraw (0.5,0.4) node[cross] {};
\filldraw (1,0) circle (1pt) node[right] {$y$};
\end{tikzpicture} }
%---------------------------------
\newcommand{\Undici}{\begin{tikzpicture}[thick,scale=1.5]
\useasboundingbox (-0.3,-0.07) rectangle (1.8,0.4);
\draw[GDoubleFleche] (0,0) -- (0.5,0);
\draw (0.5,0) -- (1,0);
\draw[DFleche] (1,0) -- (1.5,0);
\draw (0.75,0) ellipse (0.25cm and 0.15cm);
\filldraw (0,0) circle (1pt) node[left] {$x$};
\filldraw (1.5,0) circle (1pt) node[right] {$y$};
\end{tikzpicture} }
%---------------------------------
\newcommand{\Dodici}{\begin{tikzpicture}[thick,scale=1.5]
\useasboundingbox (-0.3,-0.07) rectangle (1.8,0.4);
\draw[GDoubleFleche] (0,0) -- (0.5,0);
\draw[DFleche] (0.5,0) -- (1,0);
\draw[GFleche] (1,0) -- (1.5,0);
\draw[EDleche] (0.75,0) ellipse (0.25cm and 0.15cm);
\filldraw (0,0) circle (1pt) node[left] {$x$};
\filldraw (1.5,0) circle (1pt) node[right] {$y$};
\end{tikzpicture} }
%---------------------------------
\newcommand{\Tredici}{\begin{tikzpicture}[thick,scale=1.5]
\useasboundingbox (-0.3,-0.07) rectangle (1.8,0.4);
\draw[GFleche] (0,0) -- (0.5,0);
\draw (0.5,0) -- (1,0);
\draw[DDoubleFleche] (1,0) -- (1.5,0);
\draw (0.75,0) ellipse (0.25cm and 0.15cm);
\filldraw (0,0) circle (1pt) node[left] {$x$};
\filldraw (1.5,0) circle (1pt) node[right] {$y$};
\end{tikzpicture} }
%---------------------------------
\newcommand{\Quattrodici}{\begin{tikzpicture}[thick,scale=1.5]
\useasboundingbox (-0.3,-0.07) rectangle (1.8,0.4);
\draw[GFleche] (0,0) -- (0.5,0);
\draw[GFleche] (0.5,0) -- (1,0);
\draw[DDoubleFleche] (1,0) -- (1.5,0);
\draw[EGleche] (0.75,0) ellipse (0.25cm and 0.15cm);
\filldraw (0,0) circle (1pt) node[left] {$x$};
\filldraw (1.5,0) circle (1pt) node[right] {$y$};
\end{tikzpicture} }
%---------------------------------
\newcommand{\Quindici}{\begin{tikzpicture}[thick,scale=1.5]
\useasboundingbox (-0.3,-0.07) rectangle (1.8,0.4);
\draw[GDoubleFleche] (0,0) -- (0.5,0);
\draw[GFleche] (0.5,0) -- (1,0);
\draw[DDoubleFleche] (1,0) -- (1.5,0);
\draw[EGleche] (0.75,0) ellipse (0.25cm and 0.15cm);
\filldraw (0,0) circle (1pt) node[left] {$x$};
\filldraw (1.5,0) circle (1pt) node[right] {$y$};
\end{tikzpicture} }
%---------------------------------
%---------------------------------
%---------------------------------
%---------------------------------
\newcommand{\FG}{\begin{tikzpicture}[thick,scale=1.5]
\useasboundingbox (-0.1,-0.07) rectangle (0.5,0.4);
\filldraw (0,0) circle (1.3pt);
\filldraw (0.4,0) circle (1.3pt);
\end{tikzpicture} }
%---------------------------------
\newcommand{\FoneG}{\begin{tikzpicture}[thick,scale=1.5]
\useasboundingbox (-0.1,-0.07) rectangle (0.5,0.4);
\filldraw (0,0) circle (1.3pt);
\filldraw (0.4,0) circle (1.3pt);
\draw (0,0) -- (0.4,0);
\end{tikzpicture} }
%---------------------------------
\newcommand{\FtwoG}{\begin{tikzpicture}[thick,scale=1.5]
\useasboundingbox (-0.1,-0.07) rectangle (0.5,0.4);
\filldraw (0,0) circle (1.3pt);
\filldraw (0.4,0) circle (1.3pt);
\draw (0,0) edge [out=45,in=135] node[above] {} (0.4,0);
\draw (0,0) edge [out=-45,in=-135] node[above] {} (0.4,0);
\end{tikzpicture} }
%---------------------------------
\newcommand{\FthreeG}{\begin{tikzpicture}[thick,scale=1.5]
\useasboundingbox (-0.1,-0.07) rectangle (0.5,0.4);
\filldraw (0,0) circle (1.3pt);
\filldraw (0.4,0) circle (1.3pt);
\draw (0,0) edge [out=45,in=135] node[above] {} (0.4,0);
\draw (0,0) -- (0.4,0);
\draw (0,0) edge [out=-45,in=-135] node[above] {} (0.4,0);
\end{tikzpicture} }
%---------------------------------
\newcommand{\FnG}{\begin{tikzpicture}[thick,scale=1] 
\draw[dashed] (0,0) circle (1cm and 0.3cm);
\draw (0,0) circle (1cm and 0.6cm);
\draw[dashed] (-1,0) -- (1,0);
\filldraw (-1,0) circle (2pt) node[left] {$x_1$};
\filldraw (1,0) circle (2pt) node[right] {$x_2$};
\end{tikzpicture}}
%---------------------------------
%---------------------------------
%---------------------------------
%---------------------------------
\newcommand{\FGH}{\begin{tikzpicture}[thick,scale=1.5]
\useasboundingbox (-0.1,0.1) rectangle (0.5,0.4);
\filldraw (0,0) circle (1.3pt);
\filldraw (0.4,0) circle (1.3pt);
\filldraw (0.2,0.3) circle (1.3pt);
\end{tikzpicture} }
%---------------------------------
\newcommand{\FoneGHF}{\begin{tikzpicture}[thick,scale=1.5]
\useasboundingbox (-0.1,0.1) rectangle (0.5,0.4);
\filldraw (0,0) circle (1.3pt);
\filldraw (0.4,0) circle (1.3pt);
\filldraw (0.2,0.3) circle (1.3pt);
\draw (0.2,0.3) -- (0.4,0);
\end{tikzpicture} }
%---------------------------------
\newcommand{\FGoneHF}{\begin{tikzpicture}[thick,scale=1.5]
\useasboundingbox (-0.1,0.1) rectangle (0.5,0.4);
\filldraw (0,0) circle (1.3pt);
\filldraw (0.4,0) circle (1.3pt);
\filldraw (0.2,0.3) circle (1.3pt);
\draw (0,0) -- (0.4,0);
\end{tikzpicture} }
%---------------------------------
\newcommand{\FGHoneF}{\begin{tikzpicture}[thick,scale=1.5]
\useasboundingbox (-0.1,0.1) rectangle (0.5,0.4);
\filldraw (0,0) circle (1.3pt);
\filldraw (0.4,0) circle (1.3pt);
\filldraw (0.2,0.3) circle (1.3pt);
\draw (0,0) -- (0.2,0.3);
\end{tikzpicture} }
%---------------------------------
\newcommand{\FoneGoneHF}{\begin{tikzpicture}[thick,scale=1.5]
\useasboundingbox (-0.1,0.1) rectangle (0.5,0.4);
\filldraw (0,0) circle (1.3pt);
\filldraw (0.4,0) circle (1.3pt);
\filldraw (0.2,0.3) circle (1.3pt);
\draw (0,0) -- (0.2,0.3);
\draw (0.4,0) -- (0,0);
\end{tikzpicture} }
%---------------------------------
\newcommand{\FoneGHoneF}{\begin{tikzpicture}[thick,scale=1.5]
\useasboundingbox (-0.1,0.1) rectangle (0.5,0.4);
\filldraw (0,0) circle (1.3pt);
\filldraw (0.4,0) circle (1.3pt);
\filldraw (0.2,0.3) circle (1.3pt);
\draw (0,0) -- (0.2,0.3);
\draw (0.4,0) -- (0.2,0.3);
\end{tikzpicture} }
%---------------------------------
\newcommand{\FGoneHoneF}{\begin{tikzpicture}[thick,scale=1.5]
\useasboundingbox (-0.1,0.1) rectangle (0.5,0.4);
\filldraw (0,0) circle (1.3pt);
\filldraw (0.4,0) circle (1.3pt);
\filldraw (0.2,0.3) circle (1.3pt);
\draw (0.4,0) -- (0.2,0.3);
\draw (0.4,0) -- (0,0);
\end{tikzpicture} }
%---------------------------------
\newcommand{\FtwoGHF}{\begin{tikzpicture}[thick,scale=1.5]
\useasboundingbox (-0.1,0.1) rectangle (0.5,0.4);
\filldraw (0,0) circle (1.3pt);
\filldraw (0.4,0) circle (1.3pt);
\filldraw (0.2,0.3) circle (1.3pt);
\draw (0,0) edge [out=105,in=185] node[above] {} (0.2,0.3);
\draw (0,0) edge [out=5,in=-75] node[above] {} (0.2,0.3);
\end{tikzpicture} }
%---------------------------------
\newcommand{\FGtwoHF}{\begin{tikzpicture}[thick,scale=1.5]
\useasboundingbox (-0.1,0.1) rectangle (0.5,0.4);
\filldraw (0,0) circle (1.3pt);
\filldraw (0.4,0) circle (1.3pt);
\filldraw (0.2,0.3) circle (1.3pt);
\draw (0,0) edge [out=45,in=135] node[above] {} (0.4,0);
\draw (0,0) edge [out=-45,in=-135] node[above] {} (0.4,0);
\end{tikzpicture} }
%---------------------------------
\newcommand{\FGHtwoF}{\begin{tikzpicture}[thick,scale=1.5]
\useasboundingbox (-0.1,0.1) rectangle (0.5,0.4);
\filldraw (0,0) circle (1.3pt);
\filldraw (0.4,0) circle (1.3pt);
\filldraw (0.2,0.3) circle (1.3pt);
\draw (0.4,0) edge [out=75,in=-5] node[above] {} (0.2,0.3);
\draw (0.4,0) edge [out=-185,in=-105] node[above] {} (0.2,0.3);
\end{tikzpicture} }
%---------------------------------
\newcommand{\FtwoGoneHoneF}{\begin{tikzpicture}[thick,scale=1.5]
\useasboundingbox (-0.1,0.1) rectangle (0.5,0.4);
\filldraw (0,0) circle (1.3pt);
\filldraw (0.4,0) circle (1.3pt);
\filldraw (0.2,0.3) circle (1.3pt);
\draw (0,0) edge [out=105,in=185] node[above] {} (0.2,0.3);
\draw (0,0) edge [out=5,in=-75] node[above] {} (0.2,0.3);
\draw (0.4,0) -- (0.2,0.3);
\draw (0.4,0) -- (0,0);
\end{tikzpicture} }
%---------------------------------
\newcommand{\FtwoGtwoHoneF}{\begin{tikzpicture}[thick,scale=1.5]
\useasboundingbox (-0.1,0.1) rectangle (0.5,0.4);
\filldraw (0,0) circle (1.3pt);
\filldraw (0.4,0) circle (1.3pt);
\filldraw (0.2,0.3) circle (1.3pt);
\draw (0,0) edge [out=105,in=185] node[above] {} (0.2,0.3);
\draw (0,0) edge [out=5,in=-75] node[above] {} (0.2,0.3);
\draw (0.4,0) edge [out=75,in=-5] node[above] {} (0.2,0.3);
\draw (0.4,0) edge [out=-185,in=-105] node[above] {} (0.2,0.3);
\draw (0.4,0) -- (0,0);
\end{tikzpicture} }
%---------------------------------
\newcommand{\FnGnHnF}{\begin{tikzpicture}[thick,scale=2]
\useasboundingbox (-0.1,0.1) rectangle (0.5,0.4);
\filldraw (0,0) circle (1.3pt);
\filldraw (0.4,0) circle (1.3pt);
\filldraw (0.2,0.3) circle (1.3pt);
\draw (0,0) edge [out=105,in=185] node[above] {} (0.2,0.3);
\draw (0,0) edge [out=5,in=-75] node[above] {} (0.2,0.3);
\draw (0.4,0) edge [out=75,in=-5] node[above] {} (0.2,0.3);
\draw (0.4,0) edge [out=-185,in=-105] node[above] {} (0.2,0.3);
\draw[dashed] (0,0) -- (0.2,0.3);
\draw[dashed] (0.4,0) -- (0.2,0.3);
\draw[dashed] (0.4,0) -- (0,0);
\end{tikzpicture} }
%---------------------------------
%---------------------------------
%---------------------------------
%---------------------------------
\newcommand{\catseye}{\begin{tikzpicture}[thick,scale=1.5]
\useasboundingbox (0,-0.1) rectangle (1,0.4);
\filldraw (0,0) circle (1pt);
\filldraw (0.4,0.13) circle (1pt);
\filldraw (0.4,-0.13) circle (1pt);
\filldraw (0.8,0) circle (1pt);
\draw (0,0) .. controls (0.18,0.4) and (0.62,0.4) .. (0.8,0);
\draw (0,0) .. controls (0.18,-0.4) and (0.62,-0.4) .. (0.8,0);
\draw (0,0) edge [out=35,in=145] node[above] {} (0.8,0);
\draw (0,0) edge [out=-35,in=-145] node[above] {} (0.8,0);
\draw (0.4,0.13) edge [out=-25,in=30] node[above] {} (0.4,-0.13);
\draw (0.4,0.13) edge [out=-145,in=140] node[above] {} (0.4,-0.13);
\end{tikzpicture} }
%---------------------------------
%---------------------------------
%---------------------------------
%---------------------------------
\newcommand{\analytic}{\begin{tikzpicture}[thick,scale=0.4] 
\draw[color=black] (-7,-1) -- (8,3);
\draw[color=black] (8,3) -- (22,-3);
\draw[color=black] (22,-3) -- (4,-7);
\draw[color=black] (4,-7) -- (-7,-1);
\filldraw[color=black] (7,1) circle (2pt);
\draw [fill=black!40!white,opacity=1] (5.5,3.98) -- (8.5,3.98) -- (7,1) -- cycle;
\draw [fill=black!40!white,opacity=1] (7,4) circle (1.5cm and 0.3cm);
\draw [fill=black!40!white,opacity=1,dashed] (5.5,-1.98) -- (8.5,-1.98) -- (7,1) -- cycle;
\draw [fill=black!40!white,opacity=1,dashed] (7,-2) circle (1.5cm and 0.3cm);
\filldraw (4,-1) circle (2pt);
\draw [fill=black!40!white,opacity=1] (2.5,1.98) -- (5.5,1.98) -- (4,-1) -- cycle;
\draw [fill=black!40!white,opacity=1] (4,2) circle (1.5cm and 0.3cm);
\draw [fill=black!40!white,opacity=1,dashed] (2.5,-3.98) -- (5.5,-3.98) -- (4,-1) -- cycle;
\draw [fill=black!40!white,opacity=1,dashed] (4,-4) circle (1.5cm and 0.3cm);
\filldraw (13,-1) circle (2pt);
\draw [fill=black!40!white,opacity=1] (11.5,1.98) -- (14.5,1.98) -- (13,-1) -- cycle;
\draw [fill=black!40!white,opacity=1] (13,2) circle (1.5cm and 0.3cm);
\draw [fill=black!40!white,opacity=1,dashed] (11.5,-3.98) -- (14.5,-3.98) -- (13,-1) -- cycle;
\draw [fill=black!40!white,opacity=1,dashed] (13,-4) circle (1.5cm and 0.3cm);
\draw [fill=black!40!white,opacity=1] (-2.49,2.98) -- (0.5,2.98) -- (-1,0) -- cycle;
\draw [fill=black!40!white,opacity=1] (-1,3) circle (1.5cm and 0.3cm);
\draw [fill=black!40!white,opacity=1,dashed] (-2.5,-2.98) -- (0.5,-2.98) -- (-1,0) -- cycle;
\draw [fill=black!40!white,opacity=1,dashed] (-1,-3) circle (1.5cm and 0.3cm);
\end{tikzpicture}}
%---------------------------------
%---------------------------------
%---------------------------------
%---------------------------------
\newcommand{\BigFtwoGoneHoneF}{\begin{tikzpicture}[thick,scale=1] 
\draw (0,0) -- (2,0);
\draw (2,0) -- (1,1.4);
\draw [bend left] (0,0) edge (1,1.4);
\draw [bend left] (1,1.4) edge (0,0);
\filldraw (0,0) circle (2pt) node[left] {$x_1$};
\filldraw (2,0) circle (2pt) node[right] {$x_3$};
\filldraw (1,1.4) circle (2pt) node[above] {$x_2$};
\end{tikzpicture}}
%---------------------------------

%============================================================================%
\begin{document}
%============================================================================%


\pagenumbering{arabic}
\makepagetitle


%----------------------------------------------------------------------------%


\newpage
\ \vfill


\begin{flushright}
``playing field without boundaries''
\end{flushright}


\vfill


%----------------------------------------------------------------------------%
\newpage
\vspace*{100pt}
\section*{Abstract}
%----------------------------------------------------------------------------%


In this PhD thesis we develop a regularization scheme for time--ordered products in interacting quantum field theories on curved spacetimes. We recall here that time--ordered products of fields are necessary tools to treat interacting theory by means of perturbation methods. In the naive construction of this products, by means of products of Feynman propagators, divergences on diagonals occur and they need to be tamed preserving the good properties of the underlying spacetime.


The procedure is implemented into the framework of perturbative algebraic quantum field theory (pAQFT). First we recall the framework of pAQFT. Second we present our scheme. It consists of an analytic regularization of Feynman amplitudes and a minimal subtraction of the resulting pole parts. From a mathematical point of view, it consists in the extension of certain distributions on submanifolds (diagonals) preserving good covariance properties.


This scheme is directly applicable to spacetimes with Lorentzian signature, manifestly generally covariant, invariant under any spacetime isometries present and constructed to all orders in perturbation theory. Moreover, the scheme captures correctly the non--geometric state--dependent contribution of Feynman amplitudes and it is well--suited for practical computations. To illustrate this last point, we compute explicit examples on a generic curved spacetime, and demonstrate how momentum space computations in cosmological spacetimes can be performed in our scheme. In this work, we discuss only scalar fields in four spacetime dimensions, but we argue that the renormalization scheme can be directly generalized to other spacetime dimensions and field theories with higher spin, as well as to theories with local gauge invariance. This work is based on the article \cite{GHP_2015}.


%----------------------------------------------------------------------------%
\tableofcontents
%----------------------------------------------------------------------------%


%----------------------------------------------------------------------------%
\chapter*{Introduction}
\addcontentsline{toc}{chapter}{Introduction}
%----------------------------------------------------------------------------%


%----------------------------------------------------------------------------%
\section*{Quantum field theory and regularization}
\addcontentsline{toc}{section}{Quantum field theory and regularization}
%----------------------------------------------------------------------------%


Physical systems with a finite numbers of degrees of freedom, like the harmonic oscillator, are nowadays very well understood. However, in order to describe systems at very high energy,  like those necessary to describe the phenomena occurring in accelerators, infinite degrees of freedom are necessary. We can naively imagine to approximate such systems with a large number of harmonic oscillators which are interacting. Furthermore, because of the presence of interaction it is not enough to know the dynamics of a single element to find the dynamics of the whole model. In other words, those kind of systems cannot be treated as usual, the notion of fields has to be introduced. Fields are objects which associate a value to every point in space and time. One of the most popular example of field is the (classical) electromagnetic field governed by Maxwell’s equations, in this case the theoretical predictions coincide well with the experimental results.\par%


In order to describe atomic physics and electromagnetism in a comprehensive framework quantum field theories on Minkowski spacetime have been introduced in the past. We recall here that Minkowski spacetime is a four dimensional manifold (three spatial coordinates and one time) equipped with a flat Lorentzian metric. The quantization of free fields i.e. fields which satisfy a linear equation of motion, on Minkowski spacetime is well understood. However, in order to describe realistic physical models, fields satisfying non linear dynamics have to be studied. Even if there is no rigorous mathematical theory of interacting fields, the theory of perturbations developed in the last century permits to make prediction which are extremely accurate.\par%


In the perturbative treatment of interacting fields the relevant objects are described by power series in the coupling constants, however, three kind of problems arises, namely the elements of this series are plagued by
%
\begin{enumerate}
\item\label{item:1_pertubative} ultraviolet (UV) divergences occurring at very high energy,
%
\item\label{item:2_pertubative} infrared (IR) divergences occurring at very low energy,
%
\item\label{item:3_pertubative} and the series usually do not converge in any rigorous way.
\end{enumerate}
%
The first kind of problems (\ref{item:1_pertubative}) are nowadays well understood and a solution to those problems is furnished by the theory of renormalization. Although a complete and generic solution of the second kind of problems (\ref{item:2_pertubative}) is not available, the infrared divergences can be tamed restricting the theory to local portions of the spacetime. For the last point (\ref{item:3_pertubative}) no solution is available, so the elements of the theory are considered to be formal power series in the coupling constants. 
In this work we shall concentrate on the first class of problems.\par%


In order to treat UV divergences on Minkowski spacetime different methods are available, we recall here the momentum cutoff, the Pauli Villar \cite{PV_1949}, the analytic \cite{SPEER_1968}, and the dimensional regularizations. The latest attracted our attention, it has been developed by 't Hooft and Veltman \cite{THV_1972} and independently by Bollini and Giambiagi \cite{BG_1972}.\par%


In the recent years the theory of perturbations has been extended to the case of curved background see e.g. \cite{BF_2000}. It allows us to present in this work the treatment of UV divergences in the case of curved spacetime (CST), namely we shall analyze a theory where the spacetime is described by a Lorentzian manifold with some further assumptions. As expected the same kind of problems present in the flat case remains and some time becomes even more difficult. We cannot give a straightforward generalization of procedures to treat UV divergences developed in Minkowski to curved spacetime. One of the reason is that now on curved background only local objects are meaningful, whereas on Minkowski we could work with non local objects too. Quantum field theory (QFT) on curved background can be viewed as a first try to enlarge the theory to general relativity, or more modestly just to a generalization of theory in Minkowski spacetime.\par%


Several renormalization schemes which deal with UV divergences in the presence of non--trivial spacetime curvature have been discussed in the literature in the last decades, such as for example local momentum space methods \cite{BUNCH_1981}, dimensional regularization in combination with heat kernel techniques \cite{LUSCHER_1982,TOMS_1982}, differential renormalization \cite{CHL_1995,PRANGE_1999}, zeta--function renormalization \cite{BILAL_2013}, generic Epstein--Glaser renormalization \cite{BF_2000,HW_2001,HW_2005}, and, on cosmological spacetimes, Mellin--Barnes techniques \cite{HOLLANDS_2010} and dimensional regularization with respect to the comoving spatial coordinates \cite{BCK_2010}. Some of these schemes, such as heat kernel approaches, zeta--function techniques and local momentum space methods are based on constructions which are initially only well--defined for spacetimes with Euclidean signature. These constructions can be partly transported to general Lorentzian spacetimes by local Wick--rotation techniques developed in \cite{MORETTI_2000}. It appears that the problems comes from the Feynman propagator, which is essentially unique on Euclidean spacetimes, unfortunately, this is not the case on Lorentzian spacetimes where this propagator has a non--unique contribution depending on the quantum state of the field model. Consequently, the Euclidean renormalization techniques are able to capture the correct divergent and geometric parts of Feynman amplitudes, but a priori not their non--geometric and state--dependent contributions.\par%


%----------------------------------------------------------------------------%
\section*{Algebraic quantum field theory}
\addcontentsline{toc}{section}{Algebraic quantum field theory}
%----------------------------------------------------------------------------%


In order to have a full understanding of quantum field theory axiomatic formulations have been developed. One of them is the one introduced by Haag and Kastler \cite{HK_1964}, entirely based on local concepts. For this reason it is well suited to approach quantum field theory on curved background. Indeed in every curved spacetime physical quantities are only known locally. Brunetti, Fredenhagen \cite{BF_2000}, and,  Hollands and Wald \cite{HW_2001,HW_2005} have developed a procedure to treat regularization problem on curved spacetimes with Lorentzian signature, which is based on ideas of Steinmann, Epstein, and Glaser \cite{STEINMANN_1971,EG_1973}. This method is compatible with the axiomatic approach initiated by Haag and Kastler.\par%


The algebraic approach that we shall use is completely build with the requirement of using only local objects. In chapter \ref{p:PAQFT} we start by defining the off shell configuration space of real scalar fields, fields shall be view as smooth functions which map regions of spacetime to real numbers. Then we introduce the notion of observable in the functional approach which is well suited to perform explicit computations. Every observable illustrate the action of performing an experiment on a physical system, it shall be described as functional mapping field configurations to complex numbers. 


As a first step we define in section \ref{p:CLASSICAL} the off shell algebra of classical observables. The quantization procedure chosen is the formal deformation defined in section \ref{p:Q_DEFORM}. The idea is to formally deform the (classical) pointwise product between observables to a ``quantum product'' in order to implement the quantum structure. The causality in the perturbative approach is implemented by considering a time--ordered prescription. In practice it is done modifying the ``quantum product'', to obtain the time ordered product. The UV divergences comes from the fact that this time ordered product can be ill defined. Hollands and Wald in \cite{HW_2001,HW_2002} gave a set of axioms to build time ordered product in order to get a physically meaningful theory. 


An explicit construction of time ordered products which satisfied the desired axioms has been done using the generalized (Epstein--Glaser) regularization scheme developed in \cite{BF_2000,HW_2001,HW_2005}, where the problem is in fact reduced to the extension of distributions called Feynman propagator. Actually, within this scheme the time-ordered products are constructed by means of an iterative procedure over the number of local fields.
At every induction step, causal factorization, which is one of the requested axioms, permits to construct the time ordered products up to the total diagonal. The last step is the extension of the obtained products to the full space. This can be done by means of some scaling techniques towards the point where the products are not known developed by Steinmann.


The procedure is based on microlocal techniques which replace the momentum space methods available in Minkowski spacetime and have been introduced to quantum field theory in curved spacetime by the seminal work \cite{RADZIKOWSKI_1996}. In particular it uses the notion of wave front set which is well suited to work on curved spacetime owing it does not depend on the coordinates system chosen. The wave front set characterize the singularities in position and momentum spaces of distributions.\par%


However, although this scheme is conceptually clear and mathematically rigorous, it is not easily applicable in practical computations. On the other hand, Lorentzian schemes which are better suited for this purpose have not been developed to all orders in perturbation theory \cite{CHL_1995,PRANGE_1999}, are tailored to specific spacetimes \cite{HOLLANDS_2010} or are not manifestly covariant \cite{BCK_2010}. We shall mention that the the state of the art of the mathematical formulation of regularization of quantum field theory in curve spacetime can be found in \cite{DANG_2013}.\par%


\bigskip


Motivated by this, we develop a regularization scheme in interacting field theories on curved spacetimes which is directly applicable to spacetimes with Lorentzian signature. This regularization scheme is inspired by the works \cite{KELLER_2010,DFKR_2014} which deal with perturbative QFT in Minkowski spacetime and has been presented in the paper \cite{GHP_2015}.  In these works, the authors introduce an analytic regularization of the position--space Feynman propagator in Minkowski spacetime which is similar to the one discussed in \cite{BG_1972}. The scheme shall appear to be manifestly generally covariant, invariant under any spacetime isometries present and constructed to all orders in perturbation theory. Moreover, it captures correctly the non--geometric state--dependent contribution of Feynman amplitudes and it is well--suited for practical computations. In this work, we discuss only scalar fields in four spacetime dimensions, but we shall argue that the renormalization scheme can be directly generalized to other spacetime dimensions and field theories with higher spin, as well as to theories with local gauge invariance.\par%


In particular in chapter \ref{p:COV_REG} we extend the scheme proposed in \cite{DFKR_2014} to curved spacetimes. Motivated by \cite{BG_1972} and the form of Feynman propagators on curved spacetimes, we introduce an analytic regularization $\Delta^{(\alpha)}_\fsf$ of a Feynman propagator $\Delta_\fsf$ which can be written as follow using the Hadamard representation
%
\begin{equation*}
\Hsf^{(\alpha)}_\fsf := \lim_{\epsilon \downarrow 0} \frac{1}{8\pi^2} \left(\frac{\usf}{(\sigma+i\epsilon )^{1+\alpha}} + \frac{v}{\alpha} \left(1-\frac{1}{(\sigma+i\epsilon )^{\alpha}}\right)\right) + w \ ,
\end{equation*}
%
where $\usf$, $v$ and $w$ are the so--called Hadamard coefficients, $\sigma$ is half of the squared geodesic distance, and $\alpha$ a non zero complex parameter. Pointwise product of these distributions are in general ill defined on coinciding points, however, using this analytic regularization we can determine an extension by subtracting the principal part with respect to $\alpha$. This method, is called the minimal subtraction procedure. For instance if $u^\alpha$ is the analytic regularization of a distribution $u$, we can find an extension to $u$ by
%
\begin{equation*}
u_\ms = \lim_{\alpha \to 0} \left( u^\alpha - \pp(u^\alpha) \right) \ .
\end{equation*}
%
This analytic regularization, namely the construction of certain distributions by means of powers of the geodesic distance, is reminiscent of the use of Riesz distributions to define advanced and retarded Greens functions on Minkowski spacetime. A careful discussion of Riesz distributions and their extension to the curved case is presented in \cite{BGP_2007}. The regularization we use is loosely related to dimensional regularization because the leading singularity of a Feynman propagator in $d$ spacetime dimensions is proportional to $(\sigma+i\epsilon)^{1-d/2}$, see e.g. \cite[Appendix A]{MORETTI_2003}. A regularization of the Feynman propagator similar to the one above has recently been discussed in \cite{DANG_2015}. The minimal subtraction scheme can be encoded in a forest formula of the kind discussed in \cite{HOLLANDS_2010,KELLER_2010,DFKR_2014} in order to obtain a time--ordered product which satisfies the causal factorization property, i.e. a product which is indeed ``time--ordered''. In order to prove that the analytically regularized amplitudes constructed out of $\Hsf^{(\alpha)}_\fsf$ have the meromorphic structure necessary for the application of the forest formula and in order to show how the corresponding Laurent series can be computed explicitly, we shall make use of generalized Euler operators. At the end of chapter \ref{p:COV_REG} we shall verify that the regularized time ordered product obtained through our scheme satisfies the axioms introduced by Hollands and Wald \cite{HW_2001,HW_2002}.


Finally in chapter \ref{p:EXOS} we illustrate the scheme by computing examples on generic spacetime, and on cosmological spacetimes, and in particular on  Friedmann--Lemaître--Robertson--Walker (FLRW) spacetimes. We exploit the high symmetry of these spacetimes in order to evaluate analytical expressions in spatial Fourier space. However, the regularization scheme discussed in this work operates on quantities such as the geodesic distance and the Hadamard coefficients, whose explicit position space and momentum space forms are not even explicitly known in FLRW spacetimes. Notwithstanding, we devote a large part of this work in order to develop simple methods to evaluate quantities renormalized in our scheme on FLRW spacetimes in momentum space, and we illustrate these methods by explicit examples.


In appendix \ref{p:SPACETIME} we collect know results about topology, differential and Lorentzian geometry, causality, in order to give a complete definition of the notion of spacetime. We also briefly review FLRW spacetimes.


%----------------------------------------------------------------------------%
\section*{Disgression}
\addcontentsline{toc}{section}{Disgression}
%----------------------------------------------------------------------------%


Although the main subject of this PhD thesis is the development of an analytic regularization scheme for interacting quantum field theory on curved spacetime, during the three years of my PhD course I was also interested in another project. In particular I was interested in the noncommutative approach to quantum field theory. These studies have been done in collaboration with Tajron Juri\'c\footnote{Ru\dj er Bo\v{s}kovi\'c Institute, Theoretical Physics Division, Zagreb, Croatia}, Patrizia Vitale\footnote{Dipartimento di Fisica, Universit\`a di Napoli Federico II, Italy}, and Jean-Christophe Wallet\footnote{Laboratoire de Physique Th\'eorique, CNRS and Universit\'e Paris-Sud 11, Orsay, France}. The results have been subject to three articles, \cite{GW_2015}, \cite{GJW_2015}, and \cite{GVW_2014}. Here, I would like to briefly to summarize the result obtained in those papers.\par%


\bigskip


Noncommutative Geometry (NCG) introduced by Connes \cite{CONNES_1994} could be a possibility to overcome physical obstructions to the existence of continuous spacetime and commuting coordinates at the Planck scale \cite{DFR_1994}. It gives a new approach to field theories, usually called noncommutative field theories (NCFT), which have been considered long ago within string field theory, see e.g. \cite{WITTEN_1986}. Models have been built for the fuzzy spheres \cite{MADORE_1999}, and also for gauge theories on almost commutative geometries \cite{DV_1999}. In particular NCFT on noncommutative Moyal spaces received a lot of attention from the viewpoint of perturbative properties and renormalizability \cite{MVRS_2000}.\par%


\bigskip


In several NCFT developed recently we can noticed that Jacobi operators appear as kinetic operators. Therefore in \cite{GW_2015} we wanted to characterize those operators in order to be helpful for NCFT. We restricted ourselves to the case of bounded Jacobi operators. A set of tools mainly issued from operator and spectral theory is given in a way applicable to the study of NCFT. We illustrate this formal development with an application to a gauge-fixed version of the induced gauge theory on the Moyal plane expanded around a symmetric vacuum. The characterization of the spectrum of the kinetic operator is given, showing a behavior somewhat similar to a massless theory. An attempt to characterize the noncommutative geometry related to the gauge fixed action is presented. Using a Dirac operator obtained from the kinetic operator, it is shown that one can construct an even, regular, weakly real spectral triple. This spectral triple does not define a noncommutative metric space for the Connes spectral distance.\par%


\bigskip


We also work on particular NCFT models, namely scalar field theories on the noncommutative space $\Rbb^3_\lambda$, a deformation of $\mathbb{R}^3$ preserving rotation invariance, studied in \cite{VW_2013}. In the footsteps of this work we considered gauge models on this same space. 


The construction of noncommutative gauge models can be conveniently achieved by using the general framework of the noncommutative differential calculus based on the derivations of an algebra which has been introduced a long ago in \cite{DV_1999}. Mathematical details and some related applications to NCFT can be found in \cite{WALLET_2009}. We first defined the Lie algebra of real inner derivations of $\Rbb^3_\lambda$, denoted $\Gcal$. Second we consider a noncommutative extension of the notion of connection on the right-module over $\Rbb^3_\lambda$ as a linear map $\nabla : \Gcal \times \Mbb \to \Mbb$. Here we restricted ourselves to the case $\Mbb = \Cbb \otimes \Rbb^3_\lambda$. We define the gauge field as the evaluation of a connection on the identity. By defining a gauge transformation of the connection, we can notice that a new gauge coordinate can be defined which is gauge invariant, it is called the covariant coordinate.



In \cite{GVW_2014} we considered a class of gauge invariant models on $\Rbb^3_\lambda$ where we choose as relevant gauge coordinate the ``canonical'' gauge field. We focusing on massless models with no linear term in the gauge field dependence. We obtain noncommutative gauge models for which the computation of the propagator can be done in a convenient gauge. We find that the infrared singularity of the massless propagator disappears in the computation of the correlation functions. We show that massless gauge invariant models on $\Rbb^3_\lambda$ have quantum instabilities of the vacuum, signaled by the occurrence of non vanishing one--point functions for some but not all of the components of the gauge potential. Its global symmetry does not fit with the one of the classical action, reminiscent of an explicit global symmetry breaking term. 


In a second project \cite{GJW_2015}, we look again at noncommutative gauge theory models on the same space $\Rbb^3_\lambda$, but this time we restrict ourselves to positive actions with covariant coordinates as field variables. We showed that a suitable gauge-fixing leads to a family of matrix models with quartic interactions and kinetic operators with compact resolvent. Their perturbative behavior is then studied. We first compute the 2-point and 4-point functions at the one-loop order within a subfamily of these matrix models for which the interactions have a symmetric form. We find that the corresponding contributions are finite. We then extend this result to arbitrary order. We find that the amplitudes of the ribbon diagrams for the models of this subfamily are finite to all orders in perturbation. This result extends finally to any of the models of the whole family of matrix models obtained from the above gauge-fixing. The origin of this result is discussed. Finally, the existence of a particular model related to integrable hierarchies is indicated, for which the partition function is expressible as a product of ratios of determinants.\par%


%----------------------------------------------------------------------------%
\chapter{Interacting quantum field theory}
\label{p:PAQFT}
%----------------------------------------------------------------------------%


In the last decades quantum field theory (QFT) has been tested with very sophisticated experiments, and predictions made by the theory coincide with very high precision to the experimental results. One of the missing block to this robust theoretical framework is implementing gravitation. QFT on curved spacetime (CST) is a first step in that direction. In this work, for simplicity we will restrict ourselves to the case of a quantum scalar field propagating over a curved background. Despite of the simplicity of this model, its treatment will give us a clear idea of the mathematical structure necessary to describe quantum field theories on curved spacetimes.


In this chapter we shall present the mathematical framework necessary to formalize the theory of QFT on CST. In particular we shall use the approach of algebraic quantum field theory. It consists in two steps. First we shall define the algebra of observables and the relations among them. The second step consists in the determination of a state of the system by means of which we can model expectation values.


We shall focus first on the free theory, and describe the mathematical elements necessary to describe quantum field theories on a curved spacetime $\Mcal$. That is to say we shall introduce the notion of fields and observables within this approach, then the classical field theory, and finally we present the quantization procedure we shall use.


At the end of this chapter we shall analyze the case of interacting theories, where the interaction will be treated perturbatively. In the perturbative approach to interacting quantum field theories, observables are given as formal power series in the coupling constant. These kind of theories are affected by three class of problems, namely the ultraviolet divergences, the infrared problems, and the convergence problems of the perturbative series. At the present time we have the first problem under good mathematical control thanks to the theory of regularization, the second problem can be tackled using for example the algebraic adiabatic limit. Unfortunately it is still not known in which sense these series could converge if it converge at all. In this chapter we shall concentrate on the first class of problems presenting the status of the art.


%----------------------------------------------------------------------------%
\section{Off shell configuration space}
\label{p:OFF_SHELL_CONFIG_SPACE}
%----------------------------------------------------------------------------%


%----------------------------------------------------------------------------%
\subsection{Configuration space of a real scalar field theory}
\label{p:DEF_CONFIG_SPACE}
%----------------------------------------------------------------------------%


As stated in the introduction, we aim to describe a quantum field theory over a curved background. In ordinary classical theory the state of a system is described by a section of a particular vector bundle over $\Mcal$. Furthermore, if the field theory we are considering is free, usually the section mentioned above needs to satisfy a linear equation of motion.


In this thesis we would like to describe an interacting real scalar field theory, hence a possible field configuration is described by a scalar function $\phi$ over $\Mcal$. Furthermore, since we shall treat the interaction by means of perturbation theory, we need to enlarge the space of admissible field configurations dropping the requirement that an equation of motion is satisfied.


In order to be able to work easily with the theory, some regularity needs to be requested for the admissible field configurations. In particular, we shall assume that the admissible functions are smooth.   


We may summarize all these requirements in the following definition.


\begin{definition}[Off shell configuration space]\label{def:config_space}
The off shell configuration space over the spacetime $\Mcal$, as defined in \ref{def:cst}, is the space of real valued smooth maps, $\phi \in \Ecal(\Mcal)$. It is equipped with the locally convex topology.
\end{definition}


%----------------------------------------------------------------------------%
\subsection{Topological aspects of the configuration space}
\label{p:TOPO_CONFIG_SPACE}
%----------------------------------------------------------------------------%


As we already said, in the general case the configuration space over $\Mcal$ can be defined as the space of sections of some vector bundle $E$ over $\Mcal$. Since we shall work only with real scalar fields, the chosen vector bundle is simply the real line bundle $\Ecal(\Mcal)$.


We would like now to discuss some mathematical aspects of $\Ecal(\Mcal)$, i.e. we shall discuss its topology and the notion of convergence on it. We start by looking at the space of functions of the form
%
\begin{equation*}
f : N \subset \Rbb^n \to \Rbb \ 
\end{equation*}
%
Using L. Schwartz's notation we denote the space of real valued smooth functions on $N$ by $\Ecal(N)$. We equip $\Ecal(N)$ with the following family of seminorms
%
\begin{equation}
\Pcal = \left\{ p_{K,r}(f) \ , \ \mbox{ with compact subset } K \subset N \ , \mbox{ and } r \in \Nbb \right\} \ ,
\label{eq:family_seminorm}
\end{equation}
%
where the seminorms $p_{K,r}$ are defined as follow
%
\begin{eqnarray}
&& p_{K,r}(f) = \sup \left\{ \abs{\partial^\alpha f(x)} \ \mbox{ with } \ x \in N \ \mbox{ and } \ \abs{\alpha} \leq r  \right\} \ , \nonumber \\
&& \mbox{where the multiindex} \ \alpha \in \Nbb^n, \ \abs{\alpha} = \sum_{i=1}^n \alpha_i, \ \mbox{ and } \ \partial^\alpha = \frac{\partial^{\alpha_1}}{\partial x_1^{\alpha_1}} \dots \frac{\partial^{\alpha_n}}{\partial x_n^{\alpha_n}} \ .
\label{eq:conv_multiindex}
\end{eqnarray}
%
The family of seminorms $\Pcal$ endow $\Ecal(N)$ with a locally convex topology. On this space a sequence $(f_n)$ converges with limit $f$ if for all $\alpha$ and all compact $K$,
%
\begin{eqnarray*}
&& (\partial^\alpha f_n) \ \mbox{ converge uniformly on } \ K \ \mbox{ to } \ (\partial^\alpha f) \ , \\
&& \mbox{i.e. for all } \ \epsilon > 0 \ \mbox{ there is } \ r_\epsilon \in \Nbb \ \mbox{ such that for every } \  x \in K \ \mbox{ and any } \ n \geq r_\epsilon \ , \\
&& \mbox{we have } \ \abs{\partial^\alpha f_n(x) - \partial^\alpha f(x)} < \epsilon \ . 
\end{eqnarray*}
%
This space is a locally convex topological vector space which appears to be a Fréchet space. We recall that a Fréchet space is a complete locally convex topological vector space with a metrizable topology. It can be shown that a locally convex topological vector space is metrizable if and only if the topology can be induced by a countable family of seminorm. Here, as defined in \eqref{eq:family_seminorm}, the topology on $\Ecal(N)$ is defined via a family of seminorm \eqref{eq:family_seminorm} and is also complete. Thus it is true that $\Ecal(N)$ is a Fréchet space.

Now let us come back to $\Ecal(\Mcal)$, i.e. the space of smooth functions defined as follow
%
\begin{equation*}
\phi : \Mcal \to \Rbb \ . 
\end{equation*}
%
As for $\Ecal(\Rbb^n)$ the topology of $\Ecal(\Mcal)$ can be induced by a family of seminorms. Let us consider the atlas $\{(U_i,\Phi_i)\}$ on $\Mcal$, a compact set $K \subset \Phi_i(U_i) \subset \Rbb^n$, and $r \in \Nbb$. We define the seminorm $p_\Mcal$ on $\Ecal(\Mcal)$ as follow
%
\begin{equation*}
p_\Mcal(\phi) = p_{K,r}\left( \Phi\left( \phi_{|U_i}(x) \right) \right) \ ,
\end{equation*}
%
with $\phi \in \Ecal\left(\Mcal \right)$. It defines the Fréchet topology on $\Ecal\left(\Mcal \right)$. In particular, a sequence $(\phi_n)$ converges to $\phi$ with respect to this topology if and only if for any compact set $K \subset U_i$, the sequence $\left(\partial^\alpha\phi_m\right)$ converges uniformly on $K$ to $\partial^\alpha\phi$.


For later purposes let us introduce the space of real valued smooth compactly supported functions $\Dcal(\Mcal)$, endowed with the locally convex topology implemented as follow
%
\begin{equation*}
\Dcal(\Mcal) = \bigcup_{K\subset\Mcal} \Dcal_K(\Mcal) \ ,
\end{equation*}
%
where in the right hand side the union is taken over all compact set $K \subset \Mcal$. $\Dcal_K(\Mcal) \subset \Ecal(\Mcal)$ is the space of all smooth functions supported in $K$, endowed with the topology induced from $\Ecal(\Mcal)$. On $\Dcal(\Mcal)$ we have the inductive limit topology. It is a locally convex topological vector space but non metrizable, therefore it is not a Fréchet space. 


%----------------------------------------------------------------------------%
\section{Observables as functionals over field configurations}\label{p:OBS}
%----------------------------------------------------------------------------%


In this section we shall introduce the set of observables of a scalar field theory over a curved background. We recall that an observable models a possible experimental apparatus by means of which we can test the physical system we are considering. In other words, it must be possible to associate an observable to every detectable property of an element of the configuration space over $\Mcal$. Examples of observables are the local energy density, the local temperature, the local momentum and the local density of the field.  Hence, roughly speaking, observables are functions on the configuration space with values in ordinary numbers.


Once we have detected a sensible set of observables which describes a quantum field theory on curved background, we shall analyze the relation among them. Here we aim to individuate the algebraic properties satisfied by the set of observables, this is necessary in order to use standard algebraic method to quantize the system. Notice that the algebra of observables we want to obtain will not depend on the particular state of the system (cf. section \ref{p:STATES} for an introduction to states).


We shall use the functional approach to define mathematically the observables, since this method is less abstract and more computationally friendly \cite{BFLR_2012,BDF_2009}. Hence, we shall view from now on an observable $\Fsf$ as a functional 
%
\begin{equation*}
\Fsf : \left\{
\begin{array}{ccc}
\Ecal(\Mcal) & \to     & \Cbb \\
\phi  & \mapsto & \Fsf(\phi)
\end{array}
\right. \ . 
\end{equation*}


The generic space of observables is thus the set of all possible functionals and it is denoted by $\Fcal(\Mcal)$. We shall now discuss some further restrictions we have to require in order to be able to work with these objects.


The laboratory where the measurements are made is finite dimensional in space and time, thus we shall consider only observables defined on a finite subspace of $\Mcal$. For this reason we need a concept which permits to localize functionals in a certain region of spacetime.


\begin{definition}[Spacetime support] \label{def:spacetime_supp}
Let $\Fsf$ be a complex valued functional. The spacetime support of $\Fsf$ is defined as follow
%
\begin{equation*}
\supp(\Fsf) \doteq \left\{ x \in \Mcal \ \bigg| \ 
\begin{array}{l}
\forall \ U \ni x , \ \exists \ \phi, \psi \in \Ecal(\Mcal) \ \mbox{ with } \ \supp(\psi) \subset U, \\
\mbox{such that } \Fsf(\phi + \psi) \neq \Fsf(\phi)
\end{array}
\right\} \ .
\end{equation*}
It is a closed subset of $\Mcal$.
%
\end{definition}


According to definition \ref{def:spacetime_supp} a functional $\Fsf$ does not ``feel'' field configurations which are localized outside $\supp(\Fsf)$. In other words the spacetime support of $\Fsf$ is the subset of $\Mcal$ where $\Fsf$ does see the ``fluctuations'' of the field configuration.


An important and simple example of functionals are the linear ones. 


\begin{definition}[Linear functionals]\label{def:linear_obs}
We define a linear functional $\Fsf(\phi)$ as follow
%
\begin{equation*}
\Fsf(\phi) = \int_\Mcal \dsf x \ \sqrt{\abs{g}} \ f(x) \ \phi(x) \ , 
\end{equation*}
%
with $f \in \Dcal(\Mcal)$.
\end{definition}



As discussed above, the space $\Fcal(\Mcal)$ is still too large. We should consider \textbf{functionals with compact spacetime support} over $\Mcal$, this set is denoted by $\Fcal_0(\Mcal)$. Notice that this new space can be endowed with an algebraic structure, introduced in \cite{BFLR_2012}.


\begin{definition}[Algebra of compactly supported functionals] \label{def:algebra_comp_supp_func}
Let $\Fsf$ and $\Gsf$ be compactly supported functionals, i.e. elements of $\Fcal_0(\Mcal)$ then we define the following operations.
%
\begin{itemize}
\item Sum : $(\Fsf+\Gsf)(\phi) = \Fsf(\phi) + \Gsf(\phi)$ ,
\item Multiplication by a scalar $z\in\Cbb$ : $(z \cdot \Fsf)(\phi) = z \Fsf(\phi)$ ,
\item Pointwise product : $(\Fsf \cdot \Gsf)(\phi) = \Fsf(\phi) \Gsf(\phi)$ ,
\item Involution\footnote{The operation $\overline{\cdot}$ is the complex conjugation.} : $\Fsf^\ast(\phi) = \overline{\Fsf(\overline{\phi})}$,
\item Unit : $\Ibb = \Fsf(\phi) = 1$ ,
\end{itemize}
%
with $\phi \in \Ecal(\Mcal)$. 
\end{definition}


The space $\Fcal_0(\Mcal)$ equipped with the operations and elements listed above defines a commutative unital $\ast$--algebra.


In order to ensure the well posedness of the previous definition, we can check that the algebraic operations introduced in definition \ref{def:algebra_comp_supp_func} do not modify the spacetime support presented in definition \ref{def:spacetime_supp}. Actually, we have the following lemma whose proof can be found in  \cite{BFLR_2012}.


\begin{lemma}[``Rigidity'' of the spacetime support]
The algebraic operations introduced in definition \ref{def:algebra_comp_supp_func} preserve the spacetime support of a functional. In particular
%
\begin{itemize}
\item Sum : $\supp(\Fsf + \Gsf) \subseteq \supp(\Fsf) \cup \supp(\Gsf)$ ,
\item Multiplication by a scalar $z\in\Cbb$ : $\supp(z\cdot\Fsf) = \supp(\Fsf)$ ,
\item Pointwise product : $\supp(\Fsf \cdot \Gsf) \subseteq \supp(\Fsf) \cap \supp(\Gsf)$ ,
\item Involution : $\supp(\Fsf^\ast) = \supp(\Fsf)$ ,
\item Scalar multiple of the unit, $z\in\Cbb$ : $\supp(z\cdot\Ibb) = 0 $ ,
\end{itemize}
%
with $\Fsf, \Gsf \in \Fcal_0(\Mcal)$ and $\phi \in \Ecal(\Mcal)$.
\end{lemma}


Despite of these restrictions, the functionals in $\Fcal_0(\Mcal)$ are too general. In order to be able to properly work with a functional we have to be able to compute all its functional derivatives and these functional derivatives have to show some regularity.


Let us start formalizing this discussion giving the definition of functional derivative which will be used all along in the present work. Let $\Xsf$ be a locally convex topological vector space, a subset $\Usf \subseteq \Xsf$, is also said to be locally convex if every point $x \in \Usf$ has a neighborhood contained in $\Usf$. 


\begin{definition}[Functional derivative]\label{def:functional_derivative}
Let $\Xsf$ and $\Ysf$ be two locally convex topological vector spaces, and $\Usf \subseteq \Xsf$ an open subset. The functional derivative (or Gâteau derivative) of a map $\Fsf:  \Usf \to \Ysf$ at $\phi \in \Usf$ in the direction $\psi \in \Xsf$ is defined as the map $\Fsf^{(1)} : \Usf \times \Xsf \to \Ysf$,
%
\begin{equation*}
\Fsf^{(1)}(\phi)[\psi] = \lim_{t \to 0} \ \frac{1}{t} \bigg( \Fsf(\phi_n + t \psi) - \Fsf(\phi) \bigg) \ .
\end{equation*}
% 
The map $\Fsf$ is called \textbf{differentiable} (or Gâteau differentiable) at $\phi \in \Usf$ if the limit exists for all $\psi \in \Xsf$, and \textbf{continuously differentiable} if $\Fsf^{(1)}$ is jointly continuous on the product space $\Usf \times \Xsf$.
%
%
The generalization to the $n$-th functional derivative of $\Fsf$ at $\phi \in \Usf$ with respect to the directions $\psi_1, \dots, \psi_n \in \Xsf$ is defined as a map $\Fsf : \Usf \times \Xsf^{\otimes n} \to \Ysf$,
%
\begin{equation*}
\Fsf^{(n)}(\phi)[\psi_1,\dots ,\psi_n] = \lim_{t \to 0} \ \frac{1}{t} \bigg( \Fsf^{(n-1)}(\phi + t \psi_n)[\psi_1,\dots ,\psi_{n-1}] - \Fsf^{(n-1)}(\phi)[\psi_1,\dots ,\psi_{n-1}] \bigg) \ .
\end{equation*}
%
The map $\Fsf$ is said to be \textbf{smooth} at $\phi \in \Usf$ if the limit exists for all $\psi_1, \dots, \psi_n \in \Xsf$, and if $\Fsf^{(n)}$ is jointly continuous on the product space $\Usf \times \Xsf^{\otimes n}$.
\end{definition}


We recall that a map 
%
\begin{equation*}
\fsf : \Usf \times \Xsf \to \Ysf
\end{equation*}
%
is \textbf{jointly continuous} at $(x,y) \in \Usf \times \Xsf$, if 
%
\begin{eqnarray*}
&& \mbox{for each neighborhood} \ \Ysf^\prime \ \mbox{of} \ \fsf(x,y) \ , \\
&& \exists \ \mbox{a product of open sets} \ \Xsf^\prime \times \Usf^\prime \subseteq \Xsf \times \Usf \ \mbox{containing} \ (x,y) \ , \\
&& \mbox{such that} \ \fsf(\Xsf^\prime \times \Usf^\prime) \subseteq \Ysf^\prime \ .
\end{eqnarray*}
%
For later purposes we notice that the derivatives of a functional are distributions.


If instead of considering generic locally convex topological vector spaces $U$ and $V$ in the previous definition, we take $\Ecal(\Mcal)$ or $\Dcal(\Mcal)$ and $\Cbb$ respectively, then we have a precise definition of the derivative of an observable described by a functional.

\bigskip

We illustrate this definition by means of a simple example. 


\begin{example}
Here we compute the first two derivatives of a ``functional potential'' $\phi^4$. Let us consider a test function $\lambda \in \Dcal(\Mcal)$ and 
%
\begin{equation*}
\Vsf(\phi) := \int_\Mcal \dsf x \ \sqrt{\abs{g}} \ \frac{\lambda(x)}{4!} \phi(x)^4 \ ,
\end{equation*}
%
then
%
\begin{eqnarray*}
&& \Vsf^{(1)}(\phi)[\psi] = \int_\Mcal \dsf x \ \sqrt{\abs{g}} \ \frac{\lambda(x)}{3!} \phi(x)^3  \psi(x)\ , \\
%
&& \Vsf^{(2)}(\phi)[\psi_1,\psi_2] = \int_\Mcal \dsf x  \ \sqrt{\abs{g}} \ \frac{\lambda(x)}{2!} \phi(x)^2 \psi_1(x)\psi_2(x) \ ,
\end{eqnarray*}
%
or with a small common abuse of notation
%
\begin{equation*}
\Vsf^{(1)}(\phi) = \frac{\lambda(x)}{3!} \phi(x)^3 \ , \qquad
%
\Vsf^{(2)}(\phi) = \frac{\lambda(x)}{2!} \phi(x)^2 \delta(x,y)  
\end{equation*}
%
where $\delta$ is the Dirac delta function. 
\end{example}


We can show that the main results in differential calculus still hold in this framework.


\begin{lemma}
%
Let $\Fsf$ and $\Gsf$ be two smooth functionals, and let $\phi , \psi_{\sharp}$ contained in a locally convex topological vector space, e.g. $\Ecal(\Mcal)$ or $\Dcal(\Mcal)$. 
%
\begin{itemize}
%
\item fundamental theorem of calculus
%
\begin{equation*}
\Fsf(\phi + \psi) - \Fsf(\phi) = \int_0^1 \dsf t \ \Fsf^{(1)}(\phi+t\psi)[\psi] 
\end{equation*}
%
\item Taylor's formula
\begin{eqnarray*}
\Fsf(\phi + \psi) &=& \Fsf(\phi) + \Fsf^{(1)}(\phi)[\psi] + \dots + \frac{1}{n!} \Fsf^{(n)}(\phi)[\psi_1,\dots,\psi_n] \\
&& + \cfrac{1}{n!} \ \ \bigint_0^1 \dsf t \ (1-t)^n \ \Fsf^{(n+1)}(\phi+t\psi)[\psi^{\otimes n}]
\end{eqnarray*}
%
\item Leibniz formula
\begin{equation*}
\left(\Fsf \cdot \Gsf\right)^{(n)}(\phi)[\psi_1, \dots ,\phi_n] = \sum_{k=0}^{n} \binom{n}{k} \ \Fsf^{(k)}(\phi)[\psi_1, \dots , \psi_k] \ \Gsf^{(n-k)}(\phi)[\psi_1, \dots , \psi_{n-k}] \ .
\end{equation*}
%
\end{itemize}
%
\end{lemma}


We are now ready to introduce the space of observables we shall work with. It is a subset of $\Fcal_0(\Mcal)$, called the space of regular functionals. Let us give its definition.


\begin{definition}[Space of regular functionals]\label{def:obs_reg}
We define the space of regular functionals as follow
%
\begin{equation*}
\Fcal_{\mathsf{reg}}(\Mcal) = \left\{ \Fsf(\phi) \ \bigg| \ \Fsf(\phi) \in \Fcal_0(\Mcal) \ \mbox{ is smooth}, \ \Fsf^{(n)(\phi)} \in \Dcal^\prime(\Mcal^{n}) \right\} \ ,
\end{equation*}
%
where $\phi$ is a test function, i.e. an element of $\Ecal(\Mcal)$.
\end{definition}

This set still forms an algebra. Furthermore, the space $\Dcal^\prime(\Mcal^{n})$ which shall be define in details in section \ref{p:DISTRIB}. We shall for now work with regular functionals. 


%----------------------------------------------------------------------------%
\section{Classical free field theory}\label{p:CLASSICAL}
%----------------------------------------------------------------------------%


In the previous sections we have introduced the set of observables of a scalar field theory in the functional approach, we discuss now how to describe a classical field theory in this framework, therefore we have as equation of motion the \textbf{generalized Klein Gordon} equation
%
\begin{equation} 
\Psf \phi = \left( \Box + V^\prime \right) \phi = 0 \ , \
\mbox{ where } \ V^\prime = \xi R + m^2 + V \ . 
\label{eq:kg_eq}
\end{equation}
%
The coefficient $m$ denotes the (positive real) mass of the theory, $\xi \in \Rbb$ describes the coupling with the scalar curvature $R$, and $V$ is a non linear local potential.


In the present section \ref{p:CLASSICAL} we shall only consider $V=0$, i.e. we shall work only with a free theory. We also notice that, in the case of vanishing curvature ($\xi=0$), the generalized Klein Gordon equation \eqref{eq:kg_eq} reduces to the Klein Gordon equation of the free scalar field theory on Minkowski spacetime (cf. definition \ref{def:minkowski}). The case $\xi=0$ is called minimal coupling, and $\xi=\frac16$ conformal coupling (for more details look at \cite{WALD_1984}).


We recall that $\Mcal$ is chosen to be a globally hyperbolic spacetime (cf. definition \ref{def:cst}), therefore the differential equation \eqref{eq:kg_eq} admits an unique solution once we fix enough initial conditions on a generic Cauchy surface $\Sigma$. Namely, 
%
\begin{equation}
\Psf \phi = f \ , \quad \phi|_\Sigma = \phi_1 \ , \quad \nabla_n \phi |_\Sigma = \phi_2 \ ,
\label{eq:init_val_pb}
\end{equation}
%
where $f \in \Dcal(\Mcal)$, $\phi_1, \phi_2 \in \Dcal(\Sigma)$, and $n$ is a future directed timelike vector orthogonal to the Cauchy surface $\Sigma$ of $\Mcal$. It is actually a well posed initial value problem on $\Mcal$. The unique solution $\phi \in \Ecal(\Mcal)$ has the following support property
%
\begin{equation*}
\supp(\phi) \subset J\bigg( \supp(f) \cap \supp(\phi_1) \cap \supp(\phi_2),\Mcal \bigg) \ .
\end{equation*}
%
It has been shown in \cite{BGP_2007} that the \textbf{operator} $\Psf$ has unique \textbf{retarded} $\Delta_\rsf$ and \textbf{advanced} $\Delta_\asf$ \textbf{fundamental solutions}. In other words, there are unique continuous maps 
%
\begin{equation*}
\Delta_{\rsf / \asf} : \Dcal(\Mcal) \to \Ecal(\Mcal)
\end{equation*}
%
satisfying 
%
\begin{eqnarray}
&& \Psf_x \Delta_{\rsf/\asf}(x,y) = \Delta_{\rsf/\asf}(x,y) \Psf_x = \delta(x,y) \ , 
\quad \Delta_{\asf}(x,y) = \Delta_{\rsf}(y,x) \ , 
\label{eq:identity_adv_ret} \\
&& \mbox{and} \qquad \supp(\Delta_{\rsf/\asf} f) = J^{\pm} \bigg(\supp(f) , \Mcal\bigg) \ , \nonumber
\end{eqnarray}
%
where $\delta$ is the Dirac distribution, and $f$ is a test function.


\begin{wrapfigure}{r}{0.35\textwidth}
\begin{center}
\supportPropagator
\caption{Support of the causal propagator.}
\label{fig:supp_causal_prop}
\end{center}
\end{wrapfigure}


The support of $\Delta_{\mathsf{\rsf/\asf}} f$ is in the causal future (respectively past) of the test function $f$. The difference of the two fundamental solutions is called \textbf{causal propagator} of $\Psf$, actually it is defined as follow
%
\begin{equation*}
\Delta := \Delta_\rsf - \Delta_\asf \ , \quad \mbox{thus} \qquad \Psf_x \Delta(x,y) = 0 \ .
\end{equation*}
%
It is a map from the  set of compactly supported  test functions to the one of smooth test functions. Its support, as given in (2.6) is the union of the causal future and the causal past of the element applied to it (cf.figure \ref{fig:supp_causal_prop}). 


Furthermore if $\Delta f = 0$, then for some $f, g\in\Dcal(\Mcal)$ we have $f = \Psf g$, . Finally $\Psf$ is formally self adjoint, i.e. 
%
\begin{eqnarray}
&&\Delta(f,g) = - \Delta(g,f) \\
%
\mbox{with} && \Delta(f,g) := \sm{f,\Delta g} = \int_\Mcal \dsf x \ \sqrt{\abs{g}} \ f(x) \ \left(\Delta g\right)(x) \ , \nonumber
\label{eq:smearing}
\end{eqnarray}
%
actually this property holds because $\Delta_\rsf$ is the formal adjoint of $\Delta_\asf$ and vice versa.


\bigskip


Historically classical on shell field theories have been quantized starting from a symplectic structure. A such space is construct from the space of smooth real valued solutions of $\Psf \phi = 0$ \eqref{eq:kg_eq} with compactly supported initial conditions on a Cauchy surface $\Sigma$, i.e.
%
\begin{equation}
\mathcalligra{S} \ \ (\Mcal) = \left\{ 
\begin{array}{l}
\phi \in \Ecal(\Mcal) \mbox{ such that } \Psf\phi=0 \ , \\
\mbox{ and } \phi \mbox{ has compactly supported initial values on } \Sigma 
\end{array}
\right\} \ .
\label{eq:sol_space}
\end{equation}
%
We equip \eqref{eq:sol_space} with a strongly non degenerate map $\tau$, called symplectic form. It is defined as follow
%
\begin{equation*}
\tau : \Bigg\{
\begin{array}{ccl}
\mathcalligra{S} \ \ (\Mcal) \times \mathcalligra{S} \ \ (\Mcal) & \to & \Rbb \\
(\phi \ , \ \psi) & \mapsto & \bigint_\Sigma  \dsf \Sigma \ \bigg( \psi \ (\nabla_n \phi) - \phi \ (\nabla_n \psi) \bigg)
\end{array}
\ ,
\end{equation*}
%
where $\nabla_n = n_\mu \nabla^\mu$ is the normal future directed derivative on $\Sigma$, and $\phi_1, \phi_2 \in \Dcal(\Sigma)$  are the initial data on $\Sigma$ of the solution $\phi$ of the initial value problem \eqref{eq:init_val_pb} with $f=0$. We denote this space, called symplectic space, by 
%
\begin{equation*}
\left(\mathcalligra{S} \ \ (\Mcal),\tau\right) \ . 
\end{equation*}
%
The definition of the symplectic form is independent of the choice of $\Sigma$ we make. Indeed we can identify $\tau$ as the integral over $\Sigma$ of the current $j^\mu$ defined as follow
%
\begin{equation*}
j^\mu (\phi,\psi) := \psi (\nabla^\mu \phi) - \phi (\nabla^\mu \psi),  
\label{eq:current}
\end{equation*}
%
where $\phi$ and $\psi$ are two fields in the space of solutions \eqref{eq:sol_space}. Using the Klein Gordon equation we can show that $j^\mu$ is covariantly conserved, i.e. $\nabla_\mu j^\mu = 0$. Hence the divergence theorem implies that the choice of the Cauchy surface $\Sigma$ does not matter. 


Moreover the symplectic form $\tau$ can be written in terms of the causal propagator $\Delta$. Let us collect this result in the following proposition.


\begin{proposition}[Relation between the symplectic form and the causal propagator.]
Let $\phi$ and $\psi$ be two elements of the space of solutions \eqref{eq:sol_space}, with compactly supported initial conditions on a Cauchy surface $\Sigma$ of $\Mcal$. Then for every bounded region $\Ncal(\Sigma)$ of $\Sigma$ we can find $f,g \in \Dcal(\Mcal)$such that
%
\begin{equation*}
\phi = \Delta f \ , \qquad  \psi   =  \Delta g \ ,
\end{equation*}
%
with the support of $f$ and $g$ contain in $\Ncal(\Sigma)$. Second the symplectic form $\tau$ can be written in terms of the causal propagator $\Delta$ 
%
\begin{equation*}
\tau(\phi,\psi) = \Delta(f,g) \ , \quad \mbox{where} \qquad \Delta(f,g) = \sm{f,\Delta g} \ ,
\end{equation*}
%
\end{proposition}


\begin{sketch}
For any element $\phi$ of the space of solutions \eqref{eq:sol_space} we can find a compactly supported smooth function $f \in \Dcal(\Mcal)$ such that $\phi = \Delta f$. We consider a Cauchy surface $\Sigma$, and a bounded neighborhood of $\Bcal(\Sigma)$ such that it lies in the future of $\supp(f)$ (we can do the same for $\Bcal(\Sigma)$ lying in the past of $\supp(f)$). We consider two other Cauchy surfaces $\Sigma_1$ and $\Sigma_2$ contained in $\Bcal(\Sigma)$, such that $\Sigma_1$ is in the future of $\Sigma$, and $\Sigma_2$ is in the past of $\Sigma$. We define the function $\chi \in \Ecal(\Mcal)$ such that $\chi =1$ in the past of $\Sigma_1$, and $\chi = 0$ in the future of $\Sigma_2$. Then we can write
%
\begin{eqnarray*}
\Psf \chi \phi &=& (\Box + V^\prime) \chi \phi = (\Box \chi) \phi + (\nabla \chi) (\nabla \phi) := f \ , \\
\Psf (1- \chi)\phi &=& - (\Box \chi) \phi - (\nabla \chi) (\nabla \phi) = - f \ , 
\end{eqnarray*}
%
where $f \in \Dcal(\Mcal)$ because the support of both $\Box \chi$ and $\nabla \chi$ is contained in $J^+(\Sigma_1)\cap J^-(\Sigma_2)$ and the support of $\phi$ is spatially compact therein. It permits to have
%
\begin{eqnarray*}
\Delta_\rsf \Psf \chi \phi = \Delta_\rsf f \ , \quad \mbox{and} \quad  \Delta_\asf \Psf (1-\chi) \phi = - \Delta_\asf f \ .
\end{eqnarray*}
%
Therefore
%
\begin{equation*}
\Delta_\rsf \Psf \chi \phi + \Delta_\asf \Psf (1-\chi) \phi = (\Delta_\rsf - \Delta_\asf) f \ \Leftrightarrow \phi = \Delta f \ .
\end{equation*}
%
For all solutions $\phi$ of the equation of motion $\Psf \phi =0$ \eqref{eq:kg_eq}, with compactly supported initial conditions on a Cauchy surface $\Sigma$, it is then possible to find a (non unique) compactly supported smooth function $f$ such that $\phi = \Delta f$.\par%
%
We consider now $\phi$ and $\psi$ as elements of the space \eqref{eq:sol_space}, then we have for suitable test functions $f$ and $g$, $\phi=\Delta f$ and $\psi=\Delta g$.\par%
%
We write the definition of $\Delta(f,g)$ defined in \eqref{eq:smearing}, where we integrate over a subspace $\Ncal$ of $\Mcal$ according to the support of $f$ and $g$. Then we divide $\Ncal$ in two regions. The two regions are defined as follow
%
\begin{equation*}
\Sigma^+ = J^+(\Sigma,\Mcal) \setminus \Sigma \ , \quad \mbox{and} \qquad \Sigma^- = J^-(\Sigma,\Mcal) \setminus \Sigma \ 
\end{equation*}
%
where $\partial \Sigma^+ = \partial \Sigma^- = \Sigma$. 
Therefore we now have  two integrals in which we use the identity \eqref{eq:identity_adv_ret}
%
\begin{eqnarray*}
\Delta(f,g) &=& \int_{\Sigma^+} \dsf x \ \sqrt{\abs{g}} \ \psi \Psf \Delta_\asf f \ + \int_{\Sigma^-} \dsf x \ \sqrt{\abs{g}} \ \phi \Psf \Delta_\rsf g \\
%
&=& 
\int_{\Sigma^+} \dsf x \ \sqrt{\abs{g}} \ 
\left( \psi \Psf \Delta_\asf f + \Delta_\asf f \Psf \psi \right) 
+ 
\int_{\Sigma^-} \dsf x \ \sqrt{\abs{g}} \ 
\left( \phi \Psf \Delta_\rsf g + \Delta_\rsf g \Psf \phi \right) \ ,
\end{eqnarray*}
%
where the second equality is obtained due to the fact that $\phi$ and $\psi$ are solutions of the equation of motion\eqref{eq:kg_eq}. Next step we apply the Stokes' theorem, and then we identify the causal propagator to the symplectic form $\tau(\phi,\psi)$.
\end{sketch}


However the symplectic quantization is an on shell quantization, and we would like to describe an interacting real scalar field theory, where the interaction shall be treat by means of perturbation theory, thus the requirement that an equation of motion is satisfied shall be dropped, in other words we shall work off shell. 


We shall quantize the classical theory with the idea to give a large importance to the algebraic structure. The procedure used is the formal deformation present in section \ref{p:Q_DEFORM}. Classical and quantum observables shall be seen as in the same vector space but endowed with different products, which will give different algebras. Therefore we now define the classical off shell $\ast$-algebra of a free field theory on regular functionals, where the involution is defined as follow
%
\begin{equation}
\Fsf^\ast(\phi) =  \overline{\Fsf(\overline{\phi})} \ , 
\label{eq:involution}
\end{equation}
%
where the operation considered on the right hand side is the complex conjugation. 


\begin{definition}[Classical free off shell algebra]\label{def:alg_clas}
We define the classical free off shell algebra as follow
%
\begin{equation}
\Acal_\reg(\Mcal) = \left(\Fcal_\reg(\Mcal), ^\ast , \cdot\right) \ ,
\label{eq:alg_clas}
\end{equation}
%
with $\Fcal_\reg(\Mcal)$ the space of regular functionals introduced in definition \ref{def:obs_reg}.
\end{definition}


%----------------------------------------------------------------------------%
\section{Wave front set of a distribution}
%----------------------------------------------------------------------------%


We introduced in the section \ref{p:OBS} the differentiability of functionals, cf. definition \ref{def:functional_derivative}. It shall be used to define spaces of observables having the correct structure to perform quantization of the classical theory. We shall see in section \ref{p:INTERACTING_PICTURE} that we need to consider pointwise products of derivatives of non regular functionals, i.e. pointwise products of distributions. 


Therefore we shall in this section introduce the concept of distribution on a manifold, and characterize its singularities via the concept of wave front set. Finally we shall discuss a criterion due to Hörmander, see e.g. \cite{HORMANDER_1990}, which permits to define the pointwise product of distributions.


%----------------------------------------------------------------------------%
\subsection{Distributions on a manifold}
\label{p:DISTRIB}
%----------------------------------------------------------------------------%


Let us first study distributions on $\Rbb^n$ and then generalize it to $\Mcal$. 


\bigskip


The set $\Xsf$ denotes for now on an open subset of $\Rbb^n$. As defined previously the spaces $\Dcal(\Xsf)$ and $\Ecal(\Xsf)$ denote respectively the space of compactly supported smooth functions on $\Xsf$ and the space of smooth functions on $\Xsf$. We call $u$ a \textbf{distribution} on $\Xsf$, if it is a linear form on $\Dcal(\Xsf)$ such that for every compact set $\Ksf \subset \Xsf$ there exist two constants $C$ and $k$ such that
%
\begin{equation*}
\abs{u(\phi)} \leq C \sum_{\abs{\alpha} \leq k} \sup \abs{\partial^\alpha \phi} \ , \quad \phi \in \Dcal(\Xsf) \ ,
\end{equation*}
%
where we used the same conventions as in \eqref{eq:conv_multiindex}. The set of all distributions in $\Xsf$ is denoted by $\Dcal^\prime(\Xsf)$. It is actually the dual space of $\Dcal(\Xsf)$. Thanks to linearity, we can also characterize a distribution in the following equivalent way. A linear form $u$ on $\Dcal(\Xsf)$ is a distribution if and only if 
%
\begin{equation*}
u(\phi_j) \to 0 \qquad  \mbox{when} \qquad j \to \infty \ ,
\end{equation*}
%
for every sequence $\phi_j \in \Dcal(\Xsf)$ converging to $0$ in the sense that
%
\begin{equation*}
\sup_{K}\abs{\partial^\alpha\phi_j} \to 0 \ ,
\end{equation*}
%
for every fixed $\alpha$ and $\supp(\phi_j) \subset K$ for all $j$ and some fixed compact set $K \subset \Xsf$. We invite the reader to look at the proof in the monograph of L. Hörmander \cite[theorem 2.4]{HORMANDER_1990}. Another equivalent definition is given in \cite[theorem 2.15]{HORMANDER_1990}.


Notice that it is not possible to evaluate a distribution $u \in \Dcal^\prime(\Xsf)$ on a single point in $\Xsf$, but we can \textbf{restrict} $u$ to a subset $\Usf \subset \Xsf$. 


As for functions we can define the \textbf{support} of $u \in \Dcal^\prime(\Xsf)$, denoted by $\supp(u)$. It is the set of points in $\Xsf$ having no open neighborhood in which the restriction of $u$ is $0$, i.e. 
%
\begin{equation}
\supp(u) = \left\{ x \in \Xsf \ \left|
\begin{array}{l}
\forall u \in \Dcal^\prime(\Xsf) \ \mbox{ we have } \ u(\phi) \neq 0 \ \mbox{ for all test functions } \ \phi \ , \\ 
\mbox{with } \supp(\phi) \subseteq \Usf \ \mbox{ for any open neighborhood } \ \Usf \subseteq \Xsf \ \mbox{ of } \ x 
\end{array}
\right. \right\} \ .
\label{eq:supp_distribution}
\end{equation}
%
If a distribution $u$ has \textbf{compact support}, then $u \in \Ecal^\prime(\Xsf)$ is called \textbf{compactly supported distribution}. 


\bigskip


Let us now generalize this notion to \textbf{the case of a manifold} $\Mcal$. We recall that $\Ecal(\Mcal,E)$ and $\Dcal(\Mcal,E)$ are respectively the spaces of smooth and compactly supported smooth sections of the vector bundle $E$ (see section \ref{p:LORENTZIAN_STRUCTURE}). Their topologies for the case of the real line bundle have been studied in section \ref{p:TOPO_CONFIG_SPACE}. A \textbf{distribution of the vector bundle $E$ with value in $\Ysf$} is a linear continuous map 
%
\begin{equation*}
u : \Dcal(\Mcal,E^\ast) \to \Ysf \ ,
\end{equation*}
%
where $E^\ast$ is the dual space of $E$. A such linear map $u$ is called continuous if for any $(\phi_n)$ elements in $\Dcal(\Mcal,E^\ast)$ where $\phi_n \to \phi \in \Dcal(\Mcal,E^\ast)$, we have $u(\phi_n) \to u(\phi)$. The space formed by these continuous linear maps is denoted by $\Dcal^\prime(\Mcal,E,\Ysf)$. We shall restrict ourselves for now to the complex line vector bundle. Therefore with a small common abuse of notation we shall denote this space by $\Dcal^\prime(\Mcal)$. 


As above we can define the support of a distribution $u \in \Dcal^\prime(\Mcal)$ and it is actually defined in the same way as \eqref{eq:supp_distribution}. For every submanifold $\Ncal \subset \Mcal$, any distribution $u \in \Dcal^\prime(\Mcal)$ can be restricted to a distribution $\left.u\right|_{\Ncal} \in \Dcal^\prime(\Ncal)$ such that
%
\begin{equation*}
\left.u\right|_{\Ncal}(\phi) = u(\phi) \ , 
\end{equation*}
%
for all $\phi \in \Dcal(\Ncal)$. 


We denote by $\Ecal^\prime(\Mcal,E,\Ysf)$ the space of compactly supported distributions of the vector bundle $E$ with value in $\Ysf$. As said above we shall only consider the complex line bundle thus with the same abuse of notation as for the space of smooth distributions, we shall denote the space of compactly supported distributions by $\Dcal^\prime(\Mcal)$.


\bigskip


Let us illustrate the notions collected above with few examples. 


\begin{example}
\begin{itemize}

\item A first example is the Dirac distribution $\delta \in \Dcal^\prime(\Rbb)$ defined as follow
%
\begin{equation*}
\sm{\delta , \phi} = \phi(0) \ , 
\end{equation*}
%
for $\phi \in \Dcal(\Rbb)$. We now compute its derivatives
%
\begin{equation*}
\sm{\partial^k \delta , \phi} = (-1)^k \sm{\delta , \partial^k \phi} =  (-1)^k \partial^k \phi(0) \ ,
\end{equation*}
%
with $k \in \Nbb \setminus \{0\}$, and where the first equality is obtained after performing $k$ successive integrations by part. 


\item A second example can be presented with the Heaviside function for which we associate the distribution $\Theta$ as follows
%
\begin{equation*}
H(x) = \left\{
\begin{array}{l}
1 \ \mbox{ if } \ x > 0 \\
0 \ \mbox{ if } \ x \leq 0
\end{array}
\right. \ , \qquad \mbox{and} \qquad \sm{\Theta,\phi} \ \int_\Rbb \dsf x \ H(x) \phi(x) \ ,
\end{equation*}
%
where $\phi$ is a suitable test function. We compute its derivative as follow
%
\begin{equation*}
\sm{\partial \Theta , \phi} = - \sm{\Theta , \partial \phi} = - \int_\Rbb \dsf x \ H(x) \partial \phi(x) = - \int_0^\infty \dsf x \ \partial \phi(x) = \phi(0) = \sm{\delta , \phi} \ .
\end{equation*}
%
Thus
%
\begin{equation*}
\partial \Theta = \delta \ . 
\end{equation*}


\item A last example is the distribution $u \in \Dcal^\prime(\Rbb)$ given by
%
\begin{equation*}
u(x) = \lim_{\epsilon \downarrow 0} u_\epsilon \ , \quad \mbox{with} \qquad u_\epsilon(x) = \frac{1}{x^2 + i \epsilon} \ ,
\end{equation*}
%
and where the limit is taken is a weak sense, i.e. in a distributional sense.

\end{itemize}
\end{example}


We shall look a bit more to these examples is the sections \ref{p:FOURIER} and \ref{p:SING_WF}.


%----------------------------------------------------------------------------%
\subsection{Schwartz's space and tempered distributions}
\label{p:FOURIER}
%----------------------------------------------------------------------------%


We shall introduce here the concept of Fourier transform for functions and distributions on $\Rbb^n$. We shall define a new space of functions which have ``good'' property under the action of Fourier transform.


\bigskip


Let us consider an integrable function $f$ on $\Rbb^n$. The \textbf{Fourier transform} of $f$ is defined as follow
%
\begin{equation*}
\Frak : f \mapsto  \hat{f}(k) = \int_{\Rbb^n} \dsf x \ f(x) \ \esf^{-i x \cdot k} \ , 
\end{equation*}
%
where $k \in \Rbb^n$, and $x \cdot k$ is the inner product in $\Rbb^n$. The pointwise product of two Fourier transforms of two integrable functions $f$ and $g$ over $\Rbb^n$ can be written as the Fourier transform of the convolution of $f$ and $g$, i.e.
%
\begin{equation*}
\hat{f} \cdot \hat{g} = \Frak( f \ast g )(x) \ , \quad \mbox{with} \qquad (f \ast g) := \int_{\Rbb^n} \dsf y \ f(y) g(x-y) \ . 
\end{equation*}
%
There exists a particular space of functions which behaves particularly well under the action of $\Frak$. It is the subspace of $\Ecal(\Rbb^n)$ where all functions $\phi$ are \textbf{rapidly decreasing}, i.e.
%
\begin{equation}
\phi \in \Ecal(\Rbb^n) \ \ \mbox{ where } \ \ \norm{\phi}_{\alpha\beta} := \sup_x \abs{x^\alpha \partial^\beta \phi(x)} < \infty \ ,
\label{eq:norm_schwartz}
\end{equation}
%
and all multiindices $\alpha$ and $\beta$, where we used the same convention as in \eqref{eq:conv_multiindex}. We call this function space the \textbf{Schwartz space} and denote it by $\Scal(\Rbb^n)$. It is a Fréchet space with the semi norm defined in \eqref{eq:norm_schwartz}. We can say the Schwartz space is a Fréchet space. 


In addition of being a subspace of $\Ecal(\Rbb^n)$ this new space is contained in $\Dcal(\Rbb^n)$, i.e.
%
\begin{equation*}
\Dcal(\Rbb^n) \subset \Scal(\Rbb^n) \subset \Ecal(\Rbb^n) \ .
\end{equation*}


As main properties we have for $\phi \in \Scal(\Rbb^n)$ 
%
\begin{eqnarray}
&& \Frak(\partial^\alpha \phi)(k) = k^\alpha \hat{\phi}(k) \ , \qquad \Frak(x^\alpha \phi)(k) = \partial^\alpha \hat{\phi}(k) \ , \nonumber \\[6pt]
&\mbox{and}& \phi(x) = (2\pi)^{-n} \int_{\Rbb^n} \dsf x \ \hat{\phi}(k) \ \esf^{i x \cdot k} \ .
\label{eq:inversion_formula}
\end{eqnarray}
%
The relation \eqref{eq:inversion_formula} is called the inversion formula. We can show that the map
%
\begin{equation*}
\Frak : \Scal(\Rbb^n) \to \Scal(\Rbb^n) \ , \quad \mbox{and its inverse} \qquad \Frak^{-1} : \Scal(\Rbb^n) \to \Scal(\Rbb^n)
\end{equation*}
%
are two continuous isomorphisms. The Fourier transform $\Frak$ can be used to give a criterion for the smoothness of functions in $\Dcal(\Rbb^n)$. Indeed we have the following equivalence property
%
\begin{equation}
\phi \in \Dcal(\Rbb^n) \ \  \Longleftrightarrow  \ \ \abs{\hat{\phi}(k)} \leq C_n (1+\abs{k})^{-n} \ , \ \mbox{ for } \  C_n \in \Rbb \ , \ \ n \in \Nbb \setminus \{0\} \ , \ \mbox{ and } \ \ k \in \Rbb^n \ .
\label{eq:criterion_smoothness_function}
\end{equation}
%
In words \eqref{eq:criterion_smoothness_function} means a function $\phi$ is smooth compactly supported if and only if its Fourier transform decays faster than any negative power of the dual variable $k$.


We can extend the concept of Fourier transform to the distributional framework. The dual $\Scal^\prime(\Rbb^n)$ is the space of all continuous linear form on $\Scal(\Rbb^n)$. We can show that $\Scal^\prime(\Rbb^n)$ is a space of distribution with the following property
%
\begin{equation*}
\Ecal^\prime(\Rbb^n) \subset \Scal^\prime(\Rbb^n) \subset \Dcal^\prime(\Rbb^n) \ .
\end{equation*}
%
Distributions on $\Scal^\prime(\Rbb^n)$ are called a \textbf{tempered distributions}. If $u \in \Scal^\prime(\Rbb^n)$, its Fourier transform is defined for $\phi \in \Scal(\Rbb^n)$ as follow
%
\begin{equation}
\sm{\hat{u},\phi} = \langle u,\hat{\phi}\rangle \ ,
\label{eq:fourier_distrib}
\end{equation}
%
where we used the smearing $u(\phi) = \sm{u,\phi}$. The inversion formula here can be written as
%
\begin{equation}
\sm{u,\check{\phi}} = (2\pi)^{-n} \sm{\hat{u} , \hat{\phi}} \ ,
\label{eq:inversion_fourier_distrib}
\end{equation}
%
for $u \in \Scal^\prime(\Rbb^ n)$ and $\phi \in \Scal(\Rbb^n)$. In order to give an adapted version of \eqref{eq:criterion_smoothness_function}  for the distributions, we notice that the Fourier transform of $u \in \Ecal^\prime(\Rbb^n)$ can be equivalently seen as a smooth function in the dual variable $k$ defined as follow
%
\begin{equation}
\hat{u}(k) = \sm{ u , \phi(x) } \ , \quad \mbox{with} \qquad \phi(x) = \esf^{-i x \cdot k} \ .
\label{eq:distrib_smooth_funct}
\end{equation}
%
Thus we have the following lemma.


\begin{lemma}\label{lem:smooth_criterion_distrib}
For $u \in \Dcal^\prime(\Rbb^n)$ and $\Xsf \subset \Rbb^n$, using the characterization \eqref{eq:distrib_smooth_funct}, we say that the \textbf{restriction} $\left.u\right|_{\Xsf}$ is \textbf{smooth, if and only if} for all $\phi$ in $\Dcal(\Xsf)$ defined as follow
%
\begin{equation*}
\phi(x) = \esf^{-i x \cdot k} f(x) \ ,
\end{equation*}
%
where $f$ is a suitable test function, we have 
%
\begin{equation}
\abs{\sm{ u , \phi(x) }} \leq C_n (1+\abs{k})^{-n} \ , 
\label{eq:smooth_criterion_distrib}
\end{equation}
for $C_n \in \Rbb$, $n \in \Nbb \setminus \{0\}$, and $k\in \Rbb^n$.
\end{lemma}


The convolution of $u \in \Scal^\prime(\Rbb^n)$ and $v\in \Ecal^\prime(\Rbb^n)$ is in $\Scal^\prime(\Rbb^n)$ and its Fourier transform is equal to the pointwise product of the Fourier transforms of $u$ and $v$, i.e.
%
\begin{equation*}
\Frak(u \ast v) = \hat{u} \cdot \hat{v} \ .
\end{equation*}
%
The Fourier transform map $\Frak$, which maps $\Scal^\prime(\Rbb^n)$ to itself, is a sequentially continuous isomorphism, whereas its inverse is only sequentially continuous. We recall that a map $\fsf : \Xsf \to \Ysf$ where $\Xsf$ and $\Ysf$ are at least topological spaces, is said to be sequentially continuous if for any convergent sequence $(x_n) \to x$ in $\Xsf$, we have $(\fsf(x_n)) \to \fsf(x)$ in $\Ysf$. 


%----------------------------------------------------------------------------%
\subsection{Singularities and wave front set}
\label{p:SING_WF}
%----------------------------------------------------------------------------%


We will give here a way to characterize the singularities of a distribution introducing the notion of wave front set. We shall see that it tells us not only at which points a singularity occurs, but it also indicates the directions in the dual space from which the singularities are ``coming'' from. We shall first analyze the particular case of $\Rbb^n$ and then generalize it to a curved spacetime $\Mcal$. 


Before introducing the precise definition of the wavefront set, let us start recalling some preliminary important notions.


\begin{itemize}
\setlength\itemsep{0pt}
%
%
%
%
\item We define the \textbf{singular support} of a distribution $u$, $\singsupp(u)$, as the set of points in $\Xsf \subset \Rbb^n$ having no open neighborhood to which the restriction of $u$ is a smooth function (cf. lemma \ref{lem:smooth_criterion_distrib}). 
%
%
%
%
\item A set $\Gamma \subset \Rbb^n \setminus \{0\}$ is called a \textbf{conic set} if for any point $k \in \Sigma$, it contains all the points $a k$ with $a > 0$. By a \textbf{conic neighborhood} of a point $k \in \Xsf \setminus \{0\}$ we mean an open conic set that contains the point $k$. 
%
%
%
%
\item We can now define what is a \textbf{regular direction} of a compactly supported distribution $u\in\Ecal^\prime(\Xsf)$. It is a vector $k \in \Rbb\setminus\{0\}$ such that there exists an open neighborhood $\Gamma$ of $k$, over which $\hat{u}$ is rapidly decreasing. In other words, the relation \eqref{eq:smooth_criterion_distrib} holds for every $k \in \Gamma$.\par%
%
\item Conversely a \textbf{singular direction} of a distribution $u$ is a direction which is not regular. The set of singular directions is denoted by $\Sigma (u)$, i.e.
%
\begin{equation}
\Sigma(u) = \bigg\{ k \in \Rbb^n \setminus \{0\} \ \mbox{ is not a regular direction of} \ u  \bigg\} 
\label{eq:sing_direction}
\end{equation}
%
%
%
%
\end{itemize}


The singular support gives the location of the singularities in space and the set of singular directions gives the high frequencies which are the sources of these singularities. The idea is to combine these two notions. Thus the set of singular directions of $u\in\Dcal^\prime(\Xsf)$ at a point $x$ is defined as follow
%
\begin{equation*}
\Sigma_x(u) := \underset{\phi \in \Dcal(X)}{\bigcap} \ \Sigma(\phi u) \ , 
\end{equation*}
%
with $\phi(x) \neq 0$. Since it is an intersection of closed conic sets, $\Sigma_x(u)$ is also a closed conic set. The set $\Sigma_x(u)$ encodes both notions namely the singular support and the singular directions of $u$. This permits us to define the notion of wave front set (cf. \cite{HORMANDER_1990}).


\begin{definition}[Wave front set of a numerical distribution]\label{def:wf}
If $u \in \Dcal^\prime(\Xsf)$, then the closed subset of $\Xsf \times (\Rbb^n \setminus \{0\})$ defined by
%
\begin{equation*}
\WF(u) := \bigg\{ (x,k) \in X \times (\Rbb^n \setminus \{0\}) \ , \ \mbox{ with } \ k \in \Sigma_x(u) \bigg\}
\end{equation*}
%
is called the \textbf{wave front set} of $u$.  
\end{definition}


The projection of $\WF(u)$ to $\Xsf$ is its singular support $\mathsf{singsupp}(u)$. If $u \in \Ecal^\prime(\Rbb^n)$ then the projection of $\WF(u)$ on $(\Rbb^n \setminus \{0\})$ is the set of its singular directions $\Sigma(u)$. 


The main properties of the wave front set defined in definition \ref{def:wf}, which can also be extended for the situation on manifold, are the following \cite{HORMANDER_1990}.


\begin{lemma}[Properties of the wave front set]\label{lem:prop_wf}
For $u$ and $v$ in $\Dcal^\prime(\Rbb^n)$, we have the following results.
%
\begin{eqnarray*}
&1.& \WF(\phi u) \subset \WF(u) \ , \\
&2.& \WF(u+v) \subset \WF(u) \cup \WF(v) \ , \\ 
&3.& \WF(\Psf u) \subset \WF(u) \ , 
\end{eqnarray*}
%
with $\phi\in\Dcal(\Rbb^n)$, and $\Psf$ a generic linear differential operator. 
\end{lemma}


\bigskip


We would like now to be able to compute the wave front set of pointwise products of distributions. We notice first that if two distributions $u$ and $v$ have disjoint singular support then the pointwise product of $u$ and $v$ is well defined.


\bigskip


The notion of Fourier transform introduced in section \ref{p:FOURIER} can give a method to build the pointwise product of two distributions. Let us notice that the Fourier transform of a product of distributions, if it exists, is the convolution of the Fourier transforms of these distributions. Thus the idea would be to define the product of distributions as the inverse Fourier transform of the convolution of the Fourier transform of these distributions. Nonetheless we cannot define the Fourier transform of a distribution in $\Dcal^\prime(M)$ globally, hence we need to localize the distributions before computing the pointwise product in the way described above. 



\begin{definition}[Pointwise product of numerical distributions]\label{def:prod_distib_fourier} 
Let $u, v \in \Dcal^\prime(\Rbb^n)$. We say that $w\in\Dcal^\prime(\Rbb^n)$ is the \textbf{pointwise product} of $u$ and $v$ on the compact set $K \subset \Rbb^n$ \textbf{if and only if} for all $x \in \Rbb^n$ there exists a function compactly supported $\phi$ such that
%
\begin{equation*}
\widehat{u}(k) = \sm{u , \phi(x)} \ , \quad \phi(x) = \chi(x) \esf^{-ixk} \ ,
\end{equation*}
%
where $\chi$ is a suitable smooth characteristic function of the domain $K$ where we want to compute the pointwise product, and  where the following integral is \textbf{absolutely convergent} for all $k\in\Rbb^n$
%
\begin{equation*}
\widehat{w}(k) := \left(\widehat{u} \ast \widehat{v}\right)(k) = (2\pi)^{-n} \int_{\Rbb^n} \dsf q \ \widehat{u}(q) \ \widehat{v}(k-q) .
\end{equation*}
%
We defined $\widehat{v}(k)$ in the same way as we did for $\widehat{u}(k)$ with an appropriate characteristic functions of $K$.
\end{definition}


However, definition \ref{def:prod_distib_fourier} uses the notion of Fourier transform which cannot be generalized to $\Dcal^\prime(\Mcal)$ in contrast to the wave front set as just seen above. Furthermore, the obtained distribution $w$ has compact support and coincide with the pointwise product of $u$ and $v$ only on $K$.


\bigskip


Therefore let us try to give a criterion for the multiplication of distributions with the notion of wave front set. We shall first notice that the wave front set of a distributions does not depend on the particular coordinates we are considering. Actually, this observation will permit us to generalize the notion of wave front set to manifold.


\bigskip


Let $\Xsf$ be an open subset of $\Rbb^n$, and $\Gamma$ a closed cone in $\Ysf \times\left(\Rbb^n\setminus\{0\}\right)$. We define a new space of distributions taking account of the wave front set
%
\begin{equation*}
\Dcal^\prime_\Gamma(\Ysf) = \left\{ u \in \Dcal^\prime(\Ysf) \ , \ \mbox{ with } \ \WF(u) \subset \Gamma \right\} \ . 
\end{equation*}
%
For a smooth function $F$ which maps $\Xsf$ to another open subset of $\Rbb^n$ denoted $\Ysf$, we define the normal set associated to this function as follow
%
\begin{equation*}
N_F = \left. \left\{ (F(x),k) \in \Ysf \times \Rbb^n \ \right| \ x \in \Xsf \ , \ T^\ast_xF(k) = 0 \right\} \ , 
\end{equation*}
%
where $T^\ast_xF$ is the cotangent map of $F$ at $x$. It is shown in \cite[theorem 8.2.4]{HORMANDER_1990} that if we have
%
\begin{equation}
N_F \cap \Gamma = \emptyset \ , 
\label{eq:cond_pullback_wf}
\end{equation}
%
then the pull back $F^\ast : \Ecal(\Ysf) \to \Ecal(\Xsf)$ has a unique sequentially continuous\footnote{We recall that a map $\fsf : \Xsf \to \Ysf$ where $\Xsf$ and $\Ysf$ are at least topological spaces, is said to be sequentially continuous if for any convergent sequence $(x_n) \to x$ in $\Xsf$, we have $(\fsf(x_n)) \to \fsf(x)$ in $\Ysf$.} extension map 
%
\begin{equation*}
F^\ast : \Dcal^\prime_\Gamma(\Ysf) \to \Dcal^\prime_{F^\ast\Gamma}(\Xsf) \ , 
\end{equation*}
%
denoted with the same symbol, and where
%
\begin{equation*}
F^\ast\Gamma = \left\{ \left(x , T^\ast_xF(k)\right) \ \mbox{ where } \ (F(x),k) \in \Gamma \right\} \ .
\end{equation*}
%
In particular for any $u \in \Dcal^\prime(\Ysf)$ if the condition $N_F \cap \WF(u) = \emptyset$ is satisfied then
%
\begin{equation*}
\WF(F^\ast u) \subset F^\ast \WF(u) \ .
\end{equation*}


From this point it is possible to extend the definition of wave front set to the the case of distribution defined on a manifold $\Mcal$. The wave front set of a distribution $u \in \Dcal^\prime(\Mcal)$, roughly speaking, is a conic subset of the cotangent bundle, i.e.
%
\begin{equation}
\WF(u) \subset T^\ast\Mcal\setminus\{0\} 
\end{equation}
%
where the first component gives the singular support, $\singsupp(u)$, and the second gives the direction in which the Fourier transform of $u$ does not decrease rapidly. Considering an atlas $\{(U_i,\Phi_i)\}$ covering $\Mcal$, a distribution $u\in \Dcal^\prime(\Mcal)$ can restricted to $u_i \in \Dcal^\prime(U_i)$ if the conormal bundle of each open set $U_i$ does not intersect $\WF(u)$. In this case $\WF(u)$ is defined as follow 
%
\begin{equation*}
\WF(u) = \bigcup_i \WF(u_i) \ ,
\end{equation*}
%
where $\WF(u_i)$ is the wave front set of the restriction $u_i \in \Dcal^\prime(U_i)$
%
\begin{equation*}
\WF(u_i) = \bigg\{ \bigg( x, T_x^\ast\Phi(k) \bigg) \in U_i \times \left( \Rbb^n \setminus \{0\} \right) \ \left| \ \left( \Phi(x) , k \right) \in \WF\left( u \circ \Phi_i^{-1} \right) \bigg\} \ . \right.
\end{equation*}
%
We notice to obtain the wave front of a distribution on a manifold it is sufficient to work on local coordinates.


\bigskip


We would like now characterize the the pointwise product of distributions. Let $u,v \in \Dcal^\prime(\Mcal)$, we notice that the pointwise product $u \cdot v$ can be defined as the projection of the tensor product $u \otimes v$ to the diagonal. In this way the pointwise product $u \cdot v$ is now defined as the restriction of the tensor product $u \otimes v$ to the diagonal.% 
%
%\com{This idea is implemented in the following lemma whose proof can be found in ... }
%
%\com{I suggest you to cancel this part up to the lemma}
The diagonal map is defined as follow
%
\begin{equation}
\delta : \left\{
\begin{array}{ccc}
\Mcal & \to & \Mcal \otimes \Mcal \\
u \cdot v & \mapsto & u \otimes v  
\end{array}
\right. \ ,
\label{eq:diag_map}
\end{equation}
%
where the normal set associated to this map is 
%
\begin{equation*}
N_\delta = \left\{ (x,k;x,-k) \ , \ \ x \in \Mcal , \ k \in \Rbb^n \right\} \ .
\end{equation*}
%
Therefore, as explained above, if $N_\delta \cap \WF(u \otimes v) = \emptyset$ then the distribution $u \otimes v \in \Dcal^\prime(\Mcal \otimes \Mcal)$ can be pull back to the distribution $u \cdot v \in \Dcal^\prime(\Mcal)$. The condition $N_\delta \cap \WF(u \otimes v) = \emptyset$ is equivalent to say if $(x,k) \in \WF(u)$ then $(x,-k) \notin \WF(v)$. In particular we have also 
%
\begin{equation*}
\WF(u \cdot v) \subset \delta^\ast \WF(u \otimes v) \ .
\end{equation*}
%
Knowing the wave front set of the tensor product of two distributions $u, v \in\Dcal^\prime(\Mcal)$ 
%
\begin{equation*}
\WF(u \otimes v) \subset 
\bigg( \WF(u) \times \WF(v) \bigg) 
\cup 
\bigg( \left(\supp(u) \times \{0\} \right) \times \WF(v) \bigg) 
\cup 
\bigg( \WF(u) \times \left( \supp(v) \times \{0\} \right) \bigg) \ ,
\end{equation*}
%
we can characterize the wave front set of the pointwise product $u \cdot v$
%
\begin{equation*}
\WF(u \cdot v) \subseteq \bigg( \WF(u) \otimes \WF(v) \bigg) \cup \WF(u) \cup \WF(v) \ . 
\end{equation*} 


We summarize the result of these observations in the following lemma which gives us the necessary condition for the pointwise product of distributions to be well defined. 


\begin{lemma}[Pointwise product of distributions]\label{lem:prod_distrib_wf}
The pointwise product of two distributions $u\in\Dcal^\prime(\Mcal)$ and $v\in\Dcal^\prime(\Mcal)$ exists if $(x,k) \in \WF(u)$ implies $(x,-k) \notin \WF(v)$. Furthermore, if $u\cdot v$ exists, 
\begin{equation*}
\WF(u \cdot v) \subseteq \WF(u) \otimes \WF(v) \cup \WF(u) \cup \WF(v) \ . 
\end{equation*} 
%
\end{lemma}


%----------------------------------------------------------------------------%
\section{Formal deformation}
\label{p:Q_DEFORM}
%----------------------------------------------------------------------------%


We shall now pass to discuss the quantization of the classical free off shell theory we presented in section \ref{p:CLASSICAL}. In other words we shall quantize the algebra $\Acal_\reg(\Mcal)$ of classical regular observables in the functional approach in order to define the corresponding quantum algebra $\Acal_\reg(\Mcal)[[\hbar]]$. We require the following formal classical limit
%
\begin{equation*}
\Acal_\reg(\Mcal)[[\hbar]] \quad \underset{\hbar \to 0}{\longrightarrow} \quad \Acal_\reg(\Mcal) \ . 
\end{equation*}
%
The quantization procedure we choose is a formal deformation of the (classical) pointwise product. The idea is to map the classical observables to the same ones with quantum corrections. For doing it we need to define quantum observables as power series in $\hbar$ having for coefficients classical observables. Since we are working with formal power series no convergence conditions is required. We summarize this new notion in the following definition.


\begin{definition}[Quantum regular observable]\label{def:obs_reg_q}
The quantum regular observables are defined as power series in $\hbar$ with coefficients in $\Fcal_\reg(\Mcal)$. The space of quantum regular observables is denoted by $\Fcal_\reg(\Mcal)[[\hbar]]$.
\end{definition}

Notice that a generic element $\Fsf \in \Fcal_\reg(\Mcal)[[\hbar]]$ can be written as follow
%
\begin{equation*}
\Fsf = \Fsf_0 + \Fsf_1 \ \hbar + \Fsf_2 \ \hbar^2 + \Fsf_3 \ \hbar^3 + \Fsf_4 \ \hbar^4 + \dots \ ,
\end{equation*}
%
with $\Fsf_1, \Fsf_2, \dots \in \Fcal_\reg(\Mcal)$. 


The quantum algebra is obtained equipping $\Fcal_\reg(\Mcal)[[\hbar]]$ with a product which is called the quantum product and it is a formal deformation of the pointwise (classical) product.


\begin{definition}[Regular quantum product]
The quantum product on $\Fcal_\reg(\Mcal)[[\hbar]]$ is defined as follow
%
\begin{equation}
(\Fsf \star_\Delta \Gsf)(\phi) := \Fsf(\phi) \cdot \Gsf(\phi) + \sum_{n=1}^\infty \frac{\hbar^n}{n!} \sm{ \Fsf^{(n)} , \Delta^{\otimes n} \Gsf^{(n) } } \ ,
\label{eq:q_prod_reg}
\end{equation}
%
where $\Delta$ is the causal propagator introduced in section \ref{p:CLASSICAL}. This product is associative and shall be called $\star_\Delta$ product.
\end{definition}


We notice that the power series in $\hbar$ in \eqref{eq:q_prod_reg} does not converge. However, usually the physically relevant quantum observables have only a finite number of non vanishing functional derivatives, thus at least for these elements the series in the definition of $\star_\Delta$ product is always convergent. Despite of this fact, the possibility to use formal power series will be crucial in the perturbative treatment of interacting observables.


\bigskip


The product $\star_\Delta$ implements the canonical commutation relations, actually, taking two linear fields 
%
\begin{equation*}
\Fsf(\phi) = \int_\Mcal \dsf x \ \sqrt{\abs{g}} \ f(x) \ \phi(x) \ , \qquad \Gsf(\phi) = \int_\Mcal \dsf x \ \sqrt{\abs{g}} \ g(x) \ \phi(x) \ ,
\end{equation*}
%
having for functional derivatives
%
\begin{equation*}
\Fsf^{(1)}(\phi) = f(x) \ , \qquad \Gsf^{(1)}(\phi) = g(x) \ ,
\end{equation*}
%
it holds that 
%
\begin{equation}
\left[F(\phi),G(\phi)\right]_{\star_\Delta} =  F(\phi) \star_\Delta G(\phi) - G(\phi) \star_\Delta F(\phi) =  \hbar \Delta(f,g) \ ,
\label{eq:ccr_linear}
\end{equation}
%
where $\Delta$ is the causal propagator defined in \eqref{eq:smearing}.


\bigskip


Let us analyze an example to illustrate some properties of the $\star_\Delta$ product with the following regular functional
%
\begin{equation*}
\Fsf(\phi) = \int_{\Mcal^2} \dsf x \ \dsf y \ f(x,y) \ \phi(x) \ \phi(y) \ ,
\end{equation*}
%
with $f \in \Dcal(\Mcal^2)$. We compute its first and second functional derivatives
%
\begin{eqnarray*}
\Fsf^{(1)}(\phi) = 2 \int_\Mcal \dsf y \ f(x,y) \ \phi(y) \ , 
\quad 
\Fsf^{(2)}(\phi) = 2 \ f(x,y) \ , 
\quad \mbox{and} \qquad 
\Fsf^{(3)}(\phi) = 0 \ .
\end{eqnarray*}
%
Therefore using \eqref{eq:q_prod_reg} we can write explicitly the $\star_\Delta$ product of $\Fsf$ with itself as follows
%
\begin{eqnarray*}
(\Fsf \star_\Delta \Fsf)(\phi) &=& \Fsf(\phi) \cdot \Fsf(\phi) + \hbar \sm{ \Fsf^{(1)}(\phi) , \Delta \ \Fsf^{(1)}(\phi) } + \ \frac{\hbar^2}{2} \sm{ \Fsf^{(2)}(\phi) , \Delta^{\otimes 2} \ \Fsf^{(2)}(\phi)} \\
%
&=& \int_{\Mcal^4} \dsf x \ \dsf y \ \dsf z \ \dsf t \ g(x,y,z,t) \ \bigg( \phi(x) \phi(y) \phi(z) \phi(t) \\
&& \hspace*{60pt} + \ 4 \hbar \ \phi(x) \phi(z) \Delta(y,t) + 2 \hbar^2 \ \Delta(y,t) \Delta(x,z) \bigg) \ .
\end{eqnarray*}
%
where $g$ is a suitable test function. 


\bigskip


All in all we have defined the regular quantum algebra as the free off shell $\ast$--algebra of regular functionals endowed with a $\star_\Delta$ product, denoted


\begin{definition}[Quantum free regular off shell algebra]\label{def:alg_q_reg}
We define the the quantum free off shell algebra of regular quantum functionals as follow
%
\begin{equation*}
\Acal_\reg(\Mcal)[[\hbar]] = \left(\Fcal_\reg(\Mcal)[[\hbar]] , ^\ast , \star_{\Delta} \right) \ . 
\end{equation*}
%
with $\Fcal_\reg(\Mcal)[[\hbar]]$ is the space of quantum regular functionals. It is a noncommutative, associative $\ast$--algebra. The involution defined in \eqref{eq:involution} is not changed.
\end{definition}


Let us come back to formal classical limit requirement. We do have
%
\begin{equation*}
\Fcal_\reg(\Mcal)[[\hbar]] \ \underset{\hbar \to 0}{\longrightarrow} \ \Fcal_\reg(\Mcal)\qquad \mbox{and} \qquad \Fsf \star \Gsf \ \underset{\hbar \to 0}{\longrightarrow} \ \Fsf \cdot \Gsf \ . 
\end{equation*}
%
Therefore in the limit $\hbar \to 0$ the quantum free regular off shell algebra tends to the classical free regular off shell algebra.


%----------------------------------------------------------------------------%
\section{States in algebraic quantum field theory}
\label{p:STATES}
%----------------------------------------------------------------------------%


In a laboratory the \textbf{physical system} is prepared in order to be able to perform experiments on it. The prescription on how we shall prepare a system is called the \textbf{state} of the system, preparing a system in a different way correspond to change the state. The outcome of experiments in a quantum system are of statistical nature, hence the theory should be at least able to model the mean values (numbers) of the moments (powers) of every observables. In this spirit a state of a system can be seen as a functional over all possible observables with some further properties we shall discuss below.


For instance if we prepare the system such that the uncertainty of the value of the measurements is minimal, we defined a pure state. 


Up to now all the construction used to build quantum field theory on curved background is state independent. This is important to have an independent state construction in the case of theories on curved spacetime, because we do know ``physics''  on this background only locally, and it turned out that states are actually non local objects.


Let us precise what we mean by physical system.


\begin{definition}[Physical system and state] 
\begin{itemize}
\item A physical system is described by its observables, seen as elements in a certain $\ast$--algebra $\Acal$.
%
\item A state $\omega$ on the $\ast$--algebra $\Acal$ is defined as a continuous linear functional $\omega : \Acal \to \Cbb$, which is normalized and positive, i.e.
%
\begin{equation*}
\omega(\Ibb) =  1 \ , \quad \mbox{and} \qquad \omega(\Asf^\ast \Asf) \geq 0 \ . 
\end{equation*}
%
for all $\Asf \in \Acal$.
\end{itemize}
\end{definition}


Once we have a state on an algebra, an important result tells us that we can represent this algebra on an Hilbert space. This result is know as the \textbf{GNS representation}. For details on the \textbf{GNS representation} we invite the reader to look at the monograph of V. Moretti \cite{MORETTI_2013}.


\bigskip


We call the evaluation of $\Asf$ in $\Acal$ on a state $\omega$ the \textbf{expectation value} of $\Asf$ in the given state $\omega$, and we denote it by $\omega(\Asf)$. The expectation value of an observable represent the mean values of all possible outcomes of an experiment which test that observable on the state $\omega$.


\bigskip


The $\ast$--algebras we are considering in this work represent observables of a scalar field theory. In particular let us define a state over the quantum free algebra $\Acal_\reg(\Mcal)[[\hbar]]$ introduced in section \ref{p:Q_DEFORM}.


\begin{definition}[State over the free quantum algebra]
A state $\omega$ on $\Acal_\reg(\Mcal)[[\hbar]]$ is defined as a continuous linear functional $\omega : \Acal_\reg(\Mcal)[[\hbar]] \to \Cbb$, which is normalized $\omega(\Ibb)=1$ and positive $\omega(\Asf^\ast \star_\Delta \Asf) \geq 0$ for all $\Asf \in \Acal_\reg(\Mcal)[[\hbar]]$.
\end{definition}


The state $\omega$ on $\Acal_\reg(\Mcal)[[\hbar]]$ is completely determined by its $n$--point (\textbf{correlation}) functions $\omega_n \in \Dcal^\prime(\Mcal^n)$ i.e. the expectation values of the product of $n$ linear functionals, namely, for every $\Fsf_i\in \Acal_\reg(\Mcal)[[\hbar]]$ written as
%
\begin{equation*}
\Fsf_i(\phi) = \int_\Mcal \dsf x \ \sqrt{\abs{g}} \ f_i(x) \phi(x) \ , 
\end{equation*}
%
with $f_i \in \Dcal(\Mcal)$ and $1 \leq i \leq n$,
%
\begin{equation*}
\omega_n(f_1, \dots , f_n) := \omega_n(\Fsf_1 \star \dots \star \Fsf_n) \ .
\end{equation*}


\begin{definition}[Specific states over the free quantum algebra]
%
\begin{itemize}
\setlength\itemsep{0pt}
\item A state $\omega$ on $\Acal_\reg(\Mcal)[[\hbar]]$ is called \textbf{quasifree} or \textbf{gaussian} if for all even $n$ its $n-$point functions
%
\begin{equation*}
\omega_n\left(f_1,  \dots , f_n  \right) =  \sum_{\pi \in \Srak_n} \omega_2\left( f_{\pi(1)}, f_{\pi(2)} \right) \cdot \ \dots \ \cdot \omega_2\left( f_{\pi(n-1)}, f_{\pi(n)} \right) \ ,
\end{equation*}
%
where the test functions $f_i \in \Dcal(\Mcal)$ and where $\Srak_n$ is the set of ordered permutations of $n$ elements, i.e. we have 
%
\begin{equation*}
\pi(1) < \pi(3) < \dots < \pi(n-1) \ , \quad \mbox{and} \quad \pi(1) < \pi(2) \ ;  \ \pi(n-1) < \pi(n) \ . 
\end{equation*}
%
%
%
\item A state $\omega$ on $\Acal_\reg(\Mcal)[[\hbar]]$ is called \textbf{Hadamard} if its two--point function satisfies the \textbf{microlocal spectrum condition}
%
\begin{equation}
\WF(\omega_2) \ = \ \bigg\{ \bigg( x, y ; k_x, k_y \bigg) \in T^\ast\Mcal^2 \setminus \{0\} \ \bigg| \ (x,k_x) \sim (x,-k_y), \ k_x \triangleright 0 \bigg\} \ ,
\label{eq:microlocal_spectrum_condition}
\end{equation}
%
where $(x,k_x) \sim (y,k_y)$ implies that there exists a null geodesic $\gamma$ connecting $x$ to $y$ such that $k_x$ is coparallel and cotangent to $\gamma$ at $x$ and $k_y$ is the parallel transport of $k_x$ from $x$ to $y$ along $\gamma$, and where $k_x \triangleright 0$ means that the covector $k_x$ is future directed.
\end{itemize}
\end{definition}


This previous definition of Hadamard states is abstract and inconvenient for computations. We shall give another definition more computationally friendly, which are equivalent \cite{RADZIKOWSKI_1996}.


\begin{definition}\label{def:loc_form_hadamard}
A state $\omega$ is Hadamard if and only if its two point function $\omega_2$ is of local Hadamard form, i.e. if we can write $\omega_2$ as follow
%
\begin{equation*}
\omega_2(x,y) = \lim_{\epsilon \downarrow 0} \frac{1}{8\pi^2} \bigg( \frac{\usf(x,y)}{\sigma_\epsilon(x,y)} + v(x,y) \log\left( \sigma_\epsilon(x,y) \right) + w(x,y) \bigg) \ ,
\end{equation*}
%
where the limit is taken in a distributional sense, 
%
\begin{equation*}
\sigma_\epsilon(x,y) = \sigma(x,y) + 2 i \epsilon \left( t(x) -t(y)\right) + \epsilon^2 \ , 
\end{equation*}
%
with $t$ a time function on $\Mcal$, and $\sigma$ the Synge's world function defined in section \ref{p:CONNEX_GEOD_CURV}. The coefficients $\usf$, $v$, and $w$ are smooth symmetric biscalars functions regular on coinciding points and called Hadamard coefficients. $v(x,y)$ and $w(x,y)$ possess expansion of the form
%
\begin{equation*}
v(x,y) = \sum_{n=0}^{+\infty} v_n(x,y) \sigma(x,y)^n \ , \quad 
w(x,y) = \sum_{n=0}^{+\infty} w_n(x,y) \sigma(x,y)^n \ ,
\end{equation*}
%
where $w$ is the state dependency of the two point function $\omega_2$.
\end{definition}


%----------------------------------------------------------------------------%
\section{Interacting picture}
\label{p:INTERACTING_PICTURE}
%----------------------------------------------------------------------------%


We shall present now the perturbative construction of an interacting quantum field theory on a generic curved spacetime $\Mcal$. We shall work in the framework of perturbative algebraic quantum field theory (\textbf{pAQFT}) which has been recently developed in \cite{BDF_2009,FR_2014,FR_2013} and it is based on earlier ideas of Steinmann, Epstein, and Glaser \cite{STEINMANN_1971,EG_1973}.


%----------------------------------------------------------------------------%
\subsection{Enlarged space of observables}
\label{p:OBS_ENLARGED}
%----------------------------------------------------------------------------%


We shall construct observables of the interacting theory in section \ref{p:INT_Q_ALG} by means of perturbation theory, namely when the free algebra is perturbed by a non linear local potential. However, up to now, we have worked with regular functionals only, unfortunately the case of interacting potentials constructed with regular functionals are of limited interest. We aim to be able to work with local non linear interacting potentials like
%
\begin{equation}
\Vsf(\phi) = \int_\Mcal \dsf x \ \sqrt{\abs{g}} \ \frac{\lambda_n(x)}{n!} \ \phi^n(x) \ ,
\label{eq:local_pot}
\end{equation}
%
with $\lambda_n \in \Dcal(\Mcal)$ and $n\geq2$. However, the functional derivatives of $V$ are not a smooth function, hence, we have to enlarge the space of functionals we are considering to include objects like $V$. The first new space we consider called space of \textbf{microcausal functionals} and it is defined as follows
%
\begin{equation}
\Fcal_{\muc}(\Mcal) := \left\{ 
\Fsf(\phi) \ \bigg| \ 
\begin{array}{l}
\Fsf(\phi) \in \Fcal(\Mcal), \ \Fsf^{(n)}(\phi) \in \Ecal^\prime(\Mcal^{\otimes n}) \\
\mbox{ and } \ \WF(\Fsf^{(n)}(\phi)) \cap \left( \Mcal^n \times ( \overline{V^{n}_{+}} \cup \overline{V^{n}_{-}} ) \right)  = \emptyset 
\end{array}
\right\} \ ,
\label{eq:func_micro}
\end{equation}
%
where we have set that the wave front set of $\Fsf^{(n)}$ does not intersect the set $\Mcal \times (\overline{V^n_+} \cup \overline{V^n_-})$, furthermore, $\overline{V_\pm}$ denotes the closed forward and backward light cone in the cotangent space, respectively. We shall work with its quantum version $\Fcal_{\muc}(\Mcal)[[\hbar]]$, by adapting definition \ref{def:obs_reg_q}.


This space contains the functionals representing the local interaction potentials but not only. For instance the regular functionals are still contained in it. The space which contains only the local interaction potentials is called space of local functionals, denoted $\mathcal{F}_\loc$. We define it as the space of microcausal functionals having as support for their derivatives the total diagonal $d_n$
%
\begin{equation*}
\Fcal_\loc(\Mcal) := \left\{ \Fsf(\phi) \in \Fcal_{\mu\csf}(\Mcal) \ \bigg| \ \supp\left(\Fsf^{(n)}(\phi)\right) \subset d_n \right\} \ ,
\label{eq:func_loc}
\end{equation*}
%
where the \textbf{total diagonal} $d_n$ is defined as follow
%
\begin{equation}
d_n = \left\{ (x,\dots,x) \subset \Mcal^n \right\} \ .
\label{eq:total_diag}
\end{equation}
%
Again the quantum version $\Fcal_{\mathsf{loc}}(\Mcal)[[\hbar]]$ shall be considered.


\bigskip


We have restrict the space of observables via conditions on their wave front set, we have to introduce a topology to keep control on those objects. The so-called topology we shall associate is the Hörmander topology, a definition can be found in the Ph.D. thesis of T.-P. Hack \cite[Chapter III]{HACK_2010} and in article of Brunetti Fredenhagen and Koeler \cite{BFK_1996}.


%----------------------------------------------------------------------------%
\subsection{Quantum product among local functionals}
%----------------------------------------------------------------------------%


In section \ref{p:Q_DEFORM} we have defined a quantum product among quantum regular functionals $\Fcal_\reg(\Mcal)[[\hbar]]$. But as we mentioned in section \ref{p:OBS_ENLARGED} in order to treat interacting theories we shall consider local functionals. We shall work for now on with quantum microcausal functionals $\Fcal_{\mathsf{\mu c}}(\Mcal)$. For this reason we need to work with a $\star$ product like (2.21) constructed with  an Hadamard distribution $\Delta_+$ instead of the causal propagator $\Delta$. This Hadamard distribution has to satisfy the following conditions.
%
\begin{enumerate}
\item\label{item:split_causal_prop} The antisymmetric part is proportional to the causal propagator $\Delta$, i.e. 
%
\begin{eqnarray*}
i \Delta(f,g) = \Delta_+(f,g) - \Delta_-(f,g) \ , \quad \mbox{with} \qquad \Delta_+(f,g) = \overline{\Delta_-(f,g)} \ ,
\end{eqnarray*}
%
and $f,g \in \Dcal(\Mcal)$. So that the commutation relations are preserved, in particular \eqref{eq:ccr_linear}.\par%
%
Notice that this split of $\Delta$ is not unique. However, thanks to the work of Radzikowski \cite{RADZIKOWSKI_1996} two different Hadamard distributions differs only by a smooth function. 

\item The decomposition of the causal propagator defined in \ref{item:split_causal_prop} divides the wave front set of $\Delta$ in two parts, where $\WF(\Delta)$ is given as follow
%
\begin{equation}
\WF(\Delta) \ = \ \bigg\{ \left(x,y;k_x,k_y\right) \in T^{\ast}\Mcal^{2} \backslash\{0\} \ \bigg| \ (x,k_x) \sim (y,-k_y) \bigg\} \ .
\label{eq:wf_causal_prop}
\end{equation}
%
Where again $(x,k_x) \sim (y,k_y)$ implies that there exists a null geodesic $\gamma$ connecting $x$ to $y$ such that $k_x$ is coparallel and cotangent to $\gamma$ at $x$ and $k_y$ is the parallel transport of $k_x$ from $x$ to $y$ along $\gamma$. The split of $\Delta$ given in \ref{item:split_causal_prop} divides $\WF(\Delta)$ in a part containing positive frequencies and the other one the negative frequencies. The distribution $\Delta_+$ has to satisfy the \textbf{microlocal spectrum condition} introduced in \eqref{eq:microlocal_spectrum_condition}, namely
%
\begin{equation}
\WF(\Delta_+) \ = \ \bigg\{ \bigg( x, y ; k_x, k_y \bigg) \in T^\ast\Mcal^2 \setminus \{0\} \ \bigg| \ (x,k_x) \sim (x,-k_y), \ k_x \triangleright 0 \bigg\} \ ,
\label{eq:wf_hadamard}
\end{equation}
%
where $k_x \triangleright 0$ means that the covector $k_x$ is future directed. In particular as seen in section \ref{p:STATES}, $\Delta_+$ can be written using the local form introduced in definition \ref{def:loc_form_hadamard}
\end{enumerate}


Therefore the new quantum product can be written as follow 
%
\begin{equation}
(\Fsf \star_{\Delta_+} \Gsf)(\phi) := \Fsf(\phi) \cdot \Gsf(\phi) + \sum_{n=1}^\infty \frac{\hbar^n}{n!} \sm{ \Fsf^{(n)} , \Delta_+^{\otimes n} \Gsf^{(n) } } \ .
\label{eq:q_prod_loc}
\end{equation}


The quantum product \eqref{eq:q_prod_loc} among microcausal functionals is always well defined thanks to the imposed wave front set condition to the functional derivatives. A proof of this claim can be found for example in \cite{HW_2003}. In order to analyze this issue let us start considering the following example.


\begin{example}
We consider 
%
\begin{equation}
\Fsf(\phi) = \int_\Mcal \dsf x \ \sqrt{\abs{g}} \ \phi(x)^2 \ f(x)  \ ,
\label{eq:exo_loc_obs}
\end{equation}
%
with $f \in \Dcal(\Mcal)$. Its functional derivatives are
%
\begin{eqnarray*}
\Fsf^{(1)}(\phi) = 2 \ \phi(x) \ f(x) \ , \quad \Fsf^{(2)}(\phi) = 2 \ f(x) \ \delta(x,y) \ , \quad \mbox{and} \qquad \Fsf^{(3)}(\phi) = 0 \ .
\end{eqnarray*}
%
We notice that the supports of $\Fsf^{(1)}(\phi)$ and $\Fsf^{(2)}(\phi)$ are contained in their corresponding total diagonals, thus $\Fsf(\phi)$ is an element of the space of local functionals which is a subspace of $\Fcal_{\muc}(\Mcal)[[\hbar]]$. 
%
Then we write explicitly the $\star_{\Delta_+}$ product as follow
%
\begin{eqnarray}
(\Fsf \star_{\Delta_+} \Fsf)(\phi) &=& \Fsf(\phi) \cdot \Fsf(\phi) + \hbar \sm{ \Fsf^{(1)}(\phi) , \Delta_+ \ \Fsf^{(1)}(\phi) } + \ \frac{\hbar^2}{2} \sm{ \Fsf^{(2)}(\phi) , \Delta_+^{\otimes 2} \ \Fsf^{(2)}(\phi)} \nonumber \\
%
&=& \int \dsf x \ \dsf y \ f(x) f(y) \ \bigg( \phi^2(x) \phi^2(y) + 4 \hbar \phi(x) \phi(y) \Delta_+(x,y) + 2 \hbar^2 \left(\Delta_+(x,y)\right)^2 \bigg) \ . \nonumber \\
\label{eq:exo_loc_obs_prod_q}
\end{eqnarray}
%
We see that a pointwise product of distributions is present, and as we know such products can be ill defined. In order to know if this pointwise product is well defined we shall use the notion of wave front set introduced in section \ref{p:SING_WF} and in particular the result from lemma \ref{lem:prod_distrib_wf}. The wave front set of $\Delta_+$ is given by \eqref{eq:wf_hadamard}, the additional requirement $k_x \triangleright 0$ prevents the point $(x,0)$ to be contained in $\WF(\Delta_+) \oplus \WF(\Delta_+)$. Therefore owing lemma \ref{lem:prod_distrib_wf} the pointwise product is well defined.
\end{example}


In this way we have seen that the new $\star_{\Delta_+}$ product is well defined among local functionals. However, it is also well defined on all microcausal functionals. The microlocal spectrum condition \eqref{eq:wf_hadamard} ensures to have
%
\begin{equation*}
(x,0) \notin \WF(\Delta_+^{\otimes n}) \oplus \WF\left(\Fsf^{(n)}(\phi) \otimes \Gsf^{(n)}(\phi)\right) \ .
\end{equation*}
%
for $\Fsf(\phi)$, and $\Gsf(\phi)$ quantum microcausal functionals. Then by lemma \ref{lem:prod_distrib_wf} we have that the pointwise products considered above are always well defined.


\bigskip


We have changed the quantum product by considering $\Delta_+$ instead of $\Delta$, therefore the associated algebras differ. Nonetheless on regular functionals the change is realized by an isomorphism of algebras defined as follow
%
\begin{equation}
\alpha_{W}(\Fsf) = \exp\left(\hbar \sm{ W(x,y) , \frac{\delta}{\delta\phi(x)} \ \frac{\delta}{\delta\phi(y)} } \right) \Fsf \ ,
\label{eq:alpha_isomorph} 
\end{equation}
%
where $W = \Delta_+ - \Delta$. Then we can write
%
\begin{equation*}
\Fsf \star_{\Delta_+} \Gsf = \alpha_W \left(\alpha_{-W}(\Fsf) \star_{\Delta} \alpha_{-W}(\Gsf)\right) \ .
\end{equation*}
%
for $\Fsf$ and $\Gsf$ regular functionals. In this way the commutation relations are not modified, as it can be directly checked in the particular case of linear fields, introduced in definition \ref{def:linear_obs}. The isomorphism \eqref{eq:alpha_isomorph} corresponds to the normal ordering and the $\star_{\Delta_+}$ is an algebraic version of the Wick's theorem. We invite the reader to look at the monograph of Itzykson and Zuber \cite{IZ_1980} for an introduction to the normal ordering and the Wick's theorem in quantum field theory. 


\bigskip


The product introduced in \eqref{eq:q_prod_loc} seems to depend on the choice of $\Delta_+$, however, if we choose a different $\Delta^\prime_+$ with the same properties, we have that $w:=\Delta^\prime_+ - \Delta_+$, which is real, smooth and symmetric. Using \eqref{eq:alpha_isomorph} we have
%
\begin{equation}
\Fsf \star_{\Delta^\prime_+} \Gsf = \alpha_w \left(\alpha_{-w}(\Fsf) \star_{\Delta_+} \alpha_{-w}(\Gsf)\right) \ .
\end{equation}
%
Thus the algebras constructed with $\star_{\Delta_+}$ and $\star_{\Delta^\prime_+}$ are isomorphic via $\alpha_{w}$ \eqref{eq:alpha_isomorph} and the construction does not depend on the choice of $\Delta_+$.


\bigskip


The new quantum product is well defined for all functionals in $\Fcal_{\muc}(\Mcal)[[\hbar]]$, the commutation relations are preserved, and it satisfies the following properties
%
\begin{equation*}
\left(\Fsf \star \Gsf\right)^\ast = \Fsf^\ast \star \Gsf^\ast \ ,
\end{equation*}
%
where $^\ast$ denotes the involution operation. Finally we can introduce the quantum free off shell algebra among microcausal quantum functionals.


\begin{definition}[Quantum free microcausal off shell algebra]\label{def:alg_q_muc}
We define the the quantum free off shell algebra of microcausal quantum functionals as follow
%
\begin{equation*}
\Acal_{\muc}(\Mcal)[[\hbar]] = \left(\Fcal_{\muc}(\Mcal)[[\hbar]] , ^\ast , \star \right) \ . 
\end{equation*}
%
where $\Fcal_{\muc}(\Mcal)[[\hbar]]$ is the space of quantum microcausal functionals and $\Delta_+$ is a generic Hadamard function. It is a noncommutative, associative $\ast$--algebra. 
\end{definition}


%----------------------------------------------------------------------------%
\subsection{Time ordered product}
\label{p:PROD_TIME}
%----------------------------------------------------------------------------%


In every perturbative construction of an interacting theory there is the need of writing products of the perturbation potential where the factors are ordered in time. Hence, a necessary tool to construct a perturbative interacting theory is the time ordered product. Let us start constructing it in the case of regular functionals where it is completely characterized by the \textbf{causal factorization property}, namely by the condition
%
\begin{equation}
\Fsf \cdot_\Tsf \Gsf = 
\left\{
\begin{array}{ll}
\Fsf \star \Gsf \quad \mbox{if } \ \supp(\Fsf) \ \mbox{ is later than  } \ \supp(\Gsf)  \\
\Gsf \star \Fsf \quad \mbox{if } \ \supp(\Gsf) \ \mbox{ is later than  } \ \supp(\Fsf) 
\end{array}
\right. \ ,
\label{eq:causal_factorization}
\end{equation}
%
where $\supp(\Fsf)$ is said to be  later than   $\supp(\Gsf)$ if $J^+(\supp(\Fsf)) \cap \supp(\Gsf)=\emptyset$. Let us look at this product for linear functionals.


\begin{example}
We consider the following functionals
%
\begin{equation*}
\Fsf(\phi) = \int_{\Mcal^2} \dsf x \ \dsf y \ f(x,y) \ \phi(x) \phi(y)\ , \qquad \Gsf(\phi) = \int_{\Mcal^2} \dsf x \ \dsf y \ g(x,y) \ \phi(x) \phi(y) \ .
\end{equation*}
%
Therefore
%
\begin{equation*}
(\Fsf \cdot_\Tsf \Gsf)(\phi) = \Fsf(\phi) \cdot \Gsf(\phi) + 
\left\{
\begin{array}{ll}
\mbox{if } \ \supp(\Fsf) \ \mbox{ is later than  } \ \supp(\Gsf)  \\[4pt]
\quad \hbar \sm{\Fsf^{(1)}(\phi) \ , \Delta_+ \ \Gsf^{(1)}(\phi)} + \dfrac{\hbar^2}{2} \sm{\Fsf^{(2)}(\phi) \ , \Delta_+^{\otimes 2} \ \Gsf^{(2)}(\phi)} \\[8pt]
%
\mbox{if } \ \supp(\Gsf) \ \mbox{ is later than  } \ \supp(\Fsf)  \\[4pt]
\quad \hbar \sm{\Gsf^{(1)}(\phi) \ , \Delta_+ \ \Fsf^{(1)}(\phi)} + \dfrac{\hbar^2}{2} \sm{\Gsf^{(2)}(\phi) \ , \Delta_+^{\otimes 2} \ \Fsf^{(2)}(\phi)} \\
\end{array}
\right. \ ,
\end{equation*}
%
where we can set
%
\begin{equation}
\Delta_\fsf(x,y) = \Theta(t_x-t_y) \Delta_+(x,y) + \Theta(t_y-t_x) \Delta_+(x,y) \ ,
\label{eq:feynman_relation}
\end{equation}
%
with $\Theta$ the Heaviside function. In order to write
%
\begin{equation*}
(\Fsf \cdot_\Tsf \Gsf)(\phi) = \Fsf(\phi) \cdot \Gsf(\phi) + \hbar \sm{\Fsf^{(1)}(\phi) \ , \Delta_\fsf \ \Gsf^{(1)}(\phi)} + \dfrac{\hbar^2}{2} \sm{\Fsf^{(2)}(\phi) \ , \Delta_\fsf^{\otimes 2} \ \Gsf^{(2)}(\phi)} \ .
\end{equation*}
$\Delta_\fsf$ shall be called the Feynman propagator. We need to know if the term
%
\begin{equation*}
\sm{\Fsf^{(2)}(\phi) \ , \Delta_\fsf^{\otimes 2} \ \Gsf^{(2)}(\phi)} = 4 \sm{f(x,y) \ , \Delta_\fsf^{\otimes 2} \ g(x,y)} = 4 \int_{\Mcal^2} \dsf x \ \dsf y \ f(x,y) g(x,y) \Delta_\fsf(x,y)^2
\end{equation*}
%
is well defined. By a direct computation (using definition \ref{def:prod_distib_fourier}) we show that it is well defined. We can also show that the pointwise product of $\Delta_\fsf$ with itself is well defined by consideration on wave front sets (using lemma \ref{lem:prod_distrib_wf}).
\end{example}


Let us now introduce the time ordered product for generic regular functionals. The explicit form of the \textbf{time ordering product} is
%
\begin{equation}
(\Fsf \cdot_{\Tsf_{\Delta_\fsf}}  \Gsf)(\phi) = \Fsf(\phi) \cdot \Gsf(\phi) + \sum_{n=1}^\infty \frac{\hbar^n}{n!} \sm{ \Fsf(\phi)^{(n)} , \Delta_\fsf^{\otimes n} \ \Gsf(\phi)^{(n)} } \ ,
\label{eq:prod_time}
\end{equation}
%
where $\Delta_\fsf$ still denotes the Feynman propagator. The wave front set of $\Delta_\fsf$ is equal to
%
\begin{equation}
\WF(\Delta_\fsf) = \left\{ \left(x,y;k_x,k_y\right) \in T^{\ast}\Mcal^{2} \backslash\{0\} \ \left| \ 
\begin{array}{l}
(x,k_x) \sim (y,-k_y) \ , \  (x-y)^2=0 \ , \\[2pt]
k_x \in \Vcal^\pm \ \mbox{ for } \ (x-y) \in V^\pm \ , \\[2pt]
\mbox{and } \ k_x \in T^{\ast}_x\Mcal \backslash\{0\} \ \mbox{ for } \ x = y
\end{array}
\right. \right\} \ .
\end{equation}
%
Therefore using lemma \ref{lem:prod_distrib_wf} we can say that the product \eqref{eq:prod_time} is well defined among regular functionals, indeed
%
\begin{equation*}
(x,0) \notin \WF(\Delta_+^{\otimes n}) \oplus \WF\left(\Fsf^{(n)}(\phi) \otimes \Gsf^{(n)}(\phi)\right) \ ,
\end{equation*}
%
for $\Fsf(\phi)$, and $\Gsf(\phi)$ quantum regular functionals. 


\bigskip


As for the $\star$ product we can wonder if the time ordered product introduced in \eqref{eq:prod_time} depends on the choice of $\Delta_\fsf$. Again if we choose a different $\Delta^\prime_\fsf$ with the same properties, we have that $w:=\Delta^\prime_\fsf - \Delta_\fsf$, and using $\alpha_{w}$ defined in \eqref{eq:alpha_isomorph} the algebras build with $\cdot_{\Tsf_{\Delta_\fsf}}$ and $\cdot_{\Tsf_{\Delta^\prime_\fsf}}$ are isomorphic and thus the construction is independent of this choice.


\bigskip


Let us discuss how it is possible to construct observables of the interacting theory by means of perturbation theory, namely, when the free algebra is perturbed by a potential $\Vsf$, which is taken here regular. The interacting algebra is represented on the free algebra by means of the \textbf{Bogoliubov formula} which is given in terms of the local $S$ matrix, 
%
\begin{equation}
S(\Vsf) = \exp_\Tsf\left(\Vsf\right) = \sum^\infty_{n=0} \frac{i^n}{n!\ \hbar^n} \ \underbrace{\Vsf(\phi) \cdot_\Tsf \cdots \cdot_\Tsf \Vsf(\phi)}_{n \mbox{ times }} \ ,.
\label{eq:S_matrix}
\end{equation}
%
The $\star$ product and the time ordered are well defined for regular functionals. The observables are mapped from the interacting off shell algebra to the free off shell algebra with the \textbf{Bogoliubov formula} \cite{DF_2004} which is defined as follow
%
\begin{equation}
\Rsf_\Vsf(\Fsf(\phi)) := S(\Vsf)^{\star-1} \star \left( S(\Vsf) \cdot_\Tsf \Fsf(\phi) \right) \ ,
\label{eq:bogoliubov}
\end{equation}
%
where $S^{\star-1}(\Vsf)$ is the inverse of $S(V)$ with respect to the $\star$ product. 


\begin{definition}\label{def:alg_qT_reg}
We define the the (regular) interacting free quantum off shell algebra as follow
%
\begin{equation*}
\Acal_\reg(\Mcal)[[\hbar]]\{I\} = \left(\Fcal_\reg(\Mcal)[[\hbar]] , ^\ast , \star , \cdot_\Tsf \right) \ . 
\end{equation*}
%
with $\Fcal_\reg(\Mcal)[[\hbar]]$ is the space of quantum regular functionals. It is a noncommutative, associative $\ast$--algebra. The involution defined in \eqref{eq:involution} is not changed. 
\end{definition}


Because the construction of interacting field theory with regular functionals is of limited interest, we shall in the next section \ref{p:INT_Q_ALG} look at the situation where the potential $V$ is represented by a local functional.


%----------------------------------------------------------------------------%
\subsection{Interacting off shell quantum algebra}
\label{p:INT_Q_ALG}
%----------------------------------------------------------------------------%


We shall now extend the time ordered product among generic local functionals, in order to build the interacting quantum algebra. Actually, we shall see that it will not be possible to have a time ordered product among microcausal functionals. Despite of this fact the theory of renormalization we are going to introduce permits to build time ordered products of local functionals and, luckily enough this is sufficient for the perturbative construction of non linear local theory.


The product $\cdot_\Tsf$ defined in \eqref{eq:prod_time} is well defined among regular functionals, but we shall now wonder if it is the case when the factors are local functionals. Let us start to look at this issue in an example.


\begin{example}\label{exo:t_prod_loc_obs}
We keep working with the same local functionals $\Fsf(\phi)$ defined in \eqref{eq:exo_loc_obs}. Using \eqref{eq:prod_time} we can write explicitly the time ordered product for $\Fsf(\phi)$ with itself as follow
%
\begin{eqnarray*}
(\Fsf \cdot_\Tsf \Fsf)(\phi) &=& \Fsf(\phi) \cdot \Fsf(\phi) + \hbar \sm{ \Fsf^{(1)}(\phi) , \Delta_\fsf \ \Fsf^{(1)}(\phi) } + \ \frac{\hbar^2}{2} \sm{ \Fsf^{(2)}(\phi) , \Delta_\fsf^{\otimes 2} \ \Fsf^{(2)}(\phi)} \\
%
&=& \int \dsf x \ \dsf y \ f(x) f(y) \ \bigg( \phi^2(x) \phi^2(y) + 4 \hbar \phi(x) \phi(y) \Delta_\fsf(x,y) + 2 \hbar^2 \left(\Delta_\fsf(x,y)\right)^2 \bigg) \ .
\end{eqnarray*}
%
A pointwise product of distributions is present. We need to know if it is well defined or not. As mentioned in \eqref{eq:feynman_relation}, the Feynman propagator $\Delta_\fsf$ satisfies the following identities
%
\begin{equation}
\Delta_\fsf(x,y) = \Delta_+(x,y) + i \Delta_\asf(x,y) = \Delta_-(x,y) + i \Delta_\rsf(x,y) \ .
\label{eq:conv_feynman_prop}
\end{equation}
%
By lemma \ref{lem:prop_wf} we have
%
\begin{equation*}
\WF(\Delta_\fsf) \subset \WF(\Delta_+) \cup \WF(\Delta_\asf) \ .
\end{equation*}
%
The wave front set of $\Delta_+$ does not allow for  singularities which are ``too bad'', because as we know pointwise products of $\Delta_+$ are well defined. The possible issues will come from the wave front set of $\Delta_\asf$, which satisfies $\Psf \Delta_\asf = \delta$. Thanks again to lemma \ref{lem:prop_wf} we know that the wave front set of the integral kernel of $\Delta_\asf$ contains the wave front set the Dirac distribution $\delta$, and we know that pointwise product of Dirac distributions are not well defined. 
Actually, it can shown by direct computation using definition \ref{def:prod_distib_fourier} that the pointwise product of $\Delta_\fsf$ with itself is not well defined because of the presence of singularities in the diagonal. Thus we shall have to find a way to make sense of pointwise products of $\Delta_\fsf$. This issue is called the regularization problem.
\end{example}


The regularization problem will be to find an ``extension'' to pointwise products of the Feynman propagators $\Delta_\fsf$. Unfortunately this ``extension'' is not trivial and it requires the introduction of the theory of regularization. In fact giving sense of these products shall be the the core of the present work.


\bigskip


S. Hollands and R. M. Wald have successfully established the set of axioms to build time ordered product in order to get a physically meaningful theory \cite{HW_2001,HW_2002,HW_2005}. This axioms are given for the time ordered map introduced in the following definitions. The relation of the time ordered maps and the time ordered products will be given later.


\begin{definition}[Axioms of time ordered products]
Let $\{\Tcal_n \ , \ n \in \Nbb\}$ be a set of linear maps 
%
\begin{equation*}
\Tcal_n : \Fcal_\loc(\Mcal)[[\hbar]]^{\otimes n} \to \Fcal_{\muc}(\Mcal)[[\hbar]] \ . 
\end{equation*}
%
The maps $\Tcal_n$ are called time ordered products if they satisfy the following statements.
%
\begin{description}
\item[T1 -- Initial condition.]\label{item:T1} $\Tcal_0 = 0$ and $\Tcal_1(\Fsf) = \Fsf$.
%
\item[T2 -- Symmetry.]\label{item:T2} $\Tcal_n(\Fsf_1\otimes\dots\otimes\Fsf_n) = \Tcal_n(\Fsf_{\pi(1)}\otimes\dots\otimes\Fsf_{\pi(n)})$ for every permutation $\pi$ of $\{1,\dots,n\}$.
%
\item[T3 -- Unitarity.]\label{item:T3} Let $I=(I_1,\dots,I_k)$ be a partition of $\{1,\dots,n\}$ into $k$ pairwise disjoint subsets, then
%
\begin{equation*}
\Tcal_n(\Fsf_1\otimes\dots\otimes\Fsf_n)^\ast = \sum_{I} (-1)^{n+k} \Tcal_{\abs{I_1}}\left(\bigotimes_{i\in I_1} \Fsf_i^\ast \right) \star \dots \star \Tcal_{\abs{I_k}}\left(\bigotimes_{i\in I_k} \Fsf_i^\ast \right) \ ,
\end{equation*}
%
where the sum is over all partitions of $I$.
%
\item[T4 -- Causal factorization.]\label{item:T4} Already introduced in \eqref{eq:causal_factorization}, we recall it here. If the supports of $\Fsf_1,\dots,\Fsf_k$ is later than the supports of $\Fsf_{k+1} , \dots , \Fsf_{n}$ we have
\begin{equation*}
\Tcal_n(\Fsf_1\otimes\dots\otimes\Fsf_n) = \Tcal_k(\Fsf_1\otimes\dots\otimes\Fsf_k) \star \Tcal_{n-k} (\Fsf_{k+1}\otimes\dots\otimes\Fsf_n) \ .
\end{equation*}
%
\item[T5 -- Field independence.]\label{item:T5} $\Tcal_n$ should satisfy
%
\begin{equation*}
\frac{\delta}{\delta\phi} \Tcal_n(\Fsf_1\otimes\dots\otimes\Fsf_n) = \sum_{k=1}^n \Tcal_n\left(\Fsf_1\otimes\dots\otimes \frac{\delta\Fsf_k}{\delta\phi} \otimes\dots\otimes\Fsf_n\right)
\end{equation*}
%
\item[T6 -- Locality and covariance.]\label{item:T6} The time ordered products have to be local covariant fields as defined in \cite{HW_2001}.
%
\item[T7 -- Microlocal spectrum condition.]\label{item:T7} Let $\omega$ be a continuous state on a field algebra $\Acal$, we should have that the wave front set of the expectation value of $\Tcal_n(\Fsf_1\otimes\dots\otimes\Fsf_n)$ satisfy 
%
\begin{equation*}
\WF\left(\omega\left(\Tcal_n(\Fsf_1\otimes\dots\otimes\Fsf_n)\right)\right) \ \subset \ \bigg\{ \bigg( x, y ; k_x, k_y \bigg) \in T^\ast\Mcal^2 \setminus \{0\} \ \bigg| \ (x,k_x) \sim (x,-k_y), \ k_x \triangleright 0 \bigg\} \ .
\vspace*{-10pt}
\end{equation*}
%
\item[T8 -- T12.]\label{item:T8_T12} $\Tcal_n$ has to to satisfy scaling, smoothness, analyticity, commutator conditions, Leibniz rule as defined in \cite{HW_2001,HW_2002,HW_2005}.
\end{description}
%
The time ordered product $\Tcal_n(\Fsf_1\otimes\Fsf_1\otimes\Fsf_n)$ can be written using the previously introduced symbol $\cdot_\Tsf$, we therefore get $\Fsf_1\cdot_\Tsf\Fsf_1\cdot_\Tsf\Fsf_n$. 
\end{definition}

Therefore if the map
%
\begin{equation*}
\Tcal_n \ : \ 
\left\{
\begin{array}{lcl}
\Fcal_{\loc}(\Mcal)[[\hbar]]^{\otimes n} & \to & \Fcal_{\muc}(\Mcal)[[\hbar]] \\
\Fsf_1(\phi) \otimes \ ... \ \otimes \Fsf_n(\phi) & \mapsto & \Fsf_1(\phi) \cdot_{\Tsf} \ ... \ \cdot_{\Tsf} \Fsf_n(\phi)
\end{array}
\right. \ .
\end{equation*}
%
satisfies all the previous axioms, it is called time ordered product of functionals. Now that we have a defined time ordered product we can introduced a new space of functionals.  The space of \textbf{``causal'' functionals} denoted $\Fcal_\Tsf(\Mcal)[[\hbar]]$, it is the space of functionals which can be written as time ordered products of quantum local functionals. The reason to consider this new space instead of the microlocal space is that the time ordered product is \textbf{associative} among causal functionals \cite{FR_2013}.


\bigskip


We can now define the quantum free off shell algebra among causal quantum functionals endowed with a quantum product and a time ordered. 


\begin{proposition}[Interacting quantum algebra]\label{prop:alg_int}
A representation of the interacting quantum algebra can be obtained by means of the Bogoliubov map introduced in \eqref{eq:bogoliubov},
%
\begin{equation}
\Rsf_\Vsf \ : \ \Rcal_\Tsf(\Mcal)[[\hbar]]\{I\} \ \to \ \Acal_\Tsf(\Mcal)[[\hbar]]\{I\} \ .
\label{eq:alg_int}
\end{equation}
%
where $\Acal_\Tsf(\Mcal)[[\hbar]]\{I\}$ is the quantum free off shell algebra among causal quantum functionals endowed with a quantum product and a time ordered as follow
%
\begin{equation*}
\Acal_\Tsf(\Mcal)[[\hbar]]\{I\} = \left(\Fcal_\Tsf(\Mcal)[[\hbar]] , ^\ast , \star . \cdot_\Tsf \right) \ . 
\end{equation*}
%
It is a noncommutative, associative $\ast$--algebra.
\end{proposition}



\begin{remark}[Global view of the construction of the interacting theory]
Now that we just gave the definition of the interacting quantum algebra \ref{prop:alg_int}, we shall review the path we took to give this definition.
%
\begin{enumerate}
\item Fist we define the off shell configuration space of fields in \ref{def:config_space}.
%
\item Second we implemented observables as functionals over field configuration. We define in particular the regular functionals $\Fcal_\reg(\Mcal)$ in \ref{def:obs_reg}.
%
\item Third we define the classical free off shell algebra of regular observables $\Acal_\reg(\Mcal)$ in \ref{def:alg_clas}.
%
\item The quantization procedure we choose is the formal deformation discussed in section \ref{p:Q_DEFORM}. We start to define a quantum observable in \ref{def:obs_reg_q} in order to define the quantum free off shell algebra for regular functionals $\Acal_\reg(\Mcal)[[\hbar]]$ in \ref{def:alg_q_reg}. Then we extend the situation to microcausal observables \eqref{eq:func_micro} in order to define the quantum free microcausal off shell algebra $\Acal_{\muc}(\Mcal)[[\hbar]]$ in \ref{def:alg_q_muc}.
%
\item In order t construct an interacting theory perturbatively we needed to implement a time ordered product \eqref{eq:prod_time}. Finally we define the interacting quantum algebra \ref{prop:alg_int} via the Bogoliubov formula \eqref{eq:bogoliubov} which maps the interacting algebra $\Rcal_\Tsf(\Mcal)[[\hbar]]\{I\}$ to the free algebra $\Acal_\Tsf(\Mcal)[[\hbar]]\{I\} $.
%
\end{enumerate}
\end{remark}


We shall conclude this section with a practical computation.


\begin{example}
We compute here the two point function of the interacting field in a gaussian Hadamard state $\omega$ of the free field, cf section \ref{p:STATES}. We shall consider a local potential $V$ of the form given in \eqref{eq:obs_pot_calculus-2pf}, and perform the computation up to the second order in $\lambda$. \par
%
%
In practice we need to compute the $\star$ product of $\Rsf_\Vsf(\Fsf(\phi))$ and $\Rsf_\Vsf(\Gsf(\phi))$ where $\Fsf$ and $\Gsf$ are linear local functionals as defined in \ref{def:linear_obs}, and where we evaluate it on the state $\omega$
%
\begin{equation*}
\sm{ \omega_2, f \otimes g } := \omega(\Rsf_\Vsf(\Fsf) \star \Rsf_\Vsf(\Gsf)) \ .
\end{equation*}
%
%
A graphical representation of the Bogoliubov formula \eqref{eq:bogoliubov} is displayed and permits us to write this two point function as a sum of graphs.\par
%
%
The potential $\Vsf$ is given by
%
\begin{equation}
\Vsf(\phi) = \lambda \int \dsf x \ \left( \frac{\beta(x)}{4!} + \frac{\mu(x)}{2!} \right) \ \phi(x)^4 \ ,
\label{eq:obs_pot_calculus-2pf}
\end{equation}
%
where $\beta$ and $\lambda$ are compactly smooth test functions on $\Mcal$. Then we can compute the $S$ matrix \eqref{eq:S_matrix} for the potential $\Vsf$
%
\begin{eqnarray*}
S(\Vsf) &=& 1 + i \Vsf(\phi) - \frac{1}{2} \Vsf(\phi) \cdot_\Tsf \Vsf(\phi) + \Ocal(\lambda^3) \\
&=&  \ 1 + i \Vsf(\phi) - \frac{1}{2} \Vsf(\phi) \cdot \Vsf(\phi) \\
&& - \int_\Mcal \dsf x \ \dsf y \ \beta(x) \beta(y) \ \bigg( 
\frac{\hbar}{72} \phi(x)^3 \phi(x)^3 \Delta_\fsf(x,y) + \frac{\hbar^2}{16} \phi(x)^2 \phi(x)^2 \Delta_\fsf(x,y)^2 \\ 
&& \hspace*{13pt} + \frac{\hbar^3}{6} \phi(x) \phi(x) \Delta_\fsf(x,y)^3 + \frac{\hbar^4}{8} \Delta_\fsf(x,y)^4 \bigg) \\
&& + \int_\Mcal \dsf x \ \dsf y \ \mu(x) \mu(y) \ \bigg( 
\frac{\hbar}{2} \phi(x) \phi(x) \Delta_\fsf(x,y) + \frac{\hbar^2}{4} \Delta_\fsf(x,y)^2 \bigg) + \Ocal(\lambda^3) \ , 
\end{eqnarray*}
%
and also its inverse with respect to the $\star$ product
%
\begin{eqnarray*}
S(\Vsf)^{\star -1} &=& 1 \ - i \ \Vsf(\phi) \ + \frac{1}{2} \ \Vsf(\phi) \cdot_\Tsf \Vsf(\phi) \ - \ \Vsf(\phi) \star \Vsf(\phi) \ + \ \Ocal(\lambda^3) \ .
\end{eqnarray*}
%
We can now compute the corresponding interacting functional for $\Fsf$
%
\begin{eqnarray*}
\Rsf_\Vsf(\Fsf) &=& S(\Vsf)^{\star -1} \star \left( S(\Vsf) \cdot_\Tsf \Fsf(\phi) \right) \\
&=& \Fsf(\phi) - i \Vsf(\phi) \star \Fsf(\phi) + i \Vsf(\phi) \cdot_\Tsf \Fsf(\phi) + \frac{1}{2} \left( \Vsf(\phi) \cdot_\Tsf\Vsf(\phi) \right) \star \Fsf(\phi) \\
&& - \Vsf(\phi) \star \Vsf(\phi) \star \Fsf(\phi) - \frac{1}{2} \Vsf(\phi) \cdot_\Tsf \Vsf(\phi) \cdot_\Tsf\Fsf(\phi) + \Vsf(\phi) \star \left( \Vsf(\phi) \cdot_\Tsf \Fsf(\phi) \right) + \Ocal(\lambda^3) \ ,
\end{eqnarray*}
%
and in the same way we get $\Rsf_\Vsf(\Gsf)$. Then as already said we have to compute the $\star$ product of $\Rsf_\Vsf(\Fsf)$ and $\Rsf_\Vsf(\Gsf)$ and evaluate in the state $\omega$. In the computation many expressions can be shortened by using the relation \eqref{eq:conv_feynman_prop}. The resulting Feynman diagrams are depicted in \eqref{eq:2pf} where we used the conventions from \eqref{eq:prop_vertices} concerning the various propagators and vertices. Notice we choose $\mu(x)=3 w(x,x)$.
%
\begin{eqnarray}
\Delta_\fsf(x,y) &=& \DeltaF \ = \ \Delta_\fsf(y,x) \nonumber \\ 
\Delta_+(x,y) &=& \DeltaPM \ = \ \Delta_-(y,x) \nonumber \\
\Delta_\rsf(x,y) &=& \DeltaR \ = \ \Delta_\asf(y,x) \label{eq:prop_vertices} \\[4pt]
\mu(x) &=& \LineCross \nonumber \\
\lambda &=& \CrossEdges \nonumber
\end{eqnarray}
%
The up to second order contributions to the two point function of the interacting field with potential $\Vsf$. We omit the labels of the external vertices after the first line using the convention that the left external vertex is always the $x$-vertex, and then the right external vertex the $y$-vertex.
%
\vspace*{-30pt}
\begin{eqnarray}
\omega_2(x,y) &=& 
\Uno \nonumber \\
&& - \Due - \Tre \nonumber \\
&& + \Quattro + \Cinque + \Sei \nonumber \\
&& -3i \Sette +3i \Otto -3i \Nove +3i \Diece \nonumber \\
&& -6i \Undici +6i \Dodici \nonumber \\
&& -6i \Tredici +6i \Quattrodici \\
&& +6 \Quindici \nonumber
\label{eq:2pf}
\end{eqnarray}
%
Each edge corresponds to a propagator (cf. \eqref{eq:prop_vertices}), every internal vertices to $\lambda$ or $\mu(x)$ (cf. \eqref{eq:prop_vertices}), and any closed loop represent an integration of the propagator(s) in the loop against suitable test function(s). All graph are a pointwise product of ``vertices'' and ``edges''.
\end{example}


%----------------------------------------------------------------------------%
\section{The regularization problem}
\label{p:REG_PB}
%----------------------------------------------------------------------------%


Due to the causal factorization property stated as one of the axioms presented in section \ref{p:INT_Q_ALG} that every time ordered product has to satisfy, the time ordered product \eqref{eq:time_ordered_op} is well defined if the support of $\Fsf_1, \ ... \, \Fsf_n$ are pairwise disjoint
%
\begin{equation*}
\supp(\Fsf_i) \cap \supp(\Fsf_j) = \emptyset \ , \qquad \forall i , j \in \{1,...,n\}, \ i \neq j \ . 
\end{equation*}
%
However, if we restrict ourselves to functionals which have pairwise disjoint support, we are excluding the local functionals. Unfortunately, as we said earlier the functionals which represent the local non linear interacting potentials of the theory are elements of the space of local functionals. Hence we have to extend the definition of the time ordered product for functionals with overlapping supports. This section is based on the work of Brunetti, Fredenhagen, Duetsch, Keller, Rejzner \cite{BF_2000,DFKR_2014}, and on earlier ideas of Epstein and Glaser and Steinmann \cite{EG_1973,STEINMANN_1971}.


In this section first we shall first extend the time ordered product up to the total diagonal, i.e. when all supports of $\Fsf_1, \ ... \, \Fsf_n$ coincide. Second we will characterize this remaining divergence with the so called scaling degree. Third we will present two different methods to extend distributions, the $W$ extension and the minimal subtraction. We will show how these two procedures are related. Finally, we will show how the minimal subtraction procedure can be explicitly applied in the case of specific distributions.


%----------------------------------------------------------------------------%
\subsection{Extension up to the total diagonal}
\label{p:EXT_UP_TOT}
%----------------------------------------------------------------------------%


We have build successfully the interacting quantum field theory perturbatively using the functional approach. The interacting observables have been obtained with the Bogoliubov formula \eqref{eq:bogoliubov}. The problem using $\Rsf_\Vsf$ lies in the construction of the time ordered product.


We have already seen in example \ref{exo:t_prod_loc_obs} that the time time ordered product among local functionals can be ill defined. It is what we call the \textbf{regularization problem}. It can be solved using the procedure developed in \cite{BF_2000}, where the time ordered product is build recursively in the number of local fields on the full space up to the total diagonal. At each recursion step the causal factorization property permits to construct the distributions defining the time ordered product up to the total diagonal. Let us present this recursive construction. 


\bigskip


We consider $\forall i \in \{1,\dots,n\}$ the functionals $\Fsf_i(\phi) \in \Fcal_\loc(\Mcal)[[\hbar]]$. The time ordered product shall be written constructing the maps $\Tcal_n$ 
%
\begin{equation*}
\Tcal_n \ : \ 
\left\{
\begin{array}{lcl}
\Fcal_{\loc}(\Mcal)[[\hbar]]^{\otimes n} & \to & \Fcal_{\muc}(\Mcal)[[\hbar]] \\
\Fsf_1(\phi) \otimes \ ... \ \otimes \Fsf_n(\phi) & \mapsto & \Fsf_1(\phi) \cdot_{\Tsf} \ ... \ \cdot_{\Tsf} \Fsf_n(\phi)
\end{array}
\right. \ ,
\end{equation*}
%
with the properties stated in section \ref{p:INT_Q_ALG}. Here we shall present the recursive method proposed by Epstein and Glaser, where the recursion is in the number of local fields. In particular, we prove that, at each recursive step, the time ordered product is well defined up to the total diagonal \eqref{eq:total_diag}.


\begin{description}


\item[Base cases (n = 1 \& 2).] The time ordered product of a local functional is a local functional,
\begin{equation*}
\Tcal_1(\Fsf) \in \Fcal_\loc(\Mcal)[[\hbar]] \ , 
\end{equation*}
%
and the time ordered product of two ``causal'' functionals 
%
\begin{equation*}
\Tcal_2\left(\Fsf_1(\phi)\otimes\Fsf_2(\phi)\right) = \Fsf_1(\phi) \cdot_\Tsf \Fsf_2(\phi) = \Fsf_1(\phi) \star \Fsf_2(\phi) \ ,
\end{equation*}
%
coincides with the $\star$ product for $\supp(\Fsf_1)$ later that $\supp(\Fsf_2)$. It is well defined on the full space $\Mcal^2$ except on the coinciding points, which here is equal to the total diagonal $d_2$.


\item[Inductive hypothesis ($k\leq n$).] Now let us assume that we are able to construct the time ordered product of $k$ local functionals, for any $k \leq n$. We suppose it is well defined on the full space $\Mcal^k$ minus the total diagonal $d_k$, and that it satisfies the causal factorization property. 


\item[Inductive step (k=n+1).] We want now to define the time ordered product of $n+1$ functionals on $\Mcal^{n+1}\setminus d_{n+1}$.


\begin{description}


\item[Partition.] We shall need a partition of $\Mcal^{n+1}\setminus d_{n+1}$. Let $\Jcal$ be the set of all non empty proper subset $I$ of $\{1,\dots,n+1\}$, and $I^c$ the complement set of $I$ in $\{1,\dots,n+1\}$. We define the subset $\Ccal_I$ of $\Mcal^{n+1}$ as follow
%
\begin{equation*}
\Ccal_I = \left\{ (x_1,\dots,x_{n+1}) \in \Mcal^{n+1} \ | \ \forall (i,j) \in I \times I^c , \ x_i \notin J^-(x_j) \right\} \ .
\end{equation*}
%
Then we can show \cite[lemma 4.1]{BF_2000}
%
\begin{equation*}
\bigcup_{I \in \Jcal} \Ccal_I = \Mcal^{n+1} \setminus d_{n+1} \ .
\end{equation*}


\item[Induction.] On every subset $\Ccal_I$ on $\Mcal^{n+1}$ we have 
%
\begin{equation*}
\Tcal_{n+1}\left(\Fsf_1(\phi) \otimes \dots \otimes \Fsf_{n+1}(\phi) \right) = \Tcal_{I}\left(\Fsf_I(\phi)\right) \star \Tcal_{I^c}\left(\Fsf_{I^c}(\phi)\right) \ , 
\end{equation*}
%
where $\Tcal_{I}\left(\Fsf_I(\phi)\right)$ denotes the time ordered product of all $\Fsf_i(\phi)$ with $i \in I$, and in the same way $\Tcal_{I^c}\left(\Fsf_{I^c}(\phi)\right)$ is the time ordered product of all $\Fsf_i(\phi)$ with $i \in I^c$. According to the induction hypothesis the time ordered product of $k+1$ functionals is then well defined on $\Ccal_I$. 


Two sets $\Ccal_{I_1}$ and $\Ccal_{I_2}$ can overlap, but we can show \cite[proposition 4.2]{BF_2000} that for any set $I_1 , I_2 \in \Jcal$ such that $\Ccal_{I_1} \cap \Ccal_{I_2} \neq \emptyset$ we have 
%
\begin{equation*}
\left. \Tcal_{I_1}\left(\Fsf_{I_1}(\phi)\right) \right|_{\Ccal_{I_1} \cap \Ccal_{I_2}} = \left. \Tcal_{I_2}\left(\Fsf_{I_2}(\phi)\right) \right|_{\Ccal_{I_1} \cap \Ccal_{I_2}}
\end{equation*}
%
in the distributional sense on $\Mcal^{n+1}\setminus d_{n+1}$.


Therefore, if we define a smooth partition of unity $\{ g_I\}$  subordinate to $\{\Ccal_I\}_{I\in\Jcal}$, we have
%
\begin{equation*}
\Tcal_{n+1}\left(\Fsf_1(\phi) \otimes \dots \otimes \Fsf_{n+1}(\phi) \right) = \sum_{I\in\Jcal} \  \Tcal_{I}\left(\Fsf_I(g_I\phi)\right)  
\end{equation*}
%
is well defined on the full space up to the total diagonal $\Mcal^{n+1} \setminus d_{n+1}$. 


\end{description}


\item[Conclusion.] We have proved that $\Tcal_{n+1}$ is well defined up to the total diagonal, if $\Tcal_k$ is know to be also well defined for every $k\leq n$. If the time ordered map can be extended on the total diagonal we can recursively construct every time ordered map.


\end{description}


%----------------------------------------------------------------------------%
\subsection{Regularization of numerical distributions}
\label{p:REG_NUMERRIC}
%----------------------------------------------------------------------------%


In section \ref{p:EXT_UP_TOT} we have shown that the time ordered product can be defined up to the total diagonal. The remaining step, namely the the extensions of these products to the total diagonal, is done using ideas of Steinmann. Let us first of all present these methods to the problem of extending a numerical distributions on a point. The use of similar methods to extend the time ordered products to the diagonal can be found in \cite{BF_2000}. In the next chapter we shall present a complete regularization scheme which make use of similar methods.


%----------------------------------------------------------------------------%
\subsubsection{Extensions and Steinmann scaling degree}
%----------------------------------------------------------------------------%


We shall use the following notations
%
\begin{equation*}
\Dcal^\prime : = \bigg\{ \Dcal^\prime(\Rbb^n) \quad \mbox{or} \quad \Dcal^\prime(\Rbb^n\setminus\{0\}) \bigg\} \ , \quad \mbox{and} \qquad \Dcal : = \bigg\{ \Dcal(\Rbb^n) \quad \mbox{or} \quad \Dcal(\Rbb^n\setminus\{0\}) \bigg\} \ .
\end{equation*}
%
Let us consider $u\in \Dcal^\prime(\Rbb^n \setminus \{0\})$ namely a distribution for which we want to find an extension over $\{0\}$. Hence, $u$ is defined only for all test functions supported outside the $\{0\}$. We call $\exte{u} \in \Dcal^\prime(\Rbb^n)$ an \textbf{extension} of $u$ if 
%
\begin{equation*}
 \forall \phi \in \Dcal\left(\Rbb^n \setminus \{0\} \right), \quad \exte{u}(\phi) = u(\phi) \ .
\end{equation*}
%
We cannot always find an extension of a distribution, and even if extensions exists, they must not be unique. A distribution supported on the origin can be written as a polynomial in the derivatives of the Dirac distribution. Therefore we could add a polynomial in the derivatives of the Dirac distribution to the extension obtaining different extensions. We have to restrict the possibilities of extension by adding a constraint. A possible choice of a constraint is to require that the distributions we want to extend and their corresponding extensions should have the same Steinmann scaling degree. To define this Steinmann scaling degree, we shall first define a \textbf{geometric scaling transformation} towards the origin as follow,
%
\begin{equation}
\phi_\lambda = \lambda^{n} \ \phi\left(x_1(\lambda ),x_2(\lambda) , \dots , x_n(\lambda\right)) \ ,
\label{eq:geo_scaling_transfo_numeric}
\end{equation}
%
for all $\lambda \geq 0$ and all $\phi \in \Dcal(\Rbb^n)$. We can now give the definition the Steinmann scaling degree of a distribution.


\begin{definition}\label{def:steinmann_scaling_degree}
The \textbf{Steinmann scaling degree} of a distribution $u \in \Dcal^\prime$ towards the origin is defined as 
%
\begin{equation*}
\sd(u) \ \doteq \ \inf\bigg\{ \omega \in \Rbb \ \bigg| \ \lim_{\lambda \downarrow 0} \ \lambda^\omega \ \sm{u,\phi_{1/\lambda}} \ = \ 0, \ \forall \phi\in\Dcal(\Rbb^n \setminus \{0\}) \bigg\} \ .
\end{equation*}
%
\end{definition}


Let us illustrate the definition \ref{def:steinmann_scaling_degree} with an example.


\begin{example}
We consider the distribution $u \in \Dcal^\prime(\Rbb \backslash \{0\})$ defined as 
%
\begin{equation*}
\sm{u, \phi} = \int_{\Rbb} \dsf x \ u(x) \phi(x)  \ ,
\end{equation*}
%
whose integral kernel is 
%
\begin{equation*}
u(x) = \frac{1}{\abs{x}} \ .
\end{equation*}
%
In order to compute its Steinmann scaling degree we consider
%
\begin{equation*}
\lambda^\omega \sm{u,\phi_{1/\lambda}} = \lambda^\omega \int_\Rbb \dsf x \ u(x) \ \phi\left(\frac{x}{\lambda}\right) = \lambda^{\omega-1} \sm{u , \phi} 
\end{equation*}
% 
from definition \ref{def:steinmann_scaling_degree} it follows that $u$ has Steinmann scaling degree equal to $1$.
\end{example}


The scaling degree has the the following properties whose proof can be found in \cite[theorem 5.1]{BF_2000}. Let $u, v \in\Dcal^\prime(\Rbb^n)$ and $\alpha \in \Nbb^n$, then
%
\begin{itemize}
\setlength\itemsep{0pt}
\item $\sd(\partial^\alpha u) \leq \sd(u) + \abs{\alpha}$,
\item $\sd(x^\alpha u) \leq \sd(u) - \abs{\alpha}$,
\item $\sd(\phi u) \leq \sd(u)$, for all $\phi \in \Ecal(\Rbb^n)$, 
\item $\sd(u \otimes v) = \sd(u) + \sd(v)$
\end{itemize}
%
We notice that when the product $u \cdot v$ is well defined it has the same scaling degree as $u \otimes v$. The reason is that we define $u \cdot v$ starting from $u \otimes v$, cf. section \ref{p:SING_WF}.


The scaling degree permits us to predict the existence and the possible uniqueness of a distribution. According to the value of the scaling degree of the distribution $u$ and comparing it to the dimension of the total space, we shall be able to control the existence and uniqueness of an extension. Let us present this important result in the following theorem \cite[theorems 5.2 and 5.3]{BF_2000}.


\begin{theorem}[Existence and uniqueness of an extension]\label{theo:extension_distribution_numeric}
We consider a distribution $u \in \Dcal^\prime(\Rbb^n \setminus \{0\})$, then
%
\vspace*{-5pt}
\begin{itemize}
\setlength\itemsep{0pt}
\item if $\sd(u) < n$, then there exists a unique extension $\exte{u} \in \Dcal^\prime(\Mcal^n)$ with $\sd(\exte{u})=\sd(u)$,
%
\item if $n \leq \sd(u) < \infty$, then there exist several extensions $\exte{u} \in \Dcal^\prime(\Rbb^n)$ with $\sd(\exte{u})=\sd(u)$, which are uniquely defined by their values on a finite set of test functions.
\end{itemize}
%
A distribution which have scaling degree infinite cannot be extended.
\end{theorem}


%----------------------------------------------------------------------------%
\subsubsection{Regularization procedures}
\label{p:REG_PROCED}
%----------------------------------------------------------------------------%


Once we know that a distribution can be extended the next step is to find a way to obtain at least one extension. We shall discuss a method which assures us to find an extension.  For that we need to define the \textbf{partial space} of order $\lambda$. It is the space of functions which vanish up to order $\lambda \in \Rbb$ towards the origin, 
%
\begin{equation*}
\Dcal_{\lambda}(\Rbb^n) \ = \ \left\{ \phi \in \Dcal(\Rbb^n) \ | \ \forall \abs{\alpha} \leq \lambda \ , \ \ \left(\partial^{\alpha}\phi\right)(0)=0 \right\} \ ,
\end{equation*}
%
with $\Dcal_\lambda(\Rbb^n) = \Dcal(\Rbb^n)$ if $\lambda < 0$. The dual space $\Dcal^\prime_\lambda(\Rbb^n)$ is the corresponding space of distributions. 


It has been shown in \cite{BF_2000} that a distribution ill defined at the origin which has finite Steinmann scaling degree can be extended to a distribution on the whole space using the so-called $W$--projection. It is a map from $\Dcal(\Rbb^n)$ to $\Dcal_\lambda(\Rbb^n)$. The complement of $D^\prime_\lambda(\Rbb^n)$ in $\Dcal^\prime$ is the space of polynomial derivatives of the the Dirac functions up to order $\lambda$. Hence we can write $W$ as follows,
%
\begin{equation}
W : \left\{
\begin{array}{ccl}
\Dcal(\Rbb^n) & \to & \Dcal_\lambda(\Rbb^n) \\
\phi & \mapsto & W \phi = \phi - r_W(\phi)
\end{array}
\right. \ , \label{eq:w_projection}
\end{equation}
%
with $\lambda \leq \sd(u)-n$, and the remainder term $r_W(\phi)$ defined as follow
%
\begin{equation*}
r_W(\phi) := \underset{\abs{\gamma}\leq\lambda}{\sum} \omega_\gamma \ \partial^\gamma \phi(0) \ ,
\end{equation*}
%
where $\omega_\gamma$ are some fixed smooth compactly supported functions such that $\partial^\gamma\omega_\beta(0) = \delta^\gamma_\beta$. It has been shown in \cite[lemma 11]{DF_2004} that for every choice of such functions $\omega_\gamma$ we can construct the linear map $W$ from $\Dcal(\Rbb^n)$ to $\Dcal_\lambda(\Rbb^n)$, and conversely for any such projector $W$ there exists functions such as $\omega_\gamma$ defined above.


For any $\phi \in \Dcal(\Rbb^n)$, any element of $\Dcal_\lambda(\Rbb^n)$ can be written in the form
%
\begin{equation*}
W \phi(x) = \sum_{\abs{\gamma}=\lambda+1} x^\gamma \varphi_\gamma(x) \ ,
\end{equation*}
%
with $\varphi_\gamma \in \Dcal(\Rbb^d)$. Then for any distribution $u\in\Dcal^\prime(\Rbb^n\setminus\{0\})$, with $\sd(u) = \lambda +n$, an extension with same scaling degree, $\exte{u} \in \Dcal(\Rbb^n)$, can be written as follow
% 
\begin{eqnarray}
\sm{\exte{u},\phi} &=& \sm{u,W\phi} \ + \ \sm{\exte{u},r_W(\phi)} \nonumber \\
&=& \sum_{\abs{\gamma}=\lambda+1} \sm{x^\gamma u , \varphi_\gamma} \ + \ \sum_{\abs{\gamma}\leq\lambda} \partial^\gamma \phi(0) \ \sm{\exte{u},\omega_\gamma} \ .
\label{eq:w_ext_distrib}
\end{eqnarray}
%
The second term in the right hand side of \eqref{eq:w_ext_distrib} are just some constants multiplied by some derivatives of the Dirac delta functions. Actually the choice of this constants represent the available freedom in the extension. However in the first term in the right hand side of \eqref{eq:w_ext_distrib} 
% 
\begin{equation*}
\sm{\exte{u},W\phi} = \sum_{\abs{\gamma}=\lambda+1} \sm{x^\gamma u \ , \ \varphi_\gamma} \ ,
\end{equation*}
%
the term $x^\gamma u$ has scaling degree equal at most to $\sd(u) - (\lambda+1)$. We notice that in the case $\lambda = \sd(u) - n$, we have $\sd(x^\gamma u)\leq n+1 < n$, thus according to theorem \ref{theo:extension_distribution_numeric}, the distribution $x^\alpha u$ has a unique extension with the same scaling degree. We say that $\exte{u}|_{\Dcal_\lambda(\Rbb^n)}$, the restriction of $\exte{u}$ to $\Dcal_\lambda(\Rbb^n)$, has an unique extension, we denote it by $\overline{u} \in \Dcal^\prime_\lambda(\Rbb^n)$, where $\lambda = \sd(u) - n$. The distribution $\overline{u}$ is called the \textbf{direct extension} of $u$.


\bigskip


We would like now present an alternative way to extend distributions, called the minimal subtraction procedure. We shall link this new regularization procedure to the $W$ projection.


Let us now define the \textbf{regularization} of a distribution $u \in \Dcal^\prime(\Rbb^n\setminus \{0\})$ with scaling degree $\lambda+n$. It is a family $\left\{ u^{\alpha}\right\}$ of distributions $u^{\alpha}\in\Dcal^\prime(\Rbb^n)$ if
%
\begin{equation*}
\lim_{\alpha \to 0} \sm{u^{\alpha},\phi} \ = \ \sm{\overline{u},\phi} \ , \ \ \forall \phi \in \Dcal_{\lambda}(\Mcal^n) \ , 
\end{equation*}
%
where $\alpha \in \Omega\setminus\left\{ 0\right\}$ with $\Omega\subset\mathbb{C}$ a neighborhood of the origin. The family $\left\{ u^{\alpha}\right\}$ is called \textbf{analytic regularization} if the map 
% 
\begin{equation*}
\alpha \mapsto \sm{u^{\alpha},\phi} \ , \ \ \forall \phi \in \Dcal(\Rbb^n)
\end{equation*}
%
is analytic in $\alpha\in\Omega\setminus\left\{ 0\right\}$ with pole(s) of finite order at the origin. We can write the direct extension of $u\in\Dcal^\prime(\Rbb^n)$ with its analytic regularization $u^\alpha$ in terms of the $W$--projection
%
\begin{equation*}
\sm{\overline{u} , W \phi} = \lim_{\alpha \to 0} \bigg( \sm{u^\alpha , W \phi} -  \sum_{\abs{\gamma}\leq\lambda} \sm{ u^\alpha , \omega_\gamma } \partial^\gamma \phi(0) \bigg) \ .
\end{equation*}
% 
Given the analyticity of $u^\alpha$ it is possible to expand in series both term in the right hand side around $\alpha=0$. After taking the limit $\alpha \to 0$ we get a finite result, therefore the principle parts with respect to $\alpha$ of both series in the right hand side have to be equal. Thus the principal part of any analytic regularization is a local distribution. We can now define the so-called minimal subtraction \textbf{(MS)} procedure \cite{DFKR_2014}.


\begin{theorem}[Minimal subtraction]\label{theo:ms_mumeric}
Any analytic regularization $\{u^\alpha\}$ of a distribution $u \in \Dcal^\prime(\Rbb^n \backslash \{0\})$ implies an extension of $u$ which is defined by 
%
\begin{equation*}
\sm{u_\ms,\phi} = \lim_{\alpha \to 0} \ \bigg( \sm{u^\alpha , \phi} - \pp\left(\sm{u^\alpha , \phi}\right) \bigg) \ ,
\end{equation*}
%
where $\pp$ is the principal part. We require the same scaling degree, $\sd(u_\ms) = \sd(u)$. The extension $u_\ms$ is called the minimal subtraction of $u$.
\end{theorem}


\bigskip


Let us illustrate the $\MS$ procedure in an example. 


\begin{example}\label{exo:ms_numeric}
We consider the distribution $u_n \in \Dcal^\prime(\Rbb^4\setminus\{0\})$ defined as 
%
\begin{equation*}
\sm{u, \phi} = \int_{\Rbb} \dsf x \ u(x) \phi(x)  \ ,
\end{equation*}
%
whose integral kernel is 
%
\begin{equation*}
u_n(x) = \frac{1}{(x^2)^{n+\alpha}} \ , \ \mbox{ with } \ n \geq 1 \ ,
\end{equation*}
%
where the square is defined with the Minkowskian metric. Its scaling degree is $\sd(u_n)=2n$. Therefore for $n>1$ the extension is not unique. Let us regularize it with the minimal subtraction procedure. To perform the $\MS$ procedure we introduce the regularized distribution $u^\alpha_n \in \Dcal^\prime(\Rbb^4)$ defined as follow
%
\begin{equation*}
u^\alpha_n(x) = \frac{1}{(x^2)^{n+\alpha}} \ .
\end{equation*}
%
For practical reason we consider the following identity. 
%
\begin{equation*}
\Box \ u^\alpha_n(x) = - 2 \ (n+\alpha) \ (n+\alpha-2) \ \ u^\alpha_{n+1}(x)  \ , 
\label{eq:relation_exo_ms}
\end{equation*}
%
where $\Box$ is the so-called d'alembertian operator in dimension $4$. Therefore after integrating by parts few times we obtain
%
\begin{eqnarray}
\sm{u^\alpha_n \ , \ \phi} &=& \frac{(-1)^{n}}{2^{n-1}} \ \frac{\Gamma(\alpha+1)\Gamma(\alpha-1)}{\Gamma(\alpha+n)\Gamma(\alpha+n-2)} \nonumber \\
&& \cdot \ \int_{\Rbb^4} \dsf x \ \exp\left(-(1+\alpha) \ \log\left(x^2\right)\right) \ \Box^{n-1}\phi(x) \ , 
\label{eq:exo_ms}
\end{eqnarray}
%
where $\Gamma(n)$ is the Euler gamma function. We want to determine the principal part of $\sm{u^\alpha_n \ , \ \phi}$, thus we need to expand in series with respect to $\alpha$ the expression \eqref{eq:exo_ms}. We get
%
\begin{equation*}
\pp\left(\sm{u^\alpha_n \ , \ \phi}\right) = \frac{(-1)^{n}}{2^{n-1}} \ \frac{1}{\Gamma(n-2)\Gamma(n)} \ \int_{\Rbb} dx \ \frac{1}{x^2} \ \Box^{n-1}\phi(x) 
\end{equation*}
%
thus
%
\begin{eqnarray*}
\sm{u_n \ , \ \phi}_\ms &=& \frac{(-1)^{n}}{2^{n-1}} \ \frac{1}{\Gamma(n-2)\Gamma(n)} \\
&& \cdot \ \int_{\Rbb} \dsf x \ \frac{1}{x^2} \ \bigg( -1 + 2 \gamma + \log(x^2) + \Psi^{(0)}(n-2) + \Psi^{(0)}(n) \bigg) \ \Box^{n-1} \phi(x) \ ,
\end{eqnarray*}
%
where $\Psi^{(n)}$ is the $(n+1)$th derivative of the logarithm of the gamma function. Therefore
%
\begin{equation*}
\left(u_n\right)_\ms = \frac{(-1)^{n}}{2^{n-1}} \frac{1}{\Gamma(n-2)\Gamma(n)} \ \Box^{n-1} \left( \frac{-1 + 2 \gamma + \log(x^2) + \Psi^{(0)}(n-2) + \Psi^{(0)}(n)}{x^2} \right) \ .
\end{equation*}
%
\end{example}

In the next chapter, we shall use a similar method to extend the distributions which defines the time ordered products. However, we already notice in this example that the extraction of the possible pole of an analytic regularization is not an easy task. In example \ref{exo:ms_numeric}, without the relation \eqref{eq:relation_exo_ms}, we would not have been able to extract the poles. In next section we shall present a procedure to help to subtract the principal part for specific distributions.


%----------------------------------------------------------------------------%
\subsubsection{Extensions of (almost) homogeneous distributions}
%----------------------------------------------------------------------------%


As already said we shall here present a procedure to facilitate the extraction of the poles with respect to the parameter of regularization for specific distributions.


The same geometric transformation \eqref{eq:geo_scaling_transfo_numeric} can be used to introduce some homogeneity properties of a distribution.


\begin{definition}\label{def:homogeneous_numeric}
A distribution $u \in \Dcal^\prime$ is called {\bf homogeneous of degree} $\delta$, if it satisfies the equality
%
\begin{equation}
\lambda^{\delta} \sm{ u , \phi_\lambda } = \sm{ u , \phi } \ , \ \lambda > 0 \ , \ \mbox{ and } \ \delta \in \Cbb \ ,
\label{eq:homog_id}
\end{equation}
%
where the transformation \eqref{eq:geo_scaling_transfo} are applied for all $\phi \in \Dcal$.
\end{definition}


The definitions \ref{def:steinmann_scaling_degree} and \ref{def:homogeneous_numeric} imply that a distribution which is homogeneous of degree $\delta$ has scaling degree equal to $-\Re(\delta)$, indeed
%
\begin{equation}
\mbox{if} \quad u \in \Dcal^\prime \quad \mbox{is homogeneous, then} \quad \sd(u) = -\Re(\delta) \ .
\end{equation}
%
Moreover, if the distribution $u \in \Dcal^\prime(\Rbb^n\setminus \{0\})$ is homogeneous of degree $\delta$, and $\delta$ is not an integer equal or smaller to $-n$, then $u$ has a unique extension $\exte{u} \in \Dcal(\Rbb^n)$ which is homogeneous of degree $\delta$, and $\sd(\exte{u})=\sd(u)$ \cite{HORMANDER_1990}.


An homogeneous distribution $u \in \Dcal^\prime$ can also be characterized with the Euler operator for the numerical distribution,. Actually, a distribution $u$ scales homogeneously with degree $\delta$ if and only if 
%
\begin{eqnarray*}
&& \sm{\left(x_\alpha \partial^\alpha - \delta \right) u ,\phi} =  0 \ , \\
&& \mbox{for all} \ \ \phi \in \Dcal \ , \ \ \mbox{and} \ \ x_\alpha \partial^\alpha = x_1 \dfrac{\partial^{\alpha_1}}{\partial x_1^{\alpha_1}} \dots x_n \dfrac{\partial^{\alpha_n}}{\partial x_n^{\alpha_n}} \ .
\end{eqnarray*}
%
The following theorem furnishes the existence of extensions of homogeneous distributions, and distributions which still satisfy a scale property that we shall defined.
%
\begin{theorem}\label{theo:almost_homo_numeric}
Let $u \in \Dcal^\prime(\Rbb^n\setminus\{0\})$ and suppose that for some degree $\delta\in\Cbb$ and some power $m\in\Nbb\setminus\{0\}$, 
%
\begin{equation}
\sm{\left(x_\alpha \partial^\alpha - \delta \right)^{m+1} u , \phi } =  0
\label{eq:quasi_homog}
\end{equation}
%
where $m$ is the smallest natural number such that relation \eqref{eq:quasi_homog} is satisfied\footnote{Notice that in the case $m=0$ \eqref{eq:quasi_homog} is the homogeneous scaling \eqref{eq:homog_id}.}. Then $u$ has an extension $\exte{u}\in\Dcal^\prime(\Rbb^n)$ satisfying \eqref{eq:quasi_homog} for the same degree $\delta$, and power $\ell \geq m$,
%
\begin{itemize}
\item for $\delta \notin \Nbb \setminus \{0\} + \ell$, $\exte{u}$ is unique and scale under same power as $u$, i.e. $\ell=m$,
%
\item for $\delta \in \Nbb \setminus \{0\} + \ell$, $\exte{u}$ is non unique and $\ell\in\{ m , m+1 \}$
\end{itemize}
%
\end{theorem}

In \cite{DF_2004} it has been proved that the requirement \eqref{eq:quasi_homog} is equivalent to 
%
\begin{equation}
\lambda^{m+1} \ \dfrac{d^{m+1}}{d\lambda^{m+1}} \bigg( \lambda^{-\delta}  \ u(\lambda x ) \bigg) = 0 \ .
\label{eq:quasi_euler_numeric}
\end{equation}


Let us consider a distribution $u \in \Dcal^\prime(\Rbb^n\setminus\{0\})$ which satisfies \eqref{eq:quasi_homog} with degree $-\delta = n + \alpha$, we impose $\alpha\in\Cbb\setminus\{0\}$ and $\Re(\alpha)<1$. Thus we have
%
\begin{equation*}
0 = \left( x_\alpha \partial^\alpha + n + \alpha \right)^{m+1} u \ .
\end{equation*}
%
Using the binomial expansion formula we see that we can restrict ourselves to 
%
\begin{eqnarray}
0 &=&\left(x_\alpha \partial^\alpha + \alpha \right)^{m+1} u \nonumber \\
&=& \alpha^{m+1} u + \sum_{k=1}^{m+1} \binom{m+1}{k} \ \alpha^{m+1-k} \ \left(x_\alpha \partial^\alpha \right)^{k} \circ u \nonumber \\
&=& \alpha^{m+1} u + (-1)^{m+1} \sum_{k=1}^{m+1} \binom{m+1}{k} \alpha^{m+1-k} \ \bigg( \left(\partial^\beta x_\beta \right) \circ \left(x_\alpha \partial^\alpha \right)^{k-1} \circ u \bigg) \ .
\label{eq:almost_homo_expre}
\end{eqnarray}
%
The Steinmann scaling degree of the term in brackets is equal to $\sd(u) = - \Re(\delta)=n+\Re(\alpha)$ which can be bigger than $n$ thus the extension is not unique. However, we have
%
\begin{equation*}
\sd\bigg(x_\alpha \circ \left(x_\beta \partial^\beta \right)^{k-1} \circ u\bigg) = \sd(u) - 1 = n + \Re(\alpha) - 1 < n \ .
\end{equation*}
%
Thus this term can be uniquely extended, and its extension is its direct extension introduced in section \ref{p:REG_PROCED}. The poles in $\alpha$ are apparent in \eqref{eq:almost_homo_expre}, therefore we can subtract the principal part with respect to $\alpha$. Finally we have
%
\begin{equation*}
u_\ms = - \sum_{k=1}^{n+1} \binom{n+1}{k} \alpha^{-k} x_\alpha \left( \overline{ x_\alpha \circ \left(x_\beta \partial^\beta \right)^{k-1} \circ u } \right) \ , 
\end{equation*}
%
where we used the notation $\overline{v}$ for the direct extension of $v$.


\bigskip


We have just presented a method to extend certain numerical distribution using the minimal subtraction scheme. This method is useful to compute explicitly the principal part with respect to the parameter of regularization. In the next chapter we shall also use the $\MS$ method to extend distributions defined on manifolds, and to help to extract poles we will introduce a similar method that the one presented in this section.


%----------------------------------------------------------------------------%
\chapter{A covariant regularization scheme}
\label{p:COV_REG}
%----------------------------------------------------------------------------%


In the perturbative construction of models in quantum field theory on curved spacetimes we have to construct time ordered products of field polynomials which are a priori ill defined due to the appearance of UV divergences. Several regularization schemes which deal with these divergences in the presence of non trivial spacetime curvature have been discussed in the literature \cite{BUNCH_1981,LUSCHER_1982,TOMS_1982,CHL_1995,PRANGE_1999,BILAL_2013,BF_2000,HW_2001,HW_2005,HOLLANDS_2010,BCK_2010}. The Epstein--Glaser regularization discussed in section \ref{p:REG_PB} is one of them. Although this scheme is conceptually clear and mathematically rigorous, it is not easily applicable in practical computations.\par%


In this chapter we develop a regularization scheme for time ordered products in interacting field theories on curved spacetimes, manifestly generally covariant, invariant under any spacetime isometries present and constructed to all orders in perturbation theory.  As in the previous chapters, we discuss only scalar fields in four spacetime dimensions, but we shall argue that this scheme can be directly generalized to other spacetime dimensions and field theories with higher spin, as well as to theories with local gauge invariance.\par%


The renormalization scheme we propose is inspired by the works \cite{KELLER_2010,DFKR_2014} recalled in section \ref{p:REG_PB}, which deal with perturbative QFT in Minkowski spacetime. In these works, the authors introduce an analytic regularization of the position space Feynman propagator in Minkowski spacetime which is similar to the one discussed in \cite{BG_1972}. This chapter is based on the article \cite{GHP_2015}.\par%


%----------------------------------------------------------------------------%
\section{General regularization framework}
\label{p:REG_GENERAL}
%----------------------------------------------------------------------------%


We have seen in section \ref{p:EXT_UP_TOT} that the time ordered product of local functionals can be extended up to the total diagonal via a recursive method. Therefore even if the regularization scheme we shall develop in the present chapter proceed in another way we know that the only relevant divergence is the one on the total diagonal. For this reason, in this section, we shall focus to the extension of generic distributions $u \in \Dcal^\prime(\Mcal^n \setminus d_n )$ over the total diagonal.\par%


Following the path adopted in section \ref{p:REG_NUMERRIC} in section \ref{p:EXT_SD}, we shall characterize the possible divergences occurring on $d_n$. Later on in section \ref{p:EULER} we shall construct an  explicit extension over the diagonal.


%----------------------------------------------------------------------------%
\subsection{Extensions and scaling degree}
\label{p:EXT_SD}
%----------------------------------------------------------------------------%


We shall here extend the concepts of extensions and Steinmann scaling degree introduced for numerical distributions in section \ref{p:REG_NUMERRIC}. 


The distribution $u \in \Dcal^\prime(\Mcal^n \setminus d_n )$ is defined for all test functions supported outside the the total diagonal. We recall that the \textbf{total diagonal} $d_n$ is the following subspace of $\Mcal^n$
%
\begin{equation}
d_n = \left\{ (x,\dots,x) \subset \Mcal^n \right\} \ .
\label{eq:total_diag_chap_3}
\end{equation}
%
We call $\exte{u}$ an \textbf{extension} of $u$ if 
%
\begin{equation*}
\mbox{for } \ \exte{u} \in \Dcal^\prime(\Mcal^n), \ \mbox{ we have } \ \forall \phi \in \Dcal\left(\Mcal^n \setminus d_n \right), \ \exte{u}(\phi) = u(\phi) \ .
\end{equation*}
%
In order to obtain an extension $\exte{u}$ we need relevant informations about the behavior of such a distribution in the neighborhood of the total diagonal $d_n$. For this reason, and having in mind the path followed in section \ref{p:REG_NUMERRIC}, we shall introduce a geometric scaling transformation. However, before doing this, let us define the neighborhood of the total diagonal.


First we shall show that $\Mcal$ admits a covering of open geodesically convex sets. Actually, we have the following lemma taken from \cite{ONEIL_1983}.


\begin{lemma}
For any lorentzian manifold $\Mcal$ there exits a cover $\Ccal$ such that every elements $\Ncal_i$ of $\Ccal$ and their overlaps $\Ncal_i \cap \Ncal_j$ are open geodesically convex subspaces of $\Mcal$.
\end{lemma}


Therefore we consider such a cover $\Ccal$ of $\Mcal$, and then we define the set
%
\begin{equation}
\Ncal_{n} = \bigcup_{\Ncal\in\Ccal} \ \underbrace{\Ncal \times \dots \times \Ncal}_{n \mbox{ times}} \subset \Mcal^n 
\label{eq:neib_tot_diag}
\end{equation}
%
We call sets of the form $\Ncal_n$ the \textbf{normal neighborhoods of the total diagonal} $d_n$. We can deduced that for any point $x \in \Mcal$ we can find a normal neighborhood $\Ncal_x \in \Ccal$ of $x$ in $\Mcal$. 


As we know for every pair of points $x_1,x_i$ in a normal neighborhood $\Ncal \subset \Mcal$ there exists a unique geodesic $\gamma$ connecting $x_1$ and $x_i$. We shall assume that
%
\begin{equation*}
\gamma : \lambda \mapsto x_i(\lambda) 
\end{equation*}
%
is affinely parametrized and that $x_i(0) =x_1$ whereas $x_i(1) = x_i$. For all $\lambda \geq 0$ and all $\phi \in \Dcal(\Ncal_{n})$ with $\Ncal_n \subset \Mcal^n$ a normal neighborhood of the total diagonal $d_n$ \eqref{eq:neib_tot_diag}, we shall consider the following \textbf{geometric scaling transformation} 
%
\begin{equation}
\phi_\lambda = \lambda^{4(n-1)} \ \phi\left(x_1(\lambda ),x_2(\lambda ),\dots,x_n(\lambda\right)) \ \prod_{i=2}^n \sqrt{\abs{\frac{g(x_i(\lambda ))}{g(x_i)}}} \ ,
\label{eq:geo_scaling_transfo}
\end{equation}
%
where $g(x)$ is the the determinant of the metric $g$ expressed in normal coordinates. For $\lambda > 1$ it may happen that $(x_1(\lambda),\dots, x_n(\lambda))$ lies outside $\Ncal_n$ and is thus not well defined in general. In this case we may set $\phi_\lambda = 0$  for points lying outside $\Ncal_n$ by means of some smooth partition of unit.


For later purposes, we recall that the \textbf{determinant of the metric} computed in normal coordinates centered at $x_1$ is such that
%
\begin{equation*}
\sqrt{g(x_i)} = \frac{1}{\usf^2(x_1,x_i)} \ , 
\end{equation*}
%
where $\usf$ is the Hadamard coefficient introduced in the definition of the two point function of an Hadamard state in section \ref{p:STATES}, and $\usf^2$ is the van Vleck--Morette determinant, see e.g. \cite{PPV_2011} for a derivation of this formula. We notice that the geometric scaling transformation \eqref{eq:geo_scaling_transfo} is similar to the one defined for numerical distributions in \eqref{eq:geo_scaling_transfo_numeric}, the difference here is that the scaling towards the point $x_1$ is done along geodesics. We can now introduce a notion of scaling degree towards submanifolds which generalizes the  Steinmann scaling degree \ref{def:steinmann_scaling_degree}. Let us first of all introduce some notation.  Consider a distribution $u \in \Dcal^\prime(\Mcal^n)$ or $u \in \Dcal^\prime(\Mcal^n \setminus d_n)$, employing the scaling transformations \eqref{eq:geo_scaling_transfo} we might introduce the family of \textbf{scaled distributions} 
%
\begin{equation*}
\sm{u_\lambda,\phi} := \sm{u,\phi_{1/\lambda}} \ . 
\end{equation*}
% 
Furthermore, if 
%
\begin{equation}\label{eq:transversal-wf}
\overline{\WF(u)}\cap \left\{(x_1,\dots, x_n,k,0,\dots,0)\in T^*\Mcal^n, \forall k\neq 0 \right\} = \emptyset \ ,    
\end{equation}
%
we say that the $\WF(u)$ is \textbf{transversal} to $d_n$. We may now introduce a definition of scaling degree inspired from the results of section 6 in \cite{BF_2000} .


\begin{definition}\label{def:scaling_degree}
Consider a distribution $u \in \Dcal^\prime(\Mcal^n)$ or $u \in \Dcal^\prime(\Mcal^n \setminus d_n)$ with scaling degree transversal to $d_n$. The \textbf{scaling degree} of $u$ towards $d_n$ is defined as 
%
\begin{equation*}
\sd(u) \ := \ \inf\bigg\{ \omega \in \Rbb \ \bigg| \ \lim_{\lambda \downarrow 0} \ \lambda^\omega \ u_\lambda \ = \ 0 \bigg\} \ .
\end{equation*}
%
where the limit is taken in the H\"ormander topology of $\mathcal{D}^\prime_\Gamma(\Mcal^n\setminus d_n)$, and 
$\Gamma$ is transversal to $d_n$, namely \eqref{eq:transversal-wf} holds for $\Gamma$.
%
\end{definition}


We recall haw the definition of $\mathcal{D}^\prime_\Gamma(\Mcal^n\setminus d_n)$
%
\begin{equation*}
\mathcal{D}^\prime_\Gamma(\Mcal^n\setminus d_n) = \left\{ u \in \mathcal{D}^\prime(\Mcal^n\setminus d_n) \ , \ \WF(u) \subset \Gamma \right\} \ .
\end{equation*}
%
Theorem \ref{theo:extension_distribution_numeric} can be partially extended to the present situation. 

\begin{corollary}\label{corol:theo_sd_M}
If a distribution $u \in\Dcal^\prime(\Mcal^n)$ or $u\in\Dcal^\prime(\Mcal^n\setminus d_n)$ has scaling degree towards $d_n$ strictly lower than the total dimension of the scaled coordinates $4(n-1)$, then $u$ possesses a unique extension towards $d_n$ with the same scaling degree. 
\end{corollary}
%
A proof of this corollary can be obtained along the lines of the proof of theorem 6.9 in the paper of Brunetti and Fredenhagen \cite{BF_2000}.


In the same way as we did for numerical distributions in section \ref{p:REG_NUMERRIC}, we define the \textbf{analytic regularization} of a distribution $u \in \Dcal^\prime(\Mcal^n\setminus d_n)$ as a family $\left\{ u^{(\alphabd)}\right\}$ of distributions in $\Dcal^\prime(\Mcal^n)$  such that
%
\begin{center}
$u^{(\alphabd)}$ is weakly analytic with respect to $\alphabd$ in some domain with $0$ in its boundary, and $\ \underset{\alphabd\to 0}{\lim} u^{(\alphabd)} = u$ \ on $\Dcal(\Mcal^n\setminus d_n)$ ,
\end{center}
%
where $\alphabd$ is a multivariate complex parameter. We shall define the notion of multivariate analytic function.

We consider the complex space $\Cbb^d$, and an open subspace $\Omega^d \in \Cbb^d$. A generic point $\alphabd = \left(\alpha_1 , \dots , \alpha_d\right) \in \Cbb^d$ has real and imaginary part respectively denoted by the pair $(\mathbf{a},\mathbf{b})\in \Rbb^d \times \Rbb^d$. An open ball $B(\betabd,\epsilon)\subseteq\Omega^d$ centered at $\betabd \in \Omega^d$ of radius $\epsilon > 0$ is defined as follow
%
\begin{equation*}
B(\betabd,\epsilon) = \left\{ \alphabd \in \Cbb^d \ \mbox{ such that } \ \norm{\alphabd - \betabd} < \epsilon \right\} \ , \ \mbox{with } \ \norm{\alphabd} = \max_{1\leq i \leq d} \sqrt{a_i^2 + b_i^2} \ .
\end{equation*}
%
We say that \textbf{the map}
%
\begin{equation*}
f(\alphabd) : \left\{
\begin{array}{ccl}
\Omega^d & \to & \Cbb^d \\
\alpha & \mapsto & \sm{u^{(\alphabd)},\phi} \ , \quad \phi \in \Dcal(\Mcal^n)
\end{array}
\right.
\end{equation*}
%
is \textbf{analytic}, or \textbf{holomorphic} at a point $\betabd \in \Omega^d$ if $f(\alphabd)$ is given in $B(\betabd,\epsilon)$ as a power series
%
\begin{eqnarray}
&& f(\alphabd) = \sum_{k_1 , \dots , k_d \geq 0}^\infty a_{k_1 , \dots , k_d} \ (\alpha_1 - \betabd_1)^{k_1} \ \dots \ (\alpha_d - \betabd_d)^{k_d} \ ,
\label{eq:laurent_series} \\
&& \mbox{such that for } \ 0 \leq r \leq \epsilon \ , \ \mbox{we have } \ \sum_{k_1 , \dots , k_d \geq 0}^\infty a_{k_1 , \dots , k_d} \ r^{k_1+\dots+k_n} \ < \infty \ . \nonumber
\end{eqnarray}
%
Then $f(\alphabd)$ is said to be analytic on the domain $\Omega^d$ if $f(\alphabd)$ is analytic in every points in $\Omega^d$. 


If in the power series \eqref{eq:laurent_series} of $f(\alphabd)$ the condition $k_{i\in\{1,\dots,d\}} \geq 0$ is dropped for some $k_i$s, e.g. for $k_2$ and $k_3$, then we say that $f(\alphabd)$ has poles in $\alpha_2$ and is $\alpha_3$. For $k_i$ finite the pole in $\alpha_i$ is said to be of finite order. The function $f(\alphabd)$ is said to be \textbf{meromorphic} on an open subset $\Omega^d$ if $f(\alphabd)$ is a function that is analytic on all $\Omega^d$ except on a set of isolated pole(s) of finite order(s).


\bigskip


A result permits to simplify the previous definition. If $u^{(\alphabd)}$ is continuous and analytic in every single variables, where each of them is an element of the complex plane $\Cbb$, then $u^{(\alphabd)}$ is analytic in $\alphabd$. We are then back to analyticity on $\Cbb$.


Let us give a last result which is convenient for practical computations. 


\begin{theorem}[Cauchy--Riemann criterion]
We consider the complex function $f(\alphabd)$ in $\alphabd$ which is defined in an open subset $\Omega^d \subset \Cbb^d$ and which is continuously differentiable in the underlying real coordinates of $\Cbb^d$ (real and imaginary parts of $\alphabd$). Then $f(\alphabd)$ is analytic in $\Omega^d$ if and only if it satisfies the system of partial differential equations
%
\begin{equation}
\frac{\partial}{\partial \bar{\alpha}_i} f(\alphabd) := \frac12 \left( \frac{\partial}{\partial a_i} + i \frac{\partial}{\partial b_i} \right) f(\alphabd) = 0 \ ,
\end{equation}
%
for $i\in\{1,\dots,d\}$ and $\alphabd\in\Omega^d$.
\end{theorem}


Once we obtain an analytic regularization we would like to have a procedure which gives us the corresponding extension(s). In section \ref{p:REG_NUMERRIC} among the procedures discussed we defined the $\MS$ subtraction. We shall adapt theorem \ref{theo:ms_mumeric} to the present situation.


\begin{theorem}[Minimal subtraction for complex valued multivariate functions]\label{theo:ms_general}
Any analytic regularization $\{u^{(\alpha)}\}$ of a distribution $u \in \Dcal^\prime(\Mcal^n\setminus d_n)$ with $\alpha\in\Omega\subset \Cbb$ provides an extension of $u$ which is defined by 
%
\begin{equation*}
\sm{u_\ms,\phi} = \lim_{\alpha \to 0} \ \bigg( \sm{u^{(\alpha)},\phi} - \pp\sm{u^{(\alpha)},\phi}\bigg) \ ,
\end{equation*}
%
where $\pp$ is the principal part with to $\alpha$. The extension $u_\ms$ is called the minimal subtraction of $u$.
\end{theorem}

For a generic analytic regularization $\{u^{(\alphabd)}\}$ where $\alphabd$ is in $\mathbb{C}^n$ we could also find an extension in a similar way as the one introduced in the previous theorem, however this extension could not be unique, actually, the procedure of taking the limit $\alphabd \to 0$ and the extraction of the principal part in the $\MS$ subtraction can be tricky. It the case of the time ordered product it shall be done with the help of the Forest formula, cf. theorem \ref{theo:renorm_t_prod_ms_forest}.


%----------------------------------------------------------------------------%
\subsection{The generalized Euler operator}
\label{p:EULER}
%----------------------------------------------------------------------------%


In this section we analyze analytically regularized distributions satisfying a certain weaker homogeneity condition, and show how the principal part of a distribution of this type can be efficiently computed.

\bigskip


We first recall the notion of homogeneity for distributions, which can be defined in a similar way as we did in \ref{def:homogeneous_numeric}, only the geometric scaling transformation \eqref{eq:geo_scaling_transfo} differs.


\begin{definition}\label{def:homogeneous}
A distribution $u \in \Dcal^\prime(\Mcal^n)$ or $u\in \Dcal^\prime(\Mcal^n\setminus d_n)$, which satisfies the equality 
%
\begin{equation}
\lambda^{\delta} \sm{ u , \phi_\lambda } = \sm{ u , \phi } \ , \quad \forall \lambda > 0 \ , 
\label{eq:homo_distrib}
\end{equation}
%
under geometric scaling transformations of the form \eqref{eq:geo_scaling_transfo} for all $\phi\in\Dcal(\Ncal_n\setminus d_n)$ and for a $\delta\in\Cbb$, is called \textbf{homogeneous of degree} $\delta$. 
\end{definition}

According to this definition, a distribution $u$ which is homogeneous of degree $\delta$ and with $\WF{(u)}$ transversal to $d_n$ has scaling degree $-\Re(\delta)$, indeed for all $\phi\in\Dcal(\Ncal_n\setminus \{x\})$ we have
%
\begin{equation*}
\inf\bigg\{ \omega \in \Rbb \ \bigg| \ \lim_{\lambda \downarrow 0} \lambda^\omega \sm{u,\phi_{1/\lambda}} = \lim_{\lambda \downarrow 0} \lambda^{\omega+\delta} \sm{u,\phi} = \lim_{\lambda \downarrow 0} \lambda^{\omega+\Re(\delta)} \lambda^{i\Im(\delta)} \sm{u,\phi}
= 0  \bigg\} = - \Re(\delta) \ ,
\end{equation*}
%
hence, $\sd(u) -\Re(\delta)$ because, for homogeneous distribution with wave front set transversal to $d_n$, the limit $\lambda\to 0$ of $\lambda^\omega u_\lambda$ in the topology of distribution is equivalent the limit in the H\"ormander topology $\mathcal{D}'_\Gamma(\Mcal^n\setminus d_n)$ with $\Gamma$ transversal to $d_n$.


Finally, for homogeneous distributions towards $d_n$ with wave front set transversal to $d_n$ we have actually the following lemma whose proof can be obtained following the one of   \cite[theorem 3.2.3]{HORMANDER_1990} due to L. Hörmander.


\begin{lemma}\label{lemma:exte_homo}
If $u\in\Dcal(\Mcal^n\setminus d_n)$ has wave front set transversal to $d_n$, is homogeneous of degree $\delta$, and $-(\delta+4(n-1)) \notin \Nbb$, then $u$ possess a unique extension $u \in \Dcal^\prime(\Mcal^n)$ with the same degree of homogeneity.
\end{lemma}


\begin{sketch}
The existence of an extension can be obtained in two ways. It is possible to write a proof along the lines of the proof of \cite[theorem 3.2.3]{HORMANDER_1990} due to L. Hörmander or, thanks to the previous observation about the relation of the scaling degree towards $d_n$ and the degree of homogeneity, we might directly use \cite[theorem 6.9]{BF_2000}. The uniqueness of the extension descends form the uniqueness of the extension of the homogeneous distributions $u_x$ obtained restricting $u$ to the set $\{ (x,y_2,\dots, y_n)\in \Mcal^n, \forall y_i\in \Mcal\}$.  
\end{sketch}


Unfortunately the relevant distributions we have to regularize in order to extend the time ordered product are not homogeneous. Nonetheless, as we mention in section \ref{p:REG_NUMERRIC}, they are ``almost homogeneous''. Therefore we consider a family of distributions $u^{(\alpha)} \in \Dcal^\prime(\Ncal_n\setminus d_n)$ defined for $\alpha$ in $\Omega\setminus 0$ for some neighborhood $\Omega \subset \Cbb$ of $0$ and assume that $u^{(\alpha)}$ have wave front set transversal to $d_n$ and that they can be expanded as
%
\begin{equation*}
u^{(\alpha)}  = \sum_{k=0}^m u^{(\alpha)}_k + r^{(\alpha)} \ ,
\end{equation*}
%
where $u^{(\alpha)}_k$ are homogeneous terms with transversal wave front set and with degree with degree $a_k = - \delta_\alpha + k$\footnote{The choice on how to write $a_k$ is arbitrary but independent from the following construction.}, thus 
%
\begin{equation*}
\sd(u^{(\alpha)}_k) = - \Re(a_k)  =\Re\left(\delta_\alpha\right) - k \geq 4(n-1) \ .
\end{equation*}
%
We place ourselves in the worst case where $\sd(u^{(\alpha)}_k) \geq 4(n-1)$, otherwise the extension is straightforward due to corollary \ref{corol:theo_sd_M}. The remainder $r^{(\alpha)}\in \Dcal^\prime(\Ncal_n\setminus d_n)$ has scaling degree towards $d_n$ strictly smaller than $4(n-1)$ and thus according to corollary \ref{corol:theo_sd_M} it can be uniquely extended to $x$ for every $\alpha \in \Omega$.


\bigskip


As we did for numerical distributions when we introduced a notion of ``almost homogeneity'' by theorem \ref{theo:almost_homo_numeric}, here we shall define the \textbf{generalized Euler operator}. We consider a normal neighborhood $\Ncal_n$ of the total diagonal $d_n$, see e.g.\eqref{eq:neib_tot_diag} and define the operator $\Esf_p$ as follow
%
\begin{equation}
\Esf_p : \left\{
\begin{array}{lcl}
\Dcal(\Ncal_n) & \to & \Dcal(\Ncal_n) \\
\phi(x_1,\dots, x_n) & \mapsto & \left. (-1)^p \ \lambda^{p+4(n-1)} \ \dfrac{d^p}{d\lambda^p} \bigg( \lambda^{-4(n-1)}  \phi_\lambda(x) \bigg) \right|_{\lambda = 1}
\end{array}
\right. \ ,
\label{eq:euler_op}
%
\end{equation}
%
where the scaling transformation \eqref{eq:geo_scaling_transfo} is used. Owing to its homogeneity, every $u^{(\alpha)}_k$ can be rewritten by means of the relation \eqref{eq:homo_distrib} and the generalized Euler operator $\Esf_p$ \eqref{eq:euler_op} as
%
\begin{equation}
\sm{ u^{(\alpha)}_k, \phi } = \frac{1}{\overset{p-1}{\ \underset{j=0}{\prod}} \ (a_k+j+4(n-1))}   \sm{ u^{(\alpha)}_k, \Esf_p \phi } \ .
\label{eq:expose_poles}
\end{equation}
%
Note that, $\Esf_p \phi(x_1, \dots x_n)$ is smooth and vanishes for
%
\begin{equation*}
y = (y_1 , \dots , y_n) \to x = (x_1, \dots , x_n) \ , \ \mbox{ as } \ \ C|y-x|^p \ , 
\end{equation*}
%
i.e. it is in the class $\Ocal(|y-x|^{p})$. For this reason, if  $p$ is chosen sufficiently large as $p > -a_k-4(n-1)$ then 
\begin{equation*}
u^{(\alpha)}_k \circ \Esf_p 
\end{equation*}
%
possesses a unique extension towards $d_n$ by applying Lemma \ref{lemma:exte_homo}. 


\bigskip


We recall that, in order to regularize $u^{(\alpha)}$ for $\alpha=0$ in the $\MS$ procedure of theorem \ref{theo:ms_general}, we have to subtract its principal part before computing the limit of vanishing $\alpha$
%
\begin{equation*}
\sm{ (u_k)_\ms, \phi } = \lim_{\alpha \to 0} \left( \sm{ u^{(\alpha)}_k, \phi } - \pp\sm{ u^{(\alpha)}_k , \phi } \right) \ .
\end{equation*}


However, if we use the representation of $u^{(\alpha)}_k$ provided by the right hand side of equation \eqref{eq:expose_poles}, its poles are manifestly exposed and can be easily subtracted. We recall that, since the original distribution $u^{(\alpha)}_k$ is well defined on $\Ncal_n\setminus d_n$ even for $\alpha=0$, the principal part we are subtracting can only be supported on $d_n$. We summarize this discussion in the following proposition.



\begin{proposition}\label{prop:regularization}
Consider $\Ncal_n$, a normal neighborhood of the diagonal $d_n$ and $\Omega \subset \Cbb$ a neighborhood of the origin. Assume that $u^{(\alpha)}\in\Dcal^\prime(\Ncal^n\setminus d_n)$ is an \textbf{analytic regularization} of $u\in\Dcal^\prime(\Ncal^n\setminus d_n)$. Moreover, assume that $u^{(\alpha)}$ can be decomposed as 
%
\begin{equation*}
u^{(\alpha)} = \sum_{k=0}^m u^{(\alpha)}_k + r^{(\alpha)} \ ,
\end{equation*}
%
where $u^{(\alpha)}_k$ are weakly analytic distributions which scale homogeneously under transformations of the form \eqref{eq:geo_scaling_transfo} with degree $a_k = -\delta_\alpha + k$, and where $r^{(\alpha)}_k$ is a weakly analytic distribution whose scaling degree towards $d_n$ is strictly smaller than $4(n-1)$. 
If $u^{(\alpha)}_k$ and $r^{(\alpha)}_k$ have wave front set transversal to $d_n$, then the following statements hold.
%
\begin{enumerate}
\item\label{item:1_regularization} $u^{(\alpha)} \in \Dcal^\prime(\Ncal^n\setminus \{ x\})$ can be extended to $\exte{u}^{(\alpha)} \in \Dcal^\prime(\Ncal^n)$ for every $\alpha \in \Omega \setminus \{0\}$.
%
\item\label{item:2_regularization} $\exte{u}^{(\alpha)}$ is weakly meromorphic for $\alpha \in \Omega$ with possible poles for $\alpha=0$ and it is the unique weakly meromorphic extension of $u^{(\alpha)}$.
%
\item\label{item:3_regularization} The pole of $\exte{u}^{(\alpha)}$ in $0$ is supported on $d_n$.
%
\item\label{item:4_regularization} The limit $\alpha \to 0$ can be considered after subtracting the pole part, namely
%
\begin{equation*}
\sm{ u_\ms, \phi } = \lim_{\alpha \to 0} \left( \sm{ \exte{u}^{(\alpha)} , \phi } - \pp\sm{ \exte{u}^{(\alpha)} , \phi } \right) 
\end{equation*}
%
is well defined for all $\phi \in \Dcal(\Ncal^n)$ and $u_\ms$ is an extension of $u$ which preserves the scaling degree.
\end{enumerate}
%
\end{proposition}


\begin{proof}
\begin{description}
%
%
\item[\ref{item:1_regularization} \& \ref{item:2_regularization}] The distribution $u^{(\alpha)}$ is the sum of homogeneous terms $u^{(\alpha)}_k$ plus a remainder term $r^{(\alpha)}$. The latest term $r^{(\alpha)}$ has scaling degree strictly smaller than $4(n-1)$, therefore according to corollary \ref{corol:theo_sd_M} $r^{(\alpha)}$ possesses an unique extension towards $d_n$ for every $\alpha \in \Omega \setminus \{0\}$. Regarding the homogeneous terms $u^{(\alpha)}_k$, we can use lemma \ref{lemma:exte_homo}. A term $u^{(\alpha)}_k$ has degree of homogeneity equal to $a_k = -\delta_\alpha + k$, which satisfies the condition $-\left(a_k+4(n-1)\right) \notin \Nbb$, thus every $u^{(\alpha)}_k$ possesses a unique extension in $\Dcal^\prime(\Ncal_n)$ for every $\alpha \in \Omega \setminus \{0\}$. It follows that the map
%
\begin{equation}
\alpha \mapsto \sm{\exte{u}^{(\alpha)} , \phi} 
\label{eq:proof_weak_homeo}
\end{equation}
%
is analytic in $\alpha \in \Omega \setminus \{0\}$, and that $\exte{u}^{(\alpha)}$ is the unique analytic extension of $u^{(\alpha)}$. Thus the map \eqref{eq:proof_weak_homeo} is meromorphic for $\alpha \in \Omega$ with possible poles for $\alpha=0$ and $\exte{u}^{(\alpha)}$ is the unique weakly meromorphic extension of $u^{(\alpha)}$.%
%
%
\item[\ref{item:3_regularization}] We know that the original distribution $u^{(\alpha)}$ defined on $\Ncal^n\setminus d_n$ is weakly analytic. An explicit construction of the weakly meromorphic extension $\dot{u}^{(\alpha)}$ to $\Ncal^n$ is provided by 
\eqref{eq:expose_poles} that we recall here%
%
\begin{equation*}%
\sm{ u^{(\alpha)}_k, \phi } = \frac{1}{\overset{p-1}{\ \underset{j=0}{\prod}} \ (a_k+j+4(n-1))}   \sm{ u^{(\alpha)}_k, \Esf_p \phi } \ .%
\end{equation*}%
%
Indeed if we choose for every component $u^{(\alpha)}_k$ a sufficiently large $p$, then according corollary \ref{corol:theo_sd_M} we shall be able to construct a weakly meromorphic extension $\exte{u}^{(\alpha)}$ to $\Ncal^n$. Hence, the poles of $\exte{u}^{(\alpha)}$ can only be supported on $d_n$.%
%
%
\item[\ref{item:4_regularization}] Because the pole of $\exte{u}^{(\alpha)}$ in $0$ is supported on $d_n$, after subtracting the principal part of the distribution the limit $\alpha \to 0$ can be safely taken. The such obtained distribution prior to considering the limit $\alpha \to 0$ coincides with $u^{(\alpha)}$ on $\Ncal^n\setminus d_n$ and the same holds in the limit $\alpha\to 0$. Consequently $u_\ms$ is an extension of $u$.\par%
%
Finally, due to \ref{item:1_regularization} we have by the lemma \ref{lemma:exte_homo} that the extension of each $u^{(\alpha)}_k$ keeps the degree of homogeneity, and therefore keeps also the scaling degree. By corollary \ref{corol:theo_sd_M} the extension of the remainder $r^\alpha$ conserves the scaling degree. Therefore $\sd(\exte{u}^{(\alpha)})=\sd(u^{(\alpha)})$. We have by definition $\lim_{\alpha\to 0}u^{(\alpha)} = \sd(u)$, thus because the subtraction of the principal part is a linear operation we can conclude $\sd(u_\ms)=\sd(u)$.
%
%
\end{description}
\end{proof}


We now discuss how equation \eqref{eq:expose_poles} can be used in order to regularize the most singular part of a distribution $u^{(\alpha)}$ which is known to be of the form
%
\begin{equation*}
u^{(\alpha)} = \sum_{k=0}^m u^{(\alpha)}_k + r^{(\alpha)} \ ,
\end{equation*}
%
but where the distributions $u^{(\alpha)}_k$ are not explicitly known. To this end, observe that equation \eqref{eq:expose_poles} implies
%
\begin{eqnarray*}
\sm{ u^{(\alpha)} , \Esf_p \phi } &=& \sum_{k=0}^m \sm{u^{(\alpha)}_k , \Esf_p \phi } + \sm{r^{(\alpha)} , \Esf_p \phi } \\
&=& \sum_{k=0}^m \left( \prod_{j=0}^{p-1} (a_k+j+4(n-1)) \right) \sm{ u^{(\alpha)}_k , \phi } + \sm{ r^{(\alpha)} , \Esf_p \phi } \\
&=& \sum_{k=0}^m \ c_{k,p} \ \sm{ u^{(\alpha)}_k , \phi } + \sm{ r^{(\alpha)} , \Esf_p \phi } \ ,
\end{eqnarray*}
%
where we introduced the coefficients
%
\begin{equation*}
c_{k,p} = \prod_{j=0}^{p-1} \left(a_k+j+4(n-1)\right) \ .
\end{equation*}
%
Moreover, we may assume without loss of generality as in proposition \ref{prop:regularization} that the homogeneity degrees $a_k$ of $u^{(\alpha)}_k$ are of the form $a_k = -\delta_\alpha + k$ where $\Re(\delta_\alpha)$ is the scaling degree of $u^{(\alpha)}$. Consequently $u^{(\alpha)}_0$ is the contribution with the highest scaling degree which may be extracted by considering
%
\begin{equation}
\sm{ u^{(\alpha)} , \Esf_p \phi } - c_0 \sm{ u^{(\alpha)} , \phi } = \sum_{k=1}^m (c_k-c_0) \sm{ u^{(\alpha)}_k , \phi } + \sm{ r^{(\alpha)} , \Esf_p \phi } - c_0 \sm{ r^{(\alpha)}, \phi } \ ,
\label{eq:decrease_scaling_degree}
\end{equation}
%
where the distributions on the right hand side of \eqref{eq:decrease_scaling_degree} have a scaling degree smaller than $\Re(\delta_\alpha) = - \Re (a_0)$. Hence, although in general the distribution $u^{(\alpha)}$ does not scale homogeneously, equation \eqref{eq:expose_poles} still holds up to distributions with a lower scaling degree. Knowing the decreasing degree of homogeneity of the components in the expansion of $u^{(\alpha)}$, we may use a recursive procedure in order to expose the pole part of this distribution. In fact, the previous discussion implies the validity of the following proposition.


\begin{proposition}\label{prop:expose_poles}
We consider a distribution $u^{(\alpha)}$ with the properties assumed in proposition \ref{prop:regularization} and set
%
\begin{equation*}
V_0 = u^{(\alpha)} \ , \qquad V_{k+1} = c_k V_k - V_k \circ \Esf_{p_k} \ , \qquad 0 \leq k < m 
\end{equation*}
%
where $p_k$ are the smallest natural numbers chosen in such a way that 
%
\begin{equation*}
p_k+\Re(a_k)+4(n-1)>0 \ , \ \mbox{ and } \ \ c_k = \prod_{j=0}^{p_k-1} \left(a_k+j+4(n-1)\right) \ .
\end{equation*}
%
Then in order to expose the poles of $u^{(\alpha)}$, we may invert the recursive definition of $U_k$ obtaining
%
\begin{equation*}
u^{(\alpha)} = \frac{1}{c_0} \left( V_0\circ \Esf_{p_0} +  \frac{1}{c_1} \left( V_1 \circ \Esf_{p_1} +\dots + \frac{1}{c_n} \left( V_n \circ \Esf_{p_n} + V_{n+1} \right) \right) \right) \ .
\end{equation*}
%
\end{proposition}


In order to be able use the previous results for our purposes, we provide in the next proposition a criterion which is sufficient to ensure that a generic distribution can be decomposed into the sum of a homogeneous distribution and a remainder with lower scaling degree. 

\begin{proposition}\label{prop:set}
Let $\Ncal_n$ be a normal neighborhood of the total diagonal $d_n$ and $u \in \Dcal^\prime(\Ncal_n)$ with scaling degree $s_1$ towards $d_n$. If there exists an $\delta\in\Cbb$ satisfying $\Re(\alpha)=-s_1$ such that
%
\begin{equation*}
u\circ(\Esf_1+\delta+4(n-1)) 
\end{equation*}
%
has scaling degree $s_2 < s_1$, then $u$ can be decomposed into the sum of a homogeneous distribution with degree of homogeneity $\delta$ and a remainder with scaling degree smaller than or equal to $s_2$.
\end{proposition}


\begin{proof}
We start by observing that, for every test function $\phi\in\Dcal(\Ncal_n)$, 
%
\begin{equation*}
F(\lambda, \phi) := \sm{ u, \phi_{1/\lambda} } 
\end{equation*}
% 
is a continuous linear functional of $\phi$ which is smooth in $\lambda$ for $\lambda>0$. Moreover, since the scaling degree of $u$ is $s_1$, $\lambda^{a} F(\lambda, \phi)$ vanishes in the limit $\lambda \to 0$ for every $a > s_1$ and for every $\phi\in\Dcal(\Ncal_n)$. Let us now consider 
%
\begin{equation*}
G(\lambda, \phi) :=  \sm{\left(- \Esf_1+\alpha+4(n-1)\right) u \ , \ \phi_{1/\lambda} } \ .
\end{equation*}
%
$G(\lambda, \cdot)$ is again a family of distributions on $\Ncal_n$ which depends smoothly on $\lambda$ for positive $\lambda$. Furthermore, $\lambda^a G(\lambda, \phi)$ vanishes in the limit $\lambda \to 0$ for every $a > s_2$ and every $\phi\in\Dcal(N_n)$. Hence, $\lambda^{\alpha-1} G(\lambda, \cdot)$ tends to $0$ in $\Dcal^\prime(\Ncal_0)$ for $\lambda\to 0$ and, additionally, the Banach--Steinhaus theorem \cite{BOURBAKI_1987} implies that 
%
\begin{equation}
\abs{ \lambda^a G(\lambda,\phi) }  \leq C \sum_{\alpha\leq k} \abs{\partial^{\alpha} \phi} \ ,
\label{eq:bound_dist}
\end{equation}
%
for every $a>s_2$, uniformly for $\phi$ supported in a compact set $K\subset \Ncal_n$ and for suitable $C$ and $k$ which do not depend on $\lambda$.\par%
%
After these preparatory considerations, we observe that $G$ and $F$ are related by the generalized Euler operator in the following way
%
\begin{equation*}
G(\lambda, \phi) =  \lambda^{-\alpha+1} \frac{\dsf}{\dsf\lambda} \lambda^{\alpha} F(\lambda, \phi) \ . 
\end{equation*}
%
We can invert this relation to obtain
%
\begin{equation*}
F(\lambda, \phi) =  \frac{C(\phi)}{\lambda^{\alpha}} + \frac{1}{\lambda^\alpha} \int_0^\lambda  \tilde{\lambda}^{\alpha-1} G(\tilde{\lambda}, \phi) \dsf\tilde{\lambda}, 
\end{equation*}
%
where, $C(f)$ is a suitable constant which depends on $\phi$. We want to prove that $C(\cdot)$ is in fact a distribution. To this end, we note that, owing to the bound \eqref{eq:bound_dist}, the integral in $\tilde\lambda$ can be performed and the result of this integration is a distribution for every $\lambda > 0$ because $\Re(\alpha) > s_2$. This implies that 
%
\begin{equation*}
C(\phi) = F(1,\phi) - \int_0^1  \tilde{\lambda}^{\alpha-1} G(\tilde{\lambda},\phi) d\tilde{\lambda} 
\end{equation*}
%
is a distribution because it is a linear combination of distributions. By construction $F(1,\phi_{1/\lambda})=F(\lambda,\phi)$ and $C\circ (E_1+\alpha) =0$, hence $C$ is a homogeneous distribution of degree $\alpha$. By means of a direct computation we also find that the scaling degree of the remainder $F(1,\phi)-C(\phi)$ is smaller than or equal to the scaling degree of $G$ which is $s_2$.
\end{proof}



We are going to analyze the action of the generalized Euler operators $\Esf_p$ appearing in \eqref{eq:expose_poles} on test functions. In fact, we shall see that $\Esf_p$ corresponds to a particular geometric partial differential operator. To this end, we observe that\footnote{It is shown by direct computation starting from the relation \eqref{eq:expose_poles}.}
%
\begin{equation*}
\Esf_p = (\Esf_1 + (p-1)) \Esf_{p-1} \ . 
\end{equation*}
%
Hence, knowing the differential form of the generalized Euler operator $\Esf_1$, it is possible to construct recursively every $\Esf_p$. Regarding the differential form of $\Esf_1$, we note that it can be written in terms of the geodesic distance and the van Vleck--Morette determinant\footnote{Recall that the square root of the van Vleck--Morette determinant coincides with the Hadamard coefficient $\usf$ appearing in \eqref{eq:hadamard_rep}.} $\usf^2$  as
%
\begin{equation*}
\Esf_1 \phi(x_1 , \dots , x_n) = \sum_{j=2}^n \bigg( \sigma^a(x_j) \nabla^{x_j}_a  - 2  \sigma^a(x_j) \nabla^{x_j}_a  \log\left(\usf(x_j,x_1)\right) \bigg) \phi(x_1 , \dots , x_n) \ , 
\end{equation*}
%
where $\nabla^{x_j}_a$ indicates the $a$--th component of the covariant derivative computed in $x_j$ and $\sigma^a(x_j) = {\nabla^{x_j}}^a \sigma(x_1,x_j)$. Considering the adjoint $\Esf^\dagger_p$ of $\Esf_p$, we have $u \circ \Esf_p = \Esf^\dagger_p u$ where, using the relation
\begin{equation*}
\Box \sigma + 2 \sigma^a \ \nabla_a\left( \log (u) \right) = 4 \ , 
\end{equation*}
%
we find for $p=1$
%
\begin{eqnarray}
\Esf_1^\dagger  u(x_1,\dots,x_n) &=& \sum_{j=2}^n \Bigg( - \nabla^{x_j}_a \sigma^a(x_j) - 2 \sigma^a(x_j) \bigg( \nabla^{x_j}_a \log\left(u(x_j,x_1)\right) \bigg) \Bigg) u(x_1,\dots,x_n) \nonumber \\
&=& -\left( 4(n-1) + \sum_{j=2}^n \sigma^a(x_j) \nabla^{x_j}_a \right) u(x_1,\dots,x_n) \ .
\label{eq:euler_operator}
\end{eqnarray}
%
We finally observe that the recursive identity for $\Esf_p$ implies that also $\Esf^\dagger_p$ can be constructed recursively starting from $\Esf^\dagger_1$ as 
%
\begin{equation*}
\Esf_p^\dagger = \Esf_{p-1}^\dagger \left(\Esf_1^\dagger-(p-1)\right) \ . 
\end{equation*}
%
Therefore knowing $\Esf_1^\dagger$ gives us every differential operator $\Esf_p^\dagger$.


%----------------------------------------------------------------------------%
\section{Another approach to the regularization problem}
\label{p:ANOTHER_APPROACH}
%----------------------------------------------------------------------------%


Here we present the approach to treat the regularization problem we choose in the present work. First in section \ref{p:PIC_REG_PB} we introduce a ``graphical way'' to write the time ordered product of observables. To the treat the regularization issues we need to introduce a suitable regularization scheme. A possibility is the Epstein Glaser procedure discussed in the previous chapter, but even if it is mathematically well established, it is inconvenient for practical computations, thus we choose to work with the minimal subtraction scheme already discussed previously. We present in section \ref{p:FOREST FORMULA} the $\MS$ method adapted to perform computations. Then we will see in section \ref{p:GLOB_TPROD} that the time ordered is defined only locally, we will therefore give a way to define it globally. 


%----------------------------------------------------------------------------%
\subsection{``Pictorial'' perspective of the time ordered product}
\label{p:PIC_REG_PB}
%----------------------------------------------------------------------------%


We define a method to write the time ordered product in a ``graphical way'', i.e. as a sum of Feynman diagram. 


\bigskip


The  time ordered product of ``causal'' functionals can be viewed  as the following map
%
\begin{equation}
\Tcal_n \ : \ 
\left\{
\begin{array}{lcl}
\Fcal_\Tsf(\Mcal)[[\hbar]]^{\otimes n} & \to & \Fcal_\Tsf(\Mcal)[[\hbar]] \\
\Fsf_1(\phi) \otimes \ ... \ \otimes \Fsf_n(\phi) & \mapsto & \Fsf_1(\phi) \cdot_{\Tsf} \ ... \ \cdot_{\Tsf} \Fsf_n(\phi)
\end{array}
\right. \ .
\label{eq:time_ordered_op}
\end{equation}
% 
By defining the $n$-th order pointwise product $\msf_\nsf$
%
\begin{equation*}
\msf_\nsf \left( \Fsf_1 \otimes \ ... \ \otimes \Fsf_n \right)(\phi) \ = \ \Fsf_1(\phi) \cdot \ ... \ \cdot \Fsf_n(\phi) \ ,
\end{equation*}
%
we can rewrite the time ordering product of $n$ argument in terms of an exponential
%
\begin{equation*}
\Tcal_n (\Fsf_1 \otimes \ ... \ \otimes \Fsf_n)(\phi) \ := \ \msf_\nsf \circ \Tsf_n \bigg( \Fsf_1(\phi_1) \otimes \ ... \ \otimes \Fsf_n(\phi_n) \bigg) \bigg|_{\phi_1 = ... = \phi_n = \phi} \ ,
\end{equation*}
%
with 
%
\begin{eqnarray*}
&&\Tsf_n := \exp\left(\hbar \sum_{1 \leq i < j \leq n} D_{ij}\right) =
\prod_{1 \leq i < j \leq n} \ \sum_{\abs{\ell_{ij}}=1}^\infty \hbar^{\abs{\ell_{ij}}} \ \frac{D_{ij}^{\abs{\ell_{ij}}}}{\abs{\ell_{ij}} !} \ , \\
&& \mbox{ and } \ D_{ij} \ := \ \sm{\Delta_{ij} \ , \ \frac{\delta^2 \hfil}{\delta \phi_i} } \ 
\end{eqnarray*}
%
where $\phi_i = \phi(x_i)$, and $\Delta_{ij}=\Delta_\fsf(x_i,x_j)$. We shall now introduce the graphical procedure we shall use. 


\bigskip


We aim to rewrite the time ordered product as a sum of graphs. To this end we start giving some standard definitions. A generic graph $\gamma$ is a set of points, called vertices $V(\gamma)$, which can be connected by some (unoriented) lines called edges $E(\gamma)$. We shall denote by $\abs{E(\gamma)}$ and $\abs{V(\gamma)}$ the number of elements in the set $E(\gamma)$ and $V(\gamma)$ respectively. 


We define by $\Gcal_n$ the \textbf{set of graphs} with the following requirements
%
\begin{itemize}
\item $\gamma$ has $n$ vertices, 
%
\begin{equation*}
\abs{V(\gamma)} = n \ ,
\end{equation*}
%
%
\item an edge of $\gamma$ which connect the vertices $i$ and $j$, is denoted by $\ell_{ij}$. In order to avoid tadpoles, i.e. when an edge connects a single vertex, we require
%
\begin{equation*}
\ell_{ii} = 0 \ .
\end{equation*}
%
we shall use the notation $\partial \ell_{ij} = \{i,j\}$, and $\abs{\ell_{ij}}$ which denotes the number of edges between the vertices $i$ and $j$.
%
%
\item We define for each graph $\gamma$ its corresponding degree of divergence as follow
%
\begin{equation}
\omega_\gamma = 2 \abs{E(\gamma)} - 4(\abs{V(\gamma)} - 1) \ ,
\label{eq:deg_div_graph}
\end{equation}
%
and we say $\gamma$ is
\begin{enumerate}
\item superficially convergent if $\omega_\gamma  > 0$, 
\item logarithmically divergent if $\omega_\gamma = 0$,
\item divergent if $\omega_\gamma < 0$. 
\end{enumerate}
%
In the case of a graph with only two vertices (e.g. $\FtwoG$ or $\FthreeG$) the graphs is divergent for $\abs{V(\gamma)} \geq 2$. Indeed the scaling degree towards the diagonal of $\Delta_\fsf^r$ defined in definition \ref{def:scaling_degree} is lower than $4$ only for $r=1$.
\end{itemize}


\bigskip


Therefore the time ordered product can be written as follow in order to encode the full combinatorics of the Feynman diagrams,
%
\begin{eqnarray}
&& \Tsf_n \ = \ \sum_{\gamma \in \Gcal_n} \Tsf_\gamma \ , \qquad \mbox{with} \quad \Tsf_\gamma \ = \ \frac{1}{\Nsf(\gamma)} \ \sm{\tsf_\gamma \ , \ \delta_\gamma} \ ,
\label{eq:time_ordered_prod_graph} 
\\
%
&& \Nsf(\gamma) = \hbar^{-\abs{\Esf(\gamma)}} \prod_{\ell_{ij}\in\Esf(\gamma)} \abs{\ell_{ij}} ! = \hbar^{-\abs{\Esf(\gamma)}} \prod_{(i,j) \in \gamma} \abs{\ell_{ij}} ! \ , \nonumber \\
%
&& \delta_\gamma = \frac{\delta^{2\abs{E(\gamma)}}\hfill}{\underset{\substack{i \in V(\gamma), \\ j \in \partial \ell_{ij}}}{\prod} \delta \phi_i^{\abs{\ell_{ij}}}} \ := \ \frac{\delta^{2\abs{E(\gamma)}}\hfill}{\underset{(i,j)\in\gamma}{\prod} \delta \phi_i^{\abs{\ell_{ij}}}} \ , \nonumber \\[2pt]
%
\mbox{and} &&\tsf_\gamma = \prod_{\substack{i < j, \\ i \in V(\gamma), \\ j \in \partial \ell_{ij}}} \Delta_{ij}^{\abs{\ell_{ij}}} \ := \ \prod_{(i,j)\in\gamma} \Delta_{ij}^{\abs{\ell_{ij}}} \ , 
%
\label{eq:kernel}
\end{eqnarray}



We can use the notion of scaling degree previously introduced in definition \ref{def:scaling_degree} to give the scaling degree of $\tsf_\gamma$ \eqref{eq:kernel} as follow
%
\begin{equation}
\sd(\tsf_\gamma) = 2 \abs{E(\gamma)} \ . 
\label{eq:sd_graph}
\end{equation}
%
Without giving a general proof we can see that this is really the case for it by looking at the triangular graph with one internal loop, on Minkowski spacetime. In particular considering
%
\begin{eqnarray*}
\FtwoGoneHoneF \quad \mbox{ using \eqref{eq:kernel} it gives us } \quad \tsf_\gamma &=& \Delta_{12}^2 \ \Delta_{13} \ \Delta_{23} \  . 
\end{eqnarray*}
% 
We recall that $\Delta_{ij}=\Delta_\fsf(x_i,x_j)$. The Feynman propagator for free massive scalar fields on Minkowski can be written as
%
\begin{equation*}
\Delta_\fsf(x) = \frac{1}{\left(2\pi\right)^4} \int_{\Rbb^4} \dsf^4p \ \frac{\esf^{ipx}}{p^2 - m^2 + i \epsilon} \ , \quad \mbox{ and has scaling degree } \quad \sd(\Delta_\fsf) = 2 \ .
\end{equation*}
%
Thus assuming for the moment that the product in the expression of $\tsf_\gamma$ is well defined, its scaling degree is the sum of the scaling degree of each term. Therefore $\sd(\tsf_\gamma)$ is given by \eqref{eq:sd_graph}. 


\quad


Let us illustrate this graphical approach, and give some explicit expansion of the time ordered product in terms of graphs.


\begin{examples}\label{exo:t_prod_trig_exp}
First the time ordered product of two observables 
%
\begin{eqnarray*}
(\Fsf \cdot_{\Tsf} \Gsf)(\phi) &=& \Fsf(\phi) \cdot \Gsf(\phi) + \hbar \sm{\Fsf^{(1)}(\phi) , \Delta_\fsf \ \Gsf^{(1)}(\phi)} + \frac{\hbar^2}{2} \sm{\Fsf^{(2)}(\phi) , \Delta_\fsf^{\otimes2} \ \Gsf^{(2)}(\phi)} \\
&& + \frac{\hbar^3}{3} \sm{\Fsf^{(3)}(\phi) , \Delta_\fsf^{\otimes3} \ \Gsf^{(3)}(\phi)} + \Ocal(\hbar^4) \\[4pt]
&=&\FG + \hbar \FoneG + \hbar^2 \FtwoG + \hbar^3 \FthreeG + \Ocal(\hbar^4) \ .
\end{eqnarray*}
%
Second the time ordered product for three observables
%
\begin{eqnarray*}
(\Fsf \cdot_{\Tsf} \Gsf \cdot_{\Tsf} \Ksf)(\phi) 
%&=& 
%\Fsf(\phi) \cdot \Gsf(\phi) \cdot \Ksf(\phi) \\
%&& 
%+ 
%\hbar 
%\Bigg( 
%\sm{\Delta_{12},\Fsf^{(1)}(\phi_1)\Gsf^{(1)}(\phi_2)\Ksf(\phi_3)} 
%+ 
%\sm{\Delta_{13},\Fsf^{(1)}(\phi_1)\Gsf(\phi_2)\Ksf^{(1)}(\phi_3)} \\
%&&
%+ 
%\sm{\Delta_{23},\Fsf(\phi_1)\Gsf^{(1)}(\phi_2)\Ksf^{(1)}(\phi_3)} 
%\Bigg) \\
%&& 
%+ 
%\frac{\hbar^2}{2} 
%\Bigg( 
%\sm{\Delta_{12}\Delta_{13},\Fsf^{(2)}(\phi_1)\Gsf^{(1)}(\phi_2)\Ksf^{(1)}(\phi_3)} \\
%&&
%+ 
%\sm{\Delta_{12}\Delta_{23},\Fsf^{(1)}(\phi_1)\Gsf(\phi_2)^{(2)}\Ksf^{(1)}(\phi_3)} \\
%&&
%+ 
%\sm{\Delta_{13}\Delta_{23},\Fsf^{(1)}(\phi_1)\Gsf^{(1)}(\phi_2)\Ksf^{(2)}(\phi_3)} \\
%&&
%+ \frac12 
%\bigg( 
%\sm{\Delta_{12}^{\otimes2},\Fsf^{(2)}(\phi_1)\Gsf^{(2)}(\phi_2)\Ksf(\phi_3)} \\
%&&
%+ 
%\sm{\Delta_{13}^{\otimes2},\Fsf^{(2)}(\phi_1)\Gsf(\phi_2)\Ksf^{(2)}(\phi_3)} \\
%&&
%+ 
%\sm{\Delta_{23}^{\otimes2},\Fsf(\phi_1)\Gsf^{(2)}(\phi_2)\Ksf^{(2)}(\phi_3)} 
%\bigg) 
%\Bigg) 
%+ 
%\Ocal(\hbar^3)\\
&=& \FGH+\hbar\left(\FoneGHF+\FGoneHF+\FGHoneF\right) \\
&&+\hbar^2\left[\FoneGHoneF+\FoneGoneHF+\FGoneHoneF +\frac12\left(\FtwoGHF+\FGtwoHF+\FGHtwoF\right)\right]
+\mathcal{O}(\hbar^3)
\end{eqnarray*}
\end{examples}


%----------------------------------------------------------------------------%
\subsection{Forest formula for the minimal subtraction scheme}
\label{p:FOREST FORMULA}
%----------------------------------------------------------------------------%


We have just presented a convenient way to represent the product $\cdot_\Tsf$ of ``causal'' local functionals. However, in the previous expansion and in particular in equation \eqref{eq:kernel},
%
\begin{equation*}
\tsf_\gamma = \prod_{(i,j)\in\gamma} \Delta_{ij}^{\abs{\ell_{ij}}} \ , \ \mbox{ with } \ \ \Delta_{ij}=\Delta_\fsf(x_i,x_j)
\end{equation*}
%
is a distribution which is well defined outside of all  \textbf{partial diagonals}, namely on $\Dcal(\Mcal^n\setminus D_n)$, where 
%
\begin{equation}
D_n = \left. \bigg\{x_1, \dots , x_n \ \right| \ x_i = x_j \ \text{ for at least one pair } \  (i,j), \ \mbox{ with } , \ i\neq j \bigg\} \ .
\label{eq:partial_diagonals}
\end{equation}
%
The restricted domain of $\tsf_\gamma$ is the reason why $\Tsf_n$ defined in \eqref{eq:time_ordered_prod_graph} is not a well defined operation for quantum local functionals, i.e. $\Fcal_\loc(\Mcal)[[\hbar]]^{\otimes n}$. 


In order to complete the construction of \eqref{eq:time_ordered_prod_graph} we need to extend the obtained distributions $\tsf_\gamma$ to all diagonals $D_n$. The extension is not  straightforward because of the singular structure of the Feynman propagator $\Delta_\fsf$. Indeed, as already explained in section \ref{p:INT_Q_ALG}, pointwise products of Feynman distribution are ill defined on the coinciding points. Consequently we do need a regularization procedure in order to extend $\tsf_\gamma$ to the full space $\Dcal(\Mcal^n)$. We shall see that this extension is in general not unique, but subject to a so-called regularization freedom.


\bigskip


The procedure we shall use is the minimal subtraction ($\MS$) method defined presented in theorem \ref{theo:ms_general}. The $\MS$ procedure shall permit us to extend distributions $\tsf_\gamma$ \eqref{eq:time_ordered_prod_graph} to partial diagonals $D_n$. It makes use of an analytic regularization defined in section \ref{p:EXT_SD} of the Feynman propagator $\Delta_\fsf$, defined as $\Delta_\fsf^{(\alpha)}$, parametrized by complex parameters $\alpha$ contained in some neighborhood $\Omega \subset \Cbb$ of the origin. An analytic regularization of the Feynman propagator $\Delta_\fsf$ shall be studied later on. Nonetheless the idea of the $\MS$ scheme is independent of the details of the analytic regularization. Namely, given any analytic regularization $\Delta^{(\alpha)}_\fsf$ of $\Delta_\fsf$ we repeat the formal construction of $\Tsf_n$ in \eqref{eq:time_ordered_prod_graph} by replacing $\Delta_{ij}$ by $\Delta^{(\alpha_{ij})}_{ij}$ in the integral kernels $\tsf^{(\alphabd)}_\gamma$ given in \eqref{eq:kernel} in particular we obtain
%
\begin{equation}
\tsf^{(\alphabd)}_\gamma = \prod_{(i,j) \in \gamma} \Delta^{(\alpha_{ij},\abs{\ell_{ij}})}_{ij} \ ,
\label{eq:kernel_reg}
\end{equation}
%
where $\Delta^{(\alpha_{ij},\abs{\ell_{ij}})}_{ij}$ is the regularized version of $\Delta^{\abs{\ell_{ij}}}_{ij}$, and, with $\alpha_{ij} \in \Omega \subset \Cbb$, and  $\alphabd=\{\alpha_{ij}\}_{(i,j) \in \gamma}$ are multivariate complex parameters contained in some neighborhood of the origin. To each Feynman distribution $\Delta_{ij}$, corresponding to one line between the vertices $i$ and $j$, we associate a complex parameter $\alpha_{ij}$ in order to be able to treat separately the divergences occurring on partial diagonals. In the case of a triangular graph with sub divergences we have the following situation. For the particular graph
%
\begin{eqnarray*}
\FnGnHnF
&& \mbox{\eqref{eq:kernel} gives us } \ \
\tsf_\gamma = \Delta_{12}^{\abs{\ell_{12}}} \ \Delta_{13}^{\abs{\ell_{13}}} \ \Delta_{23}^{\abs{\ell_{23}}} \ , \\
&& \mbox{and the regularized form \eqref{eq:kernel_reg} gives us } \ \ \tsf_\gamma^{(\alphabd)} = \Delta_{12}^{(\alpha_{12},\abs{\ell_{12}})} \ \Delta_{13}^{(\alpha_{13},\abs{\ell_{13}})} \ \Delta_{23}^{(\alpha_{23},\abs{\ell_{23}})} , \\
&& \mbox{with } \ \alpha_{13}, \alpha_{13}, \alpha_{13} \in \Omega \subset \Cbb \ \mbox{ a neighborhood of the origin. Here,} \\
&& \alphabd = \{ \alpha_{12} , \alpha_{13} , \alpha_{23} \} \ .
\end{eqnarray*}
%
Proceeding in this way for every graph we can define 
%
\begin{equation}
\Tsf_n^{(\alphabd)} = \sum_{\gamma \in \Gcal_n} \Tsf_\gamma^{(\alphabd)} \ , \qquad \mbox{with} \quad \Tsf_\gamma^{(\alphabd)} \ = \ \frac{1}{\Nsf(\gamma)} \ \sm{\tsf_\gamma^{(\alphabd)} \ , \ \delta_\gamma}
\label{eq:time_ordered_prod_graph_reg}
\end{equation}
%
where $\alphabd=\{\alpha_{ij}\}_{(i,j) \in \gamma}$ are multivariate complex parameters contained in some neighborhood of the origin.


\bigskip


We shall show that the distributions $\tsf^{(\alphabd)}_\gamma$ introduced in \eqref{eq:kernel_reg} are \textbf{multivariate meromorphic functions} which have poles at the origin for some of the $\alpha_{ij}$, cf. section \ref{p:EXT_SD}. Hence, in order to obtain well defined distributions in the limit $\alpha_{ij}$ to $0$ and consequently a regularized time ordered product, all these poles will have to be subtracted. The analyticity property of the regularized Feynman propagator shall imply that $\tsf^{(\alphabd)}_\gamma$ is well defined on $ \Mcal^n \setminus D_n $ even if all $\alpha_{ij}$ are vanishing. Since $\tsf^{(\alphabd)}_\gamma$ is a multivariate meromorphic function in $\alphabd$  which is analytic if restricted to $\Mcal^n\setminus D_n$, we may deduce that the principal part of $\tsf^{(\alphabd)}_\gamma$ for some $\alpha_{ij}$ must be supported on a partial diagonal of $\Mcal^n$. In fact in order for the time ordered products to fulfill the factorization property \eqref{eq:causal_factorization}, the subtraction of the principal parts of $\tsf^{(\alphabd)}_\gamma$ needs to be done in such a way that at each step only local terms are subtracted. 


However the domain of analyticity of $\tsf^{(\alphabd)}_\gamma$ only implies that the support of the principal part is contained in $D_n$, i.e. the union of all the partial diagonals in $\Mcal^n$. Hence, in order to satisfy the causal factorization property, the principal part needs to be removed in a recursive way starting from the partial diagonals corresponding to two vertices and proceeding with the partial diagonals $\drak_{I}$ corresponding to an increasing number $k \leq n$ of vertices 
%
\begin{equation*}
\drak_{I} = \left\{ (x_1,\dots, x_n) \in \Mcal^n, x_i=x_j, \mbox{ with } \ i,j \in I\subset \{1,\dots, n\} , \abs{I} = k \right\} \ . 
\end{equation*}
%
We call $\drak_{I}$ \textbf{partial ordered diagonals}. 


\bigskip


The correct recursion procedure is implemented by the so called \textbf{Epstein Glaser forest formula}, which is a position space analogue of the Zimmerman forest formula \cite{DFKR_2014}. Notice that \textbf{this method does not depend on the graph expansion}. In order to present this formula let us start considering the \textbf{set of indices}
%
\begin{equation*}
\overline{n} := \{1,\dots , n\}
\end{equation*}
%
and let us define a \textbf{forest} $F$ as a collection of subsets of $\overline{n}$ with at least two elements, namely 
%
\begin{equation*}
F = \{ I_1,\dots, I_k\} \ , \qquad I_j \subset \overline{n} \ , \qquad \mbox{and} \qquad \abs{I_j} \geq 2 \ ,
\end{equation*}
%
where $\abs{I_j}$ is the number of elements contained in $I_j$. Furthermore, for every pair $I_i,I_j$ of a forest $F$ we require
%
\begin{equation*}
I_i\cap I_j = \emptyset \ , \qquad \text{or} \qquad I_i \subset I_j \ , \qquad \mbox{or} \qquad  I_j\subset I_j \ .
\end{equation*}
%
The \textbf{set of all forests} of $n$ indices together with the empty forest $\{\emptyset\}$ is denoted by $\mathfrak{F}_{\overline{n}}$.


For every subset $I\subset \overline{n}$ we indicate by $\Rsf_I$ the ``\textbf{principal part operator}'' which extracts the principal part of of a multivariate meromorphic function
%
\begin{eqnarray}
&& \Rsf_I f(\alphabd) = - \pp_{\alphabd_I} f(\alphabd) \ , \ \mbox{ with } \ \ \alphabd_I = \{\alpha_{ij}\}_{i,j \in I} \ , \\
\label{eq:pp_op}
&& \mbox{if } \ I=\emptyset \ , \quad \Rsf_\emptyset = \Ibb \ . \nonumber
\end{eqnarray}
%
The operator $\Rsf_I$ applied to $f(\alphabd)$, where $\alphabd = \{\alpha_{ij}\}$, extracts the principal part with respect to every $\alpha_{ij}$ with $i,j\in I$. We can now define the regularized time ordered product in the $\MS$ scheme \cite{DFKR_2014}.


\begin{theorem}[The regularized time ordered product in the MS scheme] \label{theo:renorm_t_prod_ms_forest}
The regularized time ordered product of $n$ arguments can be written in the $\MS$ scheme by means of the \textbf{forest formula} as follow
%
\begin{equation}
\Tcal_n = \left(\Tcal_n\right)_\ms = \lim_{\alphabd \to 0} \msf_\nsf \circ \left( \sum_{F\in\Frak_{\overline{n}}} \prod_{I\in F} \Rsf_I \right) \circ \Tsf^{(\alphabd)}_n \ .
\label{eq:ms_t_forest}
\end{equation}
%
In the product over $I\in F$, the operator $\Rsf_I$ has to be applied before $\Rsf_J$ if $I\subset J$. Further the limit $\alphabd \to 0$ is taken in two steps. 
%
\begin{enumerate}
\item after applying $\Rsf_I$ we set $\alpha_{ij}=\alpha_I$ for $i,j\in I$,
%
\item we set $\alpha_{I} = \alpha_F$ for $I \in F$ before taking the sum over all forests,
%
\item then we take the limit $\alpha_F \to 0$.
\end{enumerate}
%
%
\end{theorem}


Let us now illustrate theorem \ref{theo:renorm_t_prod_ms_forest} with few examples.


\begin{example}[Triangular graphs]\label{exo:trig_graph}
We shall illustrate \ref{theo:renorm_t_prod_ms_forest} on the triangular graphs with one and two subdivergences.
%
\begin{itemize}
\item Let us look at the Forest formula construction for the case of triangular graphs (three vertices), i.e. when the set of indices is 
%
\begin{equation*}
\{1,2,3\} \ . 
\end{equation*}
%
The subsets $I$ from which we construct the forests are then
%
\begin{equation*}
\{1,2\} \ , \quad \{1,3\} \ , \quad \{2,3\} \ , \ \mbox{ and } \quad \{1,2,3\} \ .
\end{equation*}
%
Therefore the forests are the following 
%
\begin{eqnarray*}
&&
\{(\emptyset\} \ , \
\{\{1,2\}\} \ , \
\{\{1,3\}\} \ , \
\{\{2,3\}\} \ , \
\{\{1,2,3\}\} \ , \ \\
&&
\{\{1,2\},\{1,2,3\}\} \ , \
\{\{1,3\},\{1,2,3\}\} \ , \
\{\{2,3\},\{1,2,3\}\} \ .
\end{eqnarray*}
%
Then 
%
\begin{equation*}
(\Tcal_3)_\ms = \lim_{\alphabd \to 0} \bigg( 
1
+ \Rsf_{12}
+ \Rsf_{13}
+ \Rsf_{23}
+ \Rsf_{123}
+ \Rsf_{123} \Rsf_{12}
+ \Rsf_{123} \Rsf_{13}
+ \Rsf_{123} \Rsf_{23}
\bigg) \Tsf^{(\alphabd)}_3 \ ,
\end{equation*}
%
where we already looked at $\Tcal_3$ in example \ref{exo:t_prod_trig_exp}.
%
%
%
%
\item We shall look at a particular term in $\Tcal_3$, namely at the triangular term/graph with one subdivergence with vanishing degree of divergence, i.e. $\omega_\gamma = 0$. Therefore the considered graph is logarithmically divergent. In particular, we shall consider
%
\begin{eqnarray*}
\FtwoGoneHoneF 
&& \mbox{Its regularized integral kernel \eqref{eq:kernel_reg} can be written as follow} \\
&& \tsf_\gamma^{(\alphabd)} = \Delta_{12}^{(\alpha_{12},2)} \ \Delta_{13} \ \Delta_{23} \ , \\
&& \mbox{where we have introduced a complex parameter $\alpha_{12}$ for the divergent part only}
\end{eqnarray*}
%
In the particular case of the triangular graph with one fish graph as subgraph the subsets which correspond to divergent contributions are $\{1,2\}$ and $\{1,2,3\}$, therefore the relevant forests are 
%
\begin{equation*}
\{\emptyset\} \ , \quad \{\{1,2\}\} \ , \quad \{\{1,2,3\}\} \ , \quad \{\{1,2\},\{1,2,3\}\} \ .
\end{equation*} 
%
The regularized $\tsf_\gamma$ thus reads
\begin{eqnarray}
\left(\tsf_\gamma\right)_\ms &=& 
\lim_{\alphabd \to 0} \left(1+\Rsf_{12}+\Rsf_{123}+\Rsf_{123}\Rsf_{12}\right) \tsf^{(\alphabd)}_\gamma \nonumber \\
&=&\lim_{\alphabd \to 0} (1+\Rsf_{123})(1+\Rsf_{12}) \tsf^{(\alphabd)}_\gamma
\label{eq:kernel_trig_ms}
\end{eqnarray}
%
The regularization of $\tsf_\gamma$ \eqref{eq:kernel_trig_ms} is discussed in details in section \ref{p:COMPLICATED_GRAPH}. 
%
%
%
%
\item We consider another triangular graph but with two subdivergences.
%
\begin{eqnarray*}
\FtwoGtwoHoneF 
&& \mbox{\eqref{eq:deg_div_graph} gives us } \quad \omega_\gamma = -2 \\
&& \mbox{therefore the graph considered is divergent.}
\end{eqnarray*}
%
The relation \eqref{eq:kernel_reg} gives us the form of $\tsf_\gamma$
%
\begin{equation*}
\tsf_\gamma^{(\alphabd)} = \Delta_{12}^{(\alpha_{12},2)} \ \Delta_{13} \ \Delta_{23}^{(\alpha_{23},2)} \ .
\end{equation*}
%
To build the minimal subtraction of $\tsf_\gamma$ we shall use theorem \ref{theo:renorm_t_prod_ms_forest}. Therefore the subsets which correspond to divergent contributions are $\{12\}$, $\{23\}$, and $\{123\}$, thus the relevant forests are 
%
\begin{equation*}
\{\emptyset\} \ , \quad \{\{1,2\}\} \ , \quad \{\{1,2,3\}\} \ , \quad \{\{1,2\},\{1,2,3\}\} \ , \quad \{\{2,3\},\{1,2,3\}\} \ .
\end{equation*} 
%
The regularized $\tsf_\gamma$ thus reads
\begin{eqnarray*}
\left(\tsf_\gamma\right)_\ms &=& \lim_{\alphabd \to 0}
\left(1
+ \Rsf_{12}
+ \Rsf_{23}
+ \Rsf_{123}
+ \Rsf_{123}\Rsf_{12}
+ \Rsf_{123}\Rsf_{23}
\right) \tsf^{(\alphabd)}_\gamma \\
&=& \lim_{\alphabd \to 0} (1+\Rsf_{123})(1+\Rsf_{12}+\Rsf_{23}) \tsf^{(\alphabd)}_\gamma
\end{eqnarray*}
%
\end{itemize}
%
\end{example}


\begin{example}[A more complicated graph]
We shall here look at a more complicated graph, with $4$ vertices. 
%
\begin{eqnarray*}
\catseye && \qquad \mbox{It degree of divergence is } \ \omega_\gamma = - 4 \ , \\
&& \qquad \mbox{therefore this graph is divergent.}
\end{eqnarray*}
% 
The corresponding regularized integral kernel can be written as follow
%
\begin{equation*}
\tsf_\gamma^{(\alphabd)} = \Delta_{12} \ \Delta_{13}^{2+\alpha_{13}} \ \Delta_{14} \ \Delta_{23} \ \Delta_{24}^{2+\alpha_{24}} \ \Delta_{34} \  
\end{equation*}
%
with all $\alpha_{ij}$ in a neighborhood of the origin on the complex plane. The relevant subset of indices are
%
\begin{equation*}
\{1,3\} \ , \quad 
\{2,4\} \ , \quad 
\{1,2,3\} \ , \quad
\{1,3,4\} \ , \quad
\{1,2,4\} \ , \quad
\{2,3,4\} \ , \quad
\{1,2,3,4\} \ .
\end{equation*}
%
We can then write all the relevant forests.
%
\begin{equation*}
\begin{array}{llll}
\{\emptyset\}, &\{\{1,3)\}\}, &\{\{2,4\}\}, \\
%
\{\{1,2,3\}\}, &\{\{1,3,4\}\}, &\{\{1,2,4\}\}, \\
%
\{\{2,3,4\}\}, &\{\{1,2,3,4\}\}, &\{\{1,3\},\{1,2,3\}\}, \\
%
\{\{1,3\},\{1,3,4\}\}, & \{\{2,4\},\{1,2,4\}\}, &\{\{2,4\},\{2,3,4\}\}, \\
%
\{\{1,3\},\{1,2,3,4\}\}, &\{\{2,4\},\{1,2,3,4\}\}, &\{\{1,2,3\},\{1,2,3,4\}\}, \\
%
\{\{1,3,4\},\{1,2,3,4\}\}, &\{\{1,2,4\},\{1,2,3,4\}\}, &\{\{2,3,4\},\{1,2,3,4\}\}, \\
%
\{\{1,3\},\{1,2,3\},\{1,2,3,4\}\}, &\{\{1,3),\{1,3,4\},\{1,2,3,4\}\}, &\{\{2,4\},\{1,2,4\},\{1,2,3,4\}\}, \\
%
\{\{2,4\},\{2,3,4\},\{(1,2,3,4\}\},
\end{array}
\end{equation*}
%
Finally we can write the expression of the minimal subtraction for this graph.
%
\begin{eqnarray*}
\left(\tsf_\gamma\right)_\ms &=& \lim_{\alphabd \to 0} 
\bigg( 1
+ \Rsf_{13} 
+ \Rsf_{24}
+ \Rsf_{123}
+ \Rsf_{134}
+ \Rsf_{124}
+ \Rsf_{234}
+ \Rsf_{1234} \\
&& \quad
+ \Rsf_{13} \Rsf_{123}
+ \Rsf_{13} \Rsf_{134}
+ \Rsf_{24} \Rsf_{124}
+ \Rsf_{24} \Rsf_{234}
+ \Rsf_{13} \Rsf_{1234}
+ \Rsf_{24} \Rsf_{1234} \\
&& \quad
+ \Rsf_{123} \Rsf_{1234}
+ \Rsf_{134} \Rsf_{1234}
+ \Rsf_{124} \Rsf_{1234} 
+ \Rsf_{234} \Rsf_{1234} \\
&& \quad 
+ \Rsf_{13} \Rsf_{123} \Rsf_{1234}
+ \Rsf_{13} \Rsf_{134} \Rsf_{1234}
+ \Rsf_{24} \Rsf_{124} \Rsf_{1234}
+ \Rsf_{24} \Rsf_{234} \Rsf_{1234} \bigg) \tsf_\gamma^{(\alphabd)}
\end{eqnarray*}
%
\end{example}


%----------------------------------------------------------------------------%
\subsection{Time ordered product on the whole spacetime}
\label{p:GLOB_TPROD}
%----------------------------------------------------------------------------%


In the minimal subtraction defined in theorem \ref{theo:renorm_t_prod_ms_forest} we extract the principal part of the regularized time ordered product with respect to some complex parameters previously implemented. In order to be able to extract the principal part we need to write the kernel \eqref{eq:kernel_reg} in a convenient way. We shall write $\Delta_\fsf$ using the Hadamard representation introduced in section \ref{p:STATES}. That is in a normal neighborhood of $\Mcal$ denoted $\Ncal$, a Feynman Hadamard distribution $\Hsf_\fsf$ is of the form
%
\begin{equation}
\Hsf_\fsf(x,y) = \frac{1}{8\pi^2}  \lim_{\epsilon \downarrow 0} \left( \frac{\usf(x,y)}{\sigma_\fsf(x,y)} + v(x,y) \ \log\left( M^2 \sigma_\fsf(x,y)\right) \right) + w(x,y) \ ,
\label{eq:hadamard_rep} 
\end{equation}
%
where $\sigma_\fsf(x,y) = \sigma(x,y) + i \epsilon$, where $\sigma(x,y)$ is the Synge's world function defined in section \ref{p:CONNEX_GEOD_CURV}, and $M$ is an arbitrary mass scale. The Hadamard coefficients $\usf$ and $v$ are purely geometric and thus state independent, whereas the smooth coefficient $w$ is state dependent.


The analytic regularization of the local Hadamard expansion \ref{eq:hadamard_rep} of $\Delta_\fsf$ is defined for $\alpha \in \Omega \subset \Cbb \setminus \{0\}$ as follow 
%
\begin{equation}
\Hsf^{(\alpha)}_\fsf(x,y) = \frac{1}{8\pi^2} \lim_{\epsilon \downarrow 0} \left( \frac{\usf(x,y)}{M^{2\alpha} \ \sigma_\fsf(x,y)^{1+\alpha}} + \frac{v(x,y)}{\alpha} \left( 1 - \frac{1}{ M^{2\alpha} \ \sigma_\fsf(x,y)^{\alpha} } \right) \right) + w(x,y) \ ,
\label{eq:hadamard_rep_reg}
\end{equation}
%
where we use the (arbitrary but fixed) mass scale $M$ for preserving the mass dimension of $\Hsf_\fsf$ in the regularization. We shall prove later on that $\Hsf^{(\alpha)}_\fsf$ is really an analytic regularization. The expressions \eqref{eq:hadamard_rep} and \eqref{eq:hadamard_rep_reg} are only meaningful on normal neighborhoods $\Ncal$ of $\Mcal$. We would like to define it globally. Rather than providing general and cumbersome formulas, we prefer to illustrate the idea at the example of the triangular graph discussed in example \ref{exo:trig_graph},
%
\begin{eqnarray*}
\FtwoGoneHoneF 
&& \mbox{\eqref{eq:kernel_reg} and \eqref{eq:feynman_reg} give us} \quad \tsf_\gamma^{(\alphabd)} = \Hsf^{(\alpha_{13})}_{13} \ \Hsf^{(\alpha_{23})}_{23} \left(\Hsf^{(\alpha_{12})}_{12}\right)^2 \ , \\
&& \mbox{ where } \ \Hsf_{ij} = \Hsf_\fsf(x_i,x_j) \ .
\end{eqnarray*}
%
This expression is only well defined locally because $\sigma$ used in the expression of $\Hsf_\fsf$ \eqref{eq:hadamard_rep} is defined on geodesically convex normal neighborhood. However, in section \ref{p:EXT_SD} we recalled that $\Mcal$ admits a covering of open geodesically convex sets. Therefore, in order to define $\tsf_\gamma$ globally, we may employ suitable partitions of unity. 


We recall that we denote normal neighborhood of the total diagonal $d_n$ by $\Ncal_n$, cf. \eqref{eq:neib_tot_diag}. The squared geodesic distance $\sigma$ is then well defined on $\mathcal{N}_{2}$. We set 
%
\begin{equation*}
\sigma_{ij} = \sigma(x_i,x_j) \ .
\end{equation*}
%
We observe that $\sigma_{12}$ is well defined on $\Ncal_{2}$, and that $\sigma_{12}$, $\sigma_{13}$ and $\sigma_{23}$ are well defined on $\Ncal_{3}$. We now consider the following smooth and compactly supported functions
%
\begin{eqnarray*}
&& \chi_{12} \in \Dcal(\Ncal_{2}) \ , \ \mbox{ with } \ \ \chi_{12} = 1 \ \mbox{ on } \ d_2 \subset \Ncal_{2} \ , \\[6pt]
&\mbox{and}& \chi_{123} \in \Dcal(\Ncal_{3}) \ , \ \mbox{ with } \ \ \chi_{123} = 1 \mbox{ on } d_3 \subset \Ncal_{3} \ . 
\end{eqnarray*}
%
Note that by construction $\chi_{12}$ and $\chi_{123}$ vanish outside of $\Ncal_{2}$ and $\Ncal_{3}$ respectively. We may now define the analytically regularized distribution $\tsf^{(\alphabd)}_\gamma$ by setting
%
\begin{eqnarray}
\tsf^{(\alphabd)}_\gamma &=& \Hsf^{(\alpha_{13})}_{13} \ \Hsf^{(\alpha_{23})}_{23} \left(\Hsf^{(\alpha_{12})}_{12}\right)^2 \ \chi_{12} \ \chi_{123} \ + \ \Hsf_{13} \ \Hsf_{23} \ \Hsf_{12}^2 \ (1-\chi_{12}) \nonumber \\[3pt]
&& + \ \Hsf_{13} \ \Hsf_{23} \ \left(\Hsf^{(\alpha_{12})}_{12}\right)^2 \ \chi_{12} \ (1-\chi_{123}) \ .
\label{eq:kernel_reg_glob}
\end{eqnarray}
%
By construction $\tsf^{(\alphabd)}_\gamma$ is globally well defined.


Keeping in mind this approach to define global analytically regularized quantities, we shall for simplicity work only with local quantities in the following.


%----------------------------------------------------------------------------%
\section{The regularization scheme}
%----------------------------------------------------------------------------%


%----------------------------------------------------------------------------%
\subsection{Program}
\label{p:PROGRAM}
%----------------------------------------------------------------------------%


Now that we present the new approach to the regularization problem in section \ref{p:ANOTHER_APPROACH} and that we introduced the generalized Euler operator in \ref{p:REG_GENERAL} let us applied this to our particular situation.


As we know the problem we have is using the Bogoliubov formula \eqref{eq:bogoliubov} for constructing interacting fields perturbatively, in particular, it is given in terms of the $S$, which is the time--ordered exponential \eqref{eq:S_matrix}. Unfortunately, the time ordered product defined in terms of a ``deformation'' written by means of a Feynman propagator $\Delta_\fsf$ is well defined only on regular functionals because the singularities present in $\Delta_\fsf$ forbid their application to more general functionals. Therefore we shall here look for a the regularization scheme to be applied to distributions build from the Feynman propagator.


We recalled and introduced in section \ref{p:REG_GENERAL} definitions and properties in the general case to construct an extension of an ill defined distribution. Then in section \ref{p:ANOTHER_APPROACH} we presented the regularization procedure we shall use in this work. For implementing the minimal subtraction scheme as presented in theorem \ref{theo:renorm_t_prod_ms_forest} we need to specify an analytic regularization $\Delta^{(\alpha)}_\fsf$ of the Feynman propagator $\Delta_\fsf$ on generic curved spacetimes, and show that for all graphs $\gamma \in \Gcal_n$ the analytically regularized integral kernels appearing in \eqref{eq:kernel_reg} satisfy the necessary properties for the implementation of the $\MS$ scheme.


In particular we need to show that the distribution $\tsf^{(\alphabd)}_\gamma$ \eqref{eq:kernel_reg}, which is a priori defined only on $\Mcal^n \setminus D_n$, can be uniquely extended to the full space $\Mcal^n$ without regularization. The uniqueness of this extension is important in order to obtain a definite regularization scheme. Moreover, we need to show that the distribution $\tsf^{(\alphabd)}_\gamma \in \Dcal^\prime(\Mcal^n)$ is weakly meromorphic in $\alphabd$ in a neighborhood $\Omega \subset \Cbb$ of 0. Due to the fact we shall work with the forest formula we just need to show that, setting $\alpha_{ij} = \alpha_I$ for all $i,j\in I$, \eqref{eq:kernel_reg} is weakly meromorphic in $\alpha_I$. Additionally, we need to prove that if $\tsf_\gamma$ in \eqref{eq:kernel} is well defined outside all the partial ordered diagonal $\drak_I$, then the pole of $\tsf^{(\alphabd)}_\gamma$ in \eqref{eq:kernel_reg} with $\alpha_{ij} = \alpha_I$ for all $i,j\in I$ in $\alpha_I$ is supported on $\drak_I$ and thus local. The local pole contributions are independent of the choice of $\chi_{ij}$, and $\Ncal_{ij}$ in order to write \eqref{eq:kernel_reg_glob}, such that the $\MS$ regularized amplitude $(\tsf_\gamma)_\ms$ is both globally well defined and independent of the quantities used for the global definition of the analytic regularization. Finally, we need to prove that our $\MS$ scheme satisfies all properties given in \cite{HW_2001,HW_2002} and discussed in section \ref{p:INT_Q_ALG} which a physically meaningful regularization scheme on curved spacetimes should satisfy.


\bigskip


Our plan to construct the mentioned quantities and to prove their required properties is as follows.


\begin{itemize}


\item In section \ref{p:REG_FEYNMAN_PROP} we shall build an analytic regularization $\Delta^{(\alpha)}_\fsf$ of the Feynman propagator based on the observation that locally $\Delta_\fsf$ can be written using the Hadamard representation \eqref{eq:feynman}. Indeed in a normal neighborhood of $\Mcal$ denoted $\Ncal$, $\Delta_\fsf$ can be written of the form
%
\begin{equation}
\Hsf_\fsf(x,y) = \frac{1}{8\pi^2}  \lim_{\epsilon \downarrow 0} \left( \frac{\usf(x,y)}{\sigma_\fsf(x,y)} + v(x,y) \ \log\left( M^2 \sigma_\fsf(x,y)\right) \right) + w(x,y) \ ,
\label{eq:feynman} 
\end{equation}
%
where $\sigma_\fsf(x,y) = \sigma(x,y) + i \epsilon$, where $\sigma(x,y)$ is the Synge's world function defined in section \ref{p:CONNEX_GEOD_CURV}, and $M$ is an arbitrary mass scale. The Hadamard coefficients $u$ and $v$ are purely geometric and thus state independent, whereas the smooth coefficient $w$ is state dependent. The analytic regularization of the local Hadamard expansion of $\Hsf_\fsf$ is defined for $\alpha \in \Omega \subset \Cbb \setminus \{0\}$ as follow 
%
\begin{equation}
\Hsf^{(\alpha)}_\fsf(x,y) = \frac{1}{8\pi^2} \lim_{\epsilon \downarrow 0} \left( \frac{\usf(x,y)}{M^{2\alpha} \ \sigma_\fsf(x,y)^{1+\alpha}} + \frac{v(x,y)}{\alpha} \left( 1 - \frac{1}{ M^{2\alpha} \ \sigma_\fsf(x,y)^{\alpha} } \right) \right) + w(x,y) \ ,
\label{eq:feynman_reg}
\end{equation}
%
where we use the (arbitrary but fixed) mass scale $M$ for preserving the mass dimension of $\Hsf_\fsf$ in the regularization. 


\begin{itemize}


\item In order to show that $\Hsf^{(\alpha)}_\fsf$ is well defined we prove in proposition \ref{prop:amplitude_sigma_prop_analyt} that the relevant distributions 
%
\begin{equation*}
\Asf_\gamma^{(\alphabd)} = \prod_{(i,j)\in\gamma} \frac{1}{\sigma_\fsf^{\ell_{ij}(1+\alpha_{ij})}} \in \Dcal^\prime(\Mcal^n\setminus D_n)
\end{equation*}
%
are multivariate analytic functions. The distribution of that form only shows the most singular contribution of \eqref{eq:feynman_reg}. The others contributions are of the same form up to replacing some of the factors $(1+\alpha_{ij})$ in the exponents by $\alpha_{ij}$ or $0$.


\item In order to show that $\Asf_\gamma^{(\alphabd)}$ \eqref{eq:amplitude_sigma_reg} can be uniquely extended from $\Mcal^n\setminus D_n$ to $\Mcal^n$ in a weakly meromorphic way, i.e. that the singularities relevant for the forest formula are poles of finite order, we follow a strategy similar to the one used in \cite{HW_2001} and consider a scaling expansion with respect to a suitable scaling transformation. 

We first argued in proposition \ref{prop:regularization} that an analytically regularized distribution $u^{(\alpha)}\in\Dcal^\prime(\Mcal^n\setminus d_n)$, which can be written as a sum of homogeneous terms with respect to this scaling transformation plus a sufficiently regular remainder, can be uniquely extended to $\Mcal^n$ in a weakly meromorphic way.

In proposition \ref{prop:set} we gave a sufficient condition for the existence of such a homogeneous expansion and we shall show in proposition \ref{prop:almost_homo} that the distributions $\Asf_\gamma^{(\alphabd)}$ satisfy this condition.

\end{itemize}


\item The above mentioned results are proved by means of generalized Euler operators which can be written abstractly in terms of a scaling transformation, but also in terms of covariant differential operators whose explicit form can be straightforwardly computed as we argued in section \ref{p:EULER}. 

In proposition \ref{prop:expose_poles} we used these operators in order to demonstrate how the full relevant pole structure of $\Asf_\gamma^{(\alphabd)}$ can be computed, thus showing the practical feasibility of the $\MS$ scheme. We have found that our regularization scheme corresponds in fact to a particular form of differential regularization.


\end{itemize}


%----------------------------------------------------------------------------%
\subsection{Regularization of the Feynman propagator}
\label{p:REG_FEYNMAN_PROP}
%----------------------------------------------------------------------------%


We would like to define an analytic regularization $\Hsf^{(\alpha)}_\fsf$ \eqref{eq:feynman_reg} of $\Hsf_\fsf$ \eqref{eq:feynman}. To this end we start our analysis by constructing the distribution 
%
\begin{equation}
\frac{1}{\sigma_\fsf^{1+\alpha}}
\label{eq:inverse_sigma_1+a}
\end{equation}
%
in $\Mcal^2$ for $\alpha \in \Cbb \setminus \Nbb$. We shall make use of scaling properties of \eqref{eq:inverse_sigma_1+a} and the induced quantities 
%
\begin{equation}
\Asf_\gamma^{(\alphabd)} = \prod_{(i,j)\in\gamma} \frac{1}{\sigma_\fsf^{\ell_{ij}(1+\alpha_{ij})}} \ , \ \mbox{ for } \ \gamma \in \Gcal_n \ , \ \mbox{ i.e. } \ \abs{V(\gamma)}=n \ ,
\label{eq:amplitude_sigma_reg}
\end{equation}
%
with respect to the geometric scaling transformation \eqref{eq:geo_scaling_transfo}. 

\begin{remark}
The construction we are going to present, namely the construction of certain distributions by means of powers of the geodesic distance, is in some way similar to the extension of Riesz distributions to curved spaces presented in \cite{BGP_2007}. In particular, here we shall discuss the boundary value of 
%
\begin{equation*}
\frac{1}{\sigma_\fsf^\alpha} = \frac{1}{(\sigma+i\epsilon)^{\alpha}} 
\end{equation*}
%
for $\epsilon\to0$ while ordinary Riesz distributions are related to the antisymmetric part of a different boundary value of the functions $1/\sigma^\alpha$.
\end{remark}


Therefore we introduce the distributions we shall use as building blocks for the construction of regularized Feynman propagators on $\Mcal$, introduced in definition \ref{def:cst}. 


\begin{proposition}\label{prop:sigma_1}
Consider a normal neighborhood $\Ncal_2 \subset\Mcal^2$ of $d_2$ and the following expression for $\alpha \in \Cbb$ and $\phi \in \Dcal(\Ncal_2)$
%
\begin{equation*}
\sm{ \frac{1}{\sigma^\alpha_\fsf} , \phi } = \lim_{\epsilon \to 0^+ } \ \int_{\Mcal^2} \dsf x \ \dsf y \ \frac{1}{(\sigma(x,y)+i\epsilon)^{\alpha}} \ \phi(x,y) \ .
\end{equation*}
%
Then the distribution
%
\begin{equation}
\frac{1}{\sigma_\fsf^\alpha}
\label{eq:sigma_reg}
\end{equation}
%
satisfies the following statements.
%
\begin{enumerate}
%
\item\label{item:1_sigma_1} The distribution \eqref{eq:sigma_reg} restricted to $\Dcal(\Ncal_2 \setminus d_2)$ is a distribution which is weakly analytic in $\alpha$.
%
\item\label{item:2_sigma_1} The distribution \eqref{eq:sigma_reg} is homogeneous of degree $-2\alpha$ with respect to transformations of the form \eqref{eq:geo_scaling_transfo} for $\phi \in \Dcal(\Ncal_2\setminus d_2)$.
%
\item\label{item:3_sigma_1} The distribution \eqref{eq:sigma_reg} is well defined as a distribution on $\Ncal_2$ for $2\alpha-4 \notin \Nbb$. Furthermore
%
\begin{equation*}
\mbox{for all } \ \phi \in \Dcal(\Ncal_2) \ , \ \ \sm{ \frac{1}{\sigma^\alpha_\fsf} , \phi } \ \mbox{ is analytic for } \ 2\alpha-4\notin \Nbb 
\end{equation*}
%
and meromorphic for $\alpha \in \Cbb$ with simple poles at $2\alpha-4\in \mathbb{N}$. 
%
\end{enumerate}
%
\end{proposition}


\begin{proof}
\begin{description}
\item[\ref{item:1_sigma_1}] For every $x \in \Mcal$ we fix a normal coordinate system as follow
%
\begin{equation}
\xi_x : y \to \Rbb^4 
\label{eq:normal_coord}
\end{equation}
%
in order to parametrize the point $y$ in a normal neighborhood of $x$. Consequently, on $\Ncal_2$ half of the squared geodesic distance, also called the Synge's world function, can be expressed as follow
%
\begin{equation}
\sigma(x,y)= \frac{1}{2}\eta(\xi_x(y),\xi_x(y)) = \frac{1}{2}\xi_x^{a}{\xi_x}_{a} \ , 
\label{eq:sigma_normal_coord}
\end{equation}
%
where $\eta$ is the standard Minkowski metric (cf. definition \ref{def:minkowski}) given in cartesian coordinates. We shall first show that the function $f$ defined as follow

\begin{equation}
f(\xi^a\xi_{a},\alpha) = \frac{1}{(\xi^{a}\xi_{a})^\alpha} \ , 
\label{sigma_reg_normal_coord}
\end{equation}

for $\xi^{a} \in \{z\in \mathbb{C}^4 \,|\,\Im(z) \in V^\pm \}$, where $V^\pm$ is the forward or past light cone (taken without the tip) with respect to the Minkowski metric, is  analytic both in $\xi$ and $\alpha$.


We can write $\xi^{a}\xi_{a}$ and $f$, for $\xi \in \Sigma \subset \Cbb^4$, $n \in \Nbb$, and $\alpha \in \Omega \subset \Cbb$, as follow,
%
\begin{eqnarray}
&& \xi^{a}\xi_{a} = (a+ib)^2 = a^2 - b^2 + 2iab \ , \ \mbox{ with } \ \Re(\xi)=a \ \mbox{ and } \ \Im(\xi)=b \ , \nonumber \\[2pt]
%
&& f(\xi^a\xi_{a},\alpha) = \exp\left(-\alpha \ \log(\xi^{a}\xi_{a}) \right) \ ,
\label{eq:f_log}
\end{eqnarray}
%
where in \eqref{eq:f_log} we used the ``complex logarithm'' 
%
\begin{equation}
\mathsf{log}: \left\{
\begin{array}{rcl}
\mathbb{C} \backslash \{0\}  & \to & \mathbb{R} \oplus (-\pi,\pi] \\
r \ \mathsf{e}^{i\theta} & \mapsto & \mathsf{log}(r \ \mathsf{e}^{i\theta}) \ = \ \mathsf{log}(r) \ + \ i\theta \ + \ 2 i k \pi \ .
\end{array}\right.
\label{eq:log_complex}
\end{equation}
%
The logarithm \eqref{eq:log_complex} gives a single valued function on the principal branch $(k=0)$ in the complex plane. The principal branch has a discontinuity along the negative real axis ($\theta=\pi$). Therefore $f$ is analytic if $\xi^{a}\xi_{a} \notin \mathbb{R}^0_{-}$
%
\begin{eqnarray*}
\xi^{a}\xi_{a} \in \mathbb{C} \backslash \mathbb{R}_{-}^0 &\Longleftrightarrow&
\left\{
\begin{array}{cll}
& \Im(\xi^{a}\xi_{a}) \neq 0 \\
\text{or} & \Im(\xi^{a}\xi_{a}) = 0 \ \mathsf{and} \ \Re(\xi^{a}\xi_{a}) > 0  
\end{array}
\right. \ . \\
&& \\
\Im(\xi^{a}\xi_{a}) = 0 &\Longleftrightarrow& ab=0 \\
&\Longleftrightarrow& a^0 b^0 = \mathbf{ab} \\
&& \\
\Re(\xi^{a}\xi_{a}) > 0 &\Longleftrightarrow& a^2 - b^2 > 0 \\
&\Longleftrightarrow& -(a^0)^2 + \mathbf{a}^2 + (b^0)^2 - \mathbf{b}^2 > 0 \nonumber \\
&\Longleftrightarrow& (b^0)^4 \ + \ (b^0)^2 \ (\mathbf{a}^2-\mathbf{b}^2) \ - \ \left(\mathbf{a}\mathbf{b}\right)^2 > 0 \nonumber \\
&\Longleftrightarrow& (b^0)^2 \ > \ \mathbf{b}^2 \nonumber \\
&\Longleftrightarrow& b \in V^{\pm}_b
\end{eqnarray*}
%
Therefore the functions $f$ is analytic in $\xi$ in the following set
%
\begin{equation}
\mathbb{T}^\pm = \left. \left\{ \xi \in \mathbb{C}^4 \ \right| \ \Re(\xi)\in\mathbb{R}^4 , \ \Im(\xi)\in V^\pm_{\Im(\xi)} \right\} \ .
\label{eq:cone_analytic}
\end{equation}


\begin{figure}[h]
\centering
\analytic
\label{fig:FigExp}
\caption{Domain of analyticity of $1/\xi^a\xi_a$.}
\end{figure}


Furthermore, by theorem 3.1.15 of Hörmander in \cite{HORMANDER_1990}, noticing that
%
\begin{equation*}
\abs{f(\xi^{a}\xi_{a},\alpha)} \leq C \frac{1}{\Im(\xi^{a}\xi_{a})} \ , 
\end{equation*}
we can see \eqref{sigma_reg_normal_coord} as a distribution defined as the boundary value of $f$ as $\Im(\xi)$ approach the origin $(\Im(\xi)\to0)$ from the future or past cone (we set $\Im(\xi)=\left(\frac{\epsilon}{2},0\right)$ and then consider the limit $\epsilon \to 0$).


Let us now check the analyticity in $\alpha \in \Omega \subset \Cbb$. Setting $\alpha = \beta + i \gamma$, with $\beta, \gamma \in \Rbb$, we show that $f$ satisfies the Cauchy Riemann equation with respect to $\alpha$, 
%
\begin{equation*}
\frac{\partial f(\alpha)}{\partial \beta} = - i \frac{\partial f(\alpha)}{\partial \gamma} \ \bigg( = \log(\xi^a \xi_a) f(\alpha) \bigg) \ .
\end{equation*}
%
Thus $f$ is analytic in $\alpha$.


The analytic dependence on $\alpha$ is weakly preserved in the limit $\epsilon\downarrow0$, and thus the resulting distribution is weakly analytic in $\alpha$.


\item[\ref{item:2_sigma_1}] The transformation defined in \eqref{eq:geo_scaling_transfo}
acts on points parametrized by normal coordinates as
%
\begin{equation}
\xi \to \lambda \xi \ .
\label{eq:geo_transfo_norm_coord}
\end{equation}
%
Furthermore, $1/(\xi^{a}{\xi}_{a})^\alpha$ on a subset of $\mathbb{C}^4$ is homogeneous of degree $-2\alpha$ with respect to the transformation \eqref{eq:geo_transfo_norm_coord}, indeed
%
\begin{equation*}
\sm{\frac{1}{(\xi^{a}{\xi}_{a})^\alpha},\phi_\lambda} = \lambda^{-2\alpha} \sm{\frac{1}{(\xi^{a}{\xi}_{a})^\alpha},\phi}
\end{equation*}
%
Taking into account
%
\begin{equation*}
\sm{ \frac{1}{\sigma^\alpha_\fsf}, \phi } = 
\lim_{\epsilon\to0^+ } \int_\Mcal \int_{\Rbb^4} \dsf^4\xi_x \ \dsf x \ \sqrt{\abs{g(\xi_x)}} \ \frac{2^{\alpha}}{(\xi_x^a{\xi_{x}}_a+i\epsilon)^{\alpha}} \phi(x,\xi_x) \ .
\end{equation*}
which is well defined for $\phi\in\Dcal(\Ncal_2)$, it follows that the distribution \eqref{eq:sigma_reg} is homogeneous of degree $-2\alpha$ with respect to transformations of the form \eqref{eq:geo_scaling_transfo} for $\phi \in \Dcal(\Ncal_2\setminus d_2)$.


\item[\ref{item:3_sigma_1}] Theorem 3.2.3 in \cite{HORMANDER_1990} ensures that the distribution 
%
\begin{equation*}
u := \frac{1}{(\xi^{a}{\xi}_{a})^\alpha} \in\Dcal^\prime(\Rbb^4\setminus  0) 
\end{equation*}
%
has a unique extension $\exte{u} \in \Dcal^\prime(\Rbb^4)$ preserving its degree of homogeneity $\delta=-2\alpha$ for every 
%
\begin{equation*}
-\delta\notin\Nbb\setminus\{0\} \ \Leftrightarrow \ 2\alpha-4 \notin \Nbb \ .
\end{equation*}
%
Hence, $\Ibb\otimes u$ defines a distribution on $\Mcal \times \Rbb^4$.
Finally, notice that there exists a neighborhood $\Ncal$ of the diagonal in $\Mcal^2$
where $\phi(x,\xi_x) \sqrt{\abs{g(\xi_x)}}$ is a smooth and compactly supported function for every $\phi\in \Dcal(\Ncal)$. Consequently, the statement follows by choosing a partition of unity adapted to $\Ncal_2$.
\end{description}
%
\end{proof}


The proposition \ref{prop:sigma_1} guarantees that \eqref{eq:sigma_reg} is weakly meromorphic in $\alpha$ with simple poles at $2\alpha-4\in\Nbb$. This property is preserved taking linear combinations and multiplication by smooth functions. Consequently, the analytically regularized Feynman propagator $\Hsf^{(\alpha)}_\fsf$ defined by \eqref{eq:feynman_reg} is well defined on a normal neighborhood of the total diagonal and weakly meromorphic in $\alpha$. 


The singularities of $\Hsf^{(\alpha)}_\fsf$ given in \eqref{eq:feynman_reg} represented by points in its wave front set are contained in the wave front set of $\Hsf_\fsf$ \eqref{eq:feynman}. Hence, this guarantees that the regularization does not add new divergences. The scaling degree of $\Hsf^{(\alpha)}_\fsf$  which characterize the existence and uniqueness of an extension of $\Hsf^{(\alpha)}_\fsf$ is infinite when the real part of the complex parameter $\alpha$ tends to $\infty$. We summarize these statements in the following proposition.


\begin{proposition}\label{prop:wf_h_reg}
Consider a normal neighborhood $\Ncal_2$ of the total diagonal $d_2\in\Mcal^2$. The following statements hold for the analytically continued Feynman propagator $\Hsf^{(\alpha)}_\fsf\in\Dcal^\prime(\Ncal_2)$ defined in \eqref{eq:feynman_reg}.
%
\begin{enumerate}
\item\label{item:1_wf_h_reg} $\underset{\alpha \to 0}{\lim} \ \Hsf^{(\alpha)}_\fsf = \Hsf_\fsf$
%
\item\label{item:2_wf_h_reg} $\WF\left(\Hsf^{(\alpha)}_\fsf\right) \subset \WF\left(\Hsf_\fsf\right)$
%
\item\label{item:3_wf_h_reg} $\sd\left(\Hsf^{(\alpha)}_\fsf\right) \to \infty \ \mbox{ when } \ \Re\left(\alpha\right) \to \infty$ .
\end{enumerate}
%
\end{proposition}


\begin{proof}
\begin{description}
\item[\ref{item:1_wf_h_reg} \& \ref{item:3_wf_h_reg}] 
%
The result \ref{item:1_sigma_1} tells us that 
%
\begin{equation*}
\frac{1}{\sigma_\fsf^\alpha} \in \Dcal^\prime(\Ncal_2\setminus\{0\}) \
\label{eq:inverse_sigma_reg_distrib}
\end{equation*}
%
is weakly analytic in $\alpha$, for $\alpha\in\Cbb$, thus in particular we still have \eqref{eq:inverse_sigma_reg_distrib} for $\alpha\to0$. Thus

\begin{equation*}
\lim_{\alpha\to0} \Hsf^{(\alpha)}_\fsf(x,y) = \Hsf_\fsf(x,y) 
\end{equation*}
%
The result \ref{item:2_sigma_1} tells us that \eqref{eq:inverse_sigma_reg_distrib} is homogeneous of degree $-2\alpha$, thus as we notices in section \ref{p:EULER} we have 
%
\begin{equation*}
\sd(\Hsf^{(\alpha)}_\fsf) = 2\alpha \ ,
\end{equation*}
%
and by analyticity we can take the limit $\Re(\alpha) \to \infty$ which gives us $\sd(\Hsf^{(\alpha)}_\fsf)\to\infty$.
%
%
\item[\ref{item:2_wf_h_reg}] 
%
The result \ref{item:1_sigma_1} tells us that 
%
\begin{equation*}
\frac{1}{\sigma_\fsf^\alpha} \in \Dcal^\prime(\Ncal_2\setminus\{0\}) \ ,
\end{equation*}
%
and in particular by proving this result we have shown that
%
\begin{equation*}
f(\xi^a\xi_{a}) \leq C \ \frac{1}{\Im(\xi^{a}\xi_{a})} \ , 
\end{equation*}
%
for any $\xi \in \Tbb^\pm$ defined in \eqref{eq:cone_analytic}. Therefore by theorem 8.1.6 of Hörmander \cite{HORMANDER_1990} we have that 
%
\begin{equation}
\WF\left(\frac{1}{\sigma_\fsf^\alpha}\right) \subset  \Tbb^\pm \times \left( \Sigma \setminus \{0\} \right) \ ,
\label{eq:wf_invers_sigma_in_v+-}
\end{equation}
%
where
%
\begin{equation*}
\Sigma = \left\{ k \in \Rbb^4 \ \mbox{ such that } \ \sm{y,k} \geq 0 \ , \ \mbox{ with } \ y \in V^\pm_y \subset \Mcal \right\} \ . 
\end{equation*}
%
We recall the following relation for the wave front set of $\Hsf_\fsf$
%
\begin{equation*}
\WF(\Hsf_\fsf) \ \subset \ \WF(\Delta_+) \cup \WF(\Delta_\asf) \ \ , 
\end{equation*}
%
where
%
\begin{equation*}
\WF(\Delta_+) \ \subset \ \bigg\{ \bigg( x, y ; k_x, k_y \bigg) \in T^\ast\Msf^2 \setminus \{0\} \ \bigg| \ (x,k_x) \sim (x,-k_y), \ k_x \triangleright 0 \bigg\} \ .
\end{equation*}
% 
We notice that 
%
\begin{equation*}
\WF\left(\frac{1}{\sigma_\fsf^\alpha}\right) 
\ \subset \ 
\Tbb^\pm \times \left( \Sigma \setminus \{0\} \right) 
\ \subset \
\WF(\Delta_+)
\ \subset \
\WF(\Delta_\fsf) \ .
\end{equation*}
%
Using lemma \ref{lem:prop_wf} we have
%
\begin{equation*}
\WF(\Delta_\fsf^{(\alpha)}) 
\ \subset \ 
\WF\left(\frac{1}{\sigma_\fsf^\alpha}\right)
\ \Longrightarrow \
\WF(\Delta_\fsf^{(\alpha)}) 
\ \subset \ 
\WF(\Delta_\fsf) \ .
\end{equation*}
%
\end{description}
\end{proof}


We are now able to discuss the analytical regularization $\tsf^{(\alphabd)}_\gamma$ \eqref{eq:kernel_reg} of the distributions $\tsf_\gamma$ given in \eqref{eq:kernel} which appear in the graph expansion \eqref{eq:time_ordered_prod_graph} of the time ordered products $\Tcal_n$ \eqref{eq:time_ordered_op}. 


Owing to the form of $\Hsf^{(\alpha)}_\fsf$ given in \eqref{eq:feynman_reg} the relevant distributions which need to be discussed are $\Asf^{(\alphabd)}_\gamma$ introduced in \eqref{eq:amplitude_sigma_reg}. We analyze \eqref{eq:amplitude_sigma_reg} in the following proposition.


\begin{proposition}\label{prop:amplitude_sigma_prop_analyt}
Let consider $\Ncal$ a normal neighborhood of the union of all partial diagonals $D_n$. The operation
%
\begin{equation}
\sm{ \Asf_\gamma^{(\alphabd)} , \phi } = \int_{\Mcal^n} \ \dsf x_1 \ \dots \ \dsf x_n \ \prod_{(i,j)\in\gamma} \frac{1}{\sigma_{ij}^{\ell_{ij}(1+ \alpha_{ij})}} \ \phi(x_1,\dots,x_n) \ , 
\label{eq:amplitude_sigma_prop_analyt}
\end{equation}
%
for $\sigma_{ij}=\sigma_\fsf(x_i,x_j)$, $\ell_{ij} \in \Nbb$, $\alpha_{ij}\in\Omega\subset\Cbb$, and $\phi\in \Dcal(\Mcal^n\setminus D_n\cap \Ncal)$, has the following properties.
%
\begin{enumerate}
%
\item\label{item:1_amplitude_sigma_prop_analyt} $\Asf_\gamma^{(\alphabd)}$ is a distribution on $\Mcal^n\setminus D_n\cap \Ncal$.
%
\item\label{item:2_amplitude_sigma_prop_analyt} $\sm{ \Asf_\gamma^{(\alphabd)} , \phi }$ is a continuous function for $\alphabd \in \Cbb^{n(n-1)/2}$.
%
\item\label{item:3_amplitude_sigma_prop_analyt} $\sm{ \Asf_\gamma^{(\alphabd)} , \phi }$ is analytic for every $\alpha_{ij}$ and thus it is a multivariate analytic function.
%
\end{enumerate}
%
\end{proposition}


\begin{proof}
\begin{description}
\item The domain $\Mcal^n\setminus D_n\cap \Ncal$ is a disjoint union of connected components $\Ccal_k$
%
\begin{equation*}
\bigsqcup_k \Ccal_k = \Mcal^n\setminus D_n\cap \Ncal \ .
\end{equation*}
%
On every connected component $\Ccal_k$, $\sigma_\fsf(x_i,x_j)$ equals either $\sigma_+(x_i,x_j)$ or $\sigma_+(x_j,x_i)$ depending on the causal relation between $x_i$ and $x_j$ which is fixed in $\Ccal_k$. We shall use the notation $\sigma_\fsf(x_i,x_j)=\sigma_{ij}$.

%
%
\item[\ref{item:1_amplitude_sigma_prop_analyt}] The distribution $1/\sigma_{ij}$ is well defined on $\Ncal_2\setminus d_2$ where it coincides either with $1/\sigma_+$ or with $1/\sigma_-$. In order to analyze the wave front sets of $1/\sigma_\pm$, we pass to a normal coordinate system and obtain 
%
\begin{equation*}
\frac{1}{\sigma_\pm} = \frac{2}{\xi^a\xi_a\pm i\epsilon \xi^0} \ . 
\end{equation*}
%
This distribution can be extended to a tempered  distribution and thus its Fourier transform can be directly computed. One finds that for $1/\sigma_\pm$, only the null future/past directed directions do not decay rapidly. Hence, on $\Ccal_k$ the wave front set of $1/\sigma_{ij}$ is contained either in $\Vcal_+$ or $\Vcal_-$, where
%
\begin{equation*}
\Vcal_\pm = \bigg\{(x,y;k_x,k_y) \in T^\ast\Mcal^2\setminus\{0\} \ \left| \ (x,k_x) \sim (y,-k_y),  k_x \triangleleft / \triangleright 0 \bigg\} \right. \ .
\end{equation*}
%
We have been brief to proved the last point, but it has already been proved in the proof of proposition \ref{prop:wf_h_reg}, when we showed \eqref{eq:wf_invers_sigma_in_v+-}.\par%
%
Consequently, $1/\sigma_{ij}$ satisfies the micro local spectrum condition up to a permutation of the arguments. Due to proposition \ref{prop:sigma_1}, and also to the proof of proposition \ref{prop:wf_h_reg}, the same holds for the distributions
%
\begin{equation*}
\frac{1}{\sigma_{ij}^{\ell_{ij}(1+\alpha_{ij})}}
\end{equation*}
%
for $\ell_{ij} \in \Nbb$, $\alpha_{ij}\in\Omega\subset\Cbb$. Owing to the form of their wave front set, the pointwise products of these distributions present in $\Asf_\gamma^{(\alphabd)}$ \eqref{eq:amplitude_sigma_reg} are well defined because the Hörmander criterion for multiplication of distributions, collected in lemma \ref{lem:prod_distrib_wf}, is satisfied. In fact, up to some fixed permutation of the arguments $(x_1,\dots x_n)$, $\Asf_\gamma^{(\alphabd)}$ satisfies the micro local spectrum condition introduced in \ref{eq:wf_hadamard}. \par%
%
Hence $\Asf_\gamma^{(\alphabd)}$ is a well defined distribution on every connected component $\Ccal_k$ of $\Mcal^n\setminus D_n\cap \Ncal$ and thus it is well defined also on $\Mcal^n\setminus D_n\cap \Ncal$. 
%
%
\item[\ref{item:2_amplitude_sigma_prop_analyt}] In order to check continuity in $\alphabd = \{\alpha_{ij}\}_{(i,j)\in\gamma} \in \Cbb^{n(n-1)/2}$ of the map
%
\begin{equation*}
\alphabd \mapsto \sm{ \Asf_\gamma^{(\alphabd)} , \phi } \ ,
\end{equation*}
%
in a fixed point $\overline{\alphabd}$, we may analyze the distribution on a fixed connected component $\Ccal_k$ of the domain of $\Asf_\gamma^{(\alphabd)}$ and factorize the distribution in two parts.\par% 
%
In fact, due to the wave front set of $\Asf_\gamma^{(\alphabd)}$ on $\mathcal{C}$ the factorization 
%
\begin{equation*}
\Asf_\gamma^{(\alphabd)} = \Asf_\gamma^{(\overline{\alphabd})} \cdot \Asf_\gamma^{(\betabd)} 
\end{equation*}
%
is unique where the integral kernel of $\Asf_\gamma^{(\betabd)}$ is
%
\begin{equation*}
\prod_{1\leq i < j \leq n } \frac{1}{\sigma_{ij}^{\beta_{ij}}} \ .
\end{equation*}
%
For $\betabd$ in a sufficiently small neighborhood of $0$, $\Asf_\gamma^{(\betabd)}$ is an integrable function which is differentiable for $\betabd=0$ as can be obtained by dominated convergence. Finally, the continuity is preserved by pointwise multiplication with $\Asf_\gamma^{(\overline{\alphabd})}$.
%
%
\item[\ref{item:3_amplitude_sigma_prop_analyt}]
For an arbitrary but fixed pair of indices $i,j$, the complex parameter $\alpha_{ij}$ appears in the product displayed in \eqref{eq:amplitude_sigma_prop_analyt} as 
%
\begin{equation*}
\frac{1}{\sigma_{ij}^{\ell_{ij}(1+\alpha_{ij})}} 
\end{equation*}
%
and we have already analyzed the analyticity property of such a distribution in proposition \ref{prop:sigma_1}.\par 
We shall thus interpret $\Asf_\gamma^{(\alphabd)}$ as a composition of distributions, namely as 
%
\begin{equation*}
\frac{1}{\sigma_\fsf^{\alpha_{ij}}} \circ z \ , \quad \mbox{ with } \quad z : \Dcal(\Mcal^{n}\setminus D_{n}\cap \Ncal) \ \to \ \Dcal^\prime(\Mcal^{2}\setminus D_2\cap \Ncal_2) \ ,
\end{equation*}
%
for a suitable $\Ncal_2\supset D_2=d_2$. \par
%
The $\epsilon$--regularized integral kernel of $z$ corresponds to the product present in \eqref{eq:amplitude_sigma_prop_analyt} with the factor  $1/\sigma_\fsf^{\alpha_{ij}}$ removed. Because of the singular structure of $z$ 
%
\begin{equation*}
\sm{z,\phi} \ , \quad \forall \phi \in \Dcal\left(\Mcal^{n}\setminus D_{n}\cap \Ncal\right) \ ,  
\end{equation*}
%
is in fact a compactly supported smooth function supported on $\Mcal^{2}\setminus D_2\cap \Ncal_2$. Hence, the analysis of its composition with $1/\sigma_\fsf^{\alpha_{ij}}$ is straightforward. These considerations imply separate analyticity of $\Asf_\gamma^{(\alphabd)}$ in each $\alpha_{ij}$ whereas joint analyticity follows from the continuity proved in \ref{item:2_amplitude_sigma_prop_analyt}.
% 
\end{description}
\end{proof}


The weakly meromorphicity of $\Asf_\gamma^{(\alphabd)}$ is guaranteed by means of proposition \ref{prop:amplitude_sigma_prop_analyt}. It allows to apply the machinery developed using the generalized Euler operator to $\Asf_\gamma^{(\alphabd)}$, in order to apply the minimal subtraction procedure, i.e. to be able to explicitly subtract the principal part.


%----------------------------------------------------------------------------%
\subsection{The generalized Euler operator in practice}
\label{p:EULER_OP_PRACTICE}
%----------------------------------------------------------------------------%


In section \ref{p:EULER} we defined the generalized Euler operator, and give a recursive way to obtain the differential form of every Euler operators. We also analyzed the action of $\Esf_p$ on test functions. Here we shall look at a particular case. The distribution we shall consider $\Asf^{(\alphabd)}_\gamma$ \eqref{eq:amplitude_sigma_reg} is of the form
%
\begin{equation*}
\Asf_\gamma^{(\alphabd)}=\prod_{(i,j)\in\gamma} \frac{1}{\sigma_{ij}^{\ell_{ij}(1+ \alpha_{ij})}} 
\end{equation*}
%
and has scaling degree 
%
\begin{equation*}
\sd(\Asf_\gamma^{(\alphabd)}) = \sum_{(i,j)\in\gamma} 2 \ell_{ij}\left(1+ \Re(\alpha_{ij})\right) 
\end{equation*}
%
towards the thin diagonal $d_n$ furthermore, thanks to the form of the integral kernel of the distribution we have that the wave front set of $\Asf_\gamma^{(\alphabd)}$ is transversal to $d_n$. When we apply $\Esf^\dagger_1$, introduced in \eqref{eq:euler_operator}, to distribution $\Asf_\gamma^{(\alphabd)}$, we obtain
%
\begin{equation}
\Esf_1^\dagger  \Asf_\gamma^{(\alphabd)}(x_1,\dots,x_n) = - \left( 4(n-1) + \rho \right) \Asf_\gamma^{(\alphabd)}(x_1,\dots,x_n) \ .
\label{eq:euler_a}
\end{equation}
%
where we set
%
\begin{equation}
\rho = - \sum_{j=2}^n \sigma^a(x_j) \nabla^{x_j}_a \ .    
\label{eq:rho}
\end{equation}
%
We shall see now that the result \eqref{eq:euler_a} after applying $\Esf_1^\dagger$ to $\Asf_\gamma^{(\alphabd)}$ is equal to a term proportional to $\Asf_\gamma^{(\alphabd)}$ plus a remainder which has lower scaling degree. Hence proposition \ref{prop:set} implies that $\Asf_\gamma^{(\alphabd)}$ can be written as a homogeneous distribution plus a remainder with lower scaling degree. If the scaling degree of the remainder is not sufficiently low, we reiterate the procedure in order to obtain a full almost homogeneous expansion of the desired form.


\bigskip


In order to analyze this issue we shall only consider the relevant differential operator $\rho$ on $\Mcal^n$ appearing in $\Esf_1^\dagger$. We start by looking at the ``three dimensional case'', i.e.  $\Mcal^3$. We analyze the action of $\rho$ on $\sigma(x_2,x_3)$ for $x_2,x_3$ in a normal neighborhood $\Ncal_{x_1}$ of the point $x_1$.


\begin{lemma}\label{lem:rho_over_squared}
Let $\Ncal_{x_1}$ be a normal neighborhood of the point $x_1$ and let $x_2,x_3 \in \Ncal_{x_1}$. Then,
%
\begin{equation*}
\rho \sigma(x_2,x_3) = 2\sigma(x_2,x_3) + G(x_1,x_2,x_3) \ ,
\end{equation*}
%
where $G$ is a smooth function which vanishes in the limit $x_2,x_3 \to x_1$ as a monomial of order $4$ in the normal coordinates of $x_2$ and $x_3$ centered in $x_1$. 
\end{lemma}


\begin{proof}
For $x_1 \in \Mcal$ we fix a normal coordinate system as follow
%
\begin{equation*}
\xi_{x_1} : y \to \Rbb^4 
\end{equation*}
%
in order to parametrize points $y$ in a normal neighborhood of $x_1$. We then write the action of $\rho$ on $\sigma_{23}:=\sigma(x_2,x_3)$ as follow
%
\begin{equation*}
\rho \sigma_{23} = \xi_a(x_2) \sigma^a_{23} + \xi_{b^\prime}(x_3)\sigma^{b^\prime}_{23} \ . 
\end{equation*}
%
We recall that $\sigma_{23}^a$ is the covector in $T^*_{x_2}\Mcal$ cotangent to the unique geodesic joining $x_2$ and $x_3$, that $-\sigma^{b'}_{23}$ is equal to the parallel transport of $\sigma^a_{23}$ from $x_2$ to $x_3$ along the geodesic $\gamma$ joining the two points, and that $\xi^c(x_i):=\sigma^c(x_1,x_i)$.
%
Let us parametrize the image of $\gamma$ with an affine parameter $\lambda$ such that $x(0) = x_2$ and $x(1) = x_3$. In order to simplify the notation, we indicate by $t(\lambda)$ the tangent vector of the geodesic in $x(\lambda)$. As argued before, we have
%
\begin{equation*}
t^a(0)=\sigma^a_{23}\ , \quad \mbox{and} \qquad t^{b^\prime}(1)=-\sigma^{b^\prime}_{23} \ . 
\end{equation*}
%
Consequently,
%
\begin{eqnarray*}
\rho \sigma_{23} = \xi^a t_a (0) - \xi^b t_b(1) = - \int_{0}^{1} \frac{d}{d\lambda} (\xi^a t_a)(\lambda) d\lambda = \int_{0}^{1} t^a t^b \sigma_{ab}(x(\lambda),x_1) d\lambda \ ,
\end{eqnarray*}
where $\sigma_{ab} := \nabla_a\nabla_b \sigma$. If we now consider the covariant Taylor expansion of $\sigma_{ab}(x(\lambda),x_1)$ around $x(\lambda)$ (see e.g. \cite{PPV_2011}), we find that 
%
\begin{equation*}
\sigma_{ab}(x,x_1) - g_{ab}(x) := E_{ab}(x,x_1)
\end{equation*}
%
is a smooth function that vanishes for $x\to x_1$ as $O(\sigma(x,x_1))$, hence
%
\begin{equation*}
\rho\sigma_{23}  = \int_{0}^{1} t^a t^b g_{ab}(x(\lambda))    d\lambda +  \int_{0}^{1} t^a t^b E_{ab}(x(\lambda),x_1)    d\lambda  = 2\sigma(x_2,x_3) + G(x_1,x_2,x_3)\,,
\end{equation*}
%
where the remainder is smooth because of the smoothness of the metric $g$ and can be further expanded as 
%
\begin{eqnarray}
&& G(x_1,x_2,x_3) \ = \ 
\int_{0}^{1} t^a(\lambda) t^b(\lambda) \bigg( \sigma_{ab}\left(x(\lambda),x_1\right) - g_{ab}\left(x(\lambda)\right) \bigg) \ \dsf\lambda
\label{eq:remainder_sigma}
\\ 
&& = \ \int_{0}^{1} t^a(\lambda) \ t^b(\lambda) \ t^c(\lambda) \ t^d(\lambda) \ R_{acbd}\left(x(\lambda)\right) \ \dsf\lambda \ + \ \dots \ + \ \Ocal\left(\abs{\xi(x_2)}^4 + \abs{\xi(x_3)}^4\right) \ , \nonumber
\end{eqnarray}
where the absolute value of the normal coordinates $|\xi(x_i)|$ of $x_i$, $i=2,3$ is intended in the Euclidean sense.
\end{proof}



We are now in position to analyze the action of $\rho$ on the distribution $\Asf_\gamma^{(\alphabd)}$ introduced in \eqref{eq:amplitude_sigma_reg}. The distribution $\Asf_\gamma^{(\alphabd)}$ can be written as a sum of homogeneous distributions plus a remainder term with lower scaling degree which can be directly extended towards the total diagonal. Let us present precisely this result in the following proposition.


\begin{proposition}\label{prop:almost_homo}
The distribution $\Asf_\gamma^{(\alphabd)}$ introduced in \eqref{eq:amplitude_sigma_reg} can be written as a sum of homogeneous distributions plus a remainder towards the total diagonal $d_n$.
%
\begin{equation*}
\Asf_\gamma^{(\alphabd)} = \sum_{k=1}^m \ \Asf_{\gamma,k}^{(\alphabd)} + r_\gamma^{(\alphabd)} \ .
\end{equation*}
%
The degrees of homogeneity of the homogeneous distributions $\Asf_{\gamma,k}^{(\alphabd)}$ are contained in the following set
%
\begin{equation}
\left\{k-\sum_{(i,j)\in\gamma} 2 \ell_{ij}(1+ \alpha_{ij}) \ , \quad \mbox{with} \qquad k \in \Nbb \cup \{0\} \right\} \ .
\label{eq:deg_homo_ak}
\end{equation}
%
\end{proposition}


\begin{proof}
We perform this analysis with $\epsilon$ in $\sigma_\fsf$ taken to be strictly positive. We start by applying $\rho$ given in \eqref{eq:rho} to $\Asf_\gamma^{(\alphabd)}$. Thanks to the results stated in lemma \ref{lem:rho_over_squared} we have
%
\begin{equation*}
\rho \Asf_\gamma^{(\alphabd)} = C \Asf_\gamma^{(\alphabd)} + r^{(\alphabd)}_\gamma, 
\end{equation*}
%
where the constant $C$ is
%
\begin{equation*}
C = - \sum_{(i,j)\in\gamma} 2 \ell_{ij}(1+ \alpha_{ij}). 
\end{equation*}
%
Furthermore, lemma \ref{lem:rho_over_squared} and in particular \eqref{eq:remainder_sigma} imply that the remainder $r^{(\alphabd)}_\gamma$ has a scaling degree towards $d_n$ which is lower than the one of $\Asf_\gamma^{(\alphabd)}$ by at least two,
%
\begin{equation}\label{eq:sd_tgamma}
\sd\left(r^{(\alphabd)}_\gamma\right) \leq \sd\left(\Asf_\gamma^{(\alphabd)}\right) - 2 = \sum_{(i,j)\in\gamma} 2 \ell_{ij}\left(1+ \Re(\alpha_{ij})\right) \ - \ 2 \ .
\end{equation}
%
Proposition \ref{prop:set} then implies that the distribution $\Asf_\gamma^{(\alphabd)}$ can be written as a homogeneous distribution of degree $C$ plus a remainder with lower scaling degree.\par%
%
%
In order to finalize the proof we need to control the recursive application of $\rho$, therefore we discuss the application of $\rho$ on $\rho^n \Asf_\gamma^{(\alphabd)}$ for an arbitrary $n$.\par%
%
Let us start with $n=1$,
%
\begin{equation*}
\rho^2 \Asf_\gamma^{(\alphabd)} = C \rho \Asf_\gamma^{(\alphabd)} + \rho r^{(\alphabd)}_\gamma  = C^2 \Asf_\gamma^{(\alphabd)} + C r^{(\alphabd)}_\gamma + \rho r^{(\alphabd)}_\gamma \ .
\end{equation*}
%
In this case, we observe that the relevant contribution is the one given by the remainder $\rho r^{(\alphabd)}_\gamma$, which reads 
%
\begin{equation*}
r^{(\alphabd)}_\gamma = \sum_{(i,j)\in\gamma} \ell_{ij}(1+ \alpha_{ij}) \ \frac{G(x_1,x_i,x_j)}{\sigma_\fsf(x_i,x_j)} \ \Asf_\gamma^{(\alphabd)} \ .
\end{equation*}
%
Note that $\sigma_\fsf(x_i,x_j) \Asf_\gamma^{(\alphabd)}$ has the same structure as $\Asf_\gamma^{(\alphabd)}$, but with the scaling degree $\sd(\Asf_\gamma^{(\alphabd)}) + 2$. The term $G(x_1,x_i,x_j)$ defined in \eqref{eq:remainder_sigma} is a smooth function whose Taylor expansion for $x_i, x_j$ around $x_1$ starts with components of order $4$.\par
%
Hence, if we apply $\rho$ to the remainder $r^{(\alphabd)}_\gamma$ we obtain a constant multiple of $r^{(\alphabd)}_\gamma$ plus another remainder which has scaling degree lower or equal to $\sd(r^{(\alphabd)}_\gamma)-1$. The difference with respect to \eqref{eq:sd_tgamma} stems from the fact that $G$ can be expanded as a polynomial in $\sigma_a(x_i)$ whose lowest components are monomials of degree $4$ multiplied by curvature tensors. These monomials are homogeneous and thus contribute to the degree of homogeneity of $\rho r^{(\alphabd)}_\gamma$, while the contributions in $G$ with degree higher or equal to the scaling degree of the remainder. Repeating this analysis for a generic $n$, we find that similar results hold when $\rho$ is applied recursively to the remainder.\par
%
Consequently, an iterated application of proposition \ref{prop:set} implies that the distribution $\Asf_\gamma^{(\alphabd)}$ can be written as a finite sum of homogeneous distributions plus a remainder. Furthermore, since the scaling degree of these distributions is always finite, the degree of homogeneity of these components is finite as well. 
\end{proof}


The next step in the strategy outlined in section \ref{p:PROGRAM} is to extend the distributions $\Asf^{(\alphabd)}_\gamma$ defined in \eqref{eq:amplitude_sigma_reg} in a unique and weakly meromorphic fashion to a normal neighborhood of the union of all partial diagonal $D_n$. 


Therefore we use proposition \ref{prop:almost_homo} to write the distribution $\Asf_\gamma^{(\alphabd)}$ as a sum of homogeneous distributions of degree contained in \eqref{eq:deg_homo_ak} plus a remainder towards the total diagonal $d_n$
%
\begin{equation*}
\Asf_\gamma^{(\alphabd)} = \sum_{k=1}^m \ \Asf_{\gamma,k}^{(\alphabd)} + r_\gamma^{(\alphabd)} \ .
\end{equation*}
%
Then proposition \ref{prop:regularization} tells us $\Asf_\gamma^{(\alphabd)}$ can be extended to the full space for every $\alphabd$, and that the extension is meromorphic in $\alphabd$ with possible poles of finite order which are supported on the set of all partial diagonals, and thus that the minimal subtraction scheme introduced in theorem \ref{theo:renorm_t_prod_ms_forest} can be applied where we shall before isolate the poles using the procedure given by proposition \ref{prop:expose_poles}. 


To this avail, we stress that proposition \ref{prop:almost_homo} holds in particular for any subgraph $\gamma_I$, with $I\subset\{1,\dots,n\}$ of $\gamma$ and the corresponding distribution $\Asf_{\gamma_I}^{(\alphabd)}$ which is obtained by omitting all factors in $\Asf_{\gamma}^{(\alphabd)}$ which correspond to edges not contained in $\gamma_I$. Finally, the recursive structure of the forest formula in theorem \ref{theo:renorm_t_prod_ms_forest} implies that we are not dealing only with 
expressions of the form $\left.\Asf_{\gamma_I}^{(\alphabd)}\right|_{\alpha_{ij}=\alpha_I}, \  \forall i,j\in I$, but also with expressions which are of this form up to a subtraction of their principal part. However, our above analysis and in particular the discussion in the proof of proposition \ref{prop:regularization} implies that the propositions \ref{prop:almost_homo} and \ref{prop:expose_poles} also hold in this case.


%----------------------------------------------------------------------------%
\section{Properties of the scheme}
\label{p:PROP_SCHEME}
%----------------------------------------------------------------------------%


We conclude the general analysis of the regularization scheme introduced in this work by demonstrating that this scheme satisfies (up to one property we shall mention at the end of this section) all axioms of \cite{HW_2001,HW_2002,HW_2005} that, as argued by Hollands and Wald, any physically meaningful scheme to regularize time ordered products should satisfy. These axioms are recalled in section \ref{p:INT_Q_ALG}. In addition to showing these properties of the scheme, we also argue that it preserves invariance under any spacetime isometries present.



\begin{proposition}\label{prop:properties_scheme} 
The time ordered products
%
\begin{equation*}
\Tcal_n \ : \ 
\left\{
\begin{array}{lcl}
\Fcal_{\loc}(\Mcal)[[\hbar]]^{\otimes n} & \to & \Fcal_{\muc}(\Mcal)[[\hbar]] \\
\Fsf_1(\phi) \otimes \ ... \ \otimes \Fsf_n(\phi) & \mapsto & \Fsf_1(\phi) \cdot_{\Tsf} \ ... \ \cdot_{\Tsf} \Fsf_n(\phi)
\end{array}
\right. \ ,
\end{equation*}
%
defined by means of the relation in theorem \ref{theo:renorm_t_prod_ms_forest} have the following properties.
%
\begin{enumerate}
%
\item\label{item:1_properties_scheme} $\Tcal_n$ is symmetric and satisfies the causal factorization condition.
%
\item\label{item:2_properties_scheme} $\Tcal_n$ is unitary.
%
\item\label{item:3_properties_scheme} $\Tcal_n$ is local and covariant.
%
\item\label{item:4_properties_scheme} $\Tcal_n$ satisfies the microlocal spectrum condition.
%
\item\label{item:5_properties_scheme} $\Tcal_n$ is $\phi$ independent.
%
\item\label{item:6_properties_scheme} $\Tcal_n$ satisfies the Leibniz rule.
%
\item\label{item:7_properties_scheme} $\Tcal_n$ satisfies the Principle of Perturbative Agreement for perturbations of the generalized mass term $\mu$ in the free Klein Gordon equation
%
\begin{equation*}
\Psf \phi = \left( - \Box + \mu \right) \phi = 0 \ . 
\end{equation*}
%
\item\label{item:8_properties_scheme} If the spacetime $\Mcal$ has non trivial isometries and if the Feynman propagator $\Hsf_\fsf$ is chosen such as to be invariant under these isometries, then $\Tcal_n$ is invariant under these isometries as well.
%
\end{enumerate}
%
\end{proposition}


\begin{proof}
\begin{description}
\item[\ref{item:1_properties_scheme}] The condition holds because we constructed the renormalized time--ordered product by means of the forest formula \eqref{eq:ms_t_forest} and because, as implied by proposition \ref{prop:regularization}, all counter terms subtracted in the forest formula are local.
%
%
%
%
\item[\ref{item:2_properties_scheme}] Unitarity holds because the operation of extracting the relevant principal part of a regularized amplitude $\tsf^{(\alphabd)}_\gamma$ commutes with complex conjugation (even if $\alphabd$ is not real).
%
%
%
%
\item[\ref{item:3_properties_scheme}] The regularized amplitudes $\tsf^{(\alphabd)}_\gamma$ satisfy locality and covariance. Upon setting $\alpha_{ij}=\alpha_I$ for $i,j\in I\subset \{1,\dots,n\}$, $\tsf^{(\alphabd)}_\gamma$ is weakly meromorphic in $\alpha_I$. Thus locality and covariance holds for each term in the corresponding Laurent series and consequently also after subtracting the principal part of this series.
%
%
%
%
\item[\ref{item:4_properties_scheme}] As argued in the proof of proposition \ref{prop:amplitude_sigma_prop_analyt}, the distributions $\tsf^{(\alphabd)}_\gamma$ satisfy the microlocal spectrum condition. Consequently, the regularized amplitudes $\Asf^{(\alphabd)}_\gamma$ have the correct wave front set as well. As $\tsf^{(\alphabd)}_\gamma$ is weakly meromorphic in the sense recalled in the proof of \ref{item:3_properties_scheme}, each term in the corresponding Laurent series has a wave front set bounded by the wave front set of $\tsf^{(\alphabd)}_\gamma$. Consequently the microlocal spectrum condition holds after subtracting the principal part and considering the limit of vanishing regularization parameters.
%
%
%
%
\item[\ref{item:5_properties_scheme}] This property follows directly from the construction. In particular the subtraction of counter terms is defined in terms of numerical distributions and independent of the field $\phi$.
%
%
%
%
\item[\ref{item:6_properties_scheme}] In analogy to \ref{item:2_properties_scheme}, the Leibniz rule holds because the operation of extracting the relevant principal part of a regularized amplitude $\tsf^{(\alphabd)}_\gamma$ commutes with all partial differential operators.
%
%
%
%
\item[\ref{item:7_properties_scheme}] The Principle of Perturbative Agreement for perturbations of the generalized mass term $\mu$ demands essentially that upon setting $\mu=\mu_0 + \mu_1$, the renormalization of $\Tcal_n$ commutes with the operation of perturbatively expanding quantities in $\mu_1$ around $\mu_0$. A Feynman propagator $\Hsf_\fsf$ depends on $\mu$ only via the Hadamard coefficients $v$ and $w$ in \eqref{eq:feynman}. However, in the definition of the analytically regularized $\Hsf^{\alpha}_\fsf$ in \eqref{eq:feynman_reg} and the corresponding regularized amplitudes $\Asf^{(\alphabd)}_\gamma$, these coefficients are not altered but only the $\sigma$--dependent terms multiplying these coefficients are modified. Consequently, the analytic regularization and minimal subtraction scheme we consider commutes with a perturbative expansion in $\mu_1$ around $\mu_0$.
%
%
%
%
\item[\ref{item:8_properties_scheme}] As recalled in \ref{item:7_properties_scheme} all operations in our analytic regularization and minimal subtraction scheme act directly on quantities defined entirely in terms of the geometric quantity $\sigma$. As $\sigma$ is invariant under any spacetime isometries present, the renormalization scheme preserves this invariance.
%
%
%
%
\end{description}
\end{proof}


Note that the Principle of Perturbative Agreement (PPA) as introduced in \cite{HW_2005} also poses conditions on $\Tcal_1$, i.e. the regularization of local and covariant Wick polynomials, which we omitted in our analysis. However, given $\Tcal_n$ for $n>1$, $\Tcal_1$ can be adjusted in order to satisfy the PPA for changes of $\mu$ by using e.g. \cite[Theorem 3.3]{DHP_2015}. Moreover, the PPA as introduced in \cite{HW_2005} further demands that, setting $g = g_0 + g_1$, the regularization also commutes with perturbatively expanding quantities in $g_1$ around an arbitrary but fixed background metric $g_0$. Since $\sigma$ depends on $g$, it is not easy to check whether a perturbative expansion in $g_1$ commutes with our analytic regularization and minimal subtraction scheme and thus it might well be that the regularization scheme discussed in the present work fails to satisfy this part of the PPA. However, if this is the case, the scheme can be modified according to the construction in \cite{HW_2005} in order to satisfy also this condition while preserving the other properties in proposition \ref{prop:properties_scheme}, including the invariance under any spacetime isometries present.


\bigskip


We have omitted the explicit dependence of regularized quantities on the mass scale $M$ appearing in the analytically regularized Feynman propagator $\Hsf^{(\alpha)}_\fsf$ \eqref{eq:feynman_reg}, but our analysis implies that the dependence of these quantities on $M$ is such that all regularized quantities are polynomials of (derivatives of) $\log\left( M^2 \sigma_\fsf(x_i,x_j)\right)$, see also the examples in the next section. Thus, the regularization group flow with respect to changes of $M$ may be computed.


%----------------------------------------------------------------------------%
\chapter{Computations in our scheme}
\label{p:EXOS}
%----------------------------------------------------------------------------%


After having defined a regularization procedure in the previous chapter, we shall perform explicit computations using our regularization scheme, first on generic curved spacetime as defined in \ref{def:cst}, and then on cosmological spacetime, in particular on spatially flat Friedmann-Lemaître-Robertson-Walker spacetime, that we presented in section \ref{p:FLRW}.


%----------------------------------------------------------------------------%
\section{Examples on generic curved spacetime}
%----------------------------------------------------------------------------%


In this section we illustrate the method developed in the previous chapter to explicitly compute regularized quantities in our scheme by considering first the example of the fish graph and the sunset graph, i.e. $\Delta^n_\fsf$ for $n=2,3$. These pointwise powers of the Feynman propagator are the only ones occurring in renormalizable scalar field theories in four spacetime dimensions. Afterwards we will consider a triangular graph in section \ref{p:COMPLICATED_GRAPH} in order to illustrate the method in the case of more than two vertices. We shall work only on subsets of the spacetime where the geodesic distance is well defined without loss of generality, cf section \ref{p:GLOB_TPROD}.


%----------------------------------------------------------------------------%
\subsection{The regularized fish and sunset graphs}
%----------------------------------------------------------------------------%


In the special case of $\Delta^n_\fsf$, we are dealing with distributions which are already defined on $\Mcal^2\setminus d_2$ and have to be extended to $\Mcal^2$. In order to accomplish this task we shall use the relation \eqref{eq:expose_poles} in order to expose the poles before subtracting them. In this context, we note that $\Esf^\dagger_1$ given in \eqref{eq:euler_operator} applied to a distribution $u$ whose integral kernel $u(\sigma_\fsf)$ depends on $x,y$  only via $\sigma(x,y)$, can be further simplified. In particular, introducing $u_1(\sigma_\fsf)$ such that $\nabla^a u_1(\sigma_\fsf) = \sigma^a u(\sigma_\fsf)$, we have
%
\begin{eqnarray}
\Esf_1^\dagger u(\sigma_\fsf) &=& -\left( 4 + \sigma^a\nabla_a \right) u(\sigma_\fsf) \nonumber \\
&=& - \nabla_a \sigma^a \ u - 2 \sigma^a (\nabla_a \log (u)) \ u(\sigma_\fsf) \nonumber \\ 
&=& - \Box u_1(\sigma_\fsf) - 2 \frac{\nabla_a \usf}{\usf} \ \nabla^a u_1(\sigma_\fsf) \ , 
\label{eq:E_simplified}
\end{eqnarray}
%
where $x$ is considered to be arbitrary but fixed, and all the covariant derivatives are taken with respect to $y$.


%----------------------------------------------------------------------------%
\subsubsection{Standard approach}
%----------------------------------------------------------------------------%


We recall that the Feynman propagator $\Delta_\fsf(x,y)$ admit the Hadamard representation $\Hsf_\fsf(x,y)$ introduced in \eqref{eq:feynman},
%
\begin{equation*}
\Hsf_\fsf(x,y) = \frac{1}{8\pi^2}  \lim_{\epsilon \downarrow 0} \left( \frac{\usf(x,y)}{\sigma_\fsf(x,y)} + v(x,y) \ \log\left( M^2 \sigma_\fsf(x,y)\right) \right) + w(x,y) \ .
\end{equation*}




\begin{figure}[ht!]
\begin{center}
\FnG
\end{center}
\caption{Graphs with two vertices}
\end{figure}


From \eqref{eq:feynman} we can infer that, in order to regularize $\Hsf^2_\fsf$ and $\Hsf^3_\fsf$, i.e. in order to extend them from respectively $\Mcal^2 \setminus d_2$ to $\Mcal^2$, $\Mcal^6 \setminus d_6$, to $\Mcal^6$, we need to regularize the three following distributions
%
\begin{equation}
\frac{1}{\sigma_\fsf^2} \ , \qquad \frac{\log \left(M^2 \sigma_\fsf\right)}{\sigma_\fsf^2} \ , \qquad \frac{1}{\sigma_\fsf^3} \ ,
\label{eq:sigma_problematic}
\end{equation}
%
because all other occurring powers of $\sigma_\fsf$, i.e. $\sigma^{-m}_\fsf\log^n(\sigma_\fsf)$ for $m\in\{0,1\}$ and $n\in\{0,1,2,3\}$ have a scaling degree for $y\to x$ smaller than 4, and thus can be uniquely extended to the diagonal. To this avail, we introduce the following notation
%
\begin{equation*}
\sigma_{a_1\cdots a_n} = \nabla_{a_n} \cdots \nabla_{a_1} \sigma \ , \qquad [B](x) = B(x,x) \ , 
\end{equation*}
%
where the covariant derivatives are taken with respect to $x$ and $B$ is a generic bitensor. We recall the following identities satisfied by $\sigma$ 
%
\begin{equation}
\sigma_a \sigma^a = 2 \sigma \ , \qquad 
\sigma_{ab} \sigma^b = \sigma_a \ , \qquad 
\Box \sigma = 4 - 2 \frac{\sigma^a \nabla_a u}{u} \ .
\label{eq:sigma_identities}
\end{equation}
%
For our purposes, it will prove useful to use the last identity in the form
%
\begin{equation*}
\Box \sigma_\fsf = 4 + f \sigma_\fsf \ , \qquad
\mbox{ with } \qquad f = - 2 \frac{\sigma^a \nabla_a u}{u \ \sigma_\fsf} \ ,
\label{eq:def_f_sigma}
\end{equation*}
%
where $f$ is a distribution, which considered as a distribution in $y$ for fixed $x$, has scaling degree zero for $y \to x$ as can be seen from the covariant Taylor expansion
%
\begin{equation*}
u = [u] + \bigg( [\nabla_a u] - \nabla_a [u] \bigg) \sigma^a + \Rcal_u = 1 + \Rcal_u \ , 
\end{equation*}
%
where the remainder $\Rcal_u$ vanishes towards the diagonal faster than $\sigma_a$  (see e.g. \cite{PPV_2011}).


\begin{remark}\label{rem:fdists}
As $f$ has vanishing scaling degree for $y \to x$, the pointwise product 
%
\begin{equation*}
f(x,y) u(x,y) 
\end{equation*}
%
with any bidistribution $t$ of scaling degree for $y\to x$ lower than $4$ may be uniquely extended to the diagonal. However, we will also encounter expressions which are naively of the form 
%
\begin{equation}
f(x,y) \delta(x,y) 
\label{eq:f_delta}
\end{equation}
%
and which are a priory ill defined because $f$ is in general divergent for $x$ and $y$ light like related, and thus not continuous on the diagonal. Notwithstanding, the distribution \eqref{eq:f_delta}, which is well defined and identically vanishing outside of the diagonal $x=y$, may be extended to the diagonal. In fact, our scheme, in which expressions of the form \eqref{eq:f_delta} appear as $\alpha \to 0$ limits of particular weakly analytic expressions, provides a unique and non vanishing extension of \eqref{eq:f_delta} to the diagonal by the very analyticity of the aforementioned expressions. In particular our scheme implies the following unique and well defined definitions of distributions on $\Mcal^2$.
%
\begin{equation}
f \ \Box \left( \frac{\log^n (M^2 \sigma_\fsf) }{\sigma_\fsf} \right) = \lim_{\alpha \to 0} \ f \ \Box \left(\frac{\log^n(M^2 \sigma_\fsf)}{\sigma^{1+\alpha}_\fsf} \right) \ , \ \ n \geq 0 \ ,
\label{eq:f_dists}
\end{equation}
%
Hereby uniqueness and weak analyticity of 
%
\begin{equation*}
f \ \Box\left( \frac{ \log^n(M^2 \sigma_\fsf) }{ \sigma^{1+\alpha}_\fsf } \right) 
\end{equation*}
%
follow from arguments used throughout previous chapter.
\end{remark}


From proposition \ref{prop:sigma_1}, we know that the distribution
%
\begin{equation*}
\frac{1}{\sigma^{n+\alpha}_\fsf} 
\end{equation*}
%
is weakly meromorphic in $\alpha$. In order to compute the Laurent series, we use the above mentioned identities for $\sigma$ and obtain
%
\begin{equation*}
\frac{1}{\sigma^{n+1+\alpha}_\fsf} = \frac{1}{2(n+\alpha)(n-1+\alpha)} \left(\Box+(n+\alpha)f\right) \frac{1}{\sigma^{n+\alpha}_\fsf} \ ,
\end{equation*}
%
in accordance with \eqref{eq:expose_poles} and \eqref{eq:E_simplified}. Using this, we may compute the following Laurent series, where we recall that in $\Hsf^{(\alpha)}_\fsf$ \eqref{eq:feynman} we use the same (arbitrary) constant $M$ present in the logarithmic term  of \eqref{eq:hadamard_rep} to correct for the change of dimension and a sufficiently regular function $k$ for later purposes,
%
\begin{eqnarray}
\frac{1}{(Mk)^{2\alpha}} \ \frac{1}{\sigma^{2+\alpha}_F} &=& \frac12 \left(\Box+f\right) \left( \frac{1}{\alpha \ \sigma_\fsf} \ - \ \frac{\log\left(M^2 \sigma_\fsf\right)}{\sigma_\fsf}\right) \ - \ \frac{\log(k^2)}{2} \ \left(\Box+f\right) \ \frac{1}{\sigma_\fsf} \nonumber \\
&& - \ \Box \ \frac{1}{2\sigma_\fsf} \ + \ \Ocal(\alpha) \ , \nonumber \\
%
%
\frac{d}{d\alpha} \left( \frac{1}{(Mk)^{2\alpha}} \ \frac{1}{\sigma^{2+\alpha}_\fsf} \right) &=& \frac12 \left(\Box+f\right) \left(-\frac{1}{\alpha^2 \ \sigma_\fsf} \ + \ \frac{\log^2\left(M^2 \sigma_\fsf\right)}{2 \ \sigma_\fsf}\right)  \nonumber \\
&& + \ \Box\left(\frac{\log\left(M^2 \sigma_\fsf\right)+1}{2 \ \sigma_\fsf}\right) \ + \ \log^2(k^2) \ \left(\Box+f\right) \ \frac{1}{4\sigma_\fsf} \nonumber \\
&& + \ \log(k^2) \ \left(\Box\frac{1}{2 \ \sigma_\fsf} \ + \ \left(\Box+f\right) \ \frac{\log\left(M^2 \sigma_\fsf\right)}{2 \ \sigma_\fsf}\right) \ + \ \Ocal(\alpha) \ , \nonumber \\
%
%
\frac{1}{(M h)^{2\alpha}} \ \frac{1}{\sigma^{3+\alpha}_\fsf} &=& \frac18 \left(\Box+2f\right) \ \left(\Box+f\right) \ \left(\frac{1}{\alpha\sigma_\fsf} \ - \ \frac{\log \left(M^2 \sigma_\fsf\right)}{\sigma_\fsf}\right) \nonumber \\
&& - \ \frac{\log(k^2)}{8} \ (\Box+2f) \ (\Box+f) \ \frac{1}{\sigma_\fsf} \nonumber \\
&& - \ \frac{1}{16} \ \bigg( (5\Box+8f) \ (\Box+f) \ - \ 2 (\Box+2f) \ f \bigg) \ \frac{1}{\sigma_\fsf} \ + \ \Ocal(\alpha) \ . \nonumber \\
\label{eq:general_expansion}
\end{eqnarray}
%
Note that by means of lemma \ref{lem:product_identities} one may explicitly check that the pole terms in these Laurent series are local expressions as expected.


Using the Laurent series, the lowest regularized powers of $\sigma_\fsf$ may be defined and computed as\footnote{Note that we use here a definition of the analytic regularization of the logarithm in terms of a direct derivative rather than a limit of differences like in \eqref{eq:hadamard_rep_reg}. While the two definitions differ up to a constant factor in the principal part, they coincide in the constant regular part and thus give the same $(\sigma^{-2}_\fsf \log(M^2 \sigma_\fsf))_\ms$.}
%
\begin{eqnarray}
\left(\frac{1}{\sigma_\fsf^2}\right)_\ms &=& \lim_{\alpha\to 0} \left(\frac{1}{M^{2\alpha}} \frac{1}{\sigma^{2+\alpha}_\fsf} \ - \ \pp\left(\frac{1}{M^{2\alpha}} \frac{1}{\sigma^{2+\alpha}_\fsf}\right)\right) \nonumber \\
&=& - \frac12 (\Box+f) \ \frac{\log \left(M^2 \sigma_\fsf\right)}{\sigma_\fsf} \ - \ \Box\left(\frac{1}{2\sigma_\fsf}\right) \ , \nonumber \\
%
\left(\frac{\log\left(M^2\sigma_\fsf\right)}{\sigma_\fsf^2}\right)_\ms &=& - \ \lim_{\alpha\to 0} \Bigg( \frac{d}{d\alpha}\left(\frac{1}{M^{2\alpha}} \ \frac{1}{\sigma^{2+\alpha}_\fsf}\right) \ - \ \pp\frac{d}{d\alpha} \left( \frac{1}{M^{2\alpha}} \ \frac{1}{\sigma^{2+\alpha}_\fsf}\right) \Bigg) \nonumber \\ 
&=& - \ \frac14 \left(\Box+f\right) \ \frac{\log^2\left(M^2 \sigma_\fsf\right)}{\sigma_\fsf} \ - \ \Box\left(\frac{\log \left(M^2 \sigma_\fsf\right)+1}{2\sigma_\fsf}\right) \ , \nonumber \\
%
\left(\frac{1}{\sigma_\fsf^3}\right)_\ms &=& \lim_{\alpha\to 0} \left( \frac{1}{M^{2\alpha}} \ \frac{1}{\sigma^{3+\alpha}_\fsf} \ - \ \pp\left(\frac{1}{M^{2\alpha}} \ \frac{1}{\sigma^{3+\alpha}_\fsf} \right)\right) \ , \nonumber \\
&=& - \ \frac18 (\Box+2f) (\Box+f) \ \frac{\log\left(M^2 \sigma_\fsf\right)}{\sigma_\fsf} \nonumber \\ 
&& - \ \frac{1}{16} \bigg( (5\Box+8f)(\Box+f) \ - \ 2(\Box+2f)f \bigg) \ \frac{1}{\sigma_\fsf} \ . \nonumber \\
\label{eq:sigma_ms}
\end{eqnarray}
%
Finally $\left(\Hsf^2_\fsf\right)_\ms$ and $\left(\Hsf^3_\fsf\right)_\ms$ are defined and computed by expanding the unregularized powers $\Hsf^2_\fsf$ and $\Hsf^3_\fsf$ and replacing the three problematic expressions \eqref{eq:sigma_problematic} by their regularized versions \eqref{eq:sigma_ms}.


%----------------------------------------------------------------------------%
\subsubsection{Alternative computation}
%----------------------------------------------------------------------------%


As a preparation towards the application of our regularization scheme to \textbf{QFT} in cosmological spacetimes, we shall now derive an alternative way to compute $\left(\Hsf_\fsf^2\right)_\ms$ and $\left(\Hsf_\fsf^3\right)_\ms$ which is better suited for practical computations. We start by stating and proving a few distributional identities.


\begin{lemma}\label{lem:product_identities}
The following distributional identities hold.
%
\begin{enumerate}
\item\label{item:1_product_identities} For any continuous $F_0$ and any twice continuously differentiable $F_2$
%
\begin{eqnarray*}
&& \sigma F_0 \delta = 0 \ , \qquad \sigma_a F_0 \delta = 0 \ , \qquad F_0 \nabla_{\nabla\sigma} \delta = - [F_0 \Box \sigma] \delta \ ,  \\
&& F_2 \Box \delta \ = \ [\Box F_2]\delta \ + \ \Box [F_2]\delta \ - \ 2\nabla^a[\nabla_aF_2]\delta \ . 
\end{eqnarray*}
%
%
\item\label{item:2_product_identities} $(\Box+f)\dfrac{1}{\sigma_\fsf} \ = \ 8\pi^2i\delta$, \ and \ \ $(\Box+2f)(\Box+f)\dfrac{1}{\sigma_\fsf} \ = \ 8\pi^2i \left(\Box-\dfrac{R}{3}\right)\delta \ . $
%
%
\item\label{item:3_product_identities} For all $n_1, n_2, n_3 \in \Nbb_0$ and $n_4, n_5, n_6 \in \{0,1\}$ with $n_2-n_3+n_4 \geq - 1$,
%
\begin{eqnarray*}
&& \log^{n_1}\left(\sigma_\fsf\right) \ (\sigma^a_\fsf)^{n_4} \ \sigma^{n_2}_\fsf \ \left(\frac{1}{\sigma_\fsf^{n_3}}\right)_\ms \ = \ \log^{n_1}\left(\sigma_\fsf\right) \ (\sigma^a_\fsf)^{n_4} \ \sigma^{n_2-n_3}_\fsf \ , \\
&& \Box \log(\sigma_\fsf) \ = \ \frac{\Box \sigma -2}{\sigma_\fsf} \ , \\
&& \nabla_a\left(\frac{\log^{n_5}(\sigma_\fsf)}{\sigma^{n_6}_\fsf}\right) \ = \ \frac{\left(n_5-n_6\log^{n_5}(\sigma_\fsf)\right) \ \nabla_a \sigma}{\sigma^{n_6+1}_\fsf} \ . 
\end{eqnarray*}

\item\label{item:4_product_identities} $\sigma_\fsf\left(\dfrac{1}{\sigma_\fsf^3}\right)_\ms \ = \ \left(\dfrac{1}{\sigma_\fsf^2}\right)_\ms$.
%
%
\end{enumerate}
%
\end{lemma}


\begin{proof}
\begin{description}
\item[\ref{item:1_product_identities}] These identities follow from $B\delta=[B]\delta$ for any continuous bitensor $B$, $[\sigma]=0$, $[\sigma_a]=0$ and the definition of weak derivatives.
%
%
\item[\ref{item:2_product_identities}] The first identity holds in Minkowski spacetime because $1/(8\pi^2\sigma_\fsf)$ is the Feynman propagator of the massless vacuum state. In curved spacetimes \eqref{eq:sigma_identities} imply that
%
\begin{equation}
(\Box+f)\frac{1}{\sigma_\fsf}
\end{equation}
%
vanishes outside of the origin and thus must be a sum of derivatives of $\delta$ distributions. Because $\sigma$ depends smoothly on the metric, the coefficients in this sum must be smooth functions of the metric with appropriate mass dimension and thus $(\Box+f)1/\sigma_\fsf=c\delta$ with a constant $c$ that can be fixed in Minkowski spacetime.\par%
%
In order to prove the second identity we recall remark \ref{rem:fdists} and observe that it is sufficient to compute
%
\begin{equation*}
u:=\lim_{\alpha\to 0}f(\Box+f)\frac{1}{\sigma^{1+\alpha}_F} 
\end{equation*}
%
This expression has for $y\to x$ a scaling degree $\leq 4$, vanishes outside of $x=y$, depends smoothly on the metric, is covariant and has mass dimension $6$. Consequently $u=cR\delta$ where the dimensionless constant $c$ can be computed on any spacetime with $R\neq 0$. Moreover, in any spacetime where $f$ is actually continuous in a neighborhood of the diagonal we have $u(x,y)=8\pi i f(x,x) \delta(x,y)$. A spacetime which satisfies both properties is (the patch of) de Sitter spacetime defined in conformal coordinates by the metric line element 
%
\begin{equation*}
ds^2=\frac{1}{H^2\tau^2}\left(-d\tau^2 + d\vec{x}^2\right) 
\end{equation*}
%
on $(-\infty,0)\times\Rbb^3$, where $H$ is a constant. On this spacetime we have $R=12H^2$ and 
%
\begin{equation*}
\mu^2:=2 H^2 \sigma(\tau_1,\vec{x}_1,\tau_2,\vec{x}_2)=\cos^{-1}\left(\frac{\tau^2_1+\tau^2_2-(\vec{x}_1-\vec{x}_2)^2}{2\tau_1\tau_2}\right), 
\end{equation*}
%
see e.g. \cite{ALLEN_1985}, where analytic continuation of $\cos^{-1}$ is understood for time--like separations. From this one can infer 
%
$$
\Box \sigma = 1+3 \mu \cot (\mu) \qquad \Rightarrow \qquad f = \frac{\Box \sigma-4}{\sigma} = 6H^2 \frac{\mu \cot (\mu) - 1}{\mu^2}= -\frac{R}{6} + O(\mu^2)
$$
%
which demonstrates that on de Sitter spacetime $f$ is continuous in a neighborhood of the diagonal with $f(x,x)=-R/6$. 

\item[\ref{item:3_product_identities}]The distributions on both sides of each equation, considered as distributions in $y$ for fixed $x$, have the same scaling degree $<4$ for $y\to x$ and agree outside of the diagonal. Thus they agree also on the diagonal as unique extensions.

\item[\ref{item:4_product_identities}] As in the proof of \ref{item:1_product_identities} we observe that the potential local correction term on the right hand side must be a sum of derivatives of $\delta$ with coefficients that depend smoothly on the metric because $\sigma$ does. Thus the correction term must be of the form $c\delta$ with a constant $c$ that can be computed in Minkowski spacetime. This computation may be performed by using \eqref{eq:sigma_identities}, the previous statements of this lemma, and the following identities which are valid in Minkowski spacetime for any function $F$ such that $F(\sigma_\fsf)$ is a distribution
%
\begin{eqnarray}
&& \sigma_\fsf\Box  F(\sigma_\fsf)=\Box \sigma_\fsf F(\sigma_\fsf) - 4 F(\sigma_\fsf) - 2\nabla_{\nabla\sigma_\fsf}F(\sigma_\fsf) \ , \\
&& \sigma_\fsf\Box^2  F(\sigma_\fsf)=\Box^2 \sigma_\fsf F(\sigma_\fsf) - 4 \Box F(\sigma_\fsf) - 4\Box \nabla_{\nabla\sigma_\fsf}F(\sigma_\fsf) \ ,
\end{eqnarray}
%
whereby one finds that $c=0$.
\end{description}
\end{proof}


These identities can be used to compute $\left(\Hsf_\fsf^2\right)_\ms$ and $\left(\Hsf_\fsf^3\right)_\ms$ in an alternative way under certain conditions.


\begin{proposition}\label{prop:equivalent_scheme}
Let $\Ncal$ be a normal neighborhood in $\Mcal$ and let $\Hsf_\fsf$ be a distribution on $\Ncal \times \Ncal$ of Feynman Hadamard form \eqref{eq:hadamard_rep}. Then the following identities hold.
%
%
\begin{eqnarray}
%
&& \hspace*{-18pt} \mbox{If } \ \Hsf_\fsf^{\alpha} \ \mbox{ is a well defined distribution which is weakly meromorphic in } \ \alpha \ , \mbox{ then } \nonumber \\
%
&& 1. \hspace*{8pt} (\Hsf_\fsf^2)_\ms = \lim_{\alpha\to 0} \left( \frac{1}{M^{2\alpha}} \ \Hsf_\fsf^{2+\alpha} - \pp\left(\frac{1}{M^{2\alpha}} \ \Hsf_\fsf^{2+\alpha} \right) \right) + \frac{i\log(8\pi^2)}{16\pi^2} \delta \ , \label{eq:item_1} \\
%
&& 2. \hspace*{8pt} \left(\Hsf_\fsf^2 \ \log \left(M^{-2}\Hsf_\fsf\right)\right)_\ms = \lim_{\alpha\to 0} \Bigg(\frac{d}{d\alpha}\left(\frac{1}{M^{2\alpha}}(\Hsf_\fsf)^{2+\alpha}\right) - \pp \ \frac{d}{d\alpha}\left(\frac{1}{M^{2\alpha}}(\Hsf_\fsf)^{2+\alpha}\right)\Bigg) \nonumber \\
&& \hspace*{145pt} - \ \frac{i\log^2(8\pi^2)}{32\pi^2}\delta \ , \label{eq:item_2} \\
%
&& \hspace*{-18pt} \mbox{ and if } \ [v]=0 \ , \ \mbox{then} \nonumber \\
%
&& 3. \hspace*{12pt} (\Hsf_\fsf^3)_\ms = \lim_{\alpha\to 0}\left(\frac{1}{M^{2\alpha}} \Hsf_\fsf^{3+\alpha} - \pp\left(\frac{1}{M^{2\alpha}} \Hsf_\fsf^{3+\alpha} \right)\right) \nonumber \\
&& \hspace*{70pt} + \ \frac{i}{48(8\pi^2)^2} \ \bigg((1+2\log(8\pi^2))R+192\pi^2[w]\bigg) \ \delta \ .
\label{eq:item_3}
%
\end{eqnarray}
%
\end{proposition}


\begin{proof}
\begin{itemize}
\item We shall here prove \ref{eq:item_1}. Setting $h=8\pi^2\sigma_F \Delta_\fsf$ and $k=\sqrt{8\pi^2/h}$, we obtain
%
\begin{equation*}
\frac{1}{M^{2\alpha}} \Delta_\fsf^{2+\alpha} = \frac{h^2}{(8\pi^2)^2} \frac{1}{(Mk)^{2\alpha}} \frac{1}{\sigma_\fsf^{2+\alpha}} \ . 
\end{equation*}
%
Using \eqref{eq:general_expansion}, $[h^2]=[u^2]=1$ and lemma \ref{lem:product_identities}, we may compute
%
\begin{eqnarray*}
&& \lim_{\alpha\to 0} \left(\frac{1}{M^{2\alpha}} \Delta_\fsf^{2+\alpha} - \pp\frac{1}{M^{2\alpha}} \Delta_\fsf^{2+\alpha} \right) \\
%
&=& \frac{h^2}{(8\pi^2)^2} \lim_{\alpha\to 0} \left(\frac{1}{(Mk)^{2\alpha}} \frac{1}{\sigma_\fsf^{2+\alpha}} - \pp\frac{1}{(Mk)^{2\alpha}} \frac{1}{\sigma_\fsf^{2+\alpha}} \right) \\
%
&=& \frac{h^2}{(8\pi^2)^2} \left(\left(\frac{1}{\sigma^2_\fsf}\right)_\ms - \frac{\log (k^2)}{2} \left(\Box + f\right) \frac{1}{\sigma_\fsf}\right) = (\Delta^2_\fsf)_\ms - \frac{i\log(8\pi^2)}{16\pi^2} \delta \ . 
\end{eqnarray*}
%
%
%
%
\item We shall here prove \ref{eq:item_2}. In analogy to \ref{eq:item_1}, we may compute
%
\begin{eqnarray*}
&& \lim_{\alpha\to 0} \left(\frac{d}{d\alpha} \frac{1}{M^{2\alpha}} \Delta_\fsf^{2+\alpha} - \pp\frac{d}{d\alpha} \frac{1}{M^{2\alpha}} \Delta_\fsf^{2+\alpha} \right) \\
%
&=& \frac{h^2}{(8\pi^2)^2} \left(-\left(\frac{\log \left(M^2 \sigma_\fsf\right)}{\sigma^2_\fsf}\right)_\ms - \log\left(\frac{8\pi^2}{h^2}\right) \left(\frac{1}{\sigma^2_\fsf}\right)_\ms + \frac{\log^2 \left(\frac{8\pi^2}{h^2}\right)}{4}\left(\Box + f\right)\frac{1}{\sigma_\fsf}\right) \\
%
&=& \left(\Delta^2_\fsf \log \left(M^{-2}\Delta_\fsf \right)\right)_\ms + \frac{i\log^2(8\pi^2)}{32\pi^2} \delta \ . 
\end{eqnarray*}
%
%
%
%
\item We shall here prove \ref{eq:item_3}. This can be proven in analogy to \ref{eq:item_1} and \ref{eq:item_2}, whereby one also needs lemma \ref{lem:product_identities} and the fact that $[v]=0$ implies by means of the covariant expansion of bitensors near the diagonal (see e.g. \cite[Section 5]{PPV_2011}) that 
%
\begin{equation*}
v = [v] + \left([\nabla_a v]-\nabla_a[v]\right) \sigma^a + R_v = [\nabla_a v]\sigma^a+R_v \ , 
\end{equation*}
%
where the remainder term $R_v$ vanishes towards the diagonal fast than $\sigma_a$. Thus, the assumption $[v]=0$ implies that the term in $\Delta^3_\fsf$ proportional to $\sigma^{-2}_\fsf \log M^2\sigma_\fsf$ does not need to be renormalized, which is crucial for the present proof. The correction term arises from the  $\log\left( h/(8\pi^2)\right)$ term in the expansion of
%
\begin{equation*}
\frac{1}{(Mk)^{2\alpha}} \frac{1}{\sigma_\fsf^{3+\alpha}} 
\end{equation*}
%
whose contribution may be computed as
%
\begin{eqnarray*}
\frac{h^3}{8(8\pi^2)^3} \log\left( \frac{h}{8\pi^2}\right) (\Box+2f)(\Box+f)\frac{1}{\sigma_\fsf} 
%
&=& \frac{i}{8(8\pi^2)^2} \left(\log(8\pi^2)\frac{R}{3}+[\Box h^3 \log (h)]\right)\delta \\
%
&=& \frac{-i}{8(8\pi^2)^2} \left(\log(8\pi^2)\frac{R}{3}-[\Box u + 8\pi^2 w\Box \sigma]\right)\delta \\
%
&=& \frac{i}{48(8\pi^2)^2} \left((1+2\log(8\pi^2))R+192\pi^2[w]\right) \delta ,
\end{eqnarray*}
%
where again lemma \ref{lem:product_identities} prove to be useful.
%
%
%
\end{itemize}
\end{proof}



%----------------------------------------------------------------------------%
\subsection{A more complicated graph}
\label{p:COMPLICATED_GRAPH}
%----------------------------------------------------------------------------%


\begin{wrapfigure}{r}{0.3\textwidth}
\begin{center}
\BigFtwoGoneHoneF
\end{center}
\end{wrapfigure}


In order to show how the proposed regularization scheme works for graphs which have more than two vertices we discuss the regularization of the following triangular graph
%
\begin{equation*}
\tsf_\gamma \ = \ \Hsf_{13} \ \Hsf_{23} \ \Hsf_{12}^2 \ , 
\end{equation*}
%
where $\Hsf_{ij} = \Hsf_\fsf(x_i,x_j)$. In order to apply the forest formula from theorem \ref{theo:renorm_t_prod_ms_forest} to regularize this graph, we already know that the forests which correspond to divergent contributions are 
%
\begin{equation*}
\{12\} \ , \quad \{123\} \ , \quad \{12,123\} \ .
\end{equation*} 
%
Then as already said in \eqref{eq:kernel_trig_ms} the regularization of $\tsf_\gamma$ thus reads
%
\begin{equation*}
\left(\tsf_\gamma\right)_\ms = \left(1+\Rsf_{12}+\Rsf_{123}+\Rsf_{123}\Rsf_{12}\right) \tsf^{(\alphabd)}_\gamma = (1+\Rsf_{123})(1+\Rsf_{12}) \tsf^{(\alphabd)}_\gamma \ .
\end{equation*}
%
In order to illustrate the explicit form of the $\Rsf$, we consider only the most singular contribution to $\Asf^{(\alphabd)}_\gamma$, namely
%
\begin{equation}
\Asf_{\gamma,0}^{(\alphabd)} = \frac{1}{\sigma_{13}^{1+\alpha_{13}}} \frac{1}{\sigma_{12}^{2(1+\alpha_{12})}} \frac{1}{\sigma_{23}^{1+\alpha_{23}}} \ ,
\label{eq:amplitude_sigma_trig_0}
\end{equation}
%
where $\sigma_{ij} = \sigma_\fsf(x_i,x_j)$. Note that, with obvious notation
%
\begin{equation*}
(8\pi^2)^{-4} \ \usf_{13} \ \usf^2_{12} \ \usf_{23} \ \tsf_{\gamma,0}
\end{equation*}
%
is in fact the only contribution to $\Asf_\gamma$ which needs to be renormalized. The application  
%
\begin{equation*}
\mbox{of } \ 1 + \Rsf_{12} \ \mbox{ to } \ \eqref{eq:amplitude_sigma_trig_0}
\end{equation*}
% 
has already been discussed in the preceding sections and corresponds to the regularization of the \textbf{fish graph}. Indeed, after setting 
%
\begin{equation*}
\alpha_{12} \ , \ \ \alpha_{23} \ , \ \ \mbox{and } \ \alpha_{13} \ , \ \mbox{to } \ \alpha \ = \ \alpha_I \ \ \mbox{for } \ I=\{1,2,3\} 
\end{equation*}
%
we obtain
%
\begin{eqnarray}
\Asf_{\gamma,1}^{(\alpha)} 
&=& \lim_{\alpha_{ij}\to\alpha} (1+\Rsf_{12}) \ \Asf_{\gamma,0}^{(\alphabd)} \nonumber \\
&=& \left(\left(\frac{1}{\sigma_{12}^2}\right)_\ms + \Ocal(\alpha)\right) \frac{1}{(\sigma_{13})^{1+\alpha}} \frac{1}{(\sigma_{23})^{1+\alpha}} \ .
\label{eq:amplitude_sigma_trig_1}
\end{eqnarray}
%
The distribution 
%
\begin{equation*}
\left(\frac{1}{\sigma^2_{12}}\right)_\ms 
\end{equation*}
%
is a homogeneous distribution of degree $\delta=-4$ under scaling of $x_2$ towards $x_1$, consequently, \eqref{eq:amplitude_sigma_trig_1} has scaling degree $8+4\alpha$. Owing to proposition \ref{prop:almost_homo}, we know that \eqref{eq:amplitude_sigma_trig_1} can be decomposed into the sum of a homogeneous distribution of degree $-(8+4\alpha)$ and a remainder. Hence, in order to expose the poles of \eqref{eq:amplitude_sigma_trig_1}, we can directly apply proposition \ref{prop:expose_poles} with $m=1$ and $c_0 = -4\alpha$. To this end, we set 
%
\begin{equation*}
U_0 = \Asf_{\gamma,1}^{(\alpha)}
\end{equation*}
%
and find 
%
\begin{eqnarray*}
U_1 &=& -4 \alpha \ U_0 - \Esf^\dagger_1 U_0 \\
&=& \left(\left(\frac{1}{\sigma_{12}^{2}} \right)_\ms + \Ocal(\alpha) \right) \frac{1}{(\sigma_{13})^{1+\alpha}} \frac{1}{(\sigma_{23})^{2+\alpha}} \ G \ ,
\end{eqnarray*}
%
where $G=G(x_1,x_2,x_3)$ is the smooth function introduced in lemma \ref{lem:rho_over_squared}. From \eqref{eq:expose_poles} we can infer that the principal part of \eqref{eq:amplitude_sigma_trig_1} is
%
\begin{equation*}
\pp\left(\Asf_{\gamma,1}^{(\alpha)}\right) = - \frac{1}{4\alpha} \left(E^\dagger_1 + \frac{G}{\sigma_{23}}\right) \left( \left(\frac{1}{\sigma_{12}^2}\right)_\ms \frac{1}{\sigma_{13}} \ \frac{1}{\sigma_{23}} \right) \ ,
\end{equation*}
%
whereas the constant regular part can be easily computed as well. Consequently, the regularized distribution
%
\begin{equation*}
\left.(\Asf_{\gamma,0})\right|_\ms = \lim_{\alpha\to 0} \left( \Asf_{\gamma,1}^{(\alpha)} - \pp \Asf_{\gamma,1}^{(\alpha)} \right)
\end{equation*}
%
can be straightforwardly computed in explicit terms.

%----------------------------------------------------------------------------%
\section{Explicit computations in Friedmann-Lemaître-Robertson-Walker spacetime}
%----------------------------------------------------------------------------%


In the previous section we have applied our regularization scheme to particular graphs on generic curved spacetime. Now we shall apply this skim on Friedmann-Lemaître-Robertson-Walker spacetime (FLRW) introduced in section \ref{p:FLRW}.


%----------------------------------------------------------------------------%
\subsection{Propagators in Fourier space}
%----------------------------------------------------------------------------%


In comoving coordinates with conformal time, the Klein-Gordon operator reads
%
\begin{equation*}
\Psf \ = \ - \Box + \xi R + m^2 \ = \ \frac{1}{a(\tau)^3} \left(\partial^2_\tau-\vec{\nabla}^2 + \left(\xi-\frac16\right)R a^2+m^2a^2\right) \ a(\tau) \ .
\end{equation*}
%
It is convenient to employ Fourier transformations with respect to the spatial coordinates in order to expand quantities in QFT on FLRW spacetimes in terms of mode solutions of the free Klein-Gordon equation
%
\begin{equation*}
\phi_{\vec{k}}(\tau,\vec{x}) = \frac{\chi_k(\tau) \ \esf^{i\vec{k}\vec{x}}}{(2\pi)^{\frac32} \ a(\tau)} \ , 
\end{equation*}
%
where the temporal modes $\chi_k(\tau)$ satisfy
%
\begin{equation}
\left(\partial^2_\tau + k^2 + m^2 a^2 + \left( \xi - \frac16 \right) R \ a^2 \right) \chi_k(\tau) \ = \ 0
\label{eq:modes}
\end{equation}
%
and the normalization condition
%
\begin{equation}
\chi_k \ \partial_\tau \overline{\chi_k} \ - \ \overline{\chi_k} \ \partial_\tau{\chi_k} \ = \ i \ .
\label{eq:modes_normal}
\end{equation}
%
Here $k = \abs{\vec{k}}$ and $\overline{\cdot}$ denotes complex conjugation.


In particular we can use the mode expansion in order to give explicit expressions for the various propagators of the free Klein-Gordon quantum field in a pure, Gaussian, homogeneous and isotropic state $\Omega$ (see \cite{PINAMONTI_2011,ZSCHOCHE_2014}
for associated technical conditions on the mode functions). To this avail, we define
%
\begin{equation}
\Delta_\sharp(x_1,x_2) = \lim_{\epsilon\downarrow 0} \frac{1}{8\pi^3 \ a(\tau_1) \ a(\tau_2)} \int_{\mathbb{R}^3} \dsf^3k \ \widehat{\Delta_\sharp}(\tau_1,\tau_2,k) \ \esf^{i\vec{k}(\vec{x}_1-\vec{x}_2)-\epsilon k} \ ,
\label{eq:propagators_fourier}
\end{equation}
%
where $\Delta_\sharp$ stands for either $\Delta_+$ (two-point function), $\Delta_{\rsf/\asf}$ (retarded/advanced propagator) or $\Delta_\fsf$ (Feynman propagator). Recall that our regularization scheme preserves invariance under spacetime isometries and thus we know that regularized powers of the Feynman propagator may also be written in the form \eqref{eq:propagators_fourier}.


The Fourier versions of the single propagators read
%
\begin{eqnarray}
&& \widehat{\Delta_+}(\tau_1,\tau_2,k) = \chi_k(\tau_1) \overline{\chi_k(\tau_2)} \ , \nonumber \\
&& \widehat{\Delta_-}(\tau_1,\tau_2,k) = \overline{\widehat{\Delta_+}(\tau_1,\tau_2,k)} \ , \nonumber \\
&& \widehat{\Delta_\fsf}(\tau_1,\tau_2,k) = \Theta(\tau_1-\tau_2) \ \widehat{\Delta_+}(\tau_1,\tau_2,k) + \Theta(\tau_2-\tau_1) \ \widehat{\Delta_-}(\tau_1,\tau_2,k) \ , \nonumber \\
&& \widehat{\Delta_{\rsf/\asf}}(\tau_1,\tau_2,k) = \mp i \Theta\left(\pm(\tau_1-\tau_2)\right) \ \left(\widehat{\Delta_+}(\tau_1,\tau_2,k) - \widehat{\Delta_-}(\tau_1,\tau_2,k)\right) \ , \nonumber \\
\label{eq:propagators_fourier_exp}
\end{eqnarray}
%
whereas by the convolution theorem, we have the following Fourier versions of products and convolutions of multiple propagators, provided those products and convolutions are well defined. Defining
%
\begin{eqnarray}
&& \left[\Delta_{\sharp_1}\ast_4\Delta_{\sharp_2}\right](x,y) = \int_\Mcal \dsf^4x \ \sqrt{\abs{g}} \ \Delta_{\sharp_1}(x_1,x) \ \Delta_{\sharp_2}(x,x_2) \ , \nonumber \\
%
&& \left[\widehat{\Delta_{\sharp_1}}\ast_1\widehat{\Delta_{\sharp_2}}\right](\tau_1,\tau_2,k) = \int_I \dsf\tau \ a(\tau)^2 \ \widehat{\Delta_{\sharp_1}}(\tau_1,\tau,k) \ \widehat{\Delta_{\sharp_2}}(\tau,\tau_2,k) \ , \nonumber \\
%
&& \left[\widehat{\Delta_{\sharp_1}}\ast_3\widehat{\Delta_{\sharp_2}}\right](\tau_1,\tau_2,k) = \int_{\mathbb{R}^3} \dsf^3p \ \widehat{\Delta_{\sharp_1}}(\tau_1,\tau_2,p) \ \widehat{\Delta_{\sharp_2}}\left(\tau_1,\tau_2,|\vec{k}-\vec{p}|\right) \ , \nonumber \\
\label{eq:def_convolutions}
\end{eqnarray}
%
we have
%
\begin{eqnarray}
&& \widehat{\Delta_{\sharp_1}\ast_4\cdots\ast_4\Delta_{\sharp_n}}=\widehat{\Delta_{\sharp_1}}\ast_1\cdots\ast_1 \widehat{\Delta_{\sharp_n}} \ , \nonumber \\
&& \widehat{\prod^n_{i=1}\Delta_{\sharp_i}}(\tau_1,\tau_2,k)=\frac{1}{\left((2\pi)^3 a(\tau_1)^{2}a(\tau_2)^{2}\right)^{n-1}}\left[\widehat{\Delta_{\sharp_1}}\ast_3\cdots\ast_3\widehat{\Delta_{\sharp_n}}\right](\tau_1,\tau_2,k) \ , \nonumber \\
\label{eq:convolution_identities}
\end{eqnarray}
%
Choosing a pure, Gaussian, homogeneous and isotropic state $\Omega$ of the quantized free Klein-Gordon field on a spatially flat FLRW spacetimes amounts to choosing a solution of \eqref{eq:modes} and \eqref{eq:modes_normal} for each $k$. In order for $\Omega$ to be a Hadamard state the temporal modes $\chi_k$ have to satisfy certain conditions in the limit of large $k$ which are difficult to formulate precisely. Heuristically, a necessary but not sufficient condition is that the dominant part of $\chi_k$ for large $k$, when the mass and curvature terms in \eqref{eq:modes} are dominated by $k^2$, is 
%
\begin{equation*}
\frac{1}{\sqrt{2k}} \ \esf^{-ik\tau} \ , 
\end{equation*}
%
i.e. a positive frequency solution. Note that the retarded and advanced propagators are state-independent and thus 
%
\begin{equation*}
\widehat{\Delta_{\rsf/\asf}}(\tau_1,\tau_2,k) 
\end{equation*}
%
is independent of the particular $\chi_k$ chosen for each $k$.


%----------------------------------------------------------------------------%
\subsection{The regularized fish and sunset graphs in Fourier space}
\label{p:SUNSET_FISH_FLWR}
%----------------------------------------------------------------------------%


In perturbative calculations at low orders we encounter (pointwise) powers of $\Delta_\pm$ and $\Delta_\fsf$. While the powers of $\Delta_\pm$ are well-defined if $\Omega$ is a Hadamard state on account of the wave front set properties of these distributions, we need to regularize the powers of $\Delta_\fsf$ by means of the scheme developed in the previous sections. In order to be useful for explicit computations in FLRW spacetimes, we have to develop a spatial Fourier--space version of this scheme. Having in mind the application to $\phi^4$ theory, we shall compute 
%
\begin{equation*}
\widehat{(\Delta_\fsf)^n_\ms}(\tau_1,\tau_2,k) \ \mbox{ for } \ n = 2 , 3 \ . 
\end{equation*}
%
The difficulty in achieving this is that, to our knowledge, despite of the large symmetry of flat \textbf{FLRW} spacetimes, neither $\sigma$ nor the Hadamard coefficients $u$, $v$ and $w$ may written in a tractable form which can be Fourier transformed easily. Our strategy to circumvent this problem is the following section.


%----------------------------------------------------------------------------%
\subsubsection{Strategy}
%----------------------------------------------------------------------------%


\begin{enumerate}
%
%
\item For a general mass $m$ and coupling to the scalar curvature $\xi$ and a general homogeneous and isotropic, pure and Gaussian Hadamard state $\Omega$, split $\Hsf_\fsf$ as
%
\begin{equation}
\Hsf_\fsf = \Hsf_{\fsf,0} + d \ , \qquad d = \Hsf_\fsf - \Hsf_{\fsf,0} \ ,
\label{eq:propagator_split}
\end{equation}
%
where $\Hsf_{\fsf,0}$ must satisfy the following conditions.
%
\begin{itemize}
%
\item $\Hsf_{\fsf,0}$ is explicitly known in position space and Fourier space.
%
\item $\Hsf_{\fsf,0}$ is of the form
%
\begin{equation*}
\Hsf_{\fsf,0} = \frac{1}{8\pi^2} \left( \frac{u_0}{\sigma_\fsf} + v_0 \log\left(M^2\sigma_\fsf\right) \right) + w_0 \ , 
\end{equation*}
%
with $u_0=u$, i.e. it agrees with $\Hsf_\fsf$ in the most singular term but not necessarily in the subleading singularities.
%
\item $[v_0]=0$ and $\Hsf^{\alpha}_{\fsf,0}$ is weakly meromorphic in $\alpha$ such that
%
\begin{equation*}
\left(\Hsf^2_{\fsf,0}\right)_\ms \ , \quad 
\left(\Hsf^2_{\fsf,0} \log\left(M^{-2} \Delta_{\fsf,0}\right) \right)_\ms \ , \quad
\mbox{and} \quad \left(\Hsf^3_{\fsf,0}\right)_\ms 
\end{equation*}
%
may be computed with proposition \ref{prop:equivalent_scheme}. This is crucial for preserving the explicit knowledge of $\Hsf_{\fsf,0}$ in position space in the regularization procedure, so that one may hope to compute the Fourier transforms of the regularized powers.
%
\end{itemize}
%
%
\item With these assumptions on $\Hsf_{\fsf,0}$ it follows that the regularized fish and sunset graphs may be computed as
%
\begin{eqnarray}
&& (\Hsf_\fsf^2)_\ms = (\Hsf_{\fsf,0}^2)_\ms \ + \ 2 \ \Hsf_{\fsf,0} \ d \ + \ d^2 \ , \nonumber \\
&& (\Hsf_\fsf^3)_\ms = (\Hsf_{\fsf,0}^3)_\ms \ + \ 3 \left(\Hsf_{\fsf,0}^2 \ d \right)_\ms \ + \ 3 \Hsf_{\fsf,0} \ d^2 \ + \ d^3 \ , \nonumber \\
\label{eq:fish_sunset_reg}
\end{eqnarray}
%
because the non regularized terms in the above formulae are distributions with scaling degree lower than $4$ for $y \to x$ and thus can be directly and uniquely extended to the diagonal.
%
%
\item $\left(\Hsf_{\fsf,0}^2\right)_\ms$ and  $\left(\Hsf_{\fsf,0}^3\right)_\ms$ may be computed with proposition \ref{prop:equivalent_scheme} as anticipated. In order to compute $\left(\Hsf_{\fsf,0}^2 \ d\right)_\ms$ we further split $d$ as
%
\begin{equation*}%
d = d_1 + d_2 \ , \qquad d_1 = - \frac{[v] \log\left(M^{-2} \ \Hsf_{\fsf,0}\right)}{8\pi^2} \ ,\qquad d_2 = d - d_1 \ .
\label{eq:d_split}
\end{equation*}
%
Because 
%
\begin{equation*}
v = [v] + \Ocal(\sigma_a) \ , 
\end{equation*}
%
$d_1$ contains the leading logarithmic singularity in $d$ (and thus $\Hsf_\fsf$) which is the only logarithmic singularity relevant for the regularization of the sunset graph. Consequently
%
\begin{equation}
\left(\Hsf_{\fsf,0}^2 \ d\right)_\ms \ = \ - \frac{[v]}{8\pi^2} \ \left(\Hsf_{\fsf,0}^2 \ \log\left(M^{-2} \ \Hsf_{\fsf,0} \right) \right)_\ms \ + \ d_2 \ \left(\Hsf_{\fsf,0}^2 \right)_\ms \ ,
\label{eq:fish_sunset_reg_2}
\end{equation}
%
and thus proposition \ref{prop:equivalent_scheme} can be applied again.
%
%
\item Due to the symmetry of \textbf{FLRW} spacetimes and the assumption that the pure and Gaussian Hadamard state $\Omega$ is invariant under this symmetry, $[v]$ and $[w]$ do not depend on the spatial coordinates. Given that one succeeds to compute the spatial Fourier transforms of 
%
\begin{equation*}
\log \left(M^{-2} \ \Hsf_{\fsf,0}\right) \ , \quad
\left(\Hsf_{\fsf,0}^2\right)_\ms \ , \quad
\left(\Hsf_{\fsf,0}^2 \ \log\left(M^{-2} \ \Hsf_{\fsf,0}\right)\right)_\ms \ , \quad  \left(\Hsf_{\fsf,0}^3\right)_\ms \ , 
\end{equation*}
%
and
%
\begin{equation*}
\widehat{(\Hsf_\fsf^n)_\ms}(\tau_1,\tau_2,k)
\end{equation*}
%
may be computed by means of the convolution identities \eqref{eq:convolution_identities}, since the Fourier transforms of $\Hsf_{\fsf,0}$, $d_1$ and $d_2$ are known by construction.
%
%
\end{enumerate}


%----------------------------------------------------------------------------%
\subsubsection{Explicit computations}
%----------------------------------------------------------------------------%


In order to follow the computational strategy outlined above, we first compute $[v]$ and $[w]$.  Indeed, the coinciding point limit of the Hadamard coefficient $v$ reads (see e.g. \cite[Section III.1.2]{HACK_2010} for details)
%
\begin{equation*}
[v]=\frac{m^2+\left(\xi-\frac16\right)R}{2} \ .
\label{eq:coinciding_v}
\end{equation*}
%
Moreover, using the method of \cite{schlemmer_2010} to compute a spatial Fourier representation of the Hadamard parametrix $\hsf_\fsf$ (here considered as \eqref{eq:hadamard_rep} with $w=0$) in \textbf{FLRW} spacetimes, one can compute (see the review in \cite{degner_2013} and a related method in \cite{PINAMONTI_2011} for the conformally coupled case)
%
\begin{eqnarray}
[w] &=& \lim_{x\to y} \left(\Hsf_\fsf(x,y) - \Hsf_\fsf(x,y) \right) \nonumber \\
&=& \frac{1}{(2\pi)^3 a^2} \ \int_{\Rbb^3} \dsf^3k \ \abs{\chi_k(\tau)}^2 \ - \ \frac{1}{2\sqrt{k^2+a^2m^2+a^2\left(\xi-\frac16\right)R}} \nonumber \\
%
&& + \frac{1}{16\pi^2} \ \left(m^2+\left(\xi-\frac16\right)R\right)\left(2\gamma-1+\log\left(
\frac{m^2+\left(\xi-\frac16\right)R}{2M^2}\right)\right) \ - \ \frac{R}{36(8\pi^2)} \nonumber \\
\label{eq:coinciding_w} 
\end{eqnarray}
%
where $\gamma$ is the Euler-Mascheroni constant and $\Hsf_\fsf$ is taken with the mass scale $M$ inside of the logarithm of $\sigma$. Note that one may take instead of the function 
%
\begin{equation*}
F(k) = \frac{1}{(2\sqrt{k^2+a^2m^2+a^2\left(\xi-\frac16\right)R})}
\end{equation*}
%
in \eqref{eq:coinciding_w} any distribution $F^\prime(k)$ such that $F^\prime(k)-F(k)$ is $\Ocal(k^{-5})$ for large $k$ and integrable. By taking e.g. 
%
\begin{equation*}
F^\prime(k) = \frac{1}{2k} - \Theta(k-am)\frac{a^2 m^2 + a^2 \left( \xi - \frac16 \right) R}{4k^3}
\end{equation*}
%
one may cancel the $\log\left(R\right)$ term outside of the integral.


As anticipated we see that $[v]$ and $[w]$ are functions of time only (recall \eqref{eq:rflrw}). Moreover, we see that $[v]=0$ for a conformally coupled ($\xi=\frac16$) massless scalar field. Thus, in order to pursue our computational strategy, we should look for a candidate for $\Hsf_{\fsf,0}$ among the Feynman propagators in suitable states of this theory. In fact, choosing the conformal vacuum state of the massless conformally coupled scalar field does the job. The conformal vacuum is given by choosing the modes
%
\begin{equation*}
\chi_k(\tau) = \frac{\esf^{-ik\tau}}{\sqrt{2k}} \ , 
\end{equation*}
%
and thus the Feynman propagator $\Delta_{\fsf,0}$ in this state is of the form
%
\begin{equation}
\Delta_{\fsf,0}(x_1,x_2)=\frac{1}{8\pi^2 a(\tau_1)a(\tau_2)}\frac{1}{\sigma_{\fsf,\mathbb{M}}(x_1,x_2)} \ , \qquad 
\widehat{\Delta_{\fsf,0}}(\tau_1,\tau_2,k)=\frac{e^{-ik|\tau_1-\tau_2|}}{2k}\,.
\label{eq:propagator_conformal}
\end{equation}
%
Here, and in the following, the index $_{\Mbb}$ indicates quantities in Minkowski spacetime, in particular 
%
\begin{equation*}
\sigma_{\Mbb}(x_1,x_2) = \frac12 \left(\vec{x}_1-\vec{x_2}\right)^2 - \frac12 \left(\tau_1-\tau_2\right)^2 \ . 
\end{equation*}
%
$\Delta^\alpha_{\fsf,0}$ is weakly meromorphic in $\alpha$ because the massless vacuum Feynman propagator in Minkowski spacetime has this property and the conformal rescaling by $a$ does not violate it. Thus, we may follow our computational strategy and compute
%
\begin{equation*}
\left(\Delta^2_{F,0}\right)_\ms \ , \quad \left(\Delta^2_{F,0}\log\left(M^{-2}\Delta_{F,0}\right)\right)_\ms \ , \ \mbox{and} \ \left(\Delta^3_{F,0}\right)_\ms \ ,
\end{equation*}
%
by means of proposition \ref{prop:equivalent_scheme}. This is easily done using \eqref{eq:general_expansion} for $\sigma_{\fsf,\Mbb}$ rather than $\sigma_\fsf$ and
%
\begin{equation*}
h = \sqrt{8 \pi^2 \ a(\tau_1)a(\tau_2)} = \sqrt{8\pi^2 \  a\otimes a} \ . 
\end{equation*}
%
The results are
%
\begin{eqnarray*}
(\Delta_{\fsf,0}^2)_\ms &=& \lim_{\alpha\to 0} \left( \frac{1}{M^{2\alpha}}(\Delta_{\fsf,0})^{2+\alpha} - \pp\frac{1}{M^{2\alpha}}(\Delta_{\fsf,0})^{2+\alpha} \right) + \frac{i\log(8\pi^2)}{16\pi^2} \delta \\
&=& \lim_{\alpha\to 0} \frac{1}{(8\pi^2)^2 a^2\otimes a^2} \left(\frac{1}{(M\sqrt{8\pi^2 a\otimes a})^{2\alpha}}\frac{1}{\sigma_{\fsf,\mathbb{M}}^{2+\alpha}}-\pp \frac{1}{(M\sqrt{8\pi^2 a\otimes a})^{2\alpha}}\frac{1}{\sigma_{\fsf,\mathbb{M}}^{2+\alpha}}\right) \\
&& + \ \frac{i\log(8\pi^2)}{16\pi^2} \ \delta \\
&=& - \frac{1+2\log (a)}{16\pi^2 a^4} \ i \ \delta_\mathbb{M} \ - \ \frac{1}{2(8\pi^2)^2 a^2\otimes a^2} \ \Box_{\mathbb{M}}\left(\frac{\log\left(M^2\sigma_{\fsf,\mathbb{M}}\right)}{\sigma_{\fsf,\mathbb{M}}}\right) \ ,
\end{eqnarray*}
%
\begin{eqnarray*}
\left(\Delta^2_{\fsf,0} \log\left(M^{-2} \Delta_{\fsf,0}\right)\right)_\ms
&=& \lim_{\alpha\to 0}\left(\frac{d}{d\alpha}\frac{1}{M^{2\alpha}}(\Delta_{\fsf,0})^{2+\alpha}-\pp\frac{d}{d\alpha}\frac{1}{M^{2\alpha}}(\Delta_{\fsf,0})^{2+\alpha}\right) \ - \ \frac{i\log^2(8\pi^2)}{32\pi^2}\delta \\
%
&=& \frac{2+2\log (a^2 8\pi^2)+\log^2 (a^2)}{32\pi^2 a^4}i\delta_\mathbb{M}+\frac{1}{4(8\pi^2)^2 a^2\otimes a^2}\Box_{\mathbb{M}}\frac{\log^2\left(M^2\sigma_{\fsf,\mathbb{M}}\right)}{\sigma_{\fsf,\mathbb{M}}}\\
%
&& + \ \frac{1+\log(8\pi^2) a\otimes a}{2(8\pi^2)^2 a^2\otimes a^2}\Box_{\mathbb{M}}\frac{\log\left(M^2\sigma_{\fsf,\mathbb{M}}\right)}{\sigma_{\fsf,\mathbb{M}}} \ ,
\end{eqnarray*}
%
and
%
\begin{eqnarray*}
(\Delta_{\fsf,0})^3_\ms 
&=& \lim_{\alpha\to 0}\left(\frac{1}{M^{2\alpha}}(\Delta_{\fsf,0})^{3+\alpha}-\pp\frac{1}{M^{2\alpha}}(\Delta_{\fsf,0})^{3+\alpha}\right)+\frac{i\left((1+2\log(8\pi^2))R+192\pi^2[w]\right)}{48(8\pi^2)^2}\delta\\
%
&=&-\frac{(15+12\log (a))\Box_\mathbb{M}+6(\Box_\mathbb{M} \log (a))+2(\partial^2_\tau a)/a}{48(8\pi^2)^2a^6}i\delta_\mathbb{M} \\ 
&& - \frac{1}{8(8\pi^2)^3 a^3\otimes a^3}\Box^2_{\mathbb{M}}\frac{\log\left(M^2\sigma_{\fsf,\mathbb{M}}\right)}{\sigma_{\fsf,\mathbb{M}}}\,.
\end{eqnarray*}
%
where we have used $\delta = \dfrac{\delta_{\Mbb}}{a^4}$, $f_{\Mbb}=0$, and the fact that by \eqref{eq:coinciding_w} $8\pi^2[w_0]=-\dfrac{R}{36}$ for the conformal vacuum state of the massless, conformally coupled scalar field. 


Using these results as well as the Fourier representation of $1/\sigma_{\fsf,\Mbb}$ \eqref{eq:propagator_conformal} and $\log \left(M^2\sigma_{\fsf,\Mbb}\right)$
\eqref{eq:flog}, and convolution identities, we can finally obtain the Fourier versions of the regularized powers of $\Delta_{\fsf,0}$. For instance, we find for 
%
\begin{eqnarray}
&& \widehat{\left(\Delta^2_{F,0}\right)_\text{ms}}(\tau_1,\tau_2,k) \ = \ - \ \frac{1+2\log (a(\tau_1))}{16a(\tau_1)^2\pi^2}\delta(\tau_1-\tau_2)- \nonumber \\
&& - \ \frac{1}{16\pi^3 a(\tau_1)a(\tau_2)}(\partial^2_{\tau_1}+k^2)\bigint_{\mathbb{R}^3}d^3p\,\left(\frac12\left(\frac{1}{p^3}\right)_{\text{ren},M}  + \ \frac{i|\tau_1-\tau_2|}{2p^2}\right) \nonumber \\
&& \cdot \ \frac{1}{2|\vec{k}-\vec{p}|}e^{-i(p+|\vec{k}-\vec{p}|)|\tau_1-\tau_2|}
\label{eq:fourier_square}
\end{eqnarray}
%
where the appearing regularization of $\dfrac{1}{p^3}$ is defined in \eqref{eq:regk3}. Note that the $\vec{p}$ integral has no convergence problems for large $p$ because one may write the potentially dangerous contribution 
%
\begin{equation*}
\frac{-i|\tau_1-\tau_2|e^{-2ip|\tau_1-\tau_2|}}{p}
\quad
\mbox{as}
\quad
\partial_p\left( \dfrac{e^{-2ip|\tau_1-\tau_2|}}{2p^2}\right) 
\end{equation*}
%
plus an $\Ocal(p^{-3})$ term. Regarding the convergence for small $p$ we observe that the integral is manifestly convergent if $k\neq0$, thus yielding a well defined distribution in $\vec{k}$ on $\mathbb{R}^3\setminus\{0\}$. The scaling degree of this distribution is easily seen to be $1<3$ and thus a unique extension towards the origin exists. In practical terms this means that the integral for $k=0$ may be computed as a limit $k\to0$ of the integral with nonvanishing $k$ without any regularization. 


%----------------------------------------------------------------------------%
\subsubsection{Fourier transform on Minkowski spacetime}
%----------------------------------------------------------------------------%

In order to compute the Fourier transform of $\log\left( M^2 \sigma_{\fsf,\Mbb}\right)$, we recall that the Feynman propagator of the Klein-Gordon field with mass $m$ in the Minkowski vacuum is given by
%
\begin{eqnarray*}
\Delta_{\fsf,m,\Mbb}&=&\lim_{\epsilon\downarrow 0} \frac{1}{(2\pi)^3} \int_{\Rbb^3} \dsf^3k \ \frac{1}{2\sqrt{k^2+m^2}} \ \esf^{-i\sqrt{k^2+m^2}|\tau_1-\tau_2|} \ \esf^{i\vec{k}\left(\vec{x}-\vec{y}\right)} \ \esf^{-\epsilon k}\\
&=&\frac{1}{8\pi^2}\sqrt{\frac{2m^2}{\sigma_{F,\Mbb}}}K_1\left(\sqrt{2 m^2\sigma_{F,\Mbb}}\right)\\
&=&\frac{1}{8\pi^2} \bigg( \frac{1}{\sigma_{\fsf,\Mbb}} + \frac{m^2}{2} \left(1+\frac{m^2 \sigma_{\fsf,\Mbb}}{4} \right) \log\left(\frac{\esf^{2\gamma}m^2 \sigma_{\fsf,\Mbb}}{2}\right) \\
&& - \ \frac{m^2}{2} \left(1+\frac{5m^2\sigma_{\fsf,\Mbb}}{8}\right)\bigg) + \Ocal(m^4),
\end{eqnarray*}
%
where $K_1$  is a modified Bessel function and $\gamma$ is the Euler-Mascheroni constant. Using this, we find
%
\begin{eqnarray*}
&&\log \left(M^2 \sigma_{\fsf,\Mbb}\right) \\
%
&=&\lim_{m\to 0}\left(16\pi^2 \frac{d \ \Delta_{\fsf,m,\Mbb}}{d \,m^2} - 
\log\left(\frac{\esf^{2\gamma}m^2}{2 M^2}\right)\right)\\
%
&=& - \ \lim_{m\to 0}\bigg(\lim_{\epsilon\downarrow 0} \frac{1}{\pi} \int_{\Rbb^3} \dsf^3 k \frac{1+i\sqrt{k^2+m^2}|\tau_1-\tau_2|}{2(k^2+m^2)^{\frac32}} \ \esf^{-i\sqrt{k^2+m^2}|\tau_1-\tau_2|} \ \esf^{i\vec{k}\left(\vec{x}-\vec{y}\right)} \ \esf^{-\epsilon k} \\
&& + \ \log\left(\frac{e^{2\gamma}m^2}{2 M^2}\right)\bigg)\\
%
&=& - \ \lim_{\epsilon\downarrow 0}\frac{1}{\pi}\int_{\Rbb^3}d^3 k\left(\lim_{m\to 0}\left(\frac{1}{2(k^2+m^2)^{\frac32}}+\pi\log\left(\frac{e^{2\gamma}m^2}{2 M^2}\right)\delta(\vec{k})\right)\right.\\
%
&& \left.+ \ \frac{i|\tau_1-\tau_2|}{2k^2}\right)e^{-ik|\tau_1-\tau_2|}e^{i\vec{k}\left(\vec{x}-\vec{y}\right)}\,e^{-\epsilon k}\\
&=&-\lim_{\epsilon\downarrow 0}\frac{1}{\pi}\int_{\Rbb^3}d^3 k\left(\frac{1}{2}\left(\frac{1}{k^3}\right)_{\text{ren},M}+\frac{i|\tau_1-\tau_2|}{2k^2}\right)e^{-ik|\tau_1-\tau_2|}e^{i\vec{k}\left(\vec{x}-\vec{y}\right)}\,e^{-\epsilon k}\\
&=&\frac{1}{(2\pi)^{\frac32}}\lim_{\epsilon\downarrow 0}\int_{\Rbb^3}d^3 k \;
\text{flog}(\tau_1-\tau_2,k)\,e^{i\vec{k}\left(\vec{x}-\vec{y}\right)}\,e^{-\epsilon k}
\end{eqnarray*}
%
where the appearing regularization of the (tempered) distribution $\dfrac{1}{k^3}$ is 
%
\begin{equation}
\left(\frac{1}{k^3}\right)_{\text{ren},M}:=\lim_{m\to 0}\left(\frac{1}{(k^2+m^2)^{\frac32}}+\pi\log\left(\frac{e^{4\gamma}m^4}{4 M^4}\right)\delta(\vec{k})\right)
\label{eq:regk3}
\end{equation}
and
\begin{equation}
\text{flog}(\tau_1-\tau_2,k):=-\sqrt{8\pi}\left(\frac{1}{2}\left(\frac{1}{k^3}\right)_{\text{ren},M}+\frac{i|\tau_1-\tau_2|}{2k^2}\right)e^{-ik|\tau_1-\tau_2|}
\label{eq:flog}
\end{equation}
%
is the sought-for spatial Fourier transform of $\log \left(M^2 \sigma_{F,\Mbb}\right)$. 

%----------------------------------------------------------------------------%
\subsection{Two point function for a quartic potential up to second order}
%----------------------------------------------------------------------------%


In order to compute the analytic expressions corresponding to the graphs in figure \eqref{eq:2pf}, we may use the Fourier versions of the appearing propagators \eqref{eq:propagators_fourier_exp}, \eqref{eq:fourier_square}, and the analogous expressions for 
%
\begin{equation*}
\widehat{\left(\Delta^2_{\fsf,0}\log\left(M^{-2}\Delta^2_{\fsf,0}\right) \right)_\ms}(\tau_1,\tau_2,k) \quad \mbox{and } \quad \widehat{\left(\Delta^3_{\fsf,0}\right)_\text{ms}}(\tau_1,\tau_2,k)
\end{equation*}
%
the explicit form of $\mu(x)=3\lambda w(x,x)$ in \eqref{eq:coinciding_w}, as well as \eqref{eq:fish_sunset_reg}, \eqref{eq:fish_sunset_reg_2} and the identities for products and convolutions \eqref{eq:def_convolutions}, \eqref{eq:convolution_identities}. Note that $\mu(x)$ is in fact only time-dependent because $\Omega$ was chosen homogeneous and isotropic. Thus the integrals with $\mu$-vertices can be computed partly with the above mentioned identities by means of 
%
\begin{equation*}
\widehat{(1\otimes \mu) \Delta_{\sharp}}(\tau_1,\tau_2,k)=\mu(\tau_2)\widehat{\Delta_{\sharp}}(\tau_1,\tau_2,k) \ , \quad\widehat{(\mu\otimes 1) \Delta_{\sharp}}(\tau_1,\tau_2,k)=\mu(\tau_1)\widehat{\Delta_{\sharp}}(\tau_1,\tau_2,k) \ . 
\end{equation*}
%
Similarly, the bubbles in the third line of Figure \eqref{eq:2pf} contribute only time dependent vertex factors which can be computed as 
%
\begin{equation*}
h_\sharp(\tau):=\int_\Mcal d\tau_1 d^3x_1\; a(\tau_1)^4\mu(\tau_1)\Delta_\sharp(\tau,\tau_1,\vec{x}-\vec{x}_1)=\frac{1}{a(\tau)}\int_I d\tau_1\;a(\tau_1)^3 \mu(\tau_1)\widehat{\Delta_\sharp}(\tau,\tau_1,0) 
\end{equation*}
%
where $\Delta_\sharp$ is either $\Delta^2_+$ or $\left(\Delta^2_\fsf\right)_\ms$.


With these preparations, we can compute e.g. the first graphs of the fourth and fifth line in figure \eqref{eq:2pf} in Fourier space as
%
\begin{eqnarray*}
&& \widehat{\Delta_R\ast_4((h_\fsf \otimes 1) \Delta_+)} \ = \ \widehat{\Delta_R}\ast_1\widehat{\left((h_F\otimes 1)\Delta_+)\right)} \\
&=& \int_{I^2} d\tau_3\,d\tau_4\; a(\tau_3)a(\tau_4)^3\mu(\tau_4)\widehat{\Delta_R}(\tau_1,\tau_3,k)\widehat{\Delta_+}(\tau_3,\tau_2,k)\widehat{(\Delta^2_F)_\text{ms}}(\tau_3,\tau_4,0)
\end{eqnarray*}
%
and
%
\begin{eqnarray*}
&& \widehat{\Delta_R\ast_4(\Delta_F)^3_\ms\ast_4\Delta_+} \ = \ \widehat{\Delta_R}\ast_1\widehat{(\Delta_F)^3_\text{ms}}\ast_1\widehat{\Delta_+} \\
&=& \int_{I^2} d\tau_3\,d\tau_4\; a(\tau_3)^2a(\tau_4)^2\widehat{\Delta_R}(\tau_1,\tau_3,k)\widehat{(\Delta^3_F)_\text{ms}}(\tau_3,\tau_4,k)\widehat{\Delta_+}(\tau_4,\tau_2,k) \ .
\end{eqnarray*}


%----------------------------------------------------------------------------%
\subsection{More complicated graphs on cosmological spacetimes}
%----------------------------------------------------------------------------%


In order to compute the Fourier transforms of more complicated graphs on FLRW spacetimes, one can use a strategy generalizing the one employed in section \ref{p:SUNSET_FISH_FLWR}. Namely, one again decomposes the Feynman propagator $\Delta_\fsf$ into several pieces which capture the relevant singularities and can be expressed in terms of the conformal vacuum Feynman propagator $\Delta_{\fsf,0}$ whose explicit form in position and Fourier space is well known in contrast to the form of $\sigma$ itself. The corresponding decomposition of general Feynman amplitudes $\tsf_\gamma$ is straightforward. The only non--trivial step is to generalize proposition \ref{prop:equivalent_scheme} to the case of general amplitudes, i.e. to compute the difference between the minimal subtraction scheme used in conjunction with either analytically regularizing powers of $\sigma$ directly or analytically regularizing powers of the full propagator $\Delta_{\fsf,0}$. However, we do not foresee any problems in obtaining such a generalization by proving versions of lemma \ref{lem:rho_over_squared} and proposition \ref{prop:almost_homo} for $\Delta_{\fsf,0}$ rather than $\sigma$.


In fact, one can also skip this last step by taking a rather pragmatic approach and working directly with the regularization scheme consisting of decomposition in $\Delta_{\fsf,0}$, analytic regularization of powers of this propagator and minimal subtraction of the principal parts. This scheme, clearly applicable only to conformally flat spacetimes, satisfies all properties proved in proposition \ref{prop:properties_scheme}, with two exceptions. It is not obvious whether the Principle of Perturbative agreement with respect to generalized mass perturbations holds for this scheme, whereas locality and covariance of course only hold in the sense restricted to conformally flat spacetimes. In this respect it is essential that the Feynman propagator of the conformal vacuum $\Delta_{\fsf,0}$ on conformally flat spacetimes is manifestly ``geometric'', because the corresponding propagator of the massless Minkowski vacuum has this property.



%----------------------------------------------------------------------------%
\chapter*{Conclusion}
\addcontentsline{toc}{chapter}{Conclusion}
%----------------------------------------------------------------------------%


The regularization problem we treated in this thesis was to extend the time ordered problem for local functionals, which appears to be an extension of product of Feynman propagator when their supports overlaps. Fredenhagen and Brunetti were able in \cite{BF_2000} to extend those products up to the total diagonal, i.e. when this time all supports coincide. Unfortunately their procedure of extension to the full space was not convenient for practical computations, and to the best of our knowledge, other renormalization schemes on curved spacetimes discussed in the literature such as dimensional regularization, local momentum space methods, zeta--function regularization, heat--kernel techniques, generic Epstein--Glaser renormalization and, on cosmological spacetimes, dimensional regularization only with respect to spatial variables, lack at least one of the property we obtain with our scheme listed in section \ref{p:PROP_SCHEME}. Therefore in this thesis, we have introduced a regularization scheme on curved spacetimes consisting of a particular analytic regularization of the Feynman propagator, which is well suited to perform computations. The regularization procedure consists in three main steps. First we define an analytic regularization, second we extract the principal part of the resulting meromorphic expression with respect to the parameter of regularization, and third we perform the minimal subtraction by subtracting the principal part in order to finally take the limit of the parameter to zero. For graphs with more than two vertices it is required to use the forest formula, in this case the limit has to be taken in a certain order \ref{theo:ms_general}. The main problem was to be able to extract the principal part. It is the reason we introduced the generalized Euler operator which allowed us to isolate the poles. We have argued that this scheme has all properties that a physically meaningful renormalization scheme on curved spacetimes should have and that it is in fact a particular form of differential renormalization. The renormalization scheme discussed in this work has the advantage that it is 
%
\begin{enumerate}
\item directly applicable to spacetimes with Lorentzian signature, 
\item manifestly (local and) covariant, 
\item manifestly invariant under any spacetime isometries present,
\item capturing correctly the non--geometric and non--unique state--dependent contribution of Feynman amplitudes and not only the geometric divergent part, which is unique up to finite renormalizations,
\item well--suited for practical computations, e.g. in cosmological spacetimes,
\item constructed to all orders in perturbation theory,
\item and mathematically rigorous.
\end{enumerate}


In order to demonstrate the practical applicability of the scheme, we have computed several examples on generic curved spacetimes. Moreover, we have shown how explicit computations in cosmological spacetimes can be done, in particular, how the renormalization scheme initially defined in position space can be interpreted in terms of quantities Fourier transformed with respect to comoving spatial coordinates.

We have discussed the renormalization scheme only for scalar fields in four spacetime dimensions, however, the extension to other spacetime dimensions is straightforward. Moreover, as the analytic regularization discussed in this work consists of regularizing only inverse powers of the squared geodesic distance, it can be straightforwardly generalized to field theories with higher spin, with and without gauge--invariance. In particular, spinorial quantities can be directly regularized without the need to worry about their dependence on the dimension such as in dimensional regularization. Finally, we expect that a generalization of the scheme introduced in this work to gauge theories yields a scheme which preserves the local gauge symmetry.


%----------------------------------------------------------------------------%
\newpage
\vspace*{100pt}
\thispagestyle{empty}
\chapter*{Acknowledgements}
\addcontentsline{toc}{chapter}{Acknowledgements}
%----------------------------------------------------------------------------%


Innanzitutto vorrei ringraziare l'università di Genova, e in particolare il dipartimento di matematica che mi ha dato l'opportunità e le condizioni per realizzare il mio dottorato. Ringrazio inoltre il mio relatore di dottorato Nicola Pinamonti per la sua pazienza e disponibilità. Senza il suo aiuto questo lavoro non sarebbe stato possibile. Vorrei anche ringraziare Claudio Dapiaggi per aver accettato di essere il controrelatore della mia tesi. Infine ringraziamenti a miei amici del dipartimento.


\bigskip


I am grateful to my officemate and collaborator Thomas-Paul Hack, whose has been helpful and disponible. I would like to thank my collaborators Tajron Jurić, Patrizia Vitale, and especially Jean-Christophe Wallet for his help and enthusiasm. I am also grateful to Pierre Martinetti. Furthermore, I would like to thank my friends who made possible for me to achieve this project.


\bigskip


Pour finir je tiens à remercier ma famille, et en particulier mes parents et ma soeur, pour leur soutien. Merci.


%----------------------------------------------------------------------------%
\setcounter{section}{0}
\appendix
%----------------------------------------------------------------------------%


%----------------------------------------------------------------------------%
\chapter{Spacetime}
\label{p:SPACETIME}
%----------------------------------------------------------------------------%


The starting block for a physical theory is the notion of spacetime, it is a set of points (events) located in time and space. In Newton physics, space and time are usually treated separately and, moreover, their role encoded in the axioms of absolute time and of absolute space is passive. This two axioms simply describe the ambient where everything happens. At the beginning of the last century Einstein and others introduced a completely new point of view of these two entities where they are treated on the same level, namely as elements of an unique entity. This new point of view has been an important turning point in the understanding of ``the laws of nature''. 


After that, in the theory of General Relativity, which is nowadays used to describe gravitational interactions, the physical background (i.e. the spacetime) is an ``active actor''. Indeed gravitation roughly speaking can be viewed as a deformation of the spacetime. Therefore we shall introduce the notion of spacetime starting from the very beginning. We shall introduce the mathematical framework to describe the spacetime. First we shall define what is a manifold starting from topological considerations, then implement differential structures, and finally equip the manifold with a causality.


%----------------------------------------------------------------------------%
\section{From topology to manifold}
\label{p:TOPO_M}
%----------------------------------------------------------------------------%


The most fundamental way to define a space is to use the notion of topology. It permits to study spaces that are preserved under different type of deformations. We choose to start with the topological structure in order to define the notion of manifold.


\begin{definition}[Topological space]
Let $\Xsf$ be a set\footnote{It is simply a collection of objects.}. A topology on $\Xsf$ is a collection $\Tcal$ of subsets satisfying the three following axioms,%
%
\begin{itemize}
\item \textbf{conventions} : $\emptyset , \ \Xsf \in \Tcal$ ;
\item \textbf{arbitrary union} : $U_i \in \Tcal \mbox{ for } i \in I \Longrightarrow \bigcup_{i\in I} U_i \in \Tcal$ ;
\item \textbf{finite intersection} : $U_1 , \dots , U_n \in \Tcal \Longrightarrow U_1 \cap \dots \cap U_n \in \Tcal$ ,
\end{itemize}
%
where $I$ is an arbitrary index set.
\end{definition}
%
The pair $(\Xsf,\Tcal)$ is called a \textbf{topological space}. The element of $\Tcal$ are the open sets of $\Xsf$. We shall often omit to specify the topology $\Tcal$, and simply say that $\Xsf$ is a topological space. 


\bigskip


We shall illustrate this definitions by few examples. If $\Xsf$ is a set and $\Tcal$ is the collection of all the subsets of X , then $(\Xsf,\Tcal)$ is a topological space and this topology is called the \textbf{discrete topology}. When $\Tcal$ is just the empty set and $\Xsf$ is the entire space, then $(\Xsf,\Tcal)$ is called the \textbf{trivial topology}. In the case of $\Xsf$ corresponding to the real line $\Rbb$ all open intervals $(a,b)$ and their unions define a topology called the \textbf{usual topology}. 


\bigskip


The notion of topology is still general. It does not implement a lot of structure on theses spaces. For instance the operations between elements in $\Xsf$ are not for now considered. Nonetheless it already characterizes maps between different topological spaces.


\begin{definition}[Continuous map and homeomorphism]
%
Let $\Xsf$ and $\Ysf$ be topological spaces. We consider a map $f : \Xsf \to \Ysf$. We say
%
\begin{itemize}
\item $f$ is \textbf{continuous} if $f^{-1}(U) \subset X$ is open for every open $U \subset\Ysf$ ;
\item $f$ is a \textbf{homeomorphism} if $f$ is bijective and both $f$ and $f^{-1}$ are continuous.
\end{itemize}
%
\end{definition}


If we consider two continuous maps $f : \Xsf \to \Ysf$ and $g : \Ysf \to \Zsf$ between topological spaces, then $f \circ g$ is continuous. An example of a map which is not continuous is a function from $\Rbb$ to $\Rbb$ with the usual topology
%
\begin{equation*}
f(x) = \left\{
\begin{array}{ll}
x & \mbox{ if } \ x \leq 0 \ , \\
x + 2 & \mbox{ if } \ x > 0 \ .
\end{array}
\right.
\end{equation*}
%
In ``usual'' calculus we say that $f$ is discontinuous in $0$. Let us check that this is also the case with the previous definition. The inverse by $f$ of the open $(-1,1)$ is the set $(-1,0]$ which is not an open anymore. Thus the function $f$ is not continuous.


Notice that the previous definition permits to show that any open interval of $\Rbb$ is homeomorphic to any other open interval. In order to prove it consider the function $g : (-1,1) \to (0,2)$ such as
%
\begin{equation*}
g(x) = x + 1 \ . 
\end{equation*}
%
This function is bijective and, $g$ and $g^{-1}$ are continuous. Another possible example for a homeomorphism is the function tangent hyperbolic which maps $\Rbb$ to $(-1,1)$.


\bigskip


Suppose to have two different continuous functions $f$ and $g$, which map a topological space $\Xsf$ to another topological space $\Ysf$, and a continuous function $h : \Xsf \times [0,1] \to \Ysf$, with  $h(x,0) = f(x)$ and $h(x,1) = g(x)$ for all $x \in \Xsf$, we say that $h$ is an \textbf{homotopy}. 

\bigskip


Let us give some generic definitions which shall appear to be useful later on. Below, we shall denote by $\Xsf$ a topological space by $\Tcal$ its topology. Furthermore, $\Zsf$ is a subspace of $\Xsf$.


\begin{itemize}
%
\item $\overline{\Zsf}$ is the \textbf{closure} of $\Zsf \subset \Xsf$, which is obtained as the intersection of all closed sets\footnote{A set $\Csf \subset \Xsf$ is \textbf{closed} if $\Xsf \setminus \Csf$ is open.} containing $\Zsf$.% 
%
\item $\Zsf$ is \textbf{dense} in $\Xsf$ if $\overline{\Zsf} = \Xsf$.%
%
\item $\Bcal \subset \Tcal$ is a \textbf{topological basis} if every elements in $\Tcal$ can be written as the union of elements of the basis $\Bcal$.%
%
\item $\Xsf$ is \textbf{second countable} if it has a countable\footnote{A set is said to be \textbf{countable} if there exists a one to one correspondence between the set considered and the set of natural numbers.} topological basis.%
%
\item $\Xsf$ is \textbf{separable} if there exists a dense and countable subset.%
%
\item $\Ksf \subset \Xsf$ is \textbf{compact} if any covering of it admits a finite subcovering.%
%
\item $\Xsf$ is \textbf{locally compact} if every point in $\Xsf$ admits a neighborhood which has compact closure.%
%
\item $\Xsf$ is \textbf{connected} if it cannot be written as a disjoint union of two nonempty open subsets.%
%
\item $\Xsf$ is \textbf{Hausdorff} if every pair of points have disjoint neighborhoods.%
%
\item $\Xsf$ is \textbf{paracompact} if every open cover has a refinement\footnote{A refinement $C$ of a cover $\Ccal$ is a cover such that every element in $C$ is a subset of an element in $\Ccal$.} covering that is locally finite\footnote{A cover $\left\{U_\alpha\right\}$ of $\Xsf$ is \textbf{locally finite} if every points in $\Xsf$ has a neighborhood which has a nonempty intersection with a finite numbers of $U_\alpha$.}.% 
%
\item $\Xsf$ is said to be \textbf{contractible} if the identity map $\Xsf \to \Xsf$ is homotopic to a constant map.
\end{itemize}


In order to restrict ourselves to a less general picture, we require that the topological spaces we are considering satisfy some separability (Hausdorff) and countability (second countable) condition, so that they look locally like $\Rbb^n$. We have now enough background to introduce the notion of manifold. We start with topological manifolds and we shall implement the differential structure in the next section.%


\bigskip


\begin{definition}[Topological manifold]
A topological manifold $\Mcal$ of dimension $n$ is a Hausdorff and second countable topological space in which every point admits an open neighborhood homeomorphic to a subset of $\Rbb^n$.
\end{definition}


It is possible to show that instead of requiring $\Mcal$ to be Hausdorff and second countable, we could equivalently demand $\Mcal$ to be separable and metrizable. However, these last mentioned conditions are global extents, while what is really important for describing physical theories, is the requirement that $\Mcal$ admits locally the same topological properties as $\Rbb^n$, or in other words that  $\Mcal$ is locally compact, connected, and contractible.


An important tool to pass from a local to a global point of view is the \textbf{partition of unity} of $\Mcal$. 


\begin{definition}[Partition of unity]
Let $\Xsf$ be a topological space and $I$ an index set. A partition of unity on $\Xsf$ is a collection $\{g_i\}_{i\in I}$ of continuous functions $g_i : \Xsf \to \Rbb$ such that
%
\begin{itemize}
\item $0 \leq g_i(x) \leq 1$ for all $x \in \Xsf$ and each $i\in I$,
\item $\sum_i g_i(x) = 1$, for all $x \in \Xsf$ ,
\item for every $x \in \Xsf$, there is only a finite number of maps $g_i$ such that $g_i(x) \neq 0$.
\end{itemize}
%
\end{definition}


A partition of unity defines an open cover of $\Xsf$. We call, $\{g_i\}_{i \in I}$, \textbf{partition of unity subordinate to an open cover} $U=(U_i)_{i \in I}$, if for all $i \in I$, the support of $g_i$ is contained in $U_i$. And we have that a topological manifold $\Mcal$ is paracompact if and only if it admit a partition of unity subordinate to every open cover of $\Mcal$. This is the reason why we shall later require to work with paracompact manifold.


%----------------------------------------------------------------------------%
\section{Lorentzian manifold}
\label{p:LORENTZ_M}
%----------------------------------------------------------------------------%


%----------------------------------------------------------------------------%
\subsection{Smoothness}
\label{p:SMOOTH_M}
%----------------------------------------------------------------------------%


We concluded the previous section introducing the notion of \textbf{topological manifold}. Now we would like to implement a \textbf{differential structure} on it in order to be able for instance to define the notion of tangent spaces. In the next, we shall restrict our analysis to the case of smooth manifold. This particular case shall appear to be enough to introduce the notion of Lorentzian manifold.


\bigskip


Let $\Mcal$ be a topological manifold of dimension $n$. We would like to be able to ``localize'' points on a manifold, to this end we shall introduce the notion of chart. A \textbf{chart} on $\Mcal$ is a pair $(U,\Phi)$, where $U$ is an open subset of $\Mcal$ and $\Phi$ is a homeomorphism from $U$ to an open subset $\Phi(U) \subset \Rbb^n$. We call $U$ a \textbf{coordinate neighborhood}, $\Phi$ a \textbf{coordinate map}, and the component functions of $\Phi=(\Phi_1,\dots,\Phi_n)$ \textbf{local coordinates} on $U$.


\begin{figure}[ht]
\begin{center}
\manifold
\end{center}
\caption{Charts and transition function.}
\end{figure}



We want to define smooth manifold, therefore we need to define one more notion. Let $(\Phi,U)$ and $(\Psi,V)$ be two charts on $\Mcal$ such that $U \cap V \neq \emptyset$. We call \textbf{transition map}, the application
%
\begin{equation*}
\Psi \circ \Phi^{-1} \ : \ \Phi(U \cap V) \subset \Rbb^n \ \to \ \Psi(U \cap V) \subset \Rbb^n \ .
\end{equation*}
%
Since $\Mcal$ is a topological manifold, the previous map is surely a homeomorphism. We say that $(\Phi,U)$ and $(\Psi,V)$ are \textbf{smoothly compatible} if $U \cap V = \emptyset$ or if furthermore the transition map $\Psi \circ \Phi^{-1}$ is a diffeomorphism, i.e. bijective with smooth inverse.


We call an \textbf{atlas} of $\Mcal$ a set of charts $\left\{ (U, \Phi) \right\}$ which covers $\Mcal$. An atlas $\Acal$ is called \textbf{smooth} if any two charts in $\Acal$ are smoothly compatible. A smooth atlas $\Acal$ is called \textbf{maximal} if it is not contained in any strictly larger smooth atlas. We now can give the definition of a smooth manifold.


\begin{definition}[Smooth manifold]
$\Mcal$ is a smooth manifold if it is a topological manifold with a smooth maximal atlas $\left\{(U,\phi)\right\}$.
\end{definition}


One of the first characterization of a smooth manifold $\Mcal$ we can give is its \textbf{orientability}. An \textbf{orientation} of a smooth manifold is the choice of a maximal smooth oriented atlas. A smooth atlas $\{(U,\Phi)\}$ is called oriented if the determinant of the derivatives of all transition maps is positive.


It shall appear useful to define smooth maps between manifolds. We shall also characterize real valued maps on smooth manifolds, and say on which conditions they are smooth and compactly supported.


\begin{definition}[Smooth map]
A map $f : \Mcal \to \Ncal$ between two smooth manifolds is said to be smooth if for every charts $(U,\Phi)$ and $(V,\Psi)$ on $\Mcal$ and $\Ncal$ respectively, the transition function $\Psi \circ f \circ \Phi^{-1}$ is  smooth.
\end{definition}


We notice in particular that if the two manifolds are of the same dimension then $f$ is said to be a \textbf{diffeomorphism}.


\begin{figure}[ht!]
\begin{center}
\manifoldMaps
\end{center}
\caption{Smooth map between manifolds.}
\end{figure}


We shall now proceed defining what is a real valued smooth (compactly supported) function.


\begin{definition}[Smooth - compactly supported - function]
A function $f : \Mcal \to \Rbb$ is said to be \textbf{smooth} if and only if $f \circ \Phi^{-1} : \Phi(U) \subset \Rbb^n \to f(U) \subset \Rbb$ is smooth for each coordinate chart in the atlas.\par%
A function $f : \Mcal \to \Rbb$ is said to be \textbf{compactly supported} if the support of $f : \Mcal \to \Rbb$ (i.e. the closure of the set where $f$ does not vanish) is compact. 
\end{definition}


The set of all real valued smooth functions on $\Mcal$ is denoted by $\Ecal(\Mcal)$, and the one of all real valued smooth compactly supported functions on $\Mcal$ by $\Dcal(\Mcal)$. 


\begin{figure}[ht!]
\begin{center}
\manifoldFunction
\end{center}
\caption{Real valued smooth function.}
\end{figure}


For now on $\Mcal$ shall be understood as a smooth manifold of dimension $n$. 


%----------------------------------------------------------------------------%
\subsection{Lorentzian structure}
\label{p:LORENTZIAN_STRUCTURE}
%----------------------------------------------------------------------------%


Let us consider a smooth manifold $\Mcal$ and and define a \textbf{curve} $\gamma$ passing through $x$ as the map  
%
\begin{equation*}
\gamma : (-1,1) \to \Mcal \ , \ \ \mbox{ such that } \ \gamma(0) = x \in \Mcal \ .
\end{equation*}
%
We consider $\epsilon > 0$ small enough such that $\gamma((-1,1)) \subset U$, for a coordinate neighborhood $U$ of a chart $(U,\Phi)$. We say that two curves $\gamma$ and $\gamma^\prime$ are equivalent if
%
\begin{equation*}
\underset{t \to 0}{\lim} \ \frac{1}{t} \left( \gamma(x+t) - \gamma(x) \right) = \underset{t \to 0}{\lim} \ \frac{1}{t} \left( \gamma^\prime(x+t) - \gamma^\prime(x) \right) \ .
\end{equation*}
%
We call \textbf{tangent space} at $x$, denoted by $T_x\Mcal$, the set formed by all posible equivalence classes at $x$. The tangent space to a point of a manifold has the same dimension as the given manifold.


\begin{figure}[ht!]
\begin{center}
\tangentSpace
\end{center}
\caption{Tangent space}
\end{figure}


Before adding more structure to $T_x \Mcal$ we shall introduce the notion of \textbf{vector bundle}. A \textbf{smooth real vector bundle} is a triple $(E,\Mcal,\pi)$, where $E$ (\textbf{total space}) and $\Mcal$ (\textbf{base space} of dimension $n$) are smooth manifolds and 
%
\begin{equation*}
\pi : E \to \Mcal , 
\end{equation*}
%
a smooth surjection\footnote{A map $f : A \to B$ is a surjection if for any $b \in B$ there exists an $a\in A$ such that $b=f(a)$.} which defines a $n$ dimensional vector space $E_x = \pi^{-1}(\{x\})$ for every $x \in \Mcal$, called the \textbf{fiber} of $E$ at $x\in\Mcal$. Additionally we require that for every $x\in\Mcal$ there exists an open neighborhood $U \subset \Mcal$ and a smooth diffeomorphism $\psi : \pi^{-1}(U) \to U \times E_x$ such that its projection on the first component $\psf\rsf_1$ gives the image of $\pi$. This conditions corresponds to require the ``commutation'' of the diagram \ref{fig:vect_bund_strut}. $\psi$ is called a trivial trivialization.


\begin{figure}[ht!]
\begin{center}
\bundle
\end{center}
\caption{Vector bundle structure.}
\label{fig:vect_bund_strut}
\end{figure}


We shall omit to indicate the corresponding triple when speaking of a vector bundle, and specify only the total space $E$, and if necessary the base space.


We call \textbf{smooth section} of a vector bundle a smooth map $s : \Mcal \to E$, such that $\pi \circ s = \id$. The corresponding vector space is denoted $\Gamma(\Mcal)$. We shall later see the importance of this notion in physics.


There are important particular smooth real vector bundles, the \textbf{tangent bundle} and the \textbf{real line bundle}. This two shall appear to be useful later on. 


The \textbf{tangent bundle} is the triple $T\Mcal=(T_x\Mcal, \Mcal, \pi_t)$ with $\pi_t : T_x\Mcal \to \Mcal$. A section of the tangent bundle is $v : \Mcal \to T_x\Mcal$, it is called the \textbf{vector field}. 


The \textbf{real line bundle} is the triple $(\Rbb, \Mcal, \pi_\ell)$ with $\pi_\ell : \Rbb \to \Mcal$. A section of the real line bundle is $\phi : \Mcal \to \Rbb$, it is called the \textbf{real scalar field}. 


Let us come back to the notion of tangent space. It is a vector space therefore we can consider the dual of it. It is called the \textbf{cotangent space} of $\Mcal$ at $x$, and is denoted by $T^\ast_x\Mcal$. We recall that the dual of vector space is the vector space of the linear applications from this space to $\Rbb$, and it forms also a vector space. We can then consider the \textbf{cotangent bundle}, it is the triple
%
\begin{equation*}
T^\ast\Mcal=(T^\ast_x\Mcal, \Mcal, \pi^\ast_t) \ \mbox{ with } \ \ \pi^\ast_t : T^\ast_x\Mcal \to \Mcal \ .
\end{equation*}
%
A section of $T^\ast\Mcal$ is called a \textbf{covector field}.


Let $E$, $F$, and $G$ be vector spaces. The vector space of the multilinear forms $E \times F \to G$ is called \textbf{tensor product} and denoted $E \otimes F$. We can generalize this notion to the case of $r$ vector spaces, i.e the set of multilinear forms
%
\begin{equation*}
\underbrace{E_1 \times \dots \times E_r}_{r \ \mbox{ times}} \to G \ .
\end{equation*}
%
In the particular case where $E_1 = \dots = E_p = E$ and $E_{p+1} = \dots = E_r = E^\ast$ all these multilinear forms form the tensor product 
%
\begin{equation*}
E_1 \otimes \dots \otimes E_r = E^{\otimes p} \otimes (E^\ast)^{\otimes q} \ ,
\end{equation*}
%
with $q=r-p$. An element of this set is a tensor of type $(p,q)$, $p$ times covariant, and $q$ times contravariant. 


A tensor product is said to be symmetric if it is invariant under permutations of its arguments and it is said to be antisymmetric if it is invariant under even permutations and invariant up to a sign under odd permutations.


A $q$ \textbf{differential form} on $\Mcal$ is a antisymmetric tensor of type $(0,q)$. We shall denote by $\Omega^q(\Mcal)$ the vector space of the $q$ differential forms on $\Mcal$. If $q > n$ with $n$ the dimension of $\Mcal$ then $\Omega^q(\Mcal)=\{0\}$, and $\Omega^0(\Mcal)$ is the space of the smooth functions on $\Mcal$.


On the space of differential form we can define a product, called the \textbf{exterior product}. For every copy of differential forms, $\omega \in \Omega^r(\Mcal)$ and $\eta \in \Omega^s(\Mcal)$, the exterior product is denoted by 
$\omega \wedge \eta \in \Omega^{r+s}(\Mcal)$ and it is defined  as
%
\begin{equation*}
(\omega \wedge \eta)(x_1,\dots,x_{r+s}) = \frac{1}{r!s!} \sum_{\sigma \in \Orak_{r+s}} (-1)^{\mathsf{sign}(\sigma)} \omega(x_{\sigma{1}},\dots,x_{\sigma{r}}) \cdot \eta(x_{\sigma{r+1}},\dots,x_{\sigma{r+s}}) \ .
\end{equation*}


\bigskip


In order to be able to measure the length of tangent vectors, the angles among them and eventually the distance between different points we have to equip our spacetime with a \textbf{metric}. Locally a metric is a \textbf{scalar product}. Let us consider a vector space $E$ of dimension $n$ endowed with a non degenerate scalar product $X,Y \mapsto (X,Y)$. If $\{(e_a)\}$ is an orthonormal basis of $E$ an element $X$ of $E$ can be written as $X = X^a e_a$, where the summation over the indice $a$ is understood. The scalar product can be seen as a bilinear, symmetric, and non degenerate function $g$ such that
%
\begin{equation*}
g : \left\{ 
\begin{array}{lcl}
E \times E & \to & \Rbb \ , \\
(X,Y) & \mapsto & g_{ab} X^a Y^b
\end{array}
\right. \ ,
\end{equation*}
%
where $(g_{ab})$ is a diagonal matrix of size $n \times n$. We notice that in order to implement causality in the general case the scalar product does not need to be positive (look at \ref{p:GLOB_HYPERBOL}). In the present definition $(g_{ab})$ is diagonal\footnote{Every (non degenerate) metric can be diagonalized.} and has only $\pm 1$ on it. The signature of a scalar product is the pair $(r,s)$ where $r$ is the number of $+1$ on the diagonal, and $s$ the number $-1$ on the same diagonal. 


We shall generalized the notion of a metric tensor on a manifold. It is a non degenerate bilinear symmetric tensor of type $(0,2)$, denoted $g$, such that for smooth vector fields $X$ and $Y$, $g(X,Y)$ is a smooth function on $\Mcal$. We can show that the smoothness implies that the signature is constant on any connected component of $\Mcal$. On each point $x$ on $\Mcal$ the metric tensor $g$ defines a scalar product on $T_x\Mcal$. It is common practice to call a metric tensor a metric.


Using the metric we can relate a vector space $E$ to its dual $E^\ast$. Indeed $Y \mapsto g(X,Y) \in \Rbb$ is an element of $E^\ast$. 


We call $g$ a \textbf{riemannian metric} if for every point of the manifold $g$ has signature $n$, the dimension of $\Mcal$. If the associated scalar product has signature equal to $n-1$ for every point of the spacetime we call $g$ a \textbf{pseudo riemannian} (or \textbf{lorentzian}) metric. We shall use for now on only lorentzian metric $g$.

\begin{definition}[Lorentzian manifold]\label{def:lorentzian_M}
A Lorentzian manifold is a couple $(\Mcal,g)$ where $\Mcal$ is a $n$ dimensional smooth manifold, and $g$ is a Lorentzian metric tensor.
\end{definition}


We shall omit to say tensor metric and simply say metric.


%----------------------------------------------------------------------------%
\subsection{Integration}
\label{p:INTEGRATION}
%----------------------------------------------------------------------------%


We define the differential $\dsf$ on a form as the linear application
%
\begin{equation*}
\dsf : \Omega^r(\Mcal) \to \Omega^{r+1}(\Mcal) \ .
\end{equation*}
%
The differential $\dsf : \Omega^0(\Mcal) \to \Omega^1(\Mcal)$ is the differential of functions, i.e. the map from the tangent space to the scalars.


We first consider the particular case $\Mcal=\Rbb^n$, where we have only one chart covering the whole manifold, $\{x_i\}$ being the local coordinates of the coordinate map. Let $\omega$ be a $n$ differential form on $\Rbb^n$, then it can be written as
%
\begin{equation*}
\omega = \omega_{1,\dots,n} \ \dsf x_{1} \wedge \dots \wedge \dsf x_{n} \ .
\end{equation*}
%
The integral over a subset $U$ of $\Rbb^n$ is defined as 
%
\begin{equation*}
\int_U \omega = \int_U \omega_{1,\dots,n} \ \dsf x_{1} \dots \dsf x_{n} \ ,
\end{equation*}
%
where we omit to write the symbol $\wedge$ for the exterior product.
%
In this definition an orientation has been implicitly chosen, i.e. an order in the exterior product of $\{x_i\}$ is taken. Taking another orientation will change the sign of the integral. This is one of the reason while it is important to work with an orientable manifold.


Let us go back to the general situation, where $\Mcal$ is defined as in \ref{def:lorentzian_M}. Furthermore we require $\Mcal$ to be paracompact and oriented. As we know $\Mcal$ looks locally like $\Rbb^n$, actually this was one of the requests in our construction. Therefore it should be possible to define locally an integral on $\Mcal$. In order to map $\Mcal$ to $\Rbb^n$ we consider an atlas $\{(\Phi_i,U_i)\}$ of $\Mcal$. The support of $\omega$ does not lie entirely in one coordinate neighborhood, thus we need to use a partition of unity $\{g_i\}$ subordinate to the covering $\{U_i\}$. Then any other $n$ form $\omega$ can be written as
%
\begin{equation*}
\omega = \sum_{i} g_i \omega \ .
\end{equation*}
%
where the sum is finite because $\supp(\omega)$ is compact. We map $g_i \omega$ on each coordinate neighborhood $U_i$ with $\phi_i : U_i \to W_i \subset \Rbb^n$. Therefore the $n$ form $\phi_i ^\ast (g_i\omega)$ on $W_i$ can be integrated as we have done on $\Rbb^n$. Meaning
%
\begin{equation*}
\int_\Mcal \omega = \sum_i \int_{U_i} g_i \omega = \sum_i \int_{W_i} \phi_i ^\ast (g_i\omega) = \sum_i \int_{W_i} F_i(x_1,\dots,x_n) \ \dsf x_1 \dots \dsf x_n \ ,
\end{equation*} 
%
with $F_i(x_1,\dots,x_n) \ \dsf x_1 \dots \dsf x_n = \phi_i ^\ast (g_i\omega)$. The map $\phi_i ^\ast$ is the pull back of $\phi_i ^\ast$, we have $\phi_i ^\ast(g_i\omega) = (g_i\omega) \circ \phi_i$. It can be shown that this construction does not depend on the chosen partition of unity.


Because the manifold is assumed paracompact, $\Mcal$ is orientable if there exits a non zero smooth $n$ form $\mu$. It is called the volume form of $\Mcal$. Due to the fact that $\Mcal$ is a Lorentzian manifold with tensor metric $g$ we have 
%
\begin{equation*}
\mu = \sqrt{\abs{\det\left(g\right)}} \ \dsf^n x = \sqrt{\abs{g}} \ \dsf^n x \ ,
\end{equation*}
%
We then define the integral of a function $f$ over $\Mcal$ as follows
%
\begin{equation}
\int_\Mcal f = \sum_i \int_{W_i} (g_i f)(x_1,\dots,x_n) \ \sqrt{\abs{g}} \ \dsf x_1 \dots \dsf x_n \ .
\label{eq:int_manifold}
\end{equation}


%----------------------------------------------------------------------------%
\subsection{Covariant derivative, geodesic and curvature}
\label{p:CONNEX_GEOD_CURV}
%----------------------------------------------------------------------------%


We shall introduce the notion of covariant derivative. A \textbf{linear connection} $\nabla$ is a map 
%
\begin{equation*}
\nabla : \left\{
\begin{array}{ccl}
\Gamma(\Mcal) \times \Gamma(\Mcal) & \to & \Gamma(\Mcal) \\
(X,Y) & \mapsto & \nabla_X Y 
\end{array}
\right. \ ,
\end{equation*}
%
such that for every sections $X, Y, Z \in \Gamma(\Mcal)$ and any real valued smooth function $f \in \Ecal(\Mcal)$ we have
%
\begin{eqnarray*}
&& \nabla_{X + Y} Z = \nabla_X Z + \nabla_Y Z \ ; \\ 
&& \nabla_{f X} Y = f \nabla_X Y \ ;\\
&& \nabla_X(Y+Z) = \nabla_X Y + \nabla_X Z \ ;\\
&& \nabla_X(fY) = f \nabla_X Y + (X \cdot f) Y \ .
\end{eqnarray*}
%
The connection $\nabla_X Y$ is called the \textbf{covariant derivative} of $Y$ with respect to $X$. $\nabla$ is said to be compatible with respect to the pseudo riemannian metric $g$ if for all $X, Y, Z \in \Gamma(\Mcal)$ we have
%
\begin{equation*}
X \cdot g(Y,Z) = g(\nabla_X Y, Z) + g(Y,\nabla_X Z) \ .
\end{equation*}
%
We shall say that $\nabla$ is tensor metric connection. The \textbf{torsion} of $\nabla$, $T$, is a tensor of type $(1,2)$ such that for every vector fields $X, Y \in \Gamma(T\Mcal)$ 
%
\begin{equation*}
T(X,Y) = \nabla_X Y - \nabla_Y X - \left[ X,Y\right] \ .
\end{equation*}
%
The connection $\nabla$ is said to be torsion free if its torsion tensor is the zero tensor. A fundamental theorem in pseudo riemannian geometry (see \cite{ONEIL_1983}) tells us that on $\Mcal$ there exists a unique linear connection that is compatible with the metric and torsion free. It is called the \textbf{Levi Civita connection}.


\begin{figure}[ht!]
\centering
\expoMap
\caption{Exponential map.}
\end{figure}


A neighborhood $\Ocal_x \subset \Mcal$ is called a \textbf{geodesically starshaped} with respect to $x \in \Mcal$ if there is an open subset $\Ocal^{\prime}_x$ in $T_xMsf$ which is starshaped\footnote{A vector space $\Xsf$ is called a starshaped set if and only if for all $x\in \Xsf$  we have $-x\in\Xsf$.} with respect to $0 \in T_xM$ such that $\mathsf{exp}_x \ : \ \Ocal^{\prime}_x \ \to \ \Ocal_x$ is a diffeomorphism. $\Ocal \subset \Mcal$ is \textbf{geodesically convex} if it is starshaped with respect to all its points. Notice in particular that every two points $x,y$ in $\Ocal$ are connected by a unique geodesic which is completely contained in $\Ocal$.


If we consider $\Ocal \subset \Mcal$ to be geodesically convex, it makes sense to introduce the \textbf{Synge's world function} (or half of the square geodesic distance) \cite{PPV_2011} due to the uniqueness of the geodesic between two points.


\begin{figure}[ht]
\begin{center}
\geodesic
\end{center}
\caption{Geodesic on a geodesically convex domain.}
\end{figure}



In order to introduce its precise definition, let us fix two points $x'$ and $x$ in a $\Ocal$.
The point $x^\prime$ is called the base point, and $x$ the field point. The geodesic segment $\gamma$ that links $x$ to $x^\prime$ is described by the parametric function $z(t)$, in which $t$ is an affine parameter, $t \in \left[ t_1 , t_2 \right]$, such that $z(t_1) = x$ and $z(t_2) = x^\prime$. The tangent vector to the geodesic on a point $z$ is denoted by $\dot{z}$. The Synge's world function is a scalar function of the base point $x^\prime$ and the field point $x$. It is defined by
%
\begin{equation*}
\sigma(x,x^\prime) =  \frac{1}{2} (t_1 - t_2) \int_{t_1}^{t_2} \dsf t \ g_{\mu \nu} \left( z(t) \right) \ \dot{z}^\mu(t) \ \dot{z}^{\nu}(t) \ .
\end{equation*}
%
In flat space time we have $g_{\mu \nu} \left( z(t) \right) = \eta_{\mu \nu}$, in this case the Synge's world function simplifies to
%
\begin{eqnarray*}
\sigma(x,x^\prime) &=& \frac{1}{2} (t_1 - t_2) \ \eta_{\mu \nu} \ \int_{t_1}^{t_2} \dsf t \ \dot{z}^\mu(t) \ \dot{z}^{\nu}(t) \\
&=& \frac{1}{2} (t_1 - t_2) \ \eta_{\mu \nu} \ (y-x)^\mu \ (x^\prime-x)^\nu \ .
\end{eqnarray*}
%
On the geodesic, the quantity $g_{\mu \nu} \dot{z}^\mu \dot{z}^\nu \doteq \epsilon$ is constant, then $\sigma(x,x^\prime) = \frac{1}{2} (t_2-t_1)^2 \ \epsilon$.


The quantity which characterize how ``deformed'' is $\Mcal$ is called the \textbf{curvature} of $\Mcal$. It is a tensor of type $(1,3)$ such that for every vector fields $X, Y, Z \in \Gamma(T\Mcal)$ and any linear connection we have
%
\begin{equation*}
R(X,Y)Z = \left( \nabla_X \nabla_Y - \nabla_Y \nabla_X - \nabla_{[X,Y]} \right) Z \ .
\end{equation*}
%
We recall that we shall use for now on the Levi Civita connection. 


%----------------------------------------------------------------------------%
\section{Causality}
\label{p:CAUSALITY}
%----------------------------------------------------------------------------%


%----------------------------------------------------------------------------%
\subsection{Global hyperbolicity}
\label{p:GLOB_HYPERBOL}
%----------------------------------------------------------------------------%


We shall now work with the pair $(\Mcal,g)$ which denotes a Lorentzian manifold of dimension $n \geq 2$ together with a Lorentzian metric $g$. We associate to each point $x$ of the manifold its corresponding tangent space $T_x\Mcal$. Considering a vector over a point, namely we can evaluate its Lorentzian scalar product with itself, using the metric $g$. It divides the tangent space in three different regions.


\begin{eqnarray*}
g(v,v) &>& 0 , \ \mbox{ then $v$ is called timelike vector}, \\ 
g(v,v) &=& 0 , \ \mbox{ then $v$ is called null vector}, \\ 
g(v,v) &<& 0 , \ \mbox{ then $v$ is called spacelike vector}.
\end{eqnarray*}


In every tangent space $T_x\Mcal$ the set of timelike vectors, is called light cone, it consists of two connected components. A time orientation on $\Mcal$ is a choice of one of these connected components. Then the light cone is defined as the union of the forward and backward lightcones, 
%
\begin{equation*}
\Vcal=\Vcal^{+} \ \cup \ \Vcal^{-} \ , \quad \mbox{with } \ \Vcal^{\pm}=\left\{ x\in\Mcal \ | \ x^{2}>0, \ \pm x^{0}>0 \right\} \ . 
\end{equation*}


\begin{figure}[ht!]
\begin{center}
\lightcone
\end{center}
\caption{Light cone.}
\end{figure}


A vector $v \in T_x\Mcal$ is \textbf{future} (respectively \textbf{past}) \textbf{directed} if $v$ is a non spacelike vector and contained in $\Vcal^+$ (respectively in $\Vcal^-$). 


A differentiable curve $\gamma(\lambda)$ is said to be 
%
\begin{itemize}
\item a \textbf{future} (respectively \textbf{past}) \textbf{directed timelike curve} if at each point $x(\lambda) \in \gamma$ the tangent vector $v$ is a future (respectively past) directed timelike vector ;
\item a \textbf{future} (respectively \textbf{past}) \textbf{directed causal curve} if at each point $x(\lambda) \in \gamma$ the tangent vector $v$ is either a future (respectively past) directed timelike or null vector. 
\end{itemize}


The \textbf{chronological future} (respectively \textbf{past}) of $x \in \Mcal$ is denoted by $I^{+}(x)$. It is defined as the sets of points which can be reached by a future (respectively past) directed timelike curve starting from $x$,
%
\begin{equation*}
I^{\pm}(x) = \left\{ y \in \Mcal \ \bigg| \ \begin{array}{l} \text{There exists a future (resp. past) directed timelike} \\ \text{curve $\lambda(t)$, with $\lambda(0)=x$ and $\lambda(1)=y$} \end{array} \ \right\}.
\end{equation*}

We define $I^{+}(S) \ = \ \bigcup_{x \in S} I^{\pm}(x)$ for any subset $S \subset \Mcal$.


The causal future/past of a point of the spacetime is defined in a similar way as the chronological future/past of this point, using this time the notion of the causal curve.


The \textbf{causal future} (respectively \textbf{past}) of $x \in \Mcal$, denoted by $J^{+}(x)$, is defined as the sets of points that can be reached by a future (respectively past) directed causal curve starting from $x$,
%
\begin{equation*}
J^{\pm}(x) = \left\{ y \in M \ \bigg| \ \begin{array}{l} \text{There exists a future (respectively past)} \\ \text{directed causal curve $\lambda(t)$, with $\lambda(0)=x$ and $\lambda(1)=y$} \end{array} \; \right\},
\end{equation*}
We define $J^{\pm}(S) \ = \ \bigcup_{x \in S} J^{\pm}(x)$ for any subset $S \subset \Mcal$.


A subset $S \subset M$ is said to be \textbf{achronal} if there do not exist $x, y \in S$ such that $y \in I^{+}(x)$, i.e., if $I^{+}(S) \cap S = \emptyset$. 


We define the \textbf{future} (respectively \textbf{past}) domain of dependence of $S$, denoted by $D^{+}(S)$, by
%
\begin{equation*}
D^{\pm}(S) = \left\{ x \in M \ \bigg| \ \begin{array}{l} \text{Every past (respectively future) inextensible causal curve} \\ \text{through $x$ intersects $S$} \end{array} \; \right\}.
\end{equation*}
%
The (full) \textbf{domain of dependence} of $S$, denoted by $D(S)$, is defined as,
\begin{equation*}
D(S) \ = \ D^{+}(S) \ \cup \ D^{-}(S).
\end{equation*}
The set $S$ is a closed achronal set.


\bigskip


\begin{definition}[Cauchy surface]
A closed achronal set $\Sigma$ for which $D(\Sigma) = M$ is called a Cauchy surface. 
\end{definition}

A spacetime $(\Mcal,\gsf)$ which possesses a Cauchy surface is said to be globally hyperbolic. We invite the reader to look at the end of chapter $8$ of \cite{WALD_1984} for the equivalence of this definition of global hyperbolicity and the ones of Leray, Hawking, and Ellis. 
We have now enough background to define a curved spacetime.

\begin{definition}[Curved spacetime]\label{def:cst}
A pair $(\Mcal,g)$ is a curved spacetime if $\Mcal$ is a $n \geq 2$ dimensional Lorentzian manifold, endowed with a Lorentzian metric of signature $( - + \dots +)$. The spacetime is required to be orientable, paracompact, time orientable, and globally hyperbolic. 
\end{definition}


\begin{definition}[Minkowski spacetime]\label{def:minkowski}
A pair $(\Mbb,\eta)$ is called Minkowski spacetime if $\Mbb$ is the 4 dimensional euclidean space endowed with a metric $\eta$ of signature $(1,3)$ for which only its diagonal elements are non zero.
\end{definition}


The Minkowski spacetime is a particular case of a globally hyperbolic spacetime. 


%----------------------------------------------------------------------------%
\subsection{Friedmann--Lemaître--Robertson--Walker spacetime}
\label{p:FLRW}
%----------------------------------------------------------------------------%


If we look at our universe from different directions it looks very similar, at least on large scales. Furthermore it seems reasonable to assume that there are no special points in it. For these reason it is common to assume the universe to be homogeneous and isotropic. This two requirements imply that the spacetime describing our universe must be a warped product $I\times \Sigma$ where $I$ is an interval of time and $\Sigma$ a three dimensional riemannian manifold of constant curvature which can be open closed or flat. In this work we shall consider only a spatially flat FLRW spacetime, i.e. a curved spacetime $(\Mcal,g)$ equipped with the following metric
%
\begin{equation*}
ds^2 = dt^2 - a(t)^2 d\vec{x}^2 = a(\tau)^2\left(-d\tau^2+d\vec{x}^2\right) \ ,
\end{equation*}
%
where $t$ is the cosmological time, $\tau$ is conformal time, and $a(t)$ is the scale factor whose expansion rate is the Hubble rate
%
\begin{equation*}
H = \partial_t \log (a) = \frac{\partial_\tau a}{a^2} = \frac{\Hcal}{a} \ .
\end{equation*}
%
The time variables are related by
%
\begin{equation*}
dt = a d\tau = \frac{da}{aH} = -\frac{dz}{(1+z)H} \ . 
\end{equation*}
% 
The curvature can also be written in terms of the Hubble rate
%
\begin{equation}
R=6(\partial_t H + 2 H^2)=\frac{\partial^2_\tau a}{a^3} \ . 
\label{eq:rflrw}
\end{equation}
%
We notice that this metric is invariant under translation and rotation of the space coordinates. We have chosen a \textbf{FLRW} spacetime without spatial curvature in order to simplify computations and because observations are compatible with the assumptions of vanishing spatial curvature. 


%----------------------------------------------------------------------------%
\nocite{*}
\bibliographystyle{bibstyle}
\bibliography{biblio}
\addcontentsline{toc}{chapter}{Bibliography}
%----------------------------------------------------------------------------%


%============================================================================%
\end{document}
%============================================================================%